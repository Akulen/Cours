\documentclass[12pt,twoside,a4paper]{article}

\def\chapitre{Complexes}
\author{MPSI 2}
\def\titre{Fonctions a valeurs complexes}

\usepackage{amsfonts}
\usepackage{amsmath}
\usepackage{amsthm}
\usepackage{changepage}
\usepackage{color}
\usepackage{enumitem}
\usepackage{fancyhdr}
\usepackage{framed}
\usepackage[margin=1in]{geometry}
\usepackage{mathrsfs}
\usepackage{tikz, tkz-tab}
\usepackage{titling}

\newtheoremstyle{dotless}{}{}{\itshape}{}{\bfseries}{}{ }{}
\theoremstyle{dotless}

\newtheorem{defs}{Definition}[subsection]
\newenvironment{defi}{\definecolor{shadecolor}{RGB}{255,236,217}\begin{shaded}\begin{defs}\ \\}{\end{defs}\end{shaded}}

\newtheorem{pro}{Propriete}[subsection]
\newenvironment{prop}{\definecolor{shadecolor}{RGB}{230,230,255}\begin{shaded}\begin{pro}\ \\}{\end{pro}\end{shaded}}

\newtheorem{cor}{Corollaire}[subsection]
\newenvironment{coro}{\definecolor{shadecolor}{RGB}{245,250,255}\begin{shaded}\begin{cor}\ \\}{\end{cor}\end{shaded}}

\setlength{\droptitle}{-1in}
\predate{}
\postdate{}
\date{}
\title{\chapitre\\\titre\vspace{-.25in}}

\pagestyle{fancy}
\makeatletter
\lhead{\chapitre\ - \titre}
\rhead{\@author}
\makeatother

\newenvironment{preuve}{\begin{framed}\begin{proof}[\unskip\nopunct]}{\end{proof}\end{framed}}
\newenvironment{liste}{\begin{itemize}[leftmargin=*,noitemsep, topsep=0pt]}{\end{itemize}}
\newenvironment{tab}{\begin{adjustwidth}{.5cm}{}}{\end{adjustwidth}}

\newcommand{\uu}[1] {_{_{#1}}}
\newcommand{\lbracket}{[\![}
\newcommand{\rbracket}{]\!]}
\newcommand{\fonction}[5]{\begin{aligned}[t]#1\colon&#2&&\longrightarrow#3 \\&#4&&\longmapsto#5\end{aligned}}
\newcommand{\systeme}[1]{\left\{\begin{aligned}#1\end{aligned}\right.}
\newcommand{\cercle}[1]{\textcircled{\scriptsize{#1}}}

\newcommand{\lf}[1]{\left(#1\right)}
\newcommand{\C}{\mathbb{C}}
\newcommand{\R}{\mathbb{R}}
\newcommand{\K}{\mathbb{K}}
\newcommand{\N}{\mathbb{N}}
\newcommand{\I}{\mathcal{I}}
\newcommand{\F}{\mathcal{F}}
\newcommand{\E}{\mathcal{E}}
\newcommand{\G}{\mathcal{G}}
\newcommand{\et}{\text{ et }}
\newcommand{\ou}{\text{ ou }}
\newcommand{\xou}{\ \fbox{\text{ou}}\ }


\begin{document}
	\maketitle
	
	\section*{}
		Soit $\mathcal{I}$ un intervalle de $\mathbb{R}$. \\
		Soit $\fonction{f}{\mathcal{I}}{\mathbb{C}}{t}{f\left(t\right)}$ une application definie sur $\mathcal{I}$ et a valeurs complexes.\\
		$$
			\forall t\in\mathcal{I},f\left(t\right)=f\uu1\left(t\right)+\imath f\uu2\left(t\right)
		$$
		On definit ainsi deux fonctions sur $\mathcal{I}$ a valeurs reelles : $f\uu1$ et $f\uu2$. \\
		On a : $\forall t\in\mathcal{I},\systeme{&f\uu1\left(t\right)=\mathcal{R}e\left(f\left(t\right)\right)\\
													&f\uu2\left(t\right)=\mathcal{I}m\left(f\left(t\right)\right)}$
		\begin{defi}
			\begin{liste}
				\item On dit que $f$ admet une limite lorsque $t$ tend vers $t\uu0$ sur $\mathcal{I}$ si les fonctions composantes $f\uu1$ et $f\uu2$ admettent une limite finie quand $t$ tend vers $t\uu0$. \\
					Dans ce ca, la limite de $f\left(t\right)$ quand $t$ tend vers $t\uu0$ est : \\
					$$
						L=L\uu1+iL\uu2\ \ \ \text{ou}\ \ \ L\uu1=\lim\limits_{\substack{t\rightarrow t\uu0\\t\in\mathcal{I}}}f\uu1\left(t\right)\ \text{et}\ L\uu2=\lim\limits_{\substack{t\rightarrow t\uu0\\t\in\mathcal{I}}}f\uu2\left(t\right)
					$$
			\end{liste}
			\textbf{Notation :} $\lim\limits_{\substack{t\rightarrow t\uu0\\t\in\mathcal{I}}}f\left(t\right)=L$
			\begin{liste}
				\item $f$ est continue sur $\mathcal{I}$ si $f\uu1$ et $f\uu2$ sont continues sur $\mathcal{I}$.
				\item $f$ est derivable en $t\uu0$ si $f\uu1$ et $f\uu2$ sont derivables en $t\uu0$. \\
					Dans ce cas, le nombre derive de $f$ en $t\uu0$ est par definition :
					$$
						f'\left(t\uu0\right)=f\uu1'\left(t\uu0\right)+\imath f\uu2'\left(t\uu0\right)
					$$
				\item Soit $F\colon\mathcal{I}\longrightarrow\mathbb{C}$ une application definie sur $\mathcal{I}$ a valeurs dans $\mathbb{C}$. \\
					On dit que $F$ est une primitive de $f$ sur $\mathcal{I}$ si $F$ est derivable sur $\mathcal{I}$ et si :
					$$\forall t\in\mathcal{I},F'\left(t\right)=f\left(t\right)$$
					On a alors :
					$$\forall t\in\mathcal{I},F\uu1'\left(t\right)+\imath F\uu2'\left(t\right)=f\uu1\left(t\right)+\imath f\uu2\left(t\right)$$
				\item Si $f$ est continue sur un segment $\left[a,b\right]$, on pose :
					$$\int\limits_{a}^bf\left(t\right)dt=\int\limits_{a}^bf\uu1\left(t\right)dt+\imath\int\limits_{a}^bf\uu2\left(t\right)dt$$
			\end{liste}
		\end{defi}\ \\
		\textbf{Cas particulier de fonctions a valeurs complexes :} L'exponentielle complexe \\
		Soit $z\uu0=a+\imath b$ un nombre complexe.\\
		\textbf{Cas 1 :} $z\uu0$ reel
		\begin{tab}
			$e^{z\uu0}$ a un sens (exponentielle reelle).
		\end{tab}
		\textbf{Cas 2 :} $z\uu0$ imaginaire
		\begin{tab}
			$\begin{aligned}z\uu0&=\imath\theta \\
						e^{z\uu0}&=e^{\imath\theta}\\
								 &=\cos\theta+\imath\sin\theta\end{aligned}$
		\end{tab}
		\textbf{Cas 3 :}
		\begin{tab}
			Si $z\uu0=a+\imath b$, on pose : $e^{z\uu0}=e^ae^{\imath b}$ \\
			Soit $\fonction{f}{\mathbb{R}}{\mathbb{C}}{x}{e^{z\uu0x}}$ \\
			$f$ est une fonction a valeurs complexes et :
			$$\begin{aligned}
				\forall x\in\mathbb{R},f\left(x\right)&=e^{\left(a+\imath b\right)x} \\
													   &=e^{ax+\imath bx} \\
													   &=e^{ax}e^{\imath bx} \\
													   &=e^{ax}\left(\cos bx+\imath\sin bx\right)
			\end{aligned}$$
			Les applications composantes de $f$ sont :
			$$\fonction{f\uu1}{\mathbb{R}}{\mathbb{R}}{x}{e^{ax}\cos bx} \ \ \ \text{et} \ \ \ \fonction{f\uu1}{\mathbb{R}}{\mathbb{R}}{x}{e^{ax}\sin bx}$$
			$f\uu1$ et $f\uu2$ sont derivables su $\mathbb{R}$ donc $f$ est derivable sur $\mathbb{R}$ et:
			$$\begin{aligned}
				\forall x\in\mathbb{R},&f\uu1'\left(x\right)&&=e^{ax}\left(a\cos bx-b\sin bx\right) \\
										&f\uu2'\left(x\right)&&=e^{ax}\left(a\sin bx+b\cos bx\right) \\
										&f'\left(x\right)&&=e^{ax}\left[\left(a+\imath b\right)\cos bx+\imath\left(a+\imath b\right)\sin bx\right] \\
										&				 &&=\left(a+\imath b\right)e^{ax}\left(\cos bx+\imath\sin bx\right) \\
										&				 &&=\left(a+\imath b\right)e^{\left(a+\imath b\right)x}
			\end{aligned}$$
		\end{tab}
\end{document}
