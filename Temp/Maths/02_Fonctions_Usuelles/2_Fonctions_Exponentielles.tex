\documentclass[12pt,twoside,a4paper]{article}

\def\chapitre{Fonctions Usuelles}
\author{MPSI 2}
\def\titre{Fonctions Exponentielles}

\usepackage{amsfonts}
\usepackage{amsmath}
\usepackage{amsthm}
\usepackage{changepage}
\usepackage{color}
\usepackage{enumitem}
\usepackage{fancyhdr}
\usepackage{framed}
\usepackage[margin=1in]{geometry}
\usepackage{mathrsfs}
\usepackage{tikz, tkz-tab}
\usepackage{titling}

\newtheoremstyle{dotless}{}{}{\itshape}{}{\bfseries}{}{ }{}
\theoremstyle{dotless}

\newtheorem{defs}{Definition}[subsection]
\newenvironment{defi}{\definecolor{shadecolor}{RGB}{255,236,217}\begin{shaded}\begin{defs}\ \\}{\end{defs}\end{shaded}}

\newtheorem{pro}{Propriete}[subsection]
\newenvironment{prop}{\definecolor{shadecolor}{RGB}{230,230,255}\begin{shaded}\begin{pro}\ \\}{\end{pro}\end{shaded}}

\newtheorem{cor}{Corollaire}[subsection]
\newenvironment{coro}{\definecolor{shadecolor}{RGB}{245,250,255}\begin{shaded}\begin{cor}\ \\}{\end{cor}\end{shaded}}

\setlength{\droptitle}{-1in}
\predate{}
\postdate{}
\date{}
\title{\chapitre\\\titre\vspace{-.25in}}

\pagestyle{fancy}
\makeatletter
\lhead{\chapitre\ - \titre}
\rhead{\@author}
\makeatother

\newenvironment{preuve}{\begin{framed}\begin{proof}[\unskip\nopunct]}{\end{proof}\end{framed}}
\newenvironment{liste}{\begin{itemize}[leftmargin=*,noitemsep, topsep=0pt]}{\end{itemize}}
\newenvironment{tab}{\begin{adjustwidth}{.5cm}{}}{\end{adjustwidth}}

\newcommand{\uu}[1] {_{_{#1}}}
\newcommand{\lbracket}{[\![}
\newcommand{\rbracket}{]\!]}
\newcommand{\fonction}[5]{\begin{aligned}[t]#1\colon&#2&&\longrightarrow#3 \\&#4&&\longmapsto#5\end{aligned}}
\newcommand{\systeme}[1]{\left\{\begin{aligned}#1\end{aligned}\right.}
\newcommand{\cercle}[1]{\textcircled{\scriptsize{#1}}}

\newcommand{\lf}[1]{\left(#1\right)}
\newcommand{\C}{\mathbb{C}}
\newcommand{\R}{\mathbb{R}}
\newcommand{\K}{\mathbb{K}}
\newcommand{\N}{\mathbb{N}}
\newcommand{\I}{\mathcal{I}}
\newcommand{\F}{\mathcal{F}}
\newcommand{\E}{\mathcal{E}}
\newcommand{\G}{\mathcal{G}}
\newcommand{\et}{\text{ et }}
\newcommand{\ou}{\text{ ou }}
\newcommand{\xou}{\ \fbox{\text{ou}}\ }


\begin{document}
	\maketitle
	\begin{defi}
		\begin{liste}
			\item La fonction exponentielle est l'application reciproque de $ln$, notee $exp$.
			\item $\forall a\in\mathbb{R}^{+*}\setminus\{1\}$, la fonction exponentielle en base $a$ est l'application reciproque de $log_a$. On note $exp_e=exp$
		\end{liste}
	\end{defi}
	\begin{prop}
		\begin{liste}
			\item $exp$ realise une bijection de $\mathbb{R}^{+*}$ sur $\mathbb{R}$
			\item $\forall x\in\mathbb{R},\ exp_a'(x)=ln(a)\ exp_a(x)$
			\item $exp_a$ est de classe $\mathcal{C}^\infty$ sur $\mathbb{R}$.
		\end{liste}
	\end{prop}
	\begin{preuve}
			\begin{liste}
				\item Par le theoreme des fonctions reciproques.
				\item Par reccuerence.
			\end{liste}
	\end{preuve}
	\textbf{Valeurs remarquables:}
	\begin{tab}\begin{liste}
		\item $exp_a(0)=1$
		\item $exp_a(1)=a$
	\end{liste}\end{tab}
	\begin{prop}\begin{liste}
		\item $\forall x\in\mathbb{R}^{+*}\setminus\{1\},\ \forall(x,y)\in\mathbb{R}^2,\ exp_a(x+y)=exp_a(x)\ exp_a(y)$\\
		\item $\forall(a,b)\in\left(\mathbb{R}^{+*}\setminus\{1\}\right)^2,\ \forall x\in\mathbb{R},\ exp_{ab}(x)=exp_a(x)\ exp_b(x)$
	\end{liste}\end{prop}
	\begin{preuve}
		Application de log aux quantites:\\
		\textbf{1/} Soit a et b deux reels:
		\begin{tab}$\begin{aligned}
			&log_a(exp_a(x+y))=x+y\\
			&\begin{aligned}log_a(exp_a(x)\ exp_a(y))&=log_a(exp_a(x))+log_a(exp_a(y))\\
				&=x+y\end{aligned}\end{aligned}$\\
			Les deux quantites sont egales et \textbf{log est bijective} donc:\\
			$exp_a(x+y)=exp_a(x)\ exp_a(y)$
		\end{tab}
		\textbf{2/} On utilise $log_{ab}$\\
		\begin{tab}$\begin{aligned}
				&log_{ab}(x)=x\\
				&\begin{aligned}log_{ab}(exp_a(x\ exp_b(x))&=log_{ab}(exp_a(x))+log_{ab}(exp_b(x))\\
					&=\frac{ln(a)}{ln(ab)}x+\frac{ln(b)}{ln(ab)}x\\
					&=\frac{ln(a)+ln(b)}{ln(ab)}x\\
					&=x
		\end{aligned}\end{aligned}$\\\\
		Les deux quantites sont egales, donc:\\
		$\forall(a,b)\in\left(\mathbb{R}^{+*}\setminus\{1\}\right)^2,\ \forall x\in\mathbb{R},\ exp_{ab}(x)=exp_a(x)\ exp_b(x)$\end{tab}		
	\end{preuve}
	\textbf{Remarques}\\
	\begin{liste}\item $exp_a(x)=exp(x\ ln(a))$
		\item $\forall a\in\mathbb{R}^{+*}\setminus\{1\},\ \forall x\in\mathbb{R},\ a^x=exp_a(x)$\\
			D'ou les proprietes classiques sur les exposants
	\end{liste}
\end{document}