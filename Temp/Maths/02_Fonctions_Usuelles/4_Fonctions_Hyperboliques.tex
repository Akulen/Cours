\documentclass[12pt,twoside,a4paper]{article}

\def\chapitre{Fonctions Usuelles}
\author{MPSI 2}
\def\titre{Fonctions Hyperboliques}

\usepackage{amsfonts}
\usepackage{amsmath}
\usepackage{amsthm}
\usepackage{changepage}
\usepackage{color}
\usepackage{enumitem}
\usepackage{fancyhdr}
\usepackage{framed}
\usepackage[margin=1in]{geometry}
\usepackage{mathrsfs}
\usepackage{tikz, tkz-tab}
\usepackage{titling}

\newtheoremstyle{dotless}{}{}{\itshape}{}{\bfseries}{}{ }{}
\theoremstyle{dotless}

\newtheorem{defs}{Definition}[subsection]
\newenvironment{defi}{\definecolor{shadecolor}{RGB}{255,236,217}\begin{shaded}\begin{defs}\ \\}{\end{defs}\end{shaded}}

\newtheorem{pro}{Propriete}[subsection]
\newenvironment{prop}{\definecolor{shadecolor}{RGB}{230,230,255}\begin{shaded}\begin{pro}\ \\}{\end{pro}\end{shaded}}

\newtheorem{cor}{Corollaire}[subsection]
\newenvironment{coro}{\definecolor{shadecolor}{RGB}{245,250,255}\begin{shaded}\begin{cor}\ \\}{\end{cor}\end{shaded}}

\setlength{\droptitle}{-1in}
\predate{}
\postdate{}
\date{}
\title{\chapitre\\\titre\vspace{-.25in}}

\pagestyle{fancy}
\makeatletter
\lhead{\chapitre\ - \titre}
\rhead{\@author}
\makeatother

\newenvironment{preuve}{\begin{framed}\begin{proof}[\unskip\nopunct]}{\end{proof}\end{framed}}
\newenvironment{liste}{\begin{itemize}[leftmargin=*,noitemsep, topsep=0pt]}{\end{itemize}}
\newenvironment{tab}{\begin{adjustwidth}{.5cm}{}}{\end{adjustwidth}}

\newcommand{\uu}[1] {_{_{#1}}}
\newcommand{\lbracket}{[\![}
\newcommand{\rbracket}{]\!]}
\newcommand{\fonction}[5]{\begin{aligned}[t]#1\colon&#2&&\longrightarrow#3 \\&#4&&\longmapsto#5\end{aligned}}
\newcommand{\systeme}[1]{\left\{\begin{aligned}#1\end{aligned}\right.}
\newcommand{\cercle}[1]{\textcircled{\scriptsize{#1}}}

\newcommand{\lf}[1]{\left(#1\right)}
\newcommand{\C}{\mathbb{C}}
\newcommand{\R}{\mathbb{R}}
\newcommand{\K}{\mathbb{K}}
\newcommand{\N}{\mathbb{N}}
\newcommand{\I}{\mathcal{I}}
\newcommand{\F}{\mathcal{F}}
\newcommand{\E}{\mathcal{E}}
\newcommand{\G}{\mathcal{G}}
\newcommand{\et}{\text{ et }}
\newcommand{\ou}{\text{ ou }}
\newcommand{\xou}{\ \fbox{\text{ou}}\ }


\begin{document}
	\maketitle

	\section{Cosinus hyperbolique et Sinus hyperbolique}
		\begin{defi}
			On appelle cosinus et sinus hyperboliques les parties paires et impaires de l'exponentielle:
			$$\forall x\in\mathbb{R},ch\left(x\right)=\frac{e^x+e^{-x}}{2} \ \ \ \text{et} \ \ \ sh\left(x\right)=\frac{e^x-e^{-x}}{2}$$
		\end{defi}
		
		\subsection{Parties paires et impaires d'une fonction}
			\begin{prop}
				Soit $f\colon\mathbb{R}\longrightarrow\mathbb{R}$ une application definie sur $\mathbb{R}$ et a valeurs reelles. \\
				Alors il existe un unique couple d'applications definies sur $\mathbb{R}$ et a valeurs dans $\mathbb{R}$, $\left(g,h\right)$, telles que :\\
				$$\systeme{&\forall x\in\mathbb{R},f\left(x\right)=g\left(x\right)+h\left(x\right) \\
						   &g\ \text{est paire et }h\ \text{est impaire}}$$
				L'application $g$ s'appelle la partie paire de $f$ et $h$ la partie impaire de $f$.
			\end{prop}
			\begin{preuve}
				Soit $f\colon\mathbb{R}\longrightarrow\mathbb{R}$ une application fixee.\\ \ \\
				\cercle{1} Supposons qu'il existe deux applications definies sur $\mathbb{R}$ a valeurs dans $\mathbb{R}$ telles que:
				\begin{tab}
					$$\systeme{&f\left(x\right)=g\left(x\right)+h\left(x\right) \\
							   &g\ \text{est paire et }h\ \text{est impaire}}$$
					Alors pour x reel,
					$$\begin{aligned}
						f\left(x\right)&=g\left(x\right)+h\left(x\right) \\
						f\left(-x\right)&=g\left(x\right)-h\left(x\right)
					\end{aligned}$$
					Ainsi, $g\left(x\right)$ et $h\left(x\right)$ verifient :
					$$\systeme{f\left(x\right)&=g\left(x\right)+h\left(x\right) \\
							   f\left(-x\right)&=g\left(x\right)-h\left(x\right)}$$
					Donc : $\begin{aligned}[t]g\left(x\right)&=\frac{f\left(x\right)+f\left(-x\right)}{2} \\
								 \text{et }h\left(x\right)&=\frac{f\left(x\right)-f\left(-x\right)}{2}\end{aligned}$\newpage \ \\
					\textbf{Conclusion 1 :} 
					\begin{tab}
						Si $g$ et $h$ existent, alors :
						$$\fonction{g}{\mathbb{R}}{\mathbb{R}}{x}{\frac{f\left(x\right)+f\left(-x\right)}{2}} \ \ \ \text{et} \ \ \ \fonction{h}{\mathbb{R}}{\mathbb{R}}{x}{\frac{f\left(x\right)-f\left(-x\right)}{2}}$$
					\end{tab}
					\textbf{Conclusion 2 :} 
					\begin{tab}
						En particulier, si $g$ et $h$ existent, alors ils sont uniques.
					\end{tab}\ \\
				\end{tab}
				\cercle{2} On considere $g$ et $h$ les deux applications definies sur $\mathbb{R}$ par :
				\begin{tab}
					$$\forall x\in\mathbb{R},g\left(x\right)=\frac{f\left(x\right)+f\left(-x\right)}{2} \ \ \ \text{et} \ \ \ h\left(x\right)=\frac{f\left(x\right)-f\left(-x\right)}{2}$$
					Montrer que $g$ et $h$ verifient :
					$$\systeme{&f\left(x\right)=g\left(x\right)+h\left(x\right) \\
							   &g\ \text{est paire et }h\ \text{est impaire}}$$
					Pour $x$ reel :
					\begin{liste}
						\item $\begin{aligned}[t]g\left(x\right)+h\left(x\right)&=\frac{f\left(x\right)+f\left(-x\right)}{2}+\frac{f\left(x\right)-f\left(-x\right)}{2} \\
																				  &=\frac{1}{2}\left(f\left(x\right)+f\left(x\right)+f\left(-x\right)-f\left(-x\right)\right) \\
																				  &=f\left(x\right)\end{aligned}$
						\item $g\left(-x\right)=\frac{f\left(-x\right)+f\left(x\right)}{2}=g\left(x\right)$ \\
							$h\left(-x\right)=\frac{f\left(-x\right)-f\left(x\right)}{2}=-h\left(x\right)$
					\end{liste}
					Ceci etant valable pour tout x reel, on conclut que :
					$$\systeme{&f\left(x\right)=g\left(x\right)+h\left(x\right) \\
							   &g\ \text{est paire et }h\ \text{est impaire}}$$
					\textbf{Conclusion Generale :}
					\begin{tab}
						Il existe un unique couple d'application $\left(g,h\right)$ tel que :
						$$\systeme{&f\left(x\right)=g\left(x\right)+h\left(x\right) \\
								   &g\ \text{est paire et }h\ \text{est impaire}}$$
					\end{tab}
				\end{tab}
			\end{preuve}\ \\ \ \\
			\textbf{Retour aux fonctions hyperboliques :} \\
			$$\forall x\in\mathbb{R},ch\left(x\right)=\frac{e^x+e^{-x}}{2} \ \ \ \text{et} \ \ \ sh\left(x\right)=\frac{e^x-e^{-x}}{2}$$
			D'apres la propriete precedente, $ch$ est paire et $sh$ est impaire.
			\begin{prop}
				\begin{liste}
					\item $\forall x\in\mathbb{R},ch\left(x\right)\geq0$
					\item $ch$ et $sh$ sont deux applications definies sur $\mathbb{R}$ et : $\begin{aligned}[t]ch'=sh\\
																												   sh'=ch\end{aligned}$
					\item $ch$ et $sh$ sont deux fonctions de classe $\mathscr{C}^\infty$ sur $\mathbb{R}$
				\end{liste}
			\end{prop} \ \\
			\textbf{Etude de sh :}\\ \ \\
			\begin{tikzpicture}
				\tkzTabInit{$x$/1,$ch\left(x\right)$/1,$sh\left(x\right)$/2}{$0$,$+\infty$}
				\tkzTabLine{1,+,}
				\tkzTabVar{-/$0$,+/$+\infty$}
			\end{tikzpicture}\\ \ \\
			\textbf{Etude de ch :}\\ \ \\
			\begin{tikzpicture}
				\tkzTabInit{$x$/1,$sh\left(x\right)$/1,$ch\left(x\right)$/2}{$0$,$+\infty$}
				\tkzTabLine{0,+,}
				\tkzTabVar{-/$1$,+/$+\infty$}
			\end{tikzpicture}
			\begin{prop}
				Pour tout x dans $\mathbb{R}$ :
				\begin{liste}
					\item $ch\left(x\right)+sh\left(x\right)=e^x$
					\item $ch\left(x\right)-sh\left(x\right)=e^{-x}$
					\item $ch^2\left(x\right)-sh^2\left(x\right)=1$
				\end{liste}
			\end{prop}\newpage
	
	\section{Tangente hyperbolique}
		\begin{defi}
			La fonction tangente hyperbolique est definie par $th=\frac{sh}{ch}$.\\
			C'est une application definie sur $\mathbb{R}$ car $ch>0$ sur $\mathbb{R}$.\\
			On a, pour tout x reel, $\begin{aligned}[t]th\left(x\right)&=\frac{e^{2x}-1}{e^{2x}+1}\\
																		 &=\frac{1-e^{-2x}}{1+e^{-2x}}\end{aligned}$
		\end{defi}
		\textbf{Etude de th :}\\ \ \\
		\begin{tikzpicture}
			\tkzTabInit{$x$/1,$th'\left(x\right)$/1,$th\left(x\right)$/2}{$0$,$+\infty$}
			\tkzTabLine{1,+,}
			\tkzTabVar{-/$0$,+/$1$}
		\end{tikzpicture}
	
	\section{Fonctions circulaires reciproques}
		\subsection{arcsinus et arccosinus}
			$\sin$ est definie continue strictement croissante sur $\left[-\frac{\pi}{2},\frac{\pi}{2}\right]$ et $\sin\lf{\frac{\pi}{2}}=1$ et $\sin\lf{-\frac{\pi}{2}}=-1$. \\
			Donc $\sin$ realise une bijection de $\left[-\frac{\pi}{2},\frac{\pi}{2}\right]$ sur $[-1,1]$.
			\begin{defi}
				La fonction arcsinus est la fonction reciproque de $\sin_{|_{\left[-\frac{\pi}{2},\frac{\pi}{2}\right]}}$.
				$$\fonction{\arcsin}{[-1,1]}{\left[-\frac{\pi}{2},\frac{\pi}{2}\right]}{x}{\arcsin\lf{x}}$$
			\end{defi}\newpage
			$\cos$ est definie continue strictement decroissante sur $\left[0,\pi\right]$ et $\cos\lf{0}=1$ et $\cos\lf{\pi}=-1$.\\
			Donc $\cos$ realise une bijection de $\left[0,\pi\right]$ sur $[-1,1]$.
			\begin{defi}
				La fonction arccosinus est la fonction reciproque de $\cos_{|_{\left[0,\pi\right]}}$.
				$$\fonction{\arccos}{[-1,1]}{\left[0,\pi\right]}{x}{\arccos\lf{x}}$$
			\end{defi}
			\textbf{Remarques :}
			\begin{liste}
				\item $\forall x\in[-1,1],\sin\lf{\arcsin(x)}=x$
				\item $\forall\theta\in\left[-\frac{\pi}{2},\frac{\pi}{2}\right],\arcsin\lf{\sin(\theta)}=\theta$
				\item Si $\alpha$ appartient a $\left[\frac{\pi}{2},\frac{3\pi}{2}\right],\arcsin\lf{\sin(\alpha)}=\pi-\alpha$
				\item $\forall x\in[-1,1],\cos\lf{\arccos(x)}=x$
				\item $\forall\theta\in\left[0,\pi\right],\arccos\lf{\cos(\theta)}=\theta$
				\item Si $\alpha$ appartient a $\left[-\pi,0\right],\arccos\lf{\cos(\alpha)}=-\alpha$
				\item Pour $x$ dans $[-1,1],\cos\lf{\arcsin(x)}=\sqrt{1-x^2}$
				\item Pour $x$ dans $[-1,1],\sin\lf{\arccos(x)}=\sqrt{1-x^2}$
				\item $\forall x\in[-1,1],\arcsin(x)+\arccos(x)=\frac{\pi}{2}$
				\item $\forall x\in[-1,1],\arcsin'(x)=\frac{1}{\sqrt{1-x^2}}$
				\item $\forall x\in[-1,1],\arccos'(x)=\frac{1}{\sqrt{1-x^2}}$
			\end{liste}
		\subsection{arctangente}
			$\tan$ realise une bijection de $\left]-\frac{\pi}{2},\frac{\pi}{2}\right[$ sur $\R$.
			\begin{defi}
				La fonction arctangente est la fonction reciproque de $\tan_{|_{\left]-\frac{\pi}{2},\frac{\pi}{2}\right[}}$.
				$$\fonction{\arctan}{[-1,1]}{\left]-\frac{\pi}{2},\frac{\pi}{2}\right[}{x}{\arctan\lf{x}}$$
			\end{defi}
			\textbf{Remarques :}
			\begin{liste}
				\item $\forall x\in\R,\tan\lf{\arctan(x)}=x$
				\item $\forall\theta\in\left]-\frac{\pi}{2},\frac{\pi}{2}\right[,\arctan\lf{\tan(\theta)}=\theta$
				\item Pour $x$ reel, $\cos\lf{\arctan(x)}=\frac{1}{\sqrt{1+x^2}}$
				\item $\forall x\in\R,\arctan'(x)=\frac{1}{\sqrt{1+x^2}}$
			\end{liste}
\end{document}
