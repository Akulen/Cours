\documentclass[12pt,twoside,a4paper]{article}

\def\chapitre{Suites R\'eelles}
\author{MPSI 2}
\def\titre{Br\`eve extention aux autres suites}

\usepackage{amsfonts}
\usepackage{amsmath}
\usepackage{amsthm}
\usepackage{changepage}
\usepackage{color}
\usepackage{enumitem}
\usepackage{fancyhdr}
\usepackage{framed}
\usepackage[margin=1in]{geometry}
\usepackage{mathrsfs}
\usepackage{tikz, tkz-tab}
\usepackage{titling}

\newtheoremstyle{dotless}{}{}{\itshape}{}{\bfseries}{}{ }{}
\theoremstyle{dotless}

\newtheorem{defs}{Definition}[subsection]
\newenvironment{defi}{\definecolor{shadecolor}{RGB}{255,236,217}\begin{shaded}\begin{defs}\ \\}{\end{defs}\end{shaded}}

\newtheorem{pro}{Propriete}[subsection]
\newenvironment{prop}{\definecolor{shadecolor}{RGB}{230,230,255}\begin{shaded}\begin{pro}\ \\}{\end{pro}\end{shaded}}

\newtheorem{cor}{Corollaire}[subsection]
\newenvironment{coro}{\definecolor{shadecolor}{RGB}{245,250,255}\begin{shaded}\begin{cor}\ \\}{\end{cor}\end{shaded}}

\setlength{\droptitle}{-1in}
\predate{}
\postdate{}
\date{}
\title{\chapitre\\\titre\vspace{-.25in}}

\pagestyle{fancy}
\makeatletter
\lhead{\chapitre\ - \titre}
\rhead{\@author}
\makeatother

\newenvironment{preuve}{\begin{framed}\begin{proof}[\unskip\nopunct]}{\end{proof}\end{framed}}
\newenvironment{liste}{\begin{itemize}[leftmargin=*,noitemsep, topsep=0pt]}{\end{itemize}}
\newenvironment{tab}{\begin{adjustwidth}{.5cm}{}}{\end{adjustwidth}}

\newcommand{\uu}[1] {_{_{#1}}}
\newcommand{\lbracket}{[\![}
\newcommand{\rbracket}{]\!]}
\newcommand{\fonction}[5]{\begin{aligned}[t]#1\colon&#2&&\longrightarrow#3 \\&#4&&\longmapsto#5\end{aligned}}
\newcommand{\systeme}[1]{\left\{\begin{aligned}#1\end{aligned}\right.}
\newcommand{\cercle}[1]{\textcircled{\scriptsize{#1}}}

\newcommand{\lf}[1]{\left(#1\right)}
\newcommand{\C}{\mathbb{C}}
\newcommand{\R}{\mathbb{R}}
\newcommand{\K}{\mathbb{K}}
\newcommand{\N}{\mathbb{N}}
\newcommand{\I}{\mathcal{I}}
\newcommand{\F}{\mathcal{F}}
\newcommand{\E}{\mathcal{E}}
\newcommand{\G}{\mathcal{G}}
\newcommand{\et}{\text{ et }}
\newcommand{\ou}{\text{ ou }}
\newcommand{\xou}{\ \fbox{\text{ou}}\ }


%Auteur: Cl\'ement Phan, MPSI 2

\begin{document}
	\maketitle
	\section{Familles index\'ees par $\Z$}
		Soit $(c_n)_{n\in\Z}$\\
		On se ram\`ene \`a l'\'etude de deux suites:\\
		$\forall n\in\N,\ \systeme{&a_n=c_n\\&b_n=c_{-n}}$
	\section{Suites a valeurs complexes}
		On a:$\fonction{f}{\N}{\C}{n}{z_n}$\\
		Avec $z_n=x_n+i\,y_n$, $x$ et $y$ deux suites r\'eelles composantes.
		\begin{defi}
			Soit $z$ une suite a valeurs complexes, avec $\forall n\in\N,\ z_n=x_n+i_,y_n$\\
			On dit que $z$ converge vers $a+i\,b$ si et seulement si $x$ tend vers $a$ et $y$ tend vers $b$.
		\end{defi}
		\begin{prop}
			$z$ converge vers $\alpha$ ssi $(|z_n-\alpha|)_{n\in\N}$ converge vers $0$
		\end{prop}
		\begin{preuve}
			\begin{liste}
				\item[\cercle1]Supposons que $z$ converge vers $\alpha=a+i\,b$\\
					Montrer que $(|z_n-\alpha|)_{n\in\N}$ converge vers $0$\\
					$|z_n-\alpha|=|x_n-a+i(y_n-b)|$\\
					$\Rightarrow0\leqslant|z_n-\alpha|\leqslant|x_n-a|+|y_n-b|$\\
					Or, $x$ tend vers $a$ et $y$ tend vers $b$\\
					Donc par encadrement, on a: $(|z_n-\alpha|)\mathop{\longrightarrow}\limits_{n\rightarrow+\infty}0$
				\item[\cercle2]Supposons que $(|z_n-\alpha|)_{n\in\N}$ converge vers $0$.\\
					Montrons que $z$ converge vers $\alpha=a+i\,b$\\
					Sachant que $z=x+i\,y\Rightarrow(x\leqslant|x|\leqslant|z|)\et(y\leqslant|y|\leqslant|z|)$\\
					On obtient:$\left\{\begin{aligned}
						& 0\leqslant|x_n-a|\leqslant|z_n-\alpha|\\
						& 0\leqslant|y_n-b|\leqslant|z_n-\alpha|
					\end{aligned}\right.$\\
					Donc par encadrement, $x$ tend vers $a$ et $y$ tend vers $b$.
			\end{liste}
		\end{preuve}
		\begin{flushleft}
			\textbf{Remarque:} une suite complexe est born\'ee si son module est major\'e.
		\end{flushleft}
		\begin{prop}
			Si $z$ converge vers $\lambda$, alors:\\
			\textbullet $|z_n|\mathop{\longrightarrow}\limits_{n\rightarrow+\infty}|\lambda|$\\
			\textbullet $\overline{z_n}\mathop{\longrightarrow}\limits_{n\rightarrow+\infty}\overline{\lambda}$
		\end{prop}
		\begin{prop}
				Soit $z$ une suite complexe born\'ee.\\
				Alors il existe une suite extraite de $z$ convergente.
		\end{prop}
		\begin{preuve}
			\begin{liste}
				\item D'apr\`es le th\'eor\`eme de Bolzano-Weierstrass, il existe une suite extraite de $x$ convergente.\\
					Soit $(x_{\phi(n)})_{n\in\N}$ une telle suite, et $l_1$ sa limite.
				\item Notons $\forall n\in\N,\ v_n=y_{\phi(n)}$\\
					$y$ est born\'ee, donc $v$ est born\'ee.\\
					D'apr\`es le th\'eor\`eme de Bolzano-Weierstrass, il existe une suite extraite de $v$ convergente.\\
					Soit $(v_{\psi(n)})_{n\in\N}$ une telle suite, et $l_2$ sa limite.\\
					$\phi\circ\psi$ est d\'efinie dans $\N$ a valeurs dans $N$ et strictement croissante\\
					Donc $(y_{\phi(\psi(n))})_{n\in\N}$ est une suite extraite de $y$ et convergeant vers $l_2$.
				\item $(x_{\phi(\psi(n))})_{n\in\N}$ est une suite extraite de $x$,
					Donc elle converge vers $l_1$
				\item Donc:$\left\{\begin{aligned}& x_{\phi(\psi(n))} \mathop{\longrightarrow}\limits_{n\rightarrow+\infty} l_1\\ & y_{\phi(\psi(n))} \mathop{\longrightarrow}\limits_{n\rightarrow+\infty} l_2 \end{aligned}\right.$
			\end{liste}
			\begin{flushleft}
				Donc par d\'efinition de la convergence complexe, $z_{\phi(\psi(n))}\mathop{\longrightarrow}\limits_{n\rightarrow+\infty}l_1+i\,l_2$
			\end{flushleft}
		\end{preuve}
\end{document}