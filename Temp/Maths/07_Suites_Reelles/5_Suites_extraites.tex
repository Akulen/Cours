\documentclass[12pt,twoside,a4paper]{article}

\def\chapitre{Suites R\'eelles}
\author{MPSI 2}
\def\titre{Suites extraites}

\usepackage{amsfonts}
\usepackage{amsmath}
\usepackage{amsthm}
\usepackage{changepage}
\usepackage{color}
\usepackage{enumitem}
\usepackage{fancyhdr}
\usepackage{framed}
\usepackage[margin=1in]{geometry}
\usepackage{mathrsfs}
\usepackage{tikz, tkz-tab}
\usepackage{titling}

\newtheoremstyle{dotless}{}{}{\itshape}{}{\bfseries}{}{ }{}
\theoremstyle{dotless}

\newtheorem{defs}{Definition}[subsection]
\newenvironment{defi}{\definecolor{shadecolor}{RGB}{255,236,217}\begin{shaded}\begin{defs}\ \\}{\end{defs}\end{shaded}}

\newtheorem{pro}{Propriete}[subsection]
\newenvironment{prop}{\definecolor{shadecolor}{RGB}{230,230,255}\begin{shaded}\begin{pro}\ \\}{\end{pro}\end{shaded}}

\newtheorem{cor}{Corollaire}[subsection]
\newenvironment{coro}{\definecolor{shadecolor}{RGB}{245,250,255}\begin{shaded}\begin{cor}\ \\}{\end{cor}\end{shaded}}

\setlength{\droptitle}{-1in}
\predate{}
\postdate{}
\date{}
\title{\chapitre\\\titre\vspace{-.25in}}

\pagestyle{fancy}
\makeatletter
\lhead{\chapitre\ - \titre}
\rhead{\@author}
\makeatother

\newenvironment{preuve}{\begin{framed}\begin{proof}[\unskip\nopunct]}{\end{proof}\end{framed}}
\newenvironment{liste}{\begin{itemize}[leftmargin=*,noitemsep, topsep=0pt]}{\end{itemize}}
\newenvironment{tab}{\begin{adjustwidth}{.5cm}{}}{\end{adjustwidth}}

\newcommand{\uu}[1] {_{_{#1}}}
\newcommand{\lbracket}{[\![}
\newcommand{\rbracket}{]\!]}
\newcommand{\fonction}[5]{\begin{aligned}[t]#1\colon&#2&&\longrightarrow#3 \\&#4&&\longmapsto#5\end{aligned}}
\newcommand{\systeme}[1]{\left\{\begin{aligned}#1\end{aligned}\right.}
\newcommand{\cercle}[1]{\textcircled{\scriptsize{#1}}}

\newcommand{\lf}[1]{\left(#1\right)}
\newcommand{\C}{\mathbb{C}}
\newcommand{\R}{\mathbb{R}}
\newcommand{\K}{\mathbb{K}}
\newcommand{\N}{\mathbb{N}}
\newcommand{\I}{\mathcal{I}}
\newcommand{\F}{\mathcal{F}}
\newcommand{\E}{\mathcal{E}}
\newcommand{\G}{\mathcal{G}}
\newcommand{\et}{\text{ et }}
\newcommand{\ou}{\text{ ou }}
\newcommand{\xou}{\ \fbox{\text{ou}}\ }


%Auteur: Cl\'ement Phan, MPSI 2

\begin{document}
	\maketitle
	\section{Définition}
		\begin{defi}
			On dit que \underline{$v$ est une suite extraite de $u$} si il existe une application $\phi$ strictement croissante telle que $\forall n\in\N,\ v_n=u_{\phi(n)}$
		\end{defi}
		\begin{flushleft}
			On appelle également $v$ une sous-suite de $u$.\\
			On a notamment:\\
			\textbullet\ $(w_n)_{n\in\N}$ la suite des termes d'indices pairs de $u$\\
			\textbullet\ $(z_n)_{n\in\N}$ la suite des termes d'indices impairs de $u$
		\end{flushleft}
	\section{Propriétés de limites}
		\begin{prop}
			\textbf{Lemme:} Si $\phi\ \N\rightarrow\N$ est strictement croissante, alors $\phi(n)\geqslant n$
		\end{prop}
		\begin{preuve}
			Par récurrence, avec $\phi(0)\geqslant 0$ et $\phi(n+1)>\phi(n)$
		\end{preuve}
		\begin{prop}
			Si $u$ tend vers $l$ avec $l\in\overline{\R}$, alors toute suite extraite de $u$ tend vers $l$
		\end{prop}
		\begin{preuve}
			\begin{liste}
				\item[\cercle1] \underline{$1^{\text{er}}$ cas:} $l\in\R$\\
					Soit $v$ une suite extraite de $u$, et $\phi : n\rightarrow\N$ une application strictement croissante.\\
					Montrer que $v$ converge vers $l$\\
					Soit $\varepsilon$ un réel strictement positif.\\
					Donc $\exists n_0\in\N,\ \forall n\in\N,\ n\geqslant n_0\Rightarrow |u_n-l|<\varepsilon$\\
					Soit $n_0$ un tel entier. On a $\forall n\in\N,\ n\geqslant n_0\Rightarrow n_0\leqslant n\leqslant\phi(n)$\\
					On a donc $\forall n\in\N,\ n\geqslant n_0\Rightarrow |u_{\phi(n)}-l|<\varepsilon$\\
					Donc $v$ converge vers $l$.
				\item[\cercle2]On procède de manière analogue avec $l=\pm\infty$
			\end{liste}
		\end{preuve}
\end{document}