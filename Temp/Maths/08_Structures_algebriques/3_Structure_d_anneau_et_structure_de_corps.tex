\documentclass[12pt,twoside,a4paper]{article}

\def\chapitre{Structures Alg\'ebriqes}
\author{MPSI 2}
\def\titre{Structure d'anneau et structure de corps}

\usepackage{amsfonts}
\usepackage{amsmath}
\usepackage{amsthm}
\usepackage{changepage}
\usepackage{color}
\usepackage{enumitem}
\usepackage{fancyhdr}
\usepackage{framed}
\usepackage[margin=1in]{geometry}
\usepackage{mathrsfs}
\usepackage{tikz, tkz-tab}
\usepackage{titling}

\newtheoremstyle{dotless}{}{}{\itshape}{}{\bfseries}{}{ }{}
\theoremstyle{dotless}

\newtheorem{defs}{Definition}[subsection]
\newenvironment{defi}{\definecolor{shadecolor}{RGB}{255,236,217}\begin{shaded}\begin{defs}\ \\}{\end{defs}\end{shaded}}

\newtheorem{pro}{Propriete}[subsection]
\newenvironment{prop}{\definecolor{shadecolor}{RGB}{230,230,255}\begin{shaded}\begin{pro}\ \\}{\end{pro}\end{shaded}}

\newtheorem{cor}{Corollaire}[subsection]
\newenvironment{coro}{\definecolor{shadecolor}{RGB}{245,250,255}\begin{shaded}\begin{cor}\ \\}{\end{cor}\end{shaded}}

\setlength{\droptitle}{-1in}
\predate{}
\postdate{}
\date{}
\title{\chapitre\\\titre\vspace{-.25in}}

\pagestyle{fancy}
\makeatletter
\lhead{\chapitre\ - \titre}
\rhead{\@author}
\makeatother

\newenvironment{preuve}{\begin{framed}\begin{proof}[\unskip\nopunct]}{\end{proof}\end{framed}}
\newenvironment{liste}{\begin{itemize}[leftmargin=*,noitemsep, topsep=0pt]}{\end{itemize}}
\newenvironment{tab}{\begin{adjustwidth}{.5cm}{}}{\end{adjustwidth}}

\newcommand{\uu}[1] {_{_{#1}}}
\newcommand{\lbracket}{[\![}
\newcommand{\rbracket}{]\!]}
\newcommand{\fonction}[5]{\begin{aligned}[t]#1\colon&#2&&\longrightarrow#3 \\&#4&&\longmapsto#5\end{aligned}}
\newcommand{\systeme}[1]{\left\{\begin{aligned}#1\end{aligned}\right.}
\newcommand{\cercle}[1]{\textcircled{\scriptsize{#1}}}

\newcommand{\lf}[1]{\left(#1\right)}
\newcommand{\C}{\mathbb{C}}
\newcommand{\R}{\mathbb{R}}
\newcommand{\K}{\mathbb{K}}
\newcommand{\N}{\mathbb{N}}
\newcommand{\I}{\mathcal{I}}
\newcommand{\F}{\mathcal{F}}
\newcommand{\E}{\mathcal{E}}
\newcommand{\G}{\mathcal{G}}
\newcommand{\et}{\text{ et }}
\newcommand{\ou}{\text{ ou }}
\newcommand{\xou}{\ \fbox{\text{ou}}\ }


%Auteur: Cl\'ement Phan, MPSI 2

\begin{document}
	\maketitle
	\section{Axiomes des structures}
		\subsection{Structure d'anneau}
			\begin{flushleft}
				Soit $A$ un ensemble non vide muni de deux lois, $+$ et $\times$.
			\end{flushleft}
			\begin{defi}
				$(A,+,\times)$ est un \underline{anneau} si:
				\begin{liste}
					\item[\cercle1]$(A,+)$ est un groupe abélien.
					\item[\cercle2]$\times$ est associative.\\
						$\times$ est distributive sur $+$.\\
						$\times$ admet un élément neutre.
				\end{liste}
			\end{defi}
			\begin{flushleft}
				\textbf{Notations:} Soit $(A,+,\times)$ un anneau.
				\begin{liste}
					\item Alors $(A,+)$ est un groupe abélien. on le note avec la notation additive, et son élément neutre est noté $0_A$.
					\item $\times$ est noté multiplicativement, mais n'est pas nécessairement commutative. On note son élément neutre $1_A$. C'est \underline{l'élément unité}.
				\end{liste}
			\end{flushleft}
			\begin{defi}
				Soit $(A,+,\times)$ un anneau non réduit \`a $\{0_A\}$.
				\begin{liste}
					\item Si $a$ et $b$ deux éléments de $A$ tels que $\times b=0_A\et a\neq0\et b\neq0$, alors $a$ et $b$ sont des \underline{diviseurs de zéro}
					\item On dit que $(A,+,\times)$ est \underline{intègre} si $\forall (a,b)\in A^{2},\ (a\times b=0_A)\Rightarrow (a=0_A\ou b=0_A)$
				\end{liste}
			\end{defi}
			\begin{prop}
				\begin{liste}
					\item Si $a$ est un élément inversible de $(A,+,\times)$, alors $a$ n'est pas un diviseur de zéro.
					\item Si $(A,+,\times)$ est un anneau \underline{fini} et et si $a$ est un élément non nul de $A$, alors $a$ est inversible ssi $a$ n'est pas un diviseur de zéro.
				\end{liste}
			\end{prop}
			\begin{defi}
				Soit $a$ un élément de $A$.\\
				On dit que $a$ est \underline{nilpotent} si il existe un entier naturel $n$ non nul tel que:
				$\prod\limits_{k=1}^na=0_A$
			\end{defi}
		\subsection{Structure de corps}
			\begin{flushleft}
				Soit $K$ un ensemble non vide muni de deux lois internes $+$ et $\times$.
			\end{flushleft}
			\begin{defi}
				$(K,+,\times)$ est un corps si:
				\begin{liste}
					\item[\cercle1]$(K,+)$ est un groupe abélien.
					\item[\cercle2]$(K\setminus\{0_K\},\times)$ est un groupe abélien.\\
						$\times$ est distributive sur $+$
				\end{liste}
			\end{defi}
			\begin{flushleft}
				\textbf{Remarques:}
				\begin{liste}
					\item Si $(K,+,\times)$ est un corps, alors c'est un anneau intègre.
					\item Si $(A,+,\times)$ est un anneau intègre fini, alors c'est un corps.
				\end{liste}
			\end{flushleft}
	\section{Calculs dans un anneau}
		\begin{flushleft}
			Soit $(A,+,\times)$ un anneau.\\
			Soit $a$ un élément de $A$.\\
			Soit $n$ un élément de $\mathbb{N}$.
			\begin{liste}
				\item $a+a+...+a$, $n$ fois se note $n\,a$
				\item $(-a)+(-a)+...+(-a)$ se note $n(-a)\ou(-n)a\ou -na$ et est l'opposé de $na$
				\item Si $n=0$, alors $0\,a=0_A$
				\item $a\times a\times...\times a$, $n$ fois se note $a^{n}$
				\item Si $a$ est inversible, $(a^{-1})\times(a^{-1})\times...\times(a^{-1})$ se note $(a^{-1})^n\ou(a^{-1})^n\ou a^{-n}$ et est l'opposé de $a^n$
				\item Par convention: $a^0=1_A$
			\end{liste}
		\end{flushleft}
		\begin{prop}
			\begin{liste}
				\item $\left( 1_A+a\right) ^n=\sum\limits_{k=0}^n\binom{n}{k}a^k$
				\item \underline{Si $a$ et $b$ commutent:}\\
					$\left(a+b\right) ^n=\sum\limits_{k=0}^n\binom{n}{k}a^k\,b^{n-k}$\\
					
				\item $1_A-a^{n+1}=(1_A-a)(1_A+a+a^{2}+...+a^{n})$
				\item \underline{Si $a$ et $b$ commutent:}\\
					$a^n-b^n=(a-b)\sum\limits_{k=0}^{n-1}a^k\,b^{n-1-k}$
			\end{liste}
		\end{prop}
	\section{Homomorphismes d'anneau et Homomorphismes de corps}
		\begin{defi}
			Soit $(A,+,\times)$ et $(A',+',\times')$ deux anneaux.\\
			Soit $f:A\rightarrow A'$\\
			On dit que $f$ est un \underline{homomorphisme d'anneau} si:
			\begin{liste}
				\item $\forall(x,y)\in A^2,\ f(x+y)=f(x)+'f(y)$
				\item $\forall(x,y)\in A^2,\ f(x\times y)=f(x)\times'f(y)$
				\item $f(1_A)=1_{A'}$
			\end{liste}
		\end{defi}
		\begin{defi}
				Soit $(K,+,\times)$ et $(K',+',\times')$ deux corps.\\
				Soit $f:K\rightarrow K'$\\
				On dit que $f$ est un \underline{homomorphisme de corps} si $f$ est un homomorphisme d'anneau.
		\end{defi}
		\begin{prop}
			Tout homomorphisme de corps est injectif.
		\end{prop}
		\begin{preuve}
			En utilisant les notations de la définition:\\
			Soit $\ker(f)=f^{-1}<\{0_{K'}\}>$\\
			$f$ est en particulier un homomorphisme de groupe de $(K,+)$ dans $(K',+')$. Donc $f$ est injectif ssi $\ker(f)=\{0_K\}$\\
			Soit x un élément de $K$\\
			\underline{$1^{\text{er}}$ cas:} $x=0_K$\\
			$f(0_K)=0_{K'}$ donc $0_K\in\ker(f)$
			\underline{$2^{\text{\`eme}}$ cas:} $x\neq0_K$\\
			Alors $x$ est inversible dans $K$: $x\times x^{-1}=1_K$\\
			Donc: 
			$\begin{aligned}[t]
			& f(x\times x^{-1})=f(1_K)=1_{K'}\\
			& f(x\times x^{-1})=f(x)\times'f(x)^{-1}
			\end{aligned}$\\
			D'o\`u $f(x)\times'f(x)^{-1}=1_K'$\\
			En particulier, $f(x)\neq 0_{K'}$\\
			Donc $x\notin \ker(f)$\\
			\\
			Donc $\ker(f)=\{0_K\}$\\
			Donc $f$ est injective.
		\end{preuve}
		\begin{coro}
			Soit $f:K\rightarrow K'$ un homomorphisme de corps.\\
			Alors $K$ est isomorphe \`a un sous-corps de $K'$
		\end{coro}
		\begin{preuve}
			\begin{liste}
				\item $f$ est un homomorphisme de corps, donc $f$ est injectif.
				\item L'image d'un corps par un homomorphisme de corps est un corps.
				\item $f$ induit une bijection de $K$ sur $f(K)$:\\
					$\fonction{\tilde{f}}{K}{f(K)}{x}{f(x)}$
			\end{liste}
			\begin{flushleft}
				Finalement: $K$ est isomorphe \`a $f(K)$ et $f(K)$ est un sous-groupe de $K'$.
			\end{flushleft}
		\end{preuve}
\end{document}