\documentclass[12pt,twoside,a4paper]{article}

\newcommand{\vdashv}{\mathrel{\text{\ooalign{$\vdash$\cr$\dashv$\cr}}}}

\def\chapitre{Relations Binaires}
\author{MPSI 2}
\def\titre{Relations d'ordre}

\usepackage{amsfonts}
\usepackage{amsmath}
\usepackage{amsthm}
\usepackage{changepage}
\usepackage{color}
\usepackage{enumitem}
\usepackage{fancyhdr}
\usepackage{framed}
\usepackage[margin=1in]{geometry}
\usepackage{mathrsfs}
\usepackage{tikz, tkz-tab}
\usepackage{titling}

\newtheoremstyle{dotless}{}{}{\itshape}{}{\bfseries}{}{ }{}
\theoremstyle{dotless}

\newtheorem{defs}{Definition}[subsection]
\newenvironment{defi}{\definecolor{shadecolor}{RGB}{255,236,217}\begin{shaded}\begin{defs}\ \\}{\end{defs}\end{shaded}}

\newtheorem{pro}{Propriete}[subsection]
\newenvironment{prop}{\definecolor{shadecolor}{RGB}{230,230,255}\begin{shaded}\begin{pro}\ \\}{\end{pro}\end{shaded}}

\newtheorem{cor}{Corollaire}[subsection]
\newenvironment{coro}{\definecolor{shadecolor}{RGB}{245,250,255}\begin{shaded}\begin{cor}\ \\}{\end{cor}\end{shaded}}

\setlength{\droptitle}{-1in}
\predate{}
\postdate{}
\date{}
\title{\chapitre\\\titre\vspace{-.25in}}

\pagestyle{fancy}
\makeatletter
\lhead{\chapitre\ - \titre}
\rhead{\@author}
\makeatother

\newenvironment{preuve}{\begin{framed}\begin{proof}[\unskip\nopunct]}{\end{proof}\end{framed}}
\newenvironment{liste}{\begin{itemize}[leftmargin=*,noitemsep, topsep=0pt]}{\end{itemize}}
\newenvironment{tab}{\begin{adjustwidth}{.5cm}{}}{\end{adjustwidth}}

\newcommand{\uu}[1] {_{_{#1}}}
\newcommand{\lbracket}{[\![}
\newcommand{\rbracket}{]\!]}
\newcommand{\fonction}[5]{\begin{aligned}[t]#1\colon&#2&&\longrightarrow#3 \\&#4&&\longmapsto#5\end{aligned}}
\newcommand{\systeme}[1]{\left\{\begin{aligned}#1\end{aligned}\right.}
\newcommand{\cercle}[1]{\textcircled{\scriptsize{#1}}}

\newcommand{\lf}[1]{\left(#1\right)}
\newcommand{\C}{\mathbb{C}}
\newcommand{\R}{\mathbb{R}}
\newcommand{\K}{\mathbb{K}}
\newcommand{\N}{\mathbb{N}}
\newcommand{\I}{\mathcal{I}}
\newcommand{\F}{\mathcal{F}}
\newcommand{\E}{\mathcal{E}}
\newcommand{\G}{\mathcal{G}}
\newcommand{\et}{\text{ et }}
\newcommand{\ou}{\text{ ou }}
\newcommand{\xou}{\ \fbox{\text{ou}}\ }


\begin{document}
	\maketitle
	\section{D\'efinition}
		Soit $E$ un ensemble non vide.\\
		Soit $\mathcal{R}$ une relation binaire sur $E$.
		\begin{defi}
			$\mathcal{R}$ est une \underline{relation d'ordre} sur $E$ si:
			\begin{liste}
				\item $\mathcal{R}$ est r\'eflexive.
				\item $\mathcal{R}$ est antisym\'etrique: $\forall(x,y)\in E^2,\ (x\,\mathcal{R}\,y \text{ et }y\,\mathcal{R}\,x)\Rightarrow(x=y)$
				\item $\mathcal{R}$ est transitive.
			\end{liste}
		\end{defi}
		\begin{flushleft}
			Notations: $x\,\mathcal{R}\,y$, $x\leq y$\\
			Se note aussi $x\preccurlyeq y$
		\end{flushleft}
		\begin{defi}
			Soit $\preccurlyeq$ une relation d'ordre sur $E$.\\
			\begin{liste}
				\item On dit que l'ordre est \underline{total} si deux \'el\'ements de $E$ sont toujours en relation:\\
					$\forall(x,y)\in E^2,\ (x\preccurlyeq y)$ ou $(y\preccurlyeq x)$.
				\item Sinon, on dit que l'ordre est partiel.
			\end{liste}
		\end{defi}
		\begin{defi}
			Soit $(E,\preccurlyeq)$ un ensemble ordonn\'e.
			\begin{liste}
				\item $m\in E$ est le \underline{plus petit \'el\'ement} de $E$ si: $\forall x\in E,\ m\preccurlyeq x$
				\item $M\in E$ est le \underline{plus grand \'el\'ement} de $E$ si: $\forall x\in E,\ x\preccurlyeq M$
			\end{liste}
		\end{defi}
		\begin{defi}
		Soit $(E,\preccurlyeq)$ un ensemble ordonn\'e.
			\begin{liste}
				\item $m\in E$ est un \underline{\'el\'ement minimal} de $E$ si:\\
					$\forall x\in E,\ (x \preccurlyeq m)\Rightarrow (x=m)$
				\item $M\in E$ est un \underline{\'el\'ement maximal} de $E$ si:\\
					$\forall x\in E,\ (M\preccurlyeq x)\Rightarrow (x=M)$
			\end{liste}
		\end{defi}
		\begin{defi}
			Soit $(E,\preccurlyeq)$ un ensemble ordonn\'e.\\
			Soit $A$ un sous-ensemble de $E$
			\begin{liste}
				\item $\alpha\in E$ est un \underline{minorant de $A$ dans $E$} si:\\
					$\forall x\in E,\ (x\in A)\Rightarrow(\alpha\preccurlyeq x)$
				\item $\beta\in E$ est un \underline{majorant de $A$ dans $E$} si:\\
					$\forall x\in E,\ (x\in A)\Rightarrow(x\preccurlyeq\beta)$
			\end{liste}
		\end{defi}
	\section{Ordre naturel sur $\mathbb{N}$}
		\begin{defi}
			$\forall(x,y)\in \mathbb{N},\ x\leqslant y\iff\exists n\in\mathbb{N}, y=x+n$
		\end{defi}
		C'est un ordre total de plus petit \'el\'ement $0$.
		\begin{prop}
			Propri\'et\'e caract\'eristique de $\mathbb{N}$:\\
			Tout sous-ensemble de $\mathbb{N}$ admet un plus petit \'el\'ement.
		\end{prop}
		\begin{coro}
			Tout sous-ensemble non vide et major\'e de $\mathbb{N}$ admet un plus grand \'el\'ement.
		\end{coro}
		\begin{preuve}
			Soit $A$ un sous-ensemble non vide et major\'e de $\mathbb{N}$.\\
			On consid\`ere $B$ l'ensemble des majorants de $A$.\\
			$B=\{x\in\mathbb{N},\ \forall a\in A, x\geqslant a\}$\\
			$A$ est major\'e donc $B$ est un sous-ensemble non vide de $\mathbb{N}$.\\
			D'apr\`es la propri\'et\'e caract\'eristique de $\mathbb{N}$ $B$ admet un plus petit \'el\'ement que l'on note $\alpha$\\
			On a: $\left\lbrace\begin{aligned}&\alpha\in\mathbb{N}\\&\forall a\in A,\ a\leqslant\alpha\end{aligned}\right.$\\
			Montrer que $\alpha\in\mathbb{N}$\\
			\underline{HA}: $\alpha \notin A$\\
			Alors $\forall x\in A,\ a<\alpha$\\
			Ou encore, puisque $\alpha$ est entier: $\forall a\in A,\ a\leqslant\alpha-1$\\
			On a donc $\alpha-1$ entier naturel et $\alpha-1$ majorant de $A$.\\
			Donc $\alpha\in B$ et $\alpha-1<\alpha$, ce qui contredit $\alpha$ plus petit \'el\'ement de $B$.\\
			\\
			Donc $\alpha\in A$\\
			Conclusion: $\alpha$ est le plus grand \'el\'ement de $A$.
		\end{preuve}
		\begin{coro}
			\textbf{Principe de r\'ecurrence}\\
			\\
			Soit $P$ une proposition portant sue les entiers naturels.\\
			Soit $P(n)$ le pr\'edicat associ\'e a $n$.\\
			$$\exists n\uu0\in\mathbb{N},\ \left[P(n\uu0)\text{ et }\left(\forall n\in\mathbb{N},\ P(n)\Rightarrow P(n+1)\right)\right]\Rightarrow\left[\forall n\in\mathbb{N},\ n\geqslant n\uu0\Rightarrow P(n)\right]$$
		\end{coro}
		\begin{preuve}
			\underline{H$_1$}: Soit $n_0$ un entier naturel tel que (\underline{H$_1'$}) $P(n_0)$ et (\underline{H$_1''$}) $\forall n\in\mathbb{N},\ P(n)\Rightarrow P(n+1)$\\
			Montrer que $\forall n\in\mathbb{N},\ n\geqslant n_0\Rightarrow P(n)$\\
			On consid\`ere l'ensemble $E$, ensemble des $n\in\mathbb{N}, (n\geqslant n_0\text{ et }\neg P(n))$\\
			Montrer que $E=\varnothing$\\
			\underline{HA}: Supposons $E$ non vide.\\
			D'apr\`es la propri\'et\'e caract\'eristique, $E$ admet un plus petit \'el\'ement, pot\'e $p_0$.\\
			$p_0$ v\'erifie: $\begin{aligned}[t] & p_0\in\mathbb{N}\\ & p_0\geqslant n_0 \\ & P(p_0)\text{ est Faux}\end{aligned}$\\
			Par ailleurs, $P(n_0)$ est Vrai, donc $p_0>n_0$, ou encore $p_0-1\geqslant n_0$\\
			Or, $p_0>p_0-1$, donc $p_0-1$ n'est pas dans $E$, donc $P(p_0-1)$ est Vrai.\\
			D'apr\`es H$_1''$, avec $n=p_0-1$, $P(p_0)$ est Vrai, ce qui est en contradiction avec HA.\\
			Donc HA est fausse, $E=\varnothing$\\
			Conclusion: $\forall n\in\mathbb{N},\ n\geqslant n_0\Rightarrow P(n)$
		\end{preuve}
	\section{Ordre naturel sur $\mathbb{R}$}
		\subsection{Ordre}
			Ordre sur $\mathbb{R}$: $x\leqslant y\iff y-x\in\mathbb{R}^+$. Il est total.
			\begin{prop}
				$(\mathbb{R},+,\times,\leqslant)$ est un \underline{corps totalement ordonn\'e}.\\
				\begin{liste}
					\item $(\mathbb{R},+)$ est un groupe commutatif car:
						\begin{liste}
							\item[]$+$ est associative: $\forall(x,y,z)\in \mathbb{R}^3,\ x+(y+z)=(x+y)+z$
							\item[]$+$ admet in \'el\'ement neutre: $\forall x\in\mathbb{R},\ x+0=x$
							\item[]$+$ octroie un \'el\'ement sym\'etrique: $-x$ (car $\forall x\in\mathbb{R},\ x+(-x)=0$)
							\item[]$+$ est commutative: $\forall(x,y)\in\mathbb{R},\ x+y=y+x$
						\end{liste}
					\item $(\mathbb{R}^*,\times)$ est un groupe commutatif car:
						\begin{liste}
							\item[]$\times$ est associative: $\forall(x,y,z)\in \mathbb{R}^3,\ x\times(y\times z)=(x\times y)\times z$
							\item[]$\times$ admet in \'el\'ement neutre: $\forall x\in\mathbb{R},\ x\times1=x$
							\item[]$\times$ octroie un \'el\'ement sym\'etrique: $\frac{1}{x}$ (car $\forall x\in\mathbb{R},\ x\times\frac{1}{x}=1$)
							\item[]$\times$ est commutative: $\forall(x,y)\in\mathbb{R},\ xy=yx$
						\end{liste}
					\item La relation d'ordre est compatible avec les op\'erateurs:
						\begin{liste}
							\item[]$\forall(x,y,x',y')\in\mathbb{R},\ (x\leqslant y\text{ et }x'\leqslant y')\Rightarrow(x+x'\leqslant y+y')$
							\item[]$\forall(x,y,z)\in\mathbb{R}\times\mathbb{R}\times\mathbb{R}^+,\ (x\leqslant y)\Rightarrow (z\,x\leqslant z\,y)$
						\end{liste}
				\end{liste}
			\end{prop}
			\begin{defi}
				Soit $(E,\leqslant)$ un ensemble ordonn\'e.
				\begin{liste}
					\item Soit $A$ une partie de $E$ non vide et major\'ee, $B$ l'ensemble des majorants de $A$.\\
						$B=\left\{x\in E,\ \forall a\in E,\ (a\in A\Rightarrow a\leqslant x) \right\}$\\
						On appelle \underline{borne sup\'erieure de $A$} le plus petit \'el\'ement de $B$ (lorsqu'il existe).
					\item  Soit $A$ une partie de $E$ non vide et minor\'ee, $B$ l'ensemble des minorants de $A$.\\
						$B=\left\{x\in E,\ \forall a\in E,\ (a\in A\Rightarrow a\geqslant x) \right\}$\\
						On appelle \underline{borne inf\'erieure de $A$} le plus grand \'el\'ement de $B$ (lorsqu'il existe).
				\end{liste}
			\end{defi}
			Notation: $Sup(A)$, $Inf(A)$\\
			\begin{prop}
				Propri\'et\'e caract\'eristique de $\mathbb{R}$:
				\begin{liste}
					\item Propri\'et\'e de la borne sup\'erieure:\\
						Tout ensemble non vide et major\'e de $\mathbb{R}$ admet une borne sup\'erieure.
					\item Propri\'et\'e de la borne inf\'erieure:\\
						Tout ensemble non vide et minor\'e de $\mathbb{R}$ admet une borne inf\'erieure.
				\end{liste}
			\end{prop}
			\begin{flushleft}
				\textbf{Remarque}: Soit $A$ une partie de $\mathbb{R}$ non vide et major\'ee (minor\'ee), et $\alpha$ sa borne sup\'erieure (inf\'erieure).\\
				$\begin{aligned}
					\alpha\text{ est caract\'eris\'e par: \textbullet\ }&\alpha \text{ est un majorant (minorant) de }A\\
						\text{\textbullet\ }&\text{si }\beta\text{ est strictement inf\'erieur (sup\'erieur) a }\alpha\text{, il n'est pas majorant }\\
						&\text{(minorant) de }A\end{aligned}$
				\begin{liste}
					\item \textbf{Crit\`ere 1:} $\alpha$ est la borne sup\'erieure de $A$ si et seulement si:\\
						$\left\{\begin{aligned}
							&\forall x\in\mathbb{R},\ x\in A\Rightarrow x\leqslant\alpha\\
							&\forall\beta\in\mathbb{R},\ (\beta\leqslant\alpha)\Rightarrow(\exists x\in\R,\ x\in A\text{ et }\beta<x\leqslant\alpha)
						\end{aligned} \right.$
					\item \textbf{Crit\`ere 2:} $\alpha$ est la borne sup\'erieure de $A$ si et seulement si:\\
						$\left\{\begin{aligned}
							&\forall x\in\mathbb{R},\ x\in A\Rightarrow x\leqslant\alpha\\
							&\forall\epsilon\in\mathbb{R}^{+*},\ \exists x\in\mathbb{R},\ (x\in A\text{ et }x-\epsilon<x\leqslant\alpha)
						\end{aligned} \right.$
				\end{liste}\ \\
			\end{flushleft}
			\begin{coro}
				$\mathbb{R}$ est \underline{Archim\'edien}: $\forall x\in\mathbb{R},\ x>0\Rightarrow(\exists n\in\mathbb{N}^*,\ n>x)$
			\end{coro}
			\begin{coro}
				$\forall(a,b)\in\mathbb{R}^{+*^2},\ \exists n\in\mathbb{N},\ n\,a>b$
			\end{coro}
			\begin{preuve}
				\textbf{Corollaire 1:}\\
				Soit $x$ un r\'eel positif.\\
				On consid\`ere $A=\{n\in\mathbb{N},\ n>x\}$\\
				Montrer que $A\neq\varnothing$\\
				\underline{HA}: $A=\varnothing$\\
				Alors $\forall n\in\mathbb{N},\ n\leqslant x$\\
				Donc $\mathbb{N}$ est une partie non vide et major\'ee de $\mathbb{R}$.\\
				Donc $A$ admet une borne sup\'erieure, que l'on note $\alpha$\\
				En utilisant le critère 2 avec $\epsilon=\frac{1}{2}$:\\
				$\exists x'\in\mathbb{N},\ \alpha-\frac{1}{2}<x\leqslant\alpha$\\
				Or $x'+1$ est un entier naturel v\'erifiant $\alpha+\frac{1}{2}<x'+1$\\
				Ce qui contredit le fait que $\alpha$ soit le majorant de $\mathbb{N}$.\\
				Conclusion: $A\neq\varnothing$\\
				Conclusion g\'en\'erale: $\mathbb{R}$ est Archim\'edien.\\
				\\
				\textbf{Corollaire 2}: prendre $x=\frac{b}{a}$
			\end{preuve}
			\begin{coro}
				\underline{Partie Entière}: $\forall x\in\mathbb{R},\ \exists n\in\mathbb{Z},\ \text{unique},\ n\leqslant x<n+1$
			\end{coro}
			\begin{preuve}
				\textbf{\'Existance:}\\
				Soit $x$ un réel positif.\\
				Soit $A=\left\{ n\in\mathbb{N},\ n>x \right\}$\\
				$A$ est non vide car $\mathbb{R}$ est archimédien\\
				Donc $A$ admet un plus petit élément, noté $n_0$.\\
				On a: $0\leqslant x<n_0$ donc $1\leqslant n_0$. Donc $n_0-1\in\mathbb{N}$ et $n_0-1\notin A$\\
				Donc $n_0-1\leqslant x<n_0$.\\
				En posant $n=n_0-1$, on a: $n\leqslant x<n+1$\\\\\\
				Soit $x$ un réel strictement négatif.\\
				$-x\in\mathbb{R^+}$, donc on applique la partie précédente.\\
				$\exists p\in\mathbb{N}, p\leqslant -x<p+1$\\
				Soit $p_0$ cet entier.\\
				Donc $-p_0-1<x\leqslant p_0$\\
				\begin{liste}
					\item $x=-p_0$\\
						On peut alors écrire $-p_0\leqslant x < -p_0+1$\\
						On note $n=-p_0$
					\item $x\neq -p_0$\\
						On peut alors écrire $-p_0-1\leqslant x<-p_0$\\
						On note $n=-p_0-1$
				\end{liste}\ \\
				\textbf{Unicité:}\\
				Supposons qu'il existe deux entiers $n_1$ et $n_2$ tels que:\\
				$\left\{ \begin{aligned}
					& n_1\leqslant x<n_1+1\\
					& n_2\leqslant x<n_2+1\\
					& n_1<n_2
				\end{aligned}\right.$\\
				$n_1<n_2$ donc $x<n_1+1\leqslant n_2\leqslant x$\\
				Donc  $n$ est unique.
			\end{preuve}
		\subsection{Propriétés de $\mathbb{R}$}
			\begin{prop}
				$\mathbb{Q}$ est dense dans $\mathbb{R}$\\
				$\mathbb{Q}\setminus\mathbb{R}$ est dense dans $\mathbb{R}$
			\end{prop}
			\begin{preuve}
				Soit $x$ et $y$ deux réels tels que $x<y$\\
				$y-x\in\mathbb{R^+}\Rightarrow\exists n\in\mathbb{N},\ n(y-x)>1$\\
				Soit $n$ un tel entier naturel.\\
				Donc $y<x+\frac{1}{n}$\\
				Soit $k$ la partie entière de $nx$:\\
				$k\leqslant x<k+1$\\
				$\frac{k}{n}\leqslant x<\frac{x+1}{n}\leqslant x+\frac{1}{n}<y$\\
				Posons $q=\frac{k+1}{n}$\\
				$q$ est rationnel, donc $\mathbb{Q}$ est dense dans $\mathbb{R}$\\
				Soit $x$ et $y$ deux réels tels que $x<y$\\
				$e$ est un irrationnel strictement positif donc $\frac{x}{e}<\frac{y}{e}$\\
				$\frac{x}{e}$ et $\frac{y}{e}$ sont deux réels, donc:\\
				$\exists q\in\mathbb{Q},\ \frac{x}{e}<q<\frac{y}{e}$\\
				Soit $q$ un tel élément.\\
				Donc $x<q\,e<y$\\
				Donc $\mathbb{Q}\setminus\mathbb{R}$ est dense dans $\mathbb{R}$
			\end{preuve}
			\begin{defi}
				Soit $I$ un sous-ensemble de $\mathbb{R}$\\
				$I$ est un intervalle si $I$ est une partie convexe de $\mathbb{R}$:\\
				$$\forall(a,b)\in I^2,\ (a<b)\Rightarrow(\forall x\in\mathbb{R},\ a<x<b\Rightarrow x\in I)$$
			\end{defi}
			\begin{prop}
				Les seuls intervalles réels sont les sous-ensembles du type:\\
				$\mathbb{R};\ \varnothing\\
				\left]a,b\right[;\ \left[a,b\right[;\ \left]a,b\right];\ \left[a,b\right]\\
				\left]a,+\infty\right[;\ \left[a,+\infty\right[;\ \left]-\infty,b\right[;\ \left]-\infty,b\right]$
			\end{prop}
			\begin{preuve}
				Soit $I$ une partie convexe de $\mathbb{R}$
				\begin{tab}
					\begin{liste}
						\item[\cercle1]Montrer que si $I$ est de l'un des types ci-dessus, $I$ est un intervalle.
							\begin{liste}
								\item $[a,b]=\{x\in\mathbb{R},\ a\leqslant x\leqslant b\}$
								\item $]-\infty,b[=\{x\in\mathbb{R},\ x<b \}$
								\item Etc...
							\end{liste}
						\item[\cercle2]Montrer que si $I$ est une partie convexe de $\mathbb{R}$, alors $I$ est de l'un des ces types.
							\begin{liste}
								\item[\textbullet]$I$ n'est ni minorée, ni majorée. Montrer que $I=\mathbb{R}$
									\begin{liste}
										\item Par hypothèse sur $I$, $I\subset\mathbb{R}$
										\item Montrer que $\mathbb{R}\subset I$
											Soit $x$ un réel. $x$ n'est ni majorant, ni minorant:\\
											$\exists a\in\mathbb{R},\ a\in I\et a<x$\\
											$\exists b\in\mathbb{R},\ b\in I\et b>x$\\
											Soit $a$ et $b$ deux tels réels: $(a,b)\in i^2,\ a<x<b$\\
											Or $I$ est un intervalle, donc $x\in I$\\
											Ce raisonnement étant valable pour tout $x$, $\mathbb{R}\subset I$\\
									\end{liste}
								\item[\textbullet]$I$ est majorée et non minorée. Montrer que $I=\left]-\infty,b\right[$ ou $I=\left]-\infty,\right]$
									$I$ est non vide et majorée, donc admet une borne supérieure, notée $b$.
									\begin{liste}
										\item Montrer que $I\subset ]-\infty,b]$\\
											C'est a dire montrer que $\forall x\in\mathbb{R},\ x\in I\Rightarrow x\leqslant b$\\
											Ce qui est vrai car $b$ majore $I$.
										\item Montrer que $]-\infty,b[\subset I$\\
											C'est a dire montrer que $\forall x\in\mathbb{R},\ x<b\Rightarrow x\in I$\\
											Soit $x$ un réel de $I$.
											\begin{liste}
												\item $x$ n'est pas un minorant de $I$:\\
													$\exists a\in \mathbb{R},\ a\in I\et a<x$
												\item $b$ est une borne supérieure de $I$:\\
													$(x<b)\Rightarrow(\exists x'\in \mathbb{R},\ x'\in I\et x<x'\leqslant b)$\\
													Soit $x'$ un tel élément.
											\end{liste}
											Donc: $a<x<x'\leqslant b$\\
											Donc $x\in I$, donc $]-\infty,b[\subset I$
										\item Donc $]-\infty,b[\subset I \subset ]-\infty,b]$\\
											Donc $I=\left]-\infty,b\right[$ si $b\notin I$, ou $I=\left]-\infty,b\right]$ si $b\in I$
									\end{liste}
								\item[\textbullet] $I$ est minorée et non majorée.
								\item[\textbullet] $I$ est bornée.
							\end{liste}
					\end{liste}
				\end{tab}
			\end{preuve}
			\begin{defi}
				\textbf{Valeur absolue:} $\fonction{}{\mathbb{R}}{\mathbb{R}}{x}{\left\{\begin{aligned}x\text{ si }& x\in\mathbb{R}^+\\-x\text{ si }& x\in\mathbb{R}^- \end{aligned}\right.}$
			\end{defi}
			\begin{prop}
				$|x|=0\iff x=0\\
				|x+y|\leqslant|x|+|y|\\
				|x\,y|=|x|\,|y|$
			\end{prop}
			\begin{coro}
				$|\,|x|-|y|\,|\leqslant|x-y|$
			\end{coro}
			\begin{defi}
				Soit $(a,r)\in\mathbb{R}\times\mathbb{R}^{+*}$\\
				On appelle \underline{intervalle ouvert centré en $a$ de rayon $r$} le sous-ensemble $]a-r,a+r[$
			\end{defi}
			\begin{prop}
				$]a,b[=\bigcup\limits_{n\in\mathbb{N}^*}[a+\frac{1}{n},b-\frac{1}{n}]$\\\\
				$[a,b]=\bigcap\limits_{n\in\mathbb{N}^*}]a-\frac{1}{n},b+\frac{a}{n}[$
			\end{prop}	
\end{document}