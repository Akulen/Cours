\documentclass[12pt,twoside,a4paper]{article}

\newcommand{\vdashv}{\mathrel{\text{\ooalign{$\vdash$\cr$\dashv$\cr}}}}

\def\chapitre{Relations Binaires}
\author{MPSI 2}
\def\titre{Relations d'ordre}

\usepackage{amsfonts}
\usepackage{amsmath}
\usepackage{amsthm}
\usepackage{changepage}
\usepackage{color}
\usepackage{enumitem}
\usepackage{fancyhdr}
\usepackage{framed}
\usepackage[margin=1in]{geometry}
\usepackage{mathrsfs}
\usepackage{tikz, tkz-tab}
\usepackage{titling}

\newtheoremstyle{dotless}{}{}{\itshape}{}{\bfseries}{}{ }{}
\theoremstyle{dotless}

\newtheorem{defs}{Definition}[subsection]
\newenvironment{defi}{\definecolor{shadecolor}{RGB}{255,236,217}\begin{shaded}\begin{defs}\ \\}{\end{defs}\end{shaded}}

\newtheorem{pro}{Propriete}[subsection]
\newenvironment{prop}{\definecolor{shadecolor}{RGB}{230,230,255}\begin{shaded}\begin{pro}\ \\}{\end{pro}\end{shaded}}

\newtheorem{cor}{Corollaire}[subsection]
\newenvironment{coro}{\definecolor{shadecolor}{RGB}{245,250,255}\begin{shaded}\begin{cor}\ \\}{\end{cor}\end{shaded}}

\setlength{\droptitle}{-1in}
\predate{}
\postdate{}
\date{}
\title{\chapitre\\\titre\vspace{-.25in}}

\pagestyle{fancy}
\makeatletter
\lhead{\chapitre\ - \titre}
\rhead{\@author}
\makeatother

\newenvironment{preuve}{\begin{framed}\begin{proof}[\unskip\nopunct]}{\end{proof}\end{framed}}
\newenvironment{liste}{\begin{itemize}[leftmargin=*,noitemsep, topsep=0pt]}{\end{itemize}}
\newenvironment{tab}{\begin{adjustwidth}{.5cm}{}}{\end{adjustwidth}}

\newcommand{\uu}[1] {_{_{#1}}}
\newcommand{\lbracket}{[\![}
\newcommand{\rbracket}{]\!]}
\newcommand{\fonction}[5]{\begin{aligned}[t]#1\colon&#2&&\longrightarrow#3 \\&#4&&\longmapsto#5\end{aligned}}
\newcommand{\systeme}[1]{\left\{\begin{aligned}#1\end{aligned}\right.}
\newcommand{\cercle}[1]{\textcircled{\scriptsize{#1}}}

\newcommand{\lf}[1]{\left(#1\right)}
\newcommand{\C}{\mathbb{C}}
\newcommand{\R}{\mathbb{R}}
\newcommand{\K}{\mathbb{K}}
\newcommand{\N}{\mathbb{N}}
\newcommand{\I}{\mathcal{I}}
\newcommand{\F}{\mathcal{F}}
\newcommand{\E}{\mathcal{E}}
\newcommand{\G}{\mathcal{G}}
\newcommand{\et}{\text{ et }}
\newcommand{\ou}{\text{ ou }}
\newcommand{\xou}{\ \fbox{\text{ou}}\ }


\begin{document}
	\maketitle
	\section{D\'efinition}
		Soit $E$ un ensemble non vide.\\
		Soit $\mathcal{R}$ une relation binaire sur $E$.
		\begin{defi}
			$\mathcal{R}$ est une \underline{relation d'ordre} sur $E$ si:
			\begin{liste}
				\item $\mathcal{R}$ est r\'eflexive.
				\item $\mathcal{R}$ est antisym\'etrique: $\forall(x,y)\in E^2,\ (x\,\mathcal{R}\,y \text{ et }y\,\mathcal{R}\,x)\Rightarrow(x=y)$
				\item $\mathcal{R}$ est transitive.
			\end{liste}
		\end{defi}\ \\
		Notations: $x\,\mathcal{R}\,y$, $x\leq y$\\
		Se note aussi $x\preccurlyeq y$\\
		\begin{defi}
			Soit $\preccurlyeq$ une relation d'ordre sur $E$.\\
			\begin{liste}
				\item On dit que l'ordre est \underline{total} si deux \'el\'ements de $E$ sont toujours en relation:\\
					$\forall(x,y)\in E^2,\ (x\preccurlyeq y)$ ou $(y\preccurlyeq x)$.
				\item Sinon, on dit que l'ordre est partiel.
			\end{liste}
		\end{defi}
		\begin{defi}
			Soit $(E,\preccurlyeq)$ un ensemble ordonn\'e.
			\begin{liste}
				\item $m\in E$ est le \underline{plus petit \'el\'ement} de $E$ si: $\forall x\in E,\ m\preccurlyeq x$
				\item $M\in E$ est le \underline{plus grand \'el\'ement} de $E$ si: $\forall x\in E,\ x\preccurlyeq M$
			\end{liste}
		\end{defi}
		\begin{defi}
		Soit $(E,\preccurlyeq)$ un ensemble ordonn\'e.
			\begin{liste}
				\item $m\in E$ est un \underline{\'el\'ement minimal} de $E$ si:\\
					$\forall x\in E,\ (x \preccurlyeq m)\Rightarrow (x=m)$
				\item $M\in E$ est un \underline{\'el\'ement maximal} de $E$ si:\\
					$\forall x\in E,\ (M\preccurlyeq x)\Rightarrow (x=M)$
			\end{liste}
		\end{defi}
		\begin{defi}
			Soit $(E,\preccurlyeq)$ un ensemble ordonn\'e.\\
			Soit $A$ un sous-ensemble de $E$
			\begin{liste}
				\item $\alpha\in E$ est un \underline{minorant de $A$ dans $E$} si:\\
					$\forall x\in E,\ (x\in A)\Rightarrow(\alpha\preccurlyeq x)$
				\item $\beta\in E$ est un \underline{majorant de $A$ dans $E$} si:\\
					$\forall x\in E,\ (x\in A)\Rightarrow(x\preccurlyeq\beta)$
			\end{liste}
		\end{defi}
	\section{Ordre naturel sur $\mathbb{N}$}
		\begin{defi}
			$\forall(x,y)\in \mathbb{N},\ x\leqslant y\iff\exists n\in\mathbb{N}, y=x+n$
		\end{defi}
		C'est un ordre total de plus petit \'el\'ement $0$.
		\begin{prop}
			Tout sous-ensemble de $\mathbb{N}$ admet un plus petit \'el\'ement.
		\end{prop}
		\begin{coro}
			Tout sous-ensemble non vide et major\'e de $\mathbb{N}$ admet un plus grand \'el\'ement.
		\end{coro}
		\begin{preuve}
			Soit $A$ un sous-ensemble non vide et major\'e de $\mathbb{N}$.\\
			On consid\`ere $B$ l'ensemble des majorants de $A$.\\
			$B=\{x\in\mathbb{N},\ \forall a\in A, x\geqslant a\}$\\
			$A$ est major\'e donc $B$ est un sous-ensemble non vide de $\mathbb{N}$.\\
			D'apr\`es la propri\'et\'e caract\'eristique de $\mathbb{N}$ $B$ admet un plus petit \'el\'ement que l'on note $\alpha$\\
			On a: $\left\lbrace\begin{aligned}&\alpha\in\mathbb{N}\\&\forall a\in A,\ a\leqslant\alpha\end{aligned}\right.$\\
			Montrer que $\alpha\in\mathbb{N}$\\
			\underline{HA}: $\alpha \notin A$\\
			Alors $\forall x\in A,\ a<\alpha$\\
			Ou encore, puisque $\alpha$ est entier: $\forall a\in A,\ a\leqslant\alpha-1$\\
			On a donc $\alpha-1$ entier naturel et $\alpha-1$ majorant de $A$.\\
			Donc $\alpha\in B$ et $\alpha-1<\alpha$, ce qui contredit $\alpha$ plus petit élément de $B$.\\
			\\
			Donc $\alpha\in A$\\
			Conclusion: $\alpha$ est le plus grand élément de $A$.
		\end{preuve}
		\begin{coro}
			\textbf{Principe de récurrence}\\
			\\
			Soit $P$ une proposition portant sue les entiers naturels.\\
			Soit $P(n)$ le prédicat associé a $n$.\\
			$$\exists n\uu0\in\mathbb{N},\ \left[P(n\uu0)\text{ et }\left(\forall n\in\mathbb{N},\ P(n)\Rightarrow P(n+1)\right)\right]\Rightarrow\left[\forall n\in\mathbb{N},\ n\geqslant n\uu0\Rightarrow P(n)\right]$$
		\end{coro}
		%faire la démo, page 7-5-2
\end{document}