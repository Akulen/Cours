\documentclass[12pt,twoside,a4paper]{article}

\def\chapitre{Elements de Theorie des Ensembles}
\author{MPSI 2}
\def\titre{Graphes et Applications}

\usepackage{amsfonts}
\usepackage{amsmath}
\usepackage{amsthm}
\usepackage{changepage}
\usepackage{color}
\usepackage{enumitem}
\usepackage{fancyhdr}
\usepackage{framed}
\usepackage[margin=1in]{geometry}
\usepackage{mathrsfs}
\usepackage{tikz, tkz-tab}
\usepackage{titling}

\newtheoremstyle{dotless}{}{}{\itshape}{}{\bfseries}{}{ }{}
\theoremstyle{dotless}

\newtheorem{defs}{Definition}[subsection]
\newenvironment{defi}{\definecolor{shadecolor}{RGB}{255,236,217}\begin{shaded}\begin{defs}\ \\}{\end{defs}\end{shaded}}

\newtheorem{pro}{Propriete}[subsection]
\newenvironment{prop}{\definecolor{shadecolor}{RGB}{230,230,255}\begin{shaded}\begin{pro}\ \\}{\end{pro}\end{shaded}}

\newtheorem{cor}{Corollaire}[subsection]
\newenvironment{coro}{\definecolor{shadecolor}{RGB}{245,250,255}\begin{shaded}\begin{cor}\ \\}{\end{cor}\end{shaded}}

\setlength{\droptitle}{-1in}
\predate{}
\postdate{}
\date{}
\title{\chapitre\\\titre\vspace{-.25in}}

\pagestyle{fancy}
\makeatletter
\lhead{\chapitre\ - \titre}
\rhead{\@author}
\makeatother

\newenvironment{preuve}{\begin{framed}\begin{proof}[\unskip\nopunct]}{\end{proof}\end{framed}}
\newenvironment{liste}{\begin{itemize}[leftmargin=*,noitemsep, topsep=0pt]}{\end{itemize}}
\newenvironment{tab}{\begin{adjustwidth}{.5cm}{}}{\end{adjustwidth}}

\newcommand{\uu}[1] {_{_{#1}}}
\newcommand{\lbracket}{[\![}
\newcommand{\rbracket}{]\!]}
\newcommand{\fonction}[5]{\begin{aligned}[t]#1\colon&#2&&\longrightarrow#3 \\&#4&&\longmapsto#5\end{aligned}}
\newcommand{\systeme}[1]{\left\{\begin{aligned}#1\end{aligned}\right.}
\newcommand{\cercle}[1]{\textcircled{\scriptsize{#1}}}

\newcommand{\lf}[1]{\left(#1\right)}
\newcommand{\C}{\mathbb{C}}
\newcommand{\R}{\mathbb{R}}
\newcommand{\K}{\mathbb{K}}
\newcommand{\N}{\mathbb{N}}
\newcommand{\I}{\mathcal{I}}
\newcommand{\F}{\mathcal{F}}
\newcommand{\E}{\mathcal{E}}
\newcommand{\G}{\mathcal{G}}
\newcommand{\et}{\text{ et }}
\newcommand{\ou}{\text{ ou }}
\newcommand{\xou}{\ \fbox{\text{ou}}\ }


\begin{document}
	\maketitle
	\section{Graphes de $E$ vers $F$}
		Soit $E$ et $F$ deux ensembles. \\
		On note $E\times E=\{\lf{x,y},x\in E\et y\in F\}$
		\begin{defi}
			On appelle graphe de $E$ dans $F$ tout sous ensemble $G$ de $E\times F$
		\end{defi}
		\begin{defi}
			On appelle correspondance de $E$ vers $F$ tout triplet $\lf{G,E,F}$ o\'u $E$ et $F$ sont deux ensembles et $G$ un graphe de $E$ vers $F$.
		\end{defi}
		\begin{defi}
			Soit $E$ et $F$ deux ensembles. \\
			Soit $\lf{G,E,F}$ une correspondance de $E$ vers $F$, On dit que $\lf{G,E,F}$ est une application de $E$ dans $F$ si pour tout element de $E$, l'ensemble des $y$ de $F$ tels que $\lf{x,y}$ soit dans $G$ est reduit a un et un seul element.
		\end{defi}\ \\
		\textbf{Notations :}
		\begin{tab}
			$f\colon\lf{G,E,F}$ ou $\fonction{f}{E}{F}{x}{y\text{ tel que }\lf{x,y}\in G}$  \\
			Pour tout $x$ de $F$, l'ensemble $\{y\in F,\lf{x,y}\in G\}$ est reduit a un seul element. On notera cet element $f\lf{x}$.
		\end{tab}
		\begin{defi}
			Soit $f$ une application de $E$ dans $F$. \\
			Soit $A$ un sous-ensemble de $E$. \\
			L'ensemble $f\lf{A}=\{f\lf{x},x\in A\}$ est un sous-ensemble de $F$ que l'on appelle image de $A$ par $f$.
		\end{defi}\newpage
		\textbf{Remarques :}
		\begin{liste}
			\item $f\lf{\varnothing}=\varnothing$
			\item $f\lf{E}$ s'appelle l'image de $f$.
		\end{liste}
		\begin{defi}
			Soit $f\colon E\longrightarrow F$ une application. \\
			Soit $B$ un sous-ensemble de $F$. \\
			On appelle image reciproque de $B$ par $f$ le sous-ensemble de $E$ note $f^{-1}<B>$ tel que $f^{-1}<B>=\{x\in E,f\lf{x}\in B\}$.
		\end{defi}\ \\
		\textbf{Remarques :}
		\begin{liste}
			\item $f^{-1}<F> = E$
			\item Soit $y$ un element de $F$. On prend $B=\{y\}$
				$$\begin{aligned}
					f^{-1}<B>&=f^{-1}<\{y\}>\\
							 &=\{x\in E, f\lf{x}=y\}
				\end{aligned}$$
		\end{liste}
		\begin{defi}
			Soit $f$ une application de $E$ dans $F$.
			\begin{liste}
				\item $f$ est dite injective si deux element distincts de $E$ ont des images distinctes par $f$.
				\item $f$ est dite surjective si tout element de $F$ est dans $f\lf{E}$.
				\item $f$ est dite bijective si elle est sujective et injective.
			\end{liste}
		\end{defi}
		\begin{defi}
			\begin{liste}
				\item Soit $f$ une application de $E$ dans $F$. \\
					Soit $A$ un sous-ensemble de $E$. \\
					On appelle restriction de $f$ a $A$ l'application notee $f_{|_A}$ telle que :
					$$\fonction{f}{A}{F}{x}{f\lf{x}}$$
				\item Soit $f$ une application de $E$ dans $F$. \\
					Soit $E'$ un ensemble tel que $E$ soit un sous-ensemble de $E'$.
					Soit $g$ une application de $E'$ dans $F$. \\
					On dit que $g$ est un prolongement de $f$ par $E'$ si $g_{|_E}=f$
			\end{liste}
		\end{defi}\newpage
		\begin{defi}
			Soit $E$ un ensemble. \\
			Soit $A$ un sous-ensemble de $E$. \\
			On appelle application caracteristique de $A$ (fonction indicatrice de $A$) l'unique application notee $\mathds{1}_A$ et definie par :
			$$\fonction{\mathds{1}_A}{E}{\{0,1\}}{x}{\lf{x\in A}}$$
		\end{defi}
		\begin{prop}
			Notons $\mathcal{P}\lf{E}$ l'ensemble des sous-ensembles de $E$. \\
			Notons $\mathcal{F}\lf{E,\{0,1\}}$ l'ensemble des applications definies sur $E$ a valeurs dans $\{0,1\}$. \\
			L'application $\fonction{\phi}{\mathcal{P}\lf{E}}{\mathcal{F}\lf{E,\{0,1\}}}{A}{\mathds{1}_A}$ realise une bijection de $\mathcal{P}\lf{E}$ sur $\mathcal{F}\lf{E,\{0,1\}}$.
		\end{prop}
\end{document}
