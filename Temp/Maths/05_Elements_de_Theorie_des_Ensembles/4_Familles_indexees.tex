\documentclass[12pt,twoside,a4paper]{article}

\def\chapitre{El\'ements de Th\'eorie des Ensembles}
\author{MPSI 2}
\def\titre{Familles index\'ees}

\usepackage{amsfonts}
\usepackage{amsmath}
\usepackage{amsthm}
\usepackage{changepage}
\usepackage{color}
\usepackage{enumitem}
\usepackage{fancyhdr}
\usepackage{framed}
\usepackage[margin=1in]{geometry}
\usepackage{mathrsfs}
\usepackage{tikz, tkz-tab}
\usepackage{titling}

\newtheoremstyle{dotless}{}{}{\itshape}{}{\bfseries}{}{ }{}
\theoremstyle{dotless}

\newtheorem{defs}{Definition}[subsection]
\newenvironment{defi}{\definecolor{shadecolor}{RGB}{255,236,217}\begin{shaded}\begin{defs}\ \\}{\end{defs}\end{shaded}}

\newtheorem{pro}{Propriete}[subsection]
\newenvironment{prop}{\definecolor{shadecolor}{RGB}{230,230,255}\begin{shaded}\begin{pro}\ \\}{\end{pro}\end{shaded}}

\newtheorem{cor}{Corollaire}[subsection]
\newenvironment{coro}{\definecolor{shadecolor}{RGB}{245,250,255}\begin{shaded}\begin{cor}\ \\}{\end{cor}\end{shaded}}

\setlength{\droptitle}{-1in}
\predate{}
\postdate{}
\date{}
\title{\chapitre\\\titre\vspace{-.25in}}

\pagestyle{fancy}
\makeatletter
\lhead{\chapitre\ - \titre}
\rhead{\@author}
\makeatother

\newenvironment{preuve}{\begin{framed}\begin{proof}[\unskip\nopunct]}{\end{proof}\end{framed}}
\newenvironment{liste}{\begin{itemize}[leftmargin=*,noitemsep, topsep=0pt]}{\end{itemize}}
\newenvironment{tab}{\begin{adjustwidth}{.5cm}{}}{\end{adjustwidth}}

\newcommand{\uu}[1] {_{_{#1}}}
\newcommand{\lbracket}{[\![}
\newcommand{\rbracket}{]\!]}
\newcommand{\fonction}[5]{\begin{aligned}[t]#1\colon&#2&&\longrightarrow#3 \\&#4&&\longmapsto#5\end{aligned}}
\newcommand{\systeme}[1]{\left\{\begin{aligned}#1\end{aligned}\right.}
\newcommand{\cercle}[1]{\textcircled{\scriptsize{#1}}}

\newcommand{\lf}[1]{\left(#1\right)}
\newcommand{\C}{\mathbb{C}}
\newcommand{\R}{\mathbb{R}}
\newcommand{\K}{\mathbb{K}}
\newcommand{\N}{\mathbb{N}}
\newcommand{\I}{\mathcal{I}}
\newcommand{\F}{\mathcal{F}}
\newcommand{\E}{\mathcal{E}}
\newcommand{\G}{\mathcal{G}}
\newcommand{\et}{\text{ et }}
\newcommand{\ou}{\text{ ou }}
\newcommand{\xou}{\ \fbox{\text{ou}}\ }


\begin{document}
	\maketitle
	\section{Ensembles finis, Ensembles d\'enombrables}
		\begin{defi}
			Soit $E$ un ensemble non vide. On dit que $E$ est fini s'il existe un entier naturel non nul $n$ et une application $\phi\colon\lbracket1,n\rbracket\longrightarrow E$ bijective.
		\end{defi}
		\begin{prop}
			Si $n$ existe, alors il est unique et on l'appelle le cardinal de $E$.
		\end{prop}
		\textbf{Notation :} $card(E)=\#E=|E|$
		\begin{defi}
			$E$ est d\'enombrable s'il existe une bijection de $\N$ sur $E$.
		\end{defi}
	\section{Ensemble des familles index\'ees}
		\subsection{D\'efinitions}
			Un ensemble d'indexation est un ensemble $I$ non vide.
			\begin{defi}
				On dit que $E$ est un ensemble index\'e par $I$ s'il existe une bijection de $I$ sur $E$ $\fonction{\phi}{I}{E}{i}{x\uu i}$ bijective.
			\end{defi}
			\textbf{Notation :} $E=\left\{x\uu i,i\in I\right\}$
			\begin{defi}
				On appelle famille index\'ee par $I$ dans $E$ toute application de $I$ dans $E$.
				$$\fonction{\phi}{I}{E}{i}{x\uu i}$$
			\end{defi}
		\subsection{Operations ensemblistes sur les familles index\'ees}
			\begin{defi}
				Soit $\lf{A\uu i}\uu{i\in I}$ une famille index\'ee de sous-ensembles de $E$.
				\begin{liste}
					\item $\bigcup_{i\in I}A\uu i=\left\{x\in E,\exists i\in I,x\in A\uu i\right\}$
					\item $\bigcap_{i\in I}A\uu i=\left\{x\in E,\forall i\in I,x\in A\uu i\right\}$
				\end{liste}
			\end{defi}
			\begin{prop}
				$${^c\lf{\bigcup_{i\in I}A\uu i}}=\bigcap_{i\in I}{^cA\uu i};{^c\lf{\bigcap_{i\in I}A\uu i}}=\bigcup_{i\in I}{^cA\uu i}$$
			\end{prop}
			\begin{preuve}
				$$\begin{aligned}
					x\in{^c\lf{\bigcup_{i\in I}A\uu i}}&\iff\neg\lf{x\in\bigcup_{i\in I}A\uu i}&&\text{par d\'efinition du compl\'ementaire} \\
													   &\iff\neg\lf{\exists i\in I,x\in A\uu i}&&\text{par d\'efinition de la r\'eunion des }A\uu i \\
													   &\iff\forall i\in I,x\notin A\uu I&&\text{n\'egation de }"\exists" \\
													   &\iff\forall i\in I,x\in {^cA\uu i}&&\text{par d\'efinition du compl\'ementaire} \\
													   &\iff x\in\bigcap_{i\in I}{^cA\uu i}&&\text{par d\'efinition de l'intersection des }A\uu i
				\end{aligned}$$
			\end{preuve}
			\begin{defi}
				Soit $E$ un ensemble. \\
				Soit $\lf{A\uu i}\uu{i\in I}$ une famille de sous-ensembles de $E$ index\'ee par $I$. \\
				On dit que $\lf{A\uu i}\uu{i\in I}$ est une partition de $E$ si :
				\begin{liste}
					\item $\forall i\in I,A\uu i\neq\varnothing$
					\item $\bigcup_{i\in I}A\uu i=E$
					\item $\forall(i,j)\in I^2,A\uu i=A\uu j\ou A\uu i\cap A\uu j=\varnothing$
				\end{liste}
			\end{defi}
\end{document}
