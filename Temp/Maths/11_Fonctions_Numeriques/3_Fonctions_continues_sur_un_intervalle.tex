% !TeX encoding = UTF-8
\documentclass[12pt,twoside,a4paper]{article}


\def\chapitre{Fonctions Num\'eriques}
\author{MPSI 2}
\def\titre{Fonctions continues sur un intervalle}

\usepackage{amsfonts}
\usepackage{amsmath}
\usepackage{amsthm}
\usepackage{changepage}
\usepackage{color}
\usepackage{enumitem}
\usepackage{fancyhdr}
\usepackage{framed}
\usepackage[margin=1in]{geometry}
\usepackage{mathrsfs}
\usepackage{tikz, tkz-tab}
\usepackage{titling}

\newtheoremstyle{dotless}{}{}{\itshape}{}{\bfseries}{}{ }{}
\theoremstyle{dotless}

\newtheorem{defs}{Definition}[subsection]
\newenvironment{defi}{\definecolor{shadecolor}{RGB}{255,236,217}\begin{shaded}\begin{defs}\ \\}{\end{defs}\end{shaded}}

\newtheorem{pro}{Propriete}[subsection]
\newenvironment{prop}{\definecolor{shadecolor}{RGB}{230,230,255}\begin{shaded}\begin{pro}\ \\}{\end{pro}\end{shaded}}

\newtheorem{cor}{Corollaire}[subsection]
\newenvironment{coro}{\definecolor{shadecolor}{RGB}{245,250,255}\begin{shaded}\begin{cor}\ \\}{\end{cor}\end{shaded}}

\setlength{\droptitle}{-1in}
\predate{}
\postdate{}
\date{}
\title{\chapitre\\\titre\vspace{-.25in}}

\pagestyle{fancy}
\makeatletter
\lhead{\chapitre\ - \titre}
\rhead{\@author}
\makeatother

\newenvironment{preuve}{\begin{framed}\begin{proof}[\unskip\nopunct]}{\end{proof}\end{framed}}
\newenvironment{liste}{\begin{itemize}[leftmargin=*,noitemsep, topsep=0pt]}{\end{itemize}}
\newenvironment{tab}{\begin{adjustwidth}{.5cm}{}}{\end{adjustwidth}}

\newcommand{\uu}[1] {_{_{#1}}}
\newcommand{\lbracket}{[\![}
\newcommand{\rbracket}{]\!]}
\newcommand{\fonction}[5]{\begin{aligned}[t]#1\colon&#2&&\longrightarrow#3 \\&#4&&\longmapsto#5\end{aligned}}
\newcommand{\systeme}[1]{\left\{\begin{aligned}#1\end{aligned}\right.}
\newcommand{\cercle}[1]{\textcircled{\scriptsize{#1}}}

%Auteur: Cl\'ement Phan, MPSI 2

\begin{document}
	\maketitle
	\section{Fonctions continues}
		\begin{flushleft}
			Soit $I$ un intervalle non vide.\\
			Soit $f: I\rightarrow \R$ une fonction définie sur $I$.\\
			On dit que \underline{$f$ est continue sur $I$} si pout tout $x_0$ de $I$, $f$ est continue en $x_0$.
		\end{flushleft}
		\begin{theo}{des valeurs intermédiaires}
			L'image d'un intervalle par une fonction continue est un intervalle.
		\end{theo}
		\begin{preuve}
			Soit $I$ un intervalle.\\
			Soit $f:I\rightarrow R$ une application continue sut $I$.\\
			Montrer que $f(I)$ est un intervalle.\\
			Ou montrer que $\forall (y,y')\in \R^{2},\ ((y,y')\in f(I)^{2}\Rightarrow (\forall y''\in\R,\ y<y''<y'\Rightarrow y''\in f(I))$\\
			\\
			Soit $y\et y'$ deux éléments distincts de $f(I)$.\\
			Alors il existe $a\et b$ dans $I$ tels que: $f(a)=y\et f(b)=y'$\\
			$y\et y'$ sont distincts, donc $a\et b$ sont distincts.\\
			On suppose par exemple que $f(a)<f(b)\et a<b$\\
			\\
			Montrer que $\forall z\in\R,\ \left( f(a)<z<f(b)\right) \Rightarrow \left( \exists x\in ]a,b[,\ f(x)=z\right) $\\
			Soit $z$ un réel compris strictement entre $f(a)\et f(b)$.\\
			On considère l'ensemble $E= \left\lbrace x\in [a,b],\ f(x)<z\right\rbrace $\\
			\\
			\underline{Principe de Borne supérieure}
			\begin{tab}
				Montrer que $E$ admet une borne supérieure:\begin{liste}
					\item $E$ est non vide: $a\in E$
					\item $E$ est majoré par $b$
				\end{liste}
				Donc $E$ admet une borne supérieure que l'on notera $c$\\
				On a: $a\leqslant c\leqslant b$\\
				Et $\forall \varepsilon \in\R^{+*},\ \exists x\in E,\ c-\varepsilon<x\leqslant c$\\
				Pour tout $n$ de $\N^{*}$, on pose $\varepsilon = \frac1{n}$, et on pose $x_n$ un réel vérifiant le critère.\\
				On définit donc une suite: $\forall n\in\N^{*},\ x_n\in E\et c-\frac1{n}<x_n\geqslant c$\\
				En particulier: $\forall n\in\N^{*},\ x_n\in[a,b]\et f(x_n)<z\et |x_n-c|<\frac1{n}$\\
				Ainsi, la suite $(x_n)_{n\in\N^{*}}$ converge vers $c$.\\
				Donc $\left( f(x_n)\right) _{n\in\N^{*}}$ converge vers $f(c)$ d'après la caractérisation séquentielle de la limite.\\
				Or, $\forall n\in\N^{*},\ f(x_n)<z$\\
				Donc $f(c)\leqslant z$\\
				On a donc $f(c)\leqslant z<f(b)$\\
				D'o\`u $c<b$.\\
				Par définition de $c$: $\begin{aligned}[t]
				& \forall x\in ]c,b[,\ x\notin E\\
				& \Rightarrow \forall x\in ]c,b[,\ f(x)\geqslant z
				\end{aligned}$\\
				Par ailleurs, $f$ est continue, donc sa limite \`a droite en $c$ existe et vaut $f(c)$.\\
				Ainsi, $f(c)\geqslant z$\\
				\\
				Conclusion: $f(c)=z$
			\end{tab}	
			Conclusion générale: $\exists c\in ]a,b[,\ f(c)=z$\\
			\\
			Ce raisonnement est valable pour tout $z$ entre $a$ et $b$. On étend le raisonnement \`a $y$ et $y'$ dans $f(I)$\\
			On conclut que $f(I)$ est un intervalle.
		\end{preuve}
		\begin{prop}
			L'image d'un segment par une application continue est un segment.
		\end{prop}
		\begin{preuve}
			Soit $I$ un segment réel non vide.\\
			Soit $f:I\rightarrow \R$ continue sur $I$.\\
			D'apr\`es le TVI, $f(I)$ est un intervalle.\\
			\\
			Montrer que $f(I)$ est fermé et borné.
			\begin{liste}
				\item Montrer que $f(I)$ est borné.\\
					C'est \`a dire, montrer que $\exists M\in\R^{+},\ \forall y\in\R, y\in f(I)\Rightarrow |y|\leqslant M$\\
					\fbox{HA} Supposons que $f(I)$ ne soit pas borné.
					\begin{tab}
						Donc $\forall M\in\R^{+},\ \exists y\in\R,\ y\in I\et |y|> M$\\
						Soit $x_0$ un élément de $I$.
						\begin{liste}
							\item On considère $E_1=\{x\in I,\ |f(x)|>f(x_0)+1 \}$\\
								$f(I)$ n'est pas borné, donc $E_1$ est non vide.\\
								Notons $x_1$ un élément de $E_1$.
							\item Soit $n\in\N$, supposons construits $(x_i)_{i\in\lbrack0,n\rbrack}$,\\
								Tels que: $\forall i\in\lbrack1,n\rbrack,\ |f(x_i)|>|f(x_{i-1})|+1$\\
								Soit $E_{n+1}=\{x\in I,\ |f(x)|>|f(x_n)|+1 \}$\\
								$f(I)$ n'est pas borné, donc $E_{n+1}$ n'est pas vide.\\
								On note $x_{n+1}$ un élément de cet ensemble.
							\item Par récurrence, on construit une suite $(x_n)_{n\in \N}$.
						\end{liste}
						De plus, $\forall n\in\N^{*},\ |f(x_n)|>|f(x_{n-1})|+1$\\
						Par récurrence, on montre que: $\forall n\in\N,\ |f(x_n)|>|f(x_0)|+n$\\
						Ainsi, $\left( |f(x_n)|\right) _{n\in\N}$ diverge vers $+\infty$.\\
						\newpage
						Par ailleurs, $(x_n)_{n\in\N}$ est une suite d'éléments de $I$. Donc d'après le théorème de Bolzano-Weierstrass, il existe $\phi:\N\Rightarrow\N$ strictement croissante telle que $(x_{\phi(n)})_{n\in\N}$ converge.\\
						Notons $\phi$ une telle suite et $l$ la limite.
						\begin{liste}
							\item $\forall n\in\N,\ x_n\in I$, donc $l\in I$
							\item $f$ est continue en $l\in I$, donc (cara séquentielle de la limite) $f(x_n)\mathop{\longrightarrow}\limits_{n\rightarrow +\infty}f(l)$
							\item On a aussi \cercle1:$|f(x_n)|\mathop{\longrightarrow}\limits_{n\rightarrow +\infty}|f(l)|$\ \ \ \ (car $abs$ est continue en $f(l)$)
							\item De plus, $(|f(x_{\phi(n)})|)_{n\in\N}$ est une suite extraite de $(|f(x_{n})|)_{n\in\N}$.\\
								Donc \cercle2:$|f(x_{\phi(n)})|\mathop{\longrightarrow}\limits_{n\rightarrow +\infty}+\infty$
						\end{liste}
						On a contradiction entre \cercle1 et \cercle2. On en déduit que \fbox{HA} est fausse.
					\end{tab}
					Conclusion: $f(I)$ est in intervalle borné.
				\item Montrer que $\exists (c,d)\in\R,\ f(I)=[c,d]$\\
					On pose $c=\inf(f(I))\et d=\sup(f(I))$\\
					Montrer que $c\in I\et d\in I$
					\begin{liste}
						\item Montrons que $d\in I$\\
							Donc Montrons que $\exists x\in I,\ f(x)=d$\\
							En appliquant le principe de la borne supérieure avec $\varepsilon = \frac1{n}$ pour tout $n$ de $\N^{*}$, on construit une suite $(x_n)_{n\in\N^{*}}$ d'éléments de $I$ telle que $(f(x_n))_{n\in\N^{*}}$ converge vers $d$.
							\begin{liste}
								\item $(x_n)$ est une suite de réels bornée, donc d'après le théorème de B-W, il existe une application strictement croissante $\phi:\N\rightarrow\N$ telle que $(x_{\phi(n)})_{n\in\N}$ soit convergente.\\
									Notons $l$ sa limite.\\
									De plus, $l\in I$.
								\item $f$ est continue sur $I$ donc $(f(x_{\phi(n)}))_{n\in\N}$ converge vers $f(l)$
								\item Par ailleurs, $(f(x_{\phi(n)}))$ est une suite extraite de $(f(x_n))$, donc $(f(x_{\phi(n)}))$ converge vers $d$.
							\end{liste}
							Par unicité de la limite, $d=f(l)$. Or, $l\in I$.\\
							Finalement, $d\in f(I)$
						\item On procède de m\^eme pour montrer que $c\in f(I)$.
					\end{liste}
			\end{liste}
			\textbf{Conclusion générale:} $\exists (c,d)\in \R^{2},\ f(I)=[c,d]$
		\end{preuve}
		\newpage
		\begin{prop}
			Soit $f:I\rightarrow \R$ une fonction continue.\\
			Alors on a équivalence entre:
			\begin{liste}
				\item[\cercle1] $f$ est injective
				\item[\cercle2] $f$ est strictement monotone
			\end{liste}
		\end{prop}
		\begin{preuve}
			Soit $f:I\rightarrow \R$ une fonction continue.\\
			\underline{\cercle1$\Rightarrow$\cercle2:} Supposons $f$ injective.
			\begin{tab}
				Montrer que $f$ est strictement croissante sur $I$.\\
				Donc montrer que $f$ est croissante sur $I$.\ \ \ \ (car $f$ est injective)\\
				Montrer que pour tous $x_1<x_2<x_3$ de $I$, $f(x_2)$ soit compris entre $f(x_1)\et f(x_3)$\\
				\fbox{HA} Supposons qu'il existe $x_1<x_2<x_3$ de $I$ tels que $f(x_2)$ ne soit pas compris
				\begin{tab}
					entre $f(x_1)\et f(x_3)$.\\
					Alors il existe $y_0\in\R$, tel que $y_0$ soit compris strictement entre $f(x_1) \et f(x_2)$ et entre $f(x_2) \et f(x_3)$\\
					D'après le TVI, $\exists \alpha\in]x_1,x_2[,\ y_0=f(\alpha)$ et $\exists \beta\in]x_2,x_3[,\ y_0=f(\beta)$\\
					Les intervalles $]x_1,x_2[\et]x_2,x_3[$ sont disjoints, donc $\alpha\neq\beta$.\\
					Cependant, $f(\alpha)=f(\beta)$, ce qui contredit l'injectivité de $f$.\\
					Donc \fbox{HA} est contradictoire.
				\end{tab}
				On conclut que $f$ est monotone sur $I$\\
				Or, $f$ est injective.\\
				\\
				\textbf{Conclusion générale:} $f$ est strictement monotone sur $I$
			\end{tab}
			\underline{\cercle2$\Rightarrow$\cercle1:} Facile.
		\end{preuve}
\end{document}