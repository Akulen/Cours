% !TeX encoding = UTF-8
\documentclass[12pt,twoside,a4paper]{article}


\def\chapitre{Fonctions Num\'eriques}
\author{MPSI 2}
\def\titre{Fonctions continues sur un intervalle}

\usepackage{amsfonts}
\usepackage{amsmath}
\usepackage{amsthm}
\usepackage{changepage}
\usepackage{color}
\usepackage{enumitem}
\usepackage{fancyhdr}
\usepackage{framed}
\usepackage[margin=1in]{geometry}
\usepackage{mathrsfs}
\usepackage{tikz, tkz-tab}
\usepackage{titling}

\newtheoremstyle{dotless}{}{}{\itshape}{}{\bfseries}{}{ }{}
\theoremstyle{dotless}

\newtheorem{defs}{Definition}[subsection]
\newenvironment{defi}{\definecolor{shadecolor}{RGB}{255,236,217}\begin{shaded}\begin{defs}\ \\}{\end{defs}\end{shaded}}

\newtheorem{pro}{Propriete}[subsection]
\newenvironment{prop}{\definecolor{shadecolor}{RGB}{230,230,255}\begin{shaded}\begin{pro}\ \\}{\end{pro}\end{shaded}}

\newtheorem{cor}{Corollaire}[subsection]
\newenvironment{coro}{\definecolor{shadecolor}{RGB}{245,250,255}\begin{shaded}\begin{cor}\ \\}{\end{cor}\end{shaded}}

\setlength{\droptitle}{-1in}
\predate{}
\postdate{}
\date{}
\title{\chapitre\\\titre\vspace{-.25in}}

\pagestyle{fancy}
\makeatletter
\lhead{\chapitre\ - \titre}
\rhead{\@author}
\makeatother

\newenvironment{preuve}{\begin{framed}\begin{proof}[\unskip\nopunct]}{\end{proof}\end{framed}}
\newenvironment{liste}{\begin{itemize}[leftmargin=*,noitemsep, topsep=0pt]}{\end{itemize}}
\newenvironment{tab}{\begin{adjustwidth}{.5cm}{}}{\end{adjustwidth}}

\newcommand{\uu}[1] {_{_{#1}}}
\newcommand{\lbracket}{[\![}
\newcommand{\rbracket}{]\!]}
\newcommand{\fonction}[5]{\begin{aligned}[t]#1\colon&#2&&\longrightarrow#3 \\&#4&&\longmapsto#5\end{aligned}}
\newcommand{\systeme}[1]{\left\{\begin{aligned}#1\end{aligned}\right.}
\newcommand{\cercle}[1]{\textcircled{\scriptsize{#1}}}

%Auteur: Cl\'ement Phan, MPSI 2

\begin{document}
	\maketitle
	\section{Fonctions continues}
		\begin{flushleft}
			Soit $I$ un intervalle non vide.\\
			Soit $f: I\rightarrow \R$ une fonction définie sur $I$.\\
			On dit que \underline{$f$ est continue sur $I$} si pout tout $x_0$ de $I$, $f$ est continue en $x_0$.
		\end{flushleft}
		\begin{theo}{des valeurs intermédiaires}
			L'image d'un intervalle par une fonction continue est un intervalle.
		\end{theo}
		\begin{preuve}
			Soit $I$ un intervalle.\\
			Soit $f:I\rightarrow R$ une application continue sut $I$.\\
			Montrer que $f(I)$ est un intervalle.\\
			Ou montrer que $\forall (y,y')\in \R^{2},\ ((y,y')\in f(I)^{2}\Rightarrow (\forall y''\in\R,\ y<y''<y'\Rightarrow y''\in f(I))$\\
			\\
			Soit $y\et y'$ deux éléments distincts de $f(I)$.\\
			Alors il existe $a\et b$ dans $I$ tels que: $f(a)=y\et f(b)=y'$\\
			$y\et y'$ sont distincts, donc $a\et b$ sont distincts.\\
			On suppose par exemple que $f(a)<f(b)\et a<b$\\
			\\
			Montrer que $\forall z\in\R,\ \left( f(a)<z<f(b)\right) \Rightarrow \left( \exists x\in ]a,b[,\ f(x)=z\right) $\\
			Soit $z$ un réel compris strictement entre $f(a)\et f(b)$.\\
			On considère l'ensemble $E= \left\lbrace x\in [a,b],\ f(x)<z\right\rbrace $\\
			\\
			\underline{Principe de Borne supérieure}
			\begin{tab}
				Montrer que $E$ admet une borne supérieure:\begin{liste}
					\item $E$ est non vide: $a\in E$
					\item $E$ est majoré par $b$
				\end{liste}
				Donc $E$ admet une borne supérieure que l'on notera $c$\\
				\\
				On a: $a\leqslant c\leqslant b$\\
				Et $\forall \varepsilon \in\R^{+*},\ \exists x\in E,\ c-\varepsilon<x\leqslant c$\\
				Pour tout $n$ de $\N^{*}$, on pose $\varepsilon = \frac1{n}$, et on pose $x_n$ un réel vérifiant le critère.\\
				On définit donc une suite: $\forall n\in\N^{*},\ x_n\in E\et c-\frac1{n}<x_n\geqslant c$
				
			\end{tab}
		\end{preuve}
\end{document}