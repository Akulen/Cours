% !TeX encoding = UTF-8
\documentclass[12pt,twoside,a4paper]{article}


\def\chapitre{Fonctions Num\'eriques}
\author{MPSI 2}
\def\titre{Limites de fonctions}

\usepackage{amsfonts}
\usepackage{amsmath}
\usepackage{amsthm}
\usepackage{changepage}
\usepackage{color}
\usepackage{enumitem}
\usepackage{fancyhdr}
\usepackage{framed}
\usepackage[margin=1in]{geometry}
\usepackage{mathrsfs}
\usepackage{tikz, tkz-tab}
\usepackage{titling}

\newtheoremstyle{dotless}{}{}{\itshape}{}{\bfseries}{}{ }{}
\theoremstyle{dotless}

\newtheorem{defs}{Definition}[subsection]
\newenvironment{defi}{\definecolor{shadecolor}{RGB}{255,236,217}\begin{shaded}\begin{defs}\ \\}{\end{defs}\end{shaded}}

\newtheorem{pro}{Propriete}[subsection]
\newenvironment{prop}{\definecolor{shadecolor}{RGB}{230,230,255}\begin{shaded}\begin{pro}\ \\}{\end{pro}\end{shaded}}

\newtheorem{cor}{Corollaire}[subsection]
\newenvironment{coro}{\definecolor{shadecolor}{RGB}{245,250,255}\begin{shaded}\begin{cor}\ \\}{\end{cor}\end{shaded}}

\setlength{\droptitle}{-1in}
\predate{}
\postdate{}
\date{}
\title{\chapitre\\\titre\vspace{-.25in}}

\pagestyle{fancy}
\makeatletter
\lhead{\chapitre\ - \titre}
\rhead{\@author}
\makeatother

\newenvironment{preuve}{\begin{framed}\begin{proof}[\unskip\nopunct]}{\end{proof}\end{framed}}
\newenvironment{liste}{\begin{itemize}[leftmargin=*,noitemsep, topsep=0pt]}{\end{itemize}}
\newenvironment{tab}{\begin{adjustwidth}{.5cm}{}}{\end{adjustwidth}}

\newcommand{\uu}[1] {_{_{#1}}}
\newcommand{\lbracket}{[\![}
\newcommand{\rbracket}{]\!]}
\newcommand{\fonction}[5]{\begin{aligned}[t]#1\colon&#2&&\longrightarrow#3 \\&#4&&\longmapsto#5\end{aligned}}
\newcommand{\systeme}[1]{\left\{\begin{aligned}#1\end{aligned}\right.}
\newcommand{\cercle}[1]{\textcircled{\scriptsize{#1}}}

\newcommand{\lf}[1]{\left(#1\right)}
\newcommand{\C}{\mathbb{C}}
\newcommand{\R}{\mathbb{R}}
\newcommand{\K}{\mathbb{K}}
\newcommand{\N}{\mathbb{N}}
\newcommand{\I}{\mathcal{I}}
\newcommand{\F}{\mathcal{F}}
\newcommand{\E}{\mathcal{E}}
\newcommand{\G}{\mathcal{G}}
\newcommand{\et}{\text{ et }}
\newcommand{\ou}{\text{ ou }}
\newcommand{\xou}{\ \fbox{\text{ou}}\ }


%Auteur: Cl\'ement Phan, MPSI 2

\begin{document}
	\maketitle
	\section{D\'efinitions}
		\begin{defi}
			Soit $f\in\mathcal{F}(I,\R)$\\
			Soit $x_0\in\R$, tel que $x_0\in I$ ou $x_0$ est une extr\'emit\'e de $I$.\\
			Soit $l\in\R$\\
			\textbullet \underline{$f(x)$ tend vers $l$ quand $x$ tend vers $x_0$:}
			$$\forall \varepsilon\in\R^{+*},\ \exists \alpha \in \R^{+*},\ \forall x\in I,\ \left|x-x_0 \right|<\alpha \Rightarrow \left| f(x)-f(x_0)\right| <\varepsilon$$
		\end{defi}
		\begin{defi}
			Soit $f\in\mathcal{F}(I,\R)$\\
			Soit $x_0\in\R$, tel que $x_0\in I$ ou $x_0$ est une extr\'emit\'e de $I$.\\
			\textbullet \underline{$f(x)$ tend vers $+\infty$ quand $x$ tend vers $x_0$:}
			$$\forall K\in\R,\ \exists \alpha \in \R^{+*},\ \forall x\in I,\ \left|x-x_0 \right|<\alpha \Rightarrow K<f(x)$$\\
			\textbullet \underline{$f(x)$ tend vers $-\infty$ quand $x$ tend vers $x_0$:}
			$$\forall K\in\R,\ \exists \alpha \in \R^{+*},\ \forall x\in I,\ \left|x-x_0 \right|<\alpha \Rightarrow f(x)<K$$
		\end{defi}
		\begin{prop}
			Si $x_0\in I$, alors la seule limite \'eventuelle de $f(x)$ en $x_0$ est $f(x_0)$
		\end{prop}
		\begin{preuve}
			On suppose qu'il existe $l$ dans $\R$, tel que $f(x) \mathop{\longrightarrow}\limits_{x\rightarrow x_0} l$\\
			\fbox{HA} $l\neq f(x_0)$
			\begin{liste}
				\item[\cercle1] $l\in\R$\\
					Alors $\forall \varepsilon\in\R^{+*},\ \exists \alpha \in \R^{+*},\ \forall x\in I,\ \left|x-x_0 \right|<\alpha \Rightarrow \left| f(x)-f(x_0)\right| <\varepsilon$\\
					Supposons $l>f(x_0)$\\
					Posons $\varepsilon=\frac{l-f(x_0)}{2}$\\
					Alors $f(x_0)\notin ]l-\varepsilon,l+\varepsilon[$.\\
					Soit $\alpha$ v\'erifiant les conditions de limites.\\
					Donc $\forall x\in I,\ \left|x-x_0 \right|<\alpha \Rightarrow \left| f(x)-f(x_0)\right| <\varepsilon$\\
					En particulier, avec $x=x_0$, on a $f(x_0)\in ]l-\varepsilon,l+varepsilon[$\\
					On a donc une contradiction.
				\item[\cercle2] $l=+\infty$\\
					Alors $\forall K\in\R,\ \exists \alpha \in \R^{+*},\ \forall x\in I,\ \left|x-x_0 \right|<\alpha \Rightarrow K<f(x)$\\
					Soit $K$ un r\'eel strictement sup\'erieur \`a $f(x_0)$\\
					Soit $\alpha$ un r\'eel v\'erifiant les condition de limites.\\
					Donc $\forall x\in I,\ \left|x-x_0 \right|<\alpha \Rightarrow f(x) >K$\\
					En particulier, avec $x=x_0$, on a $f(x_0)>K$\\
					On a donc une contradiction.
				\item[\cercle3] $l=-\infty$\\
					On procède de m\^eme.
			\end{liste}
			\begin{flushleft}
				\textbf{Conclusion:} $l=f(x_0)$
			\end{flushleft}
		\end{preuve}
		\begin{defi}
			Soit $f\in \mathcal{F}(I,\R)$\\
			\textbullet \ \underline{$f(x)$ tend vers $l\in\R$ lorsque $x$ tend vers $+\infty$:}
			$$\forall \varepsilon \in \R^{+*},\ \exists k\in\R,\ \forall x\in I,\ x>k\Rightarrow \left| f(x)-l\right| <\varepsilon$$
			\textbullet \ \underline{$f(x)$ tend vers $+\infty$ lorsque $x$ tend vers $+\infty$:}
			$$\forall K \in \R,\ \exists k\in\R,\ \forall x\in I,\ x>k\Rightarrow f(x)>K$$
		\end{defi}
		\begin{prop}
			Soit $f\in \mathcal{F}(I,\R)$.\\
			Soit $x_0\in \overline{\R}$ tel que $x_0$ soit un \'el\'ement de $I$ ou une extr\'emit\'e de $I$.\\
			Soit $(l,l')\in \overline{\R}\times \overline{\R}$.\\
			Si $f$ admet $l$ et $l'$ comme limite en $x_0$, alors $l=l'$
		\end{prop}
		\begin{flushleft}
			\textbf{Notations:}$\lim\limits_{\substack{x\rightarrow x_0 \\x\in I}} f(x)=l$ et $f(x) \mathop{\longrightarrow}\limits_{\substack{x\rightarrow x_0\\x\in I}} l$\\
		\end{flushleft}
		\begin{preuve}
			Cas o\`u $x_0\in\R \et l\in\R\et l'\in\R$\\
			\cercle1 : $\forall \varepsilon\in\R^{+*},\ \exists \alpha_1 \in \R^{+*},\ \forall x\in I,\ \left|x-x_0 \right|<\alpha_1 \Rightarrow \left| f(x)-l)\right| <\varepsilon$\\
			\cercle2 : $\forall \varepsilon\in\R^{+*},\ \exists \alpha_2 \in \R^{+*},\ \forall x\in I,\ \left|x-x_0 \right|<\alpha_2 \Rightarrow \left| f(x)-l'\right| <\varepsilon$\\
			Supposons $l\neq l'$, et $l>l'$\\
			Posons $\varepsilon = \frac{l-l'}{2}$\\
			On a donc $]l-\varepsilon,l+\varepsilon [\cap]l'-\varepsilon, l'+\varepsilon [=\varnothing$\\
			\\
			Soit $\alpha_1 \et \alpha_2$ v\'erifiant \cercle1 et \cercle2.\\
			Soit $\alpha=\min(\{\alpha_1,\alpha_2 \})$\\
			Alors $\forall x\in I,\ \left|x-x_0 \right|<\alpha \Rightarrow (\left| f(x)-l\right| <\varepsilon \et \left| f(x)-l'\right| <\varepsilon)$\\
			Autrement dit: $\forall x\in I,\ \left|x-x_0 \right|<\alpha \Rightarrow f(x)\in ]l-\varepsilon,l+\varepsilon [\cap]l'-\varepsilon, l'+\varepsilon [$\\
			On a donc une contradiction.\\
			\textbf{Conclusion:} $l=l'$
		\end{preuve}
		\begin{flushleft}
			\textbf{Remarques:}
			\begin{liste}
				\item Soit $l\in\R$. Alors $f(x) \mathop{\longrightarrow}\limits_{\substack{x\rightarrow x_0\\x\in I}} l \iff f(x) -l \mathop{\longrightarrow}\limits_{\substack{x\rightarrow x_0\\x\in I}} 0$
				\item Soit $l\in\R^{+*}$. Alors f(x) $\mathop{\longrightarrow}\limits_{\substack{x\rightarrow x_0\\x\in I}} l \iff \frac{f(x)}{l}\mathop{\longrightarrow}\limits_{\substack{x\rightarrow x_0\\x\in I}} 1$
				\item Soit $x_0\in I$. Alors $f(x)\mathop{\longrightarrow}\limits_{\substack{x\rightarrow x_0\\x\in I}} l \iff f(x_0+h)\mathop{\longrightarrow}\limits_{\substack{x_0+h\in I\\h\rightarrow 0}} l$
			\end{liste}
		\end{flushleft}
		\begin{prop}
			On suppose que $f(x)$ tend vers $l\in\overline{\R}$ quand $x$ tend vers $x_0\in I$.
			\begin{liste}
				\item Si $f(x)\in [a,b]$ au voisinage de $x_0$, alors $l\in [a,b]$.
				\item Au voisinage de $x_0$: $l-1<f(x)<l+1$
				\item Si $l\neq 0$ alors au voisinage de $x_0$: $\frac{|l|}{2}<\left| f(x) \right| <\frac{3\,|l|}{2}$
			\end{liste}
		\end{prop}
		\begin{preuve}
			\begin{flushleft}
				\underline{$1^{\text{er}}$ point} dans le cas o\`u $x_0=+\infty$
				\begin{tab}
					Donc \cercle1$:\forall \varepsilon\in\R^{+*},\ \exists k\in\R,\ \forall x\in I,\ x>k\Rightarrow \left| f(x)-l \right| <\varepsilon$\\
					On suppose $\exists k\in\R,\ \forall x\in I,\ x>k\Rightarrow f(x)\in[a,b]$\\
					Montrer que $l\in[a,b]$\\
					\fbox{HA} $l\notin [a,b]$. Donc $l<a$ ou $l>b$.
					\begin{liste}
						\item Si $l<a$\\
							Soit $\varepsilon=a-l$ (car $l<a$)\\
							Donc \cercle2$:\exists k\in\R,\ \forall x\in I,\ x>k\Rightarrow f(x)\in ]2l-a,a[$\\
							Soit $k_1$ et $k_2$ deux r\'eels v\'erifiant \cercle1 et \cercle2.\\
							On pose $k=\min\{k_1,k_2\}$\\
							D'apr\`es \cercle1 et \cercle2: $\forall x\in I,\ x>k\Rightarrow f(x)\in ]2l-a,a[\cap[a,b]$\\
							Or, $]2l-a,a[\cap[a,b]=\varnothing$\\
							On a donc une contradiction.
						\item Si $l>b$, on proc\`ede de m\^eme.
					\end{liste}
					On conclut que $l\in[a,b]$
				\end{tab}
				\underline{$2^{\text{\`eme}}$ point:} On revient aux d\'efinitions avec $\varepsilon=1$\\
				\underline{$3^{\text{\`eme}}$ point:} On revient aux d\'efinitions et on prend $\varepsilon=\frac{|l|}{2}$
			\end{flushleft}
		\end{preuve}
		\begin{prop}
			Soit $x_0\in \R$.
			\begin{liste}
				\item Si $f(x)\mathop{\longrightarrow}\limits_{x\rightarrow x_0}l$, avec $l\in\R$,\\
					Alors $f$ est born\'ee au voisinage de $x_0$.
				\item Si $f(x)\mathop{\longrightarrow}\limits_{x\rightarrow x_0} l$, avec $l\in\R^{*}$,\\
					Alors $|f(x)|$ est minor\'e par un nombre strictement positif au voisinage de $x_0$
				\item Si $f(x)$ est de signe constant au voisinage de $+\infty$, et si $f(x)\mathop{\longrightarrow}\limits_{x\rightarrow x_0}l$,\\
					Alors $l$ est du m\^eme signe.
			\end{liste}
		\end{prop}
		\begin{preuve}
			Utilisation des propri\'et\'es pr\'ec\'edentes.
		\end{preuve}
		\begin{defi}
			Soit $f\in\mathcal{F}(I,\R)$.\\
			Soit $x_0\in I$.\\
			On note $I^{+}=I\ \cap\  ]x_0,+\infty[$ et $I^{-}=I\ \cap\  ]-\infty,x_0[$
			\begin{liste}
				\item On appelle \underline{limite \`a droite de $f(x)$ en $x_0$} la limite finie, si elle existe, de $f(x)$ lorsque $x$ tend vers $x_0$ sur $I^{+}$
				\item On appelle \underline{limite \`a gauche de $f(x)$ en $x_0$} la limite finie, si elle existe, de $f(x)$ lorsque $x$ tend vers $x_0$ sur $I^{-}$
			\end{liste}
		\end{defi}
		\begin{flushleft}
			\textbf{Notations:} $\lim\limits_{\substack{x\rightarrow x_0\\x\in I^{+}}} f(x) = f(x_0^{+})$ et $\lim\limits_{\substack{x\rightarrow x_0\\x\in I^{-}}} f(x) = f(x_0^{-})$
		\end{flushleft}
		\begin{prop}
			Soit $x_0\in I$
			\begin{liste}
				\item Si $f(x) \mathop{\longrightarrow}\limits_{x\rightarrow x_0} l$, alors $f({x_0}^{+})=f({x_0}^{-})=l$.
				\item Si $f(x)$ admet une limite \`a droite et \`a gauche en $x_0$, et si $f({x_0}^{+})=f({x_0}^{-})=f(x_0)$,\\
				Alors $f(x) \mathop{\longrightarrow}\limits_{x\rightarrow x_0} f(x_0)$.
			\end{liste}
		\end{prop}
		\begin{preuve}
			\underline{$1^{\text{er}}$ point}: On suppose $\forall \varepsilon\in\R^{+*},\exists \alpha \in \R^{+*},\forall x\in I,\left|x-x_0 \right|<\alpha \Rightarrow \left| f(x)-l\right| <\varepsilon$
			\begin{tab}
				Alors: $\forall \varepsilon\in\R^{+*},\exists \alpha \in \R^{+*},\forall x\in I^{+},\left|x-x_0 \right|<\alpha \Rightarrow \left| f(x)-l\right| <\varepsilon$\\
				Et: $\forall \varepsilon\in\R^{+*},\exists \alpha \in \R^{+*},\forall x\in I^{-},\left|x-x_0 \right|<\alpha \Rightarrow \left| f(x)-l\right| <\varepsilon$\\
				Donc $f(x)$ admet une limite \`a droite et \`a gauche en $x_0$, et $f({x_0}^{+})=f({x_0}^{-})=l$
			\end{tab}
			\underline{$2^{\text{\`eme}}$ point}: On suppose que $f({x_0}^{+})=f({x_0}^{-})=f(x_0)$.
			\begin{tab}
				Soit $\varepsilon\in \R^{+*}$ fix\'e.\\
				On  a: $\exists \alpha_1 \in \R^{+*},\forall x\in I^{+},\left|x-x_0 \right|<\alpha_1 \Rightarrow \left| f(x)-f(x_0)\right| <\varepsilon$\\
				Et: $\exists \alpha_2 \in \R^{+*},\forall x\in I^{-},\left|x-x_0 \right|<\alpha_2 \Rightarrow \left| f(x)-f(x_0)\right| <\varepsilon$\\
				Soit $\alpha_1\et\alpha_2$ deux tels r\'eels.\\
				Soit $\alpha=\min(\{\alpha_1, \alpha_2 \})$\\
				D'o\`u: $\forall x\in I\setminus\{x_0\},\left|x-x_0 \right|<\alpha \Rightarrow \left| f(x)-f(x_0)\right| <\varepsilon$\\
				Par ailleurs, pour $x=x_0$: $\left|x-x_0 \right|<\alpha$ et $\left| f(x)-f(x_0)\right| <\varepsilon$\\
				Donc:$\forall x\in I,\left|x-x_0 \right|<\alpha \Rightarrow \left| f(x)-f(x_0)\right| <\varepsilon$\\
				Ce raisonnement \'etant valable pour tout $\varepsilon\in\R^{+*}$, on conclut que $f(x)\mathop{\longrightarrow}\limits_{x\rightarrow x_0} f(x_0)$
			\end{tab}
		\end{preuve}
	\section{Limites et continuit\'e}
		\begin{defi}
			Soit $I$ un intervalle non vide.\\
			Soit $f$ une fonction num\'erique d\'efinie sur $I$.\\
			Soit $x_0$ un \'el\'ement de $I$.\\
			On dit que \underline{$f$ est continue en $x_0$} si:\\
			$\forall \varepsilon \in\R^{+*},\ \exists\alpha\in\R^{+*},\ \forall x\in I,\ |x-x_0|<\alpha\Rightarrow |f(x)-f(x_0)|<\varepsilon$
		\end{defi}
		\begin{flushleft}
			\textbf{Remarque}: on peut prolonger certaines fonctions par continuit\'e.
		\end{flushleft}
	\section{Limite de fonction et convergence de suites}
		\begin{prop}
			\textbf{Caract\'erisation s\'equentielle de la limite}\\
			Soit $f\in\mathcal{F}(I,\R)$.\\
			Soit $x_0\et l$ deux \'el\'ements de $\overline{\R}$\\
			On a \'equivalence entre:
			\begin{liste}
				\item $f(x) \mathop{\longrightarrow}\limits_{x\rightarrow x_0} l$
				\item Pour toute suite $\left(u_n \right)_{n\in\N}$ d'\'el\'ements de $I$ convergeant vers $x_0$, $f\left( x_n\right) \mathop{\longrightarrow}\limits_{n\rightarrow +\infty}l$
			\end{liste}
		\end{prop}
		\begin{preuve}
			\begin{liste}
				\item On suppose que $f(x)\mathop{\longrightarrow}\limits_{x\rightarrow x_0} l$.\\
					Donc \cercle1$\ :\forall \varepsilon \in\R^{+*},\ \exists\alpha\in\R^{+*},\ \forall x\in I,\ |x-x_0|<\alpha\Rightarrow |f(x)-l|<\varepsilon$\\
					Soit $\left( x_n\right) _{n\in\N}$ une suite d'\'el\'ements de $I$ convergeant vers $x_0$.\\
					Montrer que $f\left( x_n\right) \mathop{\longrightarrow}\limits_{n\rightarrow +\infty}l$\\
					Soit $\varepsilon$ un r\'eel strictement positif.\\
					Soit $\alpha$ un r\'eel v\'erifiant \cercle1 pour $\varepsilon$.\\
					De plus, $\left( x_n\right) $ converge: $\exists n_0\in\N,\ \forall n\in\N,\ n\geqslant n_0\Rightarrow \left| x_n-x_0\right| <\alpha$\\
					Soit $n_0$ un tel entier.\\
					D'o\`u: $\forall n\in\N,\ (n\geqslant n_0)\Rightarrow (\left| x_n-x_0\right| <\alpha) \Rightarrow (|f(x_n)-l|<\varepsilon)$\\
					Donc $f\left( x_n\right) \mathop{\longrightarrow}\limits_{n\rightarrow +\infty}l$\\
					C'est vrai pour toute suite d'\'el\'ements de $I$ convergeant vers $x_0$,\\
					Donc la proposition 2 est v\'erifi\'ee.
				\item On suppose que pour toute suite $\left(u_n \right)_{n\in\N}$ d'\'el\'ements de $I$ convergeant vers $x_0$, on ait: $f\left( x_n\right) \mathop{\longrightarrow}\limits_{n\rightarrow +\infty}l$\\
				\fbox{HA} Supposons que $f(x)$ ne tende pas vers $l$ quand $x$ tend vers $x_0$
				\begin{tab}
					Alors $\exists \varepsilon\in\R^{+*},\ \forall \alpha\in\R^{+*},\ \exists x\in\I,\ |x-x_0|<\alpha\et|f(x)-l|\geqslant\varepsilon$\\
					Soit $\varepsilon$ un tel réel.\\
					Pour $n\in\N^{*}$, on pose $\alpha=\frac{1}{n}$: $\exists x\in I,\ |x-x_0|<\alpha\et|f(x)-l|\geqslant\varepsilon$\\
					Soit $x_n$ un tel réel.\\
					On procède de m\^eme pour tout $n$ de $\N^{*}$. On construit donc une suite $(x_n)_{n\in\N^{*}}$\\
					On a: $\begin{aligned}[t]
					&\bullet\ (x_n)_{n\in\N^{*}} \text{ est une suite d'éléments de }I\\
					&\bullet\ (x_n)_{n\in\N^{*}} \text{ converge vers }x_0\\
					&\bullet\ (f(x_n))_{n\in\N^{*}} \text{ ne converge pas vers }l
					\end{aligned}$\\
					On contredit l'hypothèse de départ, donc HA est fausse.
				\end{tab}
				Donc $f\left( x_n\right) \mathop{\longrightarrow}\limits_{n\rightarrow +\infty}l$
			\end{liste}
		\end{preuve}
	\section{Liens entre notion de limite et de monotonie}
		\begin{prop}
			Soit $f:I\rightarrow \R$ une application monotone sur $I$.\\
			Alors $f(x)$ admet une limite \`a droite et \`a gauche en tout point de $I$.
		\end{prop}
		\begin{preuve}
			Soit $x_0$ un élément de $I$ qui ne soit pas une extrémité de $I$
			\begin{liste}
				\item Soit $I^{+}=I\ \cap\  ]x_0,+\infty[$ et $I^{-}=I\ \cap\  ]-\infty,x_0[$\\
					$I^{+}\et I^{-}$ ne sont pas vides car $x_0$ n'est pas une extrémité de $I$.\\
					On considère $A^{+}=f(I^{+})\et A^{-}=f(I^{-})$\\
					$f$ est une application, $I^{+}\et I^{-}$ ne sont pas vides, donc $A^{+}\et A^{-}$ ne sont pas vides.\\
					$f$ est croissante, donc pour tout $x$ de $I$:
					\begin{tab}
						$x>x_0\Rightarrow f(x)\geqslant f(x_0)$\\
						$x<x_0\Rightarrow f(x)\leqslant f(x_0)$
					\end{tab}
					Ainsi, $f(x_0)$ est un minorant de $A^{+}$ et un majorant de $A^{-}$.\\
					Donc $A^{+}$ admet une borne inf notée $M$, et $A^{-}$ admet une borne sup notée $m$.\\
					On a ainsi $m\leqslant f(x_0)\leqslant M$
				\item Montrer que $f(x)$ admet une limite \`a droite en $x_0$, et que $f({x_0}^{+})=M$\\
					Soit $\varepsilon\in\R^{+*}$ fixé.\\
					Critère de borne inf avec $\varepsilon$:\\
					$\exists x_1\in I^{+},\ M\leqslant f(x_1) <M+\varepsilon$\\
					Soit $x_1$ un tel réel.\\
					Donc $\forall x\in\R,\ (x_0<x<x_1)\Rightarrow M\leqslant f(x)\leqslant f(x_1)<M+\varepsilon$\\
					Posons $\alpha = x_1-x_0$\\
					Alors $\systeme{& x_0<x<x_1\\& x\in\R}\iff \systeme{& 0<x-x_0<\alpha\\& x\in I^{+}}\iff \systeme{&|x-x_0|<\alpha\\& x\in I^{+}}$\\
					Finalement: $\forall x\in\I^{+},\ (|x-x_0|<\alpha)\Rightarrow (M\leqslant f(x)<M+\varepsilon)\Rightarrow (|f(x)-M|<\varepsilon)$\\
					Ce raisonnement est valable pour tout $\varepsilon\in\R^{+*}$,\\
					Donc on conclut que $f(x)$ admet une limite \`a droite, et que $f({x_0}^{+})=M$
				\item De m\^eme, on montre que $f({x_0}^{-})$ existe et vaut $m$.
			\end{liste}
			Ce raisonnement est vrai en tout point $x_0$ de $I$.\\
			On en conclut que $f(x)$ admet une limite \`a droite et \`a gauche en tout point de $I$.
		\end{preuve}
		\begin{prop}
			Soit $f:[a,+\infty[\rightarrow \R$ une fonction numérique croissante.\\
			De deux choses l'une:
			\begin{liste}
				\item $f$ admet un majorant $K$.\\
					Alors $f(x)$ admet une limite $l$ en $+\infty$, et $l\leqslant K$
				\item $f$ n'est pas majorée.\\
					Alors $f(x)$ tend vers $+\infty$ quand $x$ tend vers $+\infty$.
			\end{liste}
		\end{prop}
\end{document}