% !TeX encoding = UTF-8
\documentclass[12pt,twoside,a4paper]{article}


\def\chapitre{Fonctions Num\'eriques}
\author{MPSI 2}
\def\titre{Limites de fonctions}

\usepackage{amsfonts}
\usepackage{amsmath}
\usepackage{amsthm}
\usepackage{changepage}
\usepackage{color}
\usepackage{enumitem}
\usepackage{fancyhdr}
\usepackage{framed}
\usepackage[margin=1in]{geometry}
\usepackage{mathrsfs}
\usepackage{tikz, tkz-tab}
\usepackage{titling}

\newtheoremstyle{dotless}{}{}{\itshape}{}{\bfseries}{}{ }{}
\theoremstyle{dotless}

\newtheorem{defs}{Definition}[subsection]
\newenvironment{defi}{\definecolor{shadecolor}{RGB}{255,236,217}\begin{shaded}\begin{defs}\ \\}{\end{defs}\end{shaded}}

\newtheorem{pro}{Propriete}[subsection]
\newenvironment{prop}{\definecolor{shadecolor}{RGB}{230,230,255}\begin{shaded}\begin{pro}\ \\}{\end{pro}\end{shaded}}

\newtheorem{cor}{Corollaire}[subsection]
\newenvironment{coro}{\definecolor{shadecolor}{RGB}{245,250,255}\begin{shaded}\begin{cor}\ \\}{\end{cor}\end{shaded}}

\setlength{\droptitle}{-1in}
\predate{}
\postdate{}
\date{}
\title{\chapitre\\\titre\vspace{-.25in}}

\pagestyle{fancy}
\makeatletter
\lhead{\chapitre\ - \titre}
\rhead{\@author}
\makeatother

\newenvironment{preuve}{\begin{framed}\begin{proof}[\unskip\nopunct]}{\end{proof}\end{framed}}
\newenvironment{liste}{\begin{itemize}[leftmargin=*,noitemsep, topsep=0pt]}{\end{itemize}}
\newenvironment{tab}{\begin{adjustwidth}{.5cm}{}}{\end{adjustwidth}}

\newcommand{\uu}[1] {_{_{#1}}}
\newcommand{\lbracket}{[\![}
\newcommand{\rbracket}{]\!]}
\newcommand{\fonction}[5]{\begin{aligned}[t]#1\colon&#2&&\longrightarrow#3 \\&#4&&\longmapsto#5\end{aligned}}
\newcommand{\systeme}[1]{\left\{\begin{aligned}#1\end{aligned}\right.}
\newcommand{\cercle}[1]{\textcircled{\scriptsize{#1}}}

%Auteur: Cl\'ement Phan, MPSI 2

\begin{document}
	\maketitle
	\section{Définitions}
		\begin{defi}
			Soit $f\in\mathcal{F}(I,\R)$\\
			Soit $x_0\in\R$, tel que $x_0\in I$ ou $x_0$ est une extrémité de $I$.\\
			Soit $l\in\R$\\
			\textbullet \underline{$f(x)$ tend vers $l$ quand $x$ tend vers $x_0$:}
			$$\forall \varepsilon\in\R^{+*},\ \exists \alpha \in \R^{+*},\ \forall x\in I,\ \left|x-x_0 \right|<\alpha \Rightarrow \left| f(x)-f(x_0)\right| <\varepsilon$$
		\end{defi}
		\begin{defi}
			Soit $f\in\mathcal{F}(I,\R)$\\
			Soit $x_0\in\R$, tel que $x_0\in I$ ou $x_0$ est une extrémité de $I$.\\
			\textbullet \underline{$f(x)$ tend vers $+\infty$ quand $x$ tend vers $x_0$:}
			$$\forall K\in\R,\ \exists \alpha \in \R^{+*},\ \forall x\in I,\ \left|x-x_0 \right|<\alpha \Rightarrow K<f(x)$$\\
			\textbullet \underline{$f(x)$ tend vers $-\infty$ quand $x$ tend vers $x_0$:}
			$$\forall K\in\R,\ \exists \alpha \in \R^{+*},\ \forall x\in I,\ \left|x-x_0 \right|<\alpha \Rightarrow f(x)<K$$
		\end{defi}
		\begin{prop}
			Si $x_0\in I$, alors la seule limite éventuelle de $f(x)$ en $x_0$ est $f(x_0)$
		\end{prop}
		\begin{preuve}
			On suppose qu'il existe $l$ dans $\R$, tel que $f(x) \mathop{\longrightarrow}\limits_{x\rightarrow x_0} l$\\
			\fbox{HA} $l\neq f(x_0)$
			\begin{liste}
				\item[\cercle1] $l\in\R$\\
					Alors $\forall \varepsilon\in\R^{+*},\ \exists \alpha \in \R^{+*},\ \forall x\in I,\ \left|x-x_0 \right|<\alpha \Rightarrow \left| f(x)-f(x_0)\right| <\varepsilon$\\
					Supposons $l>f(x_0)$\\
					Posons $\varepsilon=\frac{l-f(x_0)}{2}$\\
					Alors $f(x_0)\notin ]l-\varepsilon,l+\varepsilon[$.\\
					Soit $\alpha$ vérifiant les conditions de limites.\\
					Donc $\forall x\in I,\ \left|x-x_0 \right|<\alpha \Rightarrow \left| f(x)-f(x_0)\right| <\varepsilon$\\
					En particulier, avec $x=x_0$, on a $f(x_0)\in ]l-\varepsilon,l+varepsilon[$\\
					On a donc une contradiction.
				\item[\cercle2] $l=+\infty$\\
					Alors $\forall K\in\R,\ \exists \alpha \in \R^{+*},\ \forall x\in I,\ \left|x-x_0 \right|<\alpha \Rightarrow K<f(x)$\\
					Soit $K$ un réel strictement supérieur \`a $f(x_0)$\\
					Soit $\alpha$ un réel vérifiant les condition de limites.\\
					Donc $\forall x\in I,\ \left|x-x_0 \right|<\alpha \Rightarrow f(x) >K$\\
					En particulier, avec $x=x_0$, on a $f(x_0)>K$\\
					On a donc une contradiction.
				\item[\cercle3] $l=-\infty$\\
					On procède de m\^eme.
			\end{liste}
			\begin{flushleft}
				\textbf{Conclusion:} $l=f(x_0)$
			\end{flushleft}
		\end{preuve}
		\begin{defi}
			Soit $f\in \mathcal{F}(I,\R)$\\
			\textbullet \ \underline{$f(x)$ tend vers $l\in\R$ lorsque $x$ tend vers $+\infty$:}
			$$\forall \varepsilon \in \R^{+*},\ \exists k\in\R,\ \forall x\in I,\ x>k\Rightarrow \left| f(x)-l\right| <\varepsilon$$
			\textbullet \ \underline{$f(x)$ tend vers $+\infty$ lorsque $x$ tend vers $+\infty$:}
			$$\forall K \in \R,\ \exists k\in\R,\ \forall x\in I,\ x>k\Rightarrow f(x)>K$$
		\end{defi}
		\begin{prop}
			Soit $f\in \mathcal{F}(I,\R)$.\\
			Soit $x_0\in \overline{\R}$ tel que $x_0$ soit un élément de $I$ ou une extrémité de $I$.\\
			Soit $(l,l')\in \overline{\R}\times \overline{\R}$.\\
			Si $f$ admet $l$ et $l'$ comme limite en $x_0$, alors $l=l'$
		\end{prop}
		\begin{flushleft}
			\textbf{Notations:}$\lim\limits_{\substack{x\rightarrow x_0 \\x\in I}} f(x)=l$ et $f(x) \mathop{\longrightarrow}\limits_{\substack{x\rightarrow x_0\\x\in I}} l$\\
		\end{flushleft}
		\begin{preuve}
			Cas o\`u $x_0\in\R \et l\in\R\et l'\in\R$\\
			\cercle1 : $\forall \varepsilon\in\R^{+*},\ \exists \alpha_1 \in \R^{+*},\ \forall x\in I,\ \left|x-x_0 \right|<\alpha_1 \Rightarrow \left| f(x)-l)\right| <\varepsilon$\\
			\cercle2 : $\forall \varepsilon\in\R^{+*},\ \exists \alpha_2 \in \R^{+*},\ \forall x\in I,\ \left|x-x_0 \right|<\alpha_2 \Rightarrow \left| f(x)-l'\right| <\varepsilon$\\
			Supposons $l\neq l'$, et $l>l'$\\
			Posons $\varepsilon = \frac{l-l'}{2}$\\
			On a donc $]l-\varepsilon,l+\varepsilon [\cap]l'-\varepsilon, l'+\varepsilon [=\varnothing$\\
			\\
			Soit $\alpha_1 \et \alpha_2$ vérifiant \cercle1 et \cercle2.\\
			Soit $\alpha=\min(\{\alpha_1,\alpha_2 \})$\\
			Alors $\forall x\in I,\ \left|x-x_0 \right|<\alpha \Rightarrow (\left| f(x)-l\right| <\varepsilon \et \left| f(x)-l'\right| <\varepsilon)$\\
			Autrement dit: $\forall x\in I,\ \left|x-x_0 \right|<\alpha \Rightarrow f(x)\in ]l-\varepsilon,l+\varepsilon [\cap]l'-\varepsilon, l'+\varepsilon [$\\
			On a donc une contradiction.\\
			\textbf{Conclusion:} $l=l'$
		\end{preuve}
		\begin{flushleft}
			\textbf{Remarques:}
			\begin{liste}
				\item Soit $l\in\R$. Alors $f(x) \mathop{\longrightarrow}\limits_{\substack{x\rightarrow x_0\\x\in I}} l \iff f(x) -l \mathop{\longrightarrow}\limits_{\substack{x\rightarrow x_0\\x\in I}} 0$
				\item Soit $l\in\R^{+*}$. Alors f(x) $\mathop{\longrightarrow}\limits_{\substack{x\rightarrow x_0\\x\in I}} l \iff \frac{f(x)}{l}\mathop{\longrightarrow}\limits_{\substack{x\rightarrow x_0\\x\in I}} 1$
				\item Soit $x_0\in I$. Alors $f(x)\mathop{\longrightarrow}\limits_{\substack{x\rightarrow x_0\\x\in I}} l \iff f(x_0+h)\mathop{\longrightarrow}\limits_{\substack{x_0+h\in I\\h\rightarrow 0}} l$
			\end{liste}
		\end{flushleft}
		\begin{prop}
			On suppose que $f(x)$ tend vers $l\in\overline{\R}$ quand $x$ tend vers $x_0\in I$\\
			
		\end{prop}
\end{document}