% !TeX encoding = UTF-8
\documentclass[12pt,twoside,a4paper]{article}


\def\chapitre{Fonctions Num\'eriques}
\author{MPSI 2}
\def\titre{Limites de fonctions}

\usepackage{amsfonts}
\usepackage{amsmath}
\usepackage{amsthm}
\usepackage{changepage}
\usepackage{color}
\usepackage{enumitem}
\usepackage{fancyhdr}
\usepackage{framed}
\usepackage[margin=1in]{geometry}
\usepackage{mathrsfs}
\usepackage{tikz, tkz-tab}
\usepackage{titling}

\newtheoremstyle{dotless}{}{}{\itshape}{}{\bfseries}{}{ }{}
\theoremstyle{dotless}

\newtheorem{defs}{Definition}[subsection]
\newenvironment{defi}{\definecolor{shadecolor}{RGB}{255,236,217}\begin{shaded}\begin{defs}\ \\}{\end{defs}\end{shaded}}

\newtheorem{pro}{Propriete}[subsection]
\newenvironment{prop}{\definecolor{shadecolor}{RGB}{230,230,255}\begin{shaded}\begin{pro}\ \\}{\end{pro}\end{shaded}}

\newtheorem{cor}{Corollaire}[subsection]
\newenvironment{coro}{\definecolor{shadecolor}{RGB}{245,250,255}\begin{shaded}\begin{cor}\ \\}{\end{cor}\end{shaded}}

\setlength{\droptitle}{-1in}
\predate{}
\postdate{}
\date{}
\title{\chapitre\\\titre\vspace{-.25in}}

\pagestyle{fancy}
\makeatletter
\lhead{\chapitre\ - \titre}
\rhead{\@author}
\makeatother

\newenvironment{preuve}{\begin{framed}\begin{proof}[\unskip\nopunct]}{\end{proof}\end{framed}}
\newenvironment{liste}{\begin{itemize}[leftmargin=*,noitemsep, topsep=0pt]}{\end{itemize}}
\newenvironment{tab}{\begin{adjustwidth}{.5cm}{}}{\end{adjustwidth}}

\newcommand{\uu}[1] {_{_{#1}}}
\newcommand{\lbracket}{[\![}
\newcommand{\rbracket}{]\!]}
\newcommand{\fonction}[5]{\begin{aligned}[t]#1\colon&#2&&\longrightarrow#3 \\&#4&&\longmapsto#5\end{aligned}}
\newcommand{\systeme}[1]{\left\{\begin{aligned}#1\end{aligned}\right.}
\newcommand{\cercle}[1]{\textcircled{\scriptsize{#1}}}

\newcommand{\lf}[1]{\left(#1\right)}
\newcommand{\C}{\mathbb{C}}
\newcommand{\R}{\mathbb{R}}
\newcommand{\K}{\mathbb{K}}
\newcommand{\N}{\mathbb{N}}
\newcommand{\I}{\mathcal{I}}
\newcommand{\F}{\mathcal{F}}
\newcommand{\E}{\mathcal{E}}
\newcommand{\G}{\mathcal{G}}
\newcommand{\et}{\text{ et }}
\newcommand{\ou}{\text{ ou }}
\newcommand{\xou}{\ \fbox{\text{ou}}\ }


%Auteur: Cl\'ement Phan, MPSI 2

\begin{document}
	\maketitle
	\section{Définitions}
		\begin{defi}
			Soit $f\in\mathcal{F}(I,\R)$\\
			Soit $x_0\in\R$, tel que $x_0\in I$ ou $x_0$ est une extrémité de $I$.\\
			Soit $l\in\R$\\
			\textbullet \underline{$f(x)$ tend vers $l$ quand $x$ tend vers $x_0$:}
			$$\forall \varepsilon\in\R^{+*},\ \exists \alpha \in \R^{+*},\ \forall x\in I,\ \left|x-x_0 \right|<\alpha \Rightarrow \left| f(x)-f(x_0)\right| <\varepsilon$$
		\end{defi}
		\begin{defi}
			Soit $f\in\mathcal{F}(I,\R)$\\
			Soit $x_0\in\R$, tel que $x_0\in I$ ou $x_0$ est une extrémité de $I$.\\
			\textbullet \underline{$f(x)$ tend vers $+\infty$ quand $x$ tend vers $x_0$:}
			$$\forall K\in\R,\ \exists \alpha \in \R^{+*},\ \forall x\in I,\ \left|x-x_0 \right|<\alpha \Rightarrow K<f(x)$$\\
			\textbullet \underline{$f(x)$ tend vers $-\infty$ quand $x$ tend vers $x_0$:}
			$$\forall K\in\R,\ \exists \alpha \in \R^{+*},\ \forall x\in I,\ \left|x-x_0 \right|<\alpha \Rightarrow f(x)<K$$
		\end{defi}
		\begin{prop}
			Si $x_0\in I$, alors la seule limite éventuelle de $f(x)$ en $x_0$ est $f(x_0)$
		\end{prop}
		\begin{preuve}
			On suppose qu'il existe $l$ dans $\R$, tel que $f(x) \mathop{\longrightarrow}\limits_{x\rightarrow x_0}\limits_{plop} l$
			\end{preuve}
\end{document}