\documentclass[12pt,twoside,a4paper]{article}

\usepackage[francais]{babel}
\usepackage[ansinew]{inputenc}

\def\chapitre{Fonctions Numériques}
\author{MPSI 2}
\def\titre{Généralités}

\usepackage{amsfonts}
\usepackage{amsmath}
\usepackage{amsthm}
\usepackage{changepage}
\usepackage{color}
\usepackage{enumitem}
\usepackage{fancyhdr}
\usepackage{framed}
\usepackage[margin=1in]{geometry}
\usepackage{mathrsfs}
\usepackage{tikz, tkz-tab}
\usepackage{titling}

\newtheoremstyle{dotless}{}{}{\itshape}{}{\bfseries}{}{ }{}
\theoremstyle{dotless}

\newtheorem{defs}{Definition}[subsection]
\newenvironment{defi}{\definecolor{shadecolor}{RGB}{255,236,217}\begin{shaded}\begin{defs}\ \\}{\end{defs}\end{shaded}}

\newtheorem{pro}{Propriete}[subsection]
\newenvironment{prop}{\definecolor{shadecolor}{RGB}{230,230,255}\begin{shaded}\begin{pro}\ \\}{\end{pro}\end{shaded}}

\newtheorem{cor}{Corollaire}[subsection]
\newenvironment{coro}{\definecolor{shadecolor}{RGB}{245,250,255}\begin{shaded}\begin{cor}\ \\}{\end{cor}\end{shaded}}

\setlength{\droptitle}{-1in}
\predate{}
\postdate{}
\date{}
\title{\chapitre\\\titre\vspace{-.25in}}

\pagestyle{fancy}
\makeatletter
\lhead{\chapitre\ - \titre}
\rhead{\@author}
\makeatother

\newenvironment{preuve}{\begin{framed}\begin{proof}[\unskip\nopunct]}{\end{proof}\end{framed}}
\newenvironment{liste}{\begin{itemize}[leftmargin=*,noitemsep, topsep=0pt]}{\end{itemize}}
\newenvironment{tab}{\begin{adjustwidth}{.5cm}{}}{\end{adjustwidth}}

\newcommand{\uu}[1] {_{_{#1}}}
\newcommand{\lbracket}{[\![}
\newcommand{\rbracket}{]\!]}
\newcommand{\fonction}[5]{\begin{aligned}[t]#1\colon&#2&&\longrightarrow#3 \\&#4&&\longmapsto#5\end{aligned}}
\newcommand{\systeme}[1]{\left\{\begin{aligned}#1\end{aligned}\right.}
\newcommand{\cercle}[1]{\textcircled{\scriptsize{#1}}}

\newcommand{\lf}[1]{\left(#1\right)}
\newcommand{\C}{\mathbb{C}}
\newcommand{\R}{\mathbb{R}}
\newcommand{\K}{\mathbb{K}}
\newcommand{\N}{\mathbb{N}}
\newcommand{\I}{\mathcal{I}}
\newcommand{\F}{\mathcal{F}}
\newcommand{\E}{\mathcal{E}}
\newcommand{\G}{\mathcal{G}}
\newcommand{\et}{\text{ et }}
\newcommand{\ou}{\text{ ou }}
\newcommand{\xou}{\ \fbox{\text{ou}}\ }


%Auteur: Cl\'ement Phan, MPSI 2

\begin{document}
	\maketitle
	\begin{defi}
		On appelle \underline{Fonction numérique} toute application de $\mathcal{F}(I,\R,G)$ o\`u:
		\begin{liste}
			\item $I$ est un intervalle réel.
			\item $G$ est un graphe de $I$ dans $\R$ associé a cette application.
		\end{liste}
		On écrit $\fonction{f}{I}{\R}{x}{f(x)}$
	\end{defi}
	\begin{flushleft}
		\underline{Notation:} $\mathcal{F}(I,\R)$ désigne l'ensemble des applications de $I$ vers $\R$.
	\end{flushleft}
	\section{Opérations}
		\begin{flushleft}
			Soit $f$ et $g$ deux fonctions numérique définies sur $I$.\\
			On pose :
			$\begin{aligned}[t]
			\bullet & \fonction{f+g}{I}{\R}{x}{f(x)+g(x)}\\
			\bullet & \fonction{f\times g}{I}{\R}{x}{f(x)\times g(x)}\\
			\bullet & \forall \lambda \in \R,\ \fonction{\lambda\dot f}{I}{\R}{x}{\lambda\times f(x)}
			\end{aligned}$\\
			On définit ainsi deux lois internes et une loi externe: $\left(\mathcal{F}(I,\R),\ +,\ \times,\ \dot \right)$ est une algèbre commutative.\\
			Cela signifie que :
			$\begin{aligned}[t]
			\bullet & \left(\mathcal{F}(I,\R),\ +,\ \times\right) \text{ est un anneau commutatif.}\\
			\bullet & \left(\mathcal{F}(I,\R),\ +,\ \dot\right) \text{ est un espace véctoriel.}\\
			\bullet & \forall (f,g,\lambda)\in \mathcal{F}(I,\R)^{2}\times\R,\ \lambda\dot (f \times g) = (\lambda\dot f) \times g = f\times (\lambda\dot g)
			\end{aligned}$
		\end{flushleft}
		\begin{flushleft}
			\textbf{Notations:}
			\begin{lsite}
				\item $0_{\mathcal{F}(I,\R)}$ désigne l'application $\fonction{}{I}{\R}{x}{0}$
				\item Si $f\in \mathcal{F}(I,\R)$, on note $\fonction{-f}{I}{\R}{x}{-f(x)}$
				\item Si $f\in\mathcal{F}(I,\R)$ vérifie $\forall x\in I,\ f(x)\neq I$,\\
					Alors on note $\fonction{\frac{1}{f}}{I}{\R}{x}{\frac{1}{f(x)}}$
			\end{lsite}
		\end{flushleft}
	\section{Relation d'ordre}
		\begin{defi}
			Soit $f$ et $g$ deux fonctions numériques définies sur $I$.\\
			On note $f \leqslant g \iff \forall x \in I,\ f(x) \leqslant g(x)$
		\end{defi}
		\begin{flushleft}
			On définit alors une relation d'ordre partielle sur $\mathcal{F}(I,\R)$.
		\end{flushleft}
		\begin{defi}
			\begin{liste}
				\item On dit que $f$ est positive sur $I$ si:\\
					$\forall x\in I,\ f(x) \geqslant 0$\\
					On note alors $f\geqslant 0$
				\item On procède de manière analogue pour $f>0$
			\end{liste}
		\end{defi}
		\begin{prop}
			Soit $f_1,\ f_2,\ g_1,\ g_2$ des fonctions numériques définies sur $I$.
			\begin{liste}
				\item $f_1 \leqslant f_2 \et g_1 \leqslant g_2 \Longrightarrow f_1+g_1 \leqslant f_2 + g_2$
				\item $\left(g_1\geqslant 0 \et f_1 \geqslant f_2 \right) \Longrightarrow f_1 \times g_1 \leqslant f_2 \times g_1$
			\end{liste}
		\end{prop}
		\begin{preuve}
			Cela découle du fait que $(\R,+,\times,\leqslant)$ soit un corps totalement ordonné.
		\end{preuve}
		\begin{defi}
			Soit $f$ une fonction numérique définie sur $I$.\\
			On pose $\fonction{f^{+}}{I}{\R}{x}{\left\{\begin{aligned}& f(x) & \text{ si }f(x)\geqslant 0\\	& 0 & \text{ si }f(x)<0	\end{aligned} \right.}$ et $\fonction{f^{-}}{I}{\R}{x}{\left\{\begin{aligned}& 0 & \text{ si }f(x)> 0\\	& -f(x) & \text{ si }f(x) \leqslant 0	\end{aligned} \right.}$\\
			$f^{+}$ et $f^{-}$ sont les parties positive et négative de $f$.
			\end{defi}
			\begin{flushleft}
				\textbf{Remarques:}
				\begin{lsite}
					\item $f^{+}$ et $f^{-}$ sont toutes deux positives.
					\item $f=f^{+}-f^{-}$
					\item $|f|=f^{+}+f^{-}$
				\end{lsite}
			\end{flushleft}
			\begin{defi}
				Soit $f$ une fonction numérique définie sur $I$.\\
				L'image de $f$, $F(I)$, est l'ensemble :\\
				$f(I)=\{y\in \R,\ \exists x\in I, f(x)=y \}$
			\end{defi}
			\begin{defi}
				Soit $f\in \mathcal{F}(I,\R)$.
				\begin{liste}
					\item On dit que \underline{$f$ est bornée sur $I$} si $f(I)$ est borné.\\
						$\exists(a,b)\in \R^{2},\ f(I)\subset [a,b]$
					\item On dit que \underline{$f$ est majorée sur $I$} si $f(I)$ est majoré.\\
						$\exists K\in\R,\ f \leqslant K$
					\item On dit que \underline{$f$ est minorée sur $I$} si $f(I)$ est minoré.\\
						$\exists k\in\R,\ k\leqslant f$
				\end{liste}
			\end{defi}
			\begin{defi}
				Soit $f\in \mathcal{F}(I,\R)$.
				\begin{liste}
					\item Si $f$ est majorée sur $I$, on appelle borne supérieure de $f$ sur $I$ la borne supérieure de $f(I)$
					\item Si $f$ est minorée sur $I$, on appelle borne inférieure de $f$ sur $I$ la borne inférieure de $f(I)$
				\end{liste}
			\end{defi}
			\begin{flushleft}
				\textbf{Notation:} %12-1-4
			\end{flushleft}
\end{document}