\documentclass[12pt,twoside,a4paper]{article}

\def\chapitre{Fractions Rationnelles}
\author{MPSI 2}
\def\titre{G\'en\'eralit\'es}

\usepackage{amsfonts}
\usepackage{amsmath}
\usepackage{amsthm}
\usepackage{changepage}
\usepackage{color}
\usepackage{enumitem}
\usepackage{fancyhdr}
\usepackage{framed}
\usepackage[margin=1in]{geometry}
\usepackage{mathrsfs}
\usepackage{tikz, tkz-tab}
\usepackage{titling}

\newtheoremstyle{dotless}{}{}{\itshape}{}{\bfseries}{}{ }{}
\theoremstyle{dotless}

\newtheorem{defs}{Definition}[subsection]
\newenvironment{defi}{\definecolor{shadecolor}{RGB}{255,236,217}\begin{shaded}\begin{defs}\ \\}{\end{defs}\end{shaded}}

\newtheorem{pro}{Propriete}[subsection]
\newenvironment{prop}{\definecolor{shadecolor}{RGB}{230,230,255}\begin{shaded}\begin{pro}\ \\}{\end{pro}\end{shaded}}

\newtheorem{cor}{Corollaire}[subsection]
\newenvironment{coro}{\definecolor{shadecolor}{RGB}{245,250,255}\begin{shaded}\begin{cor}\ \\}{\end{cor}\end{shaded}}

\setlength{\droptitle}{-1in}
\predate{}
\postdate{}
\date{}
\title{\chapitre\\\titre\vspace{-.25in}}

\pagestyle{fancy}
\makeatletter
\lhead{\chapitre\ - \titre}
\rhead{\@author}
\makeatother

\newenvironment{preuve}{\begin{framed}\begin{proof}[\unskip\nopunct]}{\end{proof}\end{framed}}
\newenvironment{liste}{\begin{itemize}[leftmargin=*,noitemsep, topsep=0pt]}{\end{itemize}}
\newenvironment{tab}{\begin{adjustwidth}{.5cm}{}}{\end{adjustwidth}}

\newcommand{\uu}[1] {_{_{#1}}}
\newcommand{\lbracket}{[\![}
\newcommand{\rbracket}{]\!]}
\newcommand{\fonction}[5]{\begin{aligned}[t]#1\colon&#2&&\longrightarrow#3 \\&#4&&\longmapsto#5\end{aligned}}
\newcommand{\systeme}[1]{\left\{\begin{aligned}#1\end{aligned}\right.}
\newcommand{\cercle}[1]{\textcircled{\scriptsize{#1}}}

\newcommand{\lf}[1]{\left(#1\right)}
\newcommand{\C}{\mathbb{C}}
\newcommand{\R}{\mathbb{R}}
\newcommand{\K}{\mathbb{K}}
\newcommand{\N}{\mathbb{N}}
\newcommand{\I}{\mathcal{I}}
\newcommand{\F}{\mathcal{F}}
\newcommand{\E}{\mathcal{E}}
\newcommand{\G}{\mathcal{G}}
\newcommand{\et}{\text{ et }}
\newcommand{\ou}{\text{ ou }}
\newcommand{\xou}{\ \fbox{\text{ou}}\ }


\DeclareMathOperator{\PGCD}{pgcd}
\DeclareMathOperator{\Max}{Max}

%Auteur: Tomas Rigaux, MPSI 2

\begin{document}
	\maketitle
	On note $\K(X)$ le corps des fractionnelles sur $\K$ (le corps des fractions de l'anneau int\`egre $\K[X]$). \\
	L'existence et la construction de $\K(X)$ sont admises. \\
	\begin{liste}
		\item Soit $(P_1,Q_1)\in\K[X]^2,(P_2,Q_2)\in\K[X]^2,Q_1\neq0\et Q_2\neq0$ \\
			\begin{liste}
				\item $\frac{P_1}{Q_1}+\frac{P_2}{Q_2}=\frac{P_1Q_2+P_2Q_1}{Q_1Q_2}$ El\'ement neutre : $\frac0Q$
				\item $\frac{P_1}{Q_1}\times\frac{P_2}{Q_2}=\frac{P_1P_2}{Q_1Q_2}$ El\'ement unit\'e : $\frac{Q}Q$
				\item $\frac{P_1}{Q_1}=\frac{P_2}{Q_2}\iff P_1Q_2=P_2Q_1$ \\
					On dit que $(P_1,Q_1)$ et $(P_2,Q_2)$ sont deux repr\'esentants de la m\^eme fraction.
			\end{liste}
		\item On a $\K\overset{\varphi}\hookrightarrow\K[X]\overset{\iota}\hookrightarrow\K(X)$ ``iota`` \\
			$$\begin{aligned}
				&\fonction{\varphi}{\K}{K[X]}{\lambda}{(\lambda,0,0,\ldots)}&&\text{est un homomorphisme d'anneaux injectif} \\
				&\fonction{\iota}{\K}{K[X]}{P}{\frac P1}&&\text{est un homomorphisme d'anneaux injectif}
			\end{aligned}$$
		\item Forme irr\'eductible d'une fraction rationnelle : \\
			$\frac PQ\text{ o\`u }\left\{\begin{aligned}&P\in\K[X],Q\in\K[X],Q\neq0 \\
														&P\wedge Q=1\\
														&Q\text{ est unitaire}\end{aligned}\right.$
	\end{liste}
	\begin{prop}
		Toute fraction rationnelle de $\K[X]$ peut \^etre mise de mani\`ere unique sous forme irr\'eductible.
	\end{prop}
	\begin{preuve}
		\begin{liste}
			\item \textbf{Existence :} Soit $\begin{aligned}[t]&(P,Q)\in\K[X]\times\K[X]^* \\
															   &D=\PGCD(P,Q)\end{aligned}$ \\
				Alors $\exists(P_1,Q_1)\in\K[X]^2,P=DP_1\et Q=DQ_1\et P_1\wedge Q_1=1$ \\
				On a : $\frac PQ=\frac{P_1}{Q_1}$ \\
				Soit $\lambda$ le coefficient dominant de $Q_1$. \\
				Alors $\lambda\in\K^*$ et : $\exists Q_2\in\K[X],Q_1=\lambda Q_2\et Q_2\text{ est unitaire}$ \\
				$$\frac{P_1}{Q_1}=\frac{\lambda(\lambda^{-1}P_1)}{\lambda Q_2}=\frac{\lambda^{-1}P_1}{Q_2}$$
				Posons $P_2=\lambda^{-1}P_1$. \\
				$P_2\wedge Q_2=1$ car $P_1\wedge Q_1=1$. \\
				Finalement, le couple $(P_2,Q_2)$ convient.
			\item \textbf{Unicit\'e :} Supposons qu'il existe deux formes irr\'eductibles $(P_1,Q_1)\et(P_2,Q_2)$ d'une m\^eme fraction. \\
				Alors :$\left\{\begin{aligned}&P_1\wedge Q_1=1 \\
											  &P_2\wedge Q_2=1 \\
											  &Q1,Q2\text{ unitaires} \\
											  &P_1Q_2-P_2Q_1=0\end{aligned}\right.$ \\
				$P_1Q_2=P_2Q_1$ En utilisant un th\'eor\`eme de GAUSS, on obtient : $\left\{\begin{aligned}&Q_2|Q_1 \\
																										   &Q_1|Q_2\end{aligned}\right.$ \\
				Donc $Q_1$ et $Q_2$ sont des polynomes associ\'es. Comme $Q_1$ et $Q_2$ sont unitaires, on en d\'eduit que $Q_1=Q_2$ , d'o\`u l'unicit\'e de l'\'ecriture.
		\end{liste}
	\end{preuve}
	\begin{defi}
		Soit $F\in\K[X]:\exists(P,Q)\in\K[X]^2,Q\neq0\et F=\frac PQ$ \\
		On pose : $\deg(F)=\deg(P)-\deg(Q)$
	\end{defi}
	\textbf{Justification :} Soit $(P,Q)$ et $(P',Q')$ deux repr\'esentants de $F$. \\
	Alors $PQ'=P'Q$ \\
	Donc : $\deg(P)+\deg(Q)=\deg(P')+\deg(Q')$ \\
	D'o\`u : $\deg(P)-\deg(Q')=\deg(P')-\deg(Q)$
	\begin{prop}
		Soit $F_1$ et $F_2$ deux \'el\'ements de $\K[X]$. On a : \\
		$$\begin{aligned}
			\deg(F_1+F_2)&\leq\Max(\{\deg(F_1),\deg(F_2)\}) \\
			\text{Si }\deg{F_1}\neq\deg{F_2}\text{ alors }\deg(F_1+F_2)&=\Max(\{\deg(F_1),\deg(F_2)\}) \\
			\deg(F_1F_2)&=\deg(F_1)+\deg(F_2)
		\end{aligned}$$
	\end{prop}
\end{document}
