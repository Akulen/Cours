% !TeX encoding = UTF-8
\documentclass[12pt,twoside,a4paper]{article}


\def\chapitre{D\'erivabilit\'e}
\author{MPSI 2}
\def\titre{D\'erivations successives}

\usepackage{amsfonts}
\usepackage{amsmath}
\usepackage{amsthm}
\usepackage{changepage}
\usepackage{color}
\usepackage{enumitem}
\usepackage{fancyhdr}
\usepackage{framed}
\usepackage[margin=1in]{geometry}
\usepackage{mathrsfs}
\usepackage{tikz, tkz-tab}
\usepackage{titling}

\newtheoremstyle{dotless}{}{}{\itshape}{}{\bfseries}{}{ }{}
\theoremstyle{dotless}

\newtheorem{defs}{Definition}[subsection]
\newenvironment{defi}{\definecolor{shadecolor}{RGB}{255,236,217}\begin{shaded}\begin{defs}\ \\}{\end{defs}\end{shaded}}

\newtheorem{pro}{Propriete}[subsection]
\newenvironment{prop}{\definecolor{shadecolor}{RGB}{230,230,255}\begin{shaded}\begin{pro}\ \\}{\end{pro}\end{shaded}}

\newtheorem{cor}{Corollaire}[subsection]
\newenvironment{coro}{\definecolor{shadecolor}{RGB}{245,250,255}\begin{shaded}\begin{cor}\ \\}{\end{cor}\end{shaded}}

\setlength{\droptitle}{-1in}
\predate{}
\postdate{}
\date{}
\title{\chapitre\\\titre\vspace{-.25in}}

\pagestyle{fancy}
\makeatletter
\lhead{\chapitre\ - \titre}
\rhead{\@author}
\makeatother

\newenvironment{preuve}{\begin{framed}\begin{proof}[\unskip\nopunct]}{\end{proof}\end{framed}}
\newenvironment{liste}{\begin{itemize}[leftmargin=*,noitemsep, topsep=0pt]}{\end{itemize}}
\newenvironment{tab}{\begin{adjustwidth}{.5cm}{}}{\end{adjustwidth}}

\newcommand{\uu}[1] {_{_{#1}}}
\newcommand{\lbracket}{[\![}
\newcommand{\rbracket}{]\!]}
\newcommand{\fonction}[5]{\begin{aligned}[t]#1\colon&#2&&\longrightarrow#3 \\&#4&&\longmapsto#5\end{aligned}}
\newcommand{\systeme}[1]{\left\{\begin{aligned}#1\end{aligned}\right.}
\newcommand{\cercle}[1]{\textcircled{\scriptsize{#1}}}

%Auteur: Tomas Rigaux, MPSI 2

\begin{document}
	\maketitle
	\begin{flushleft}
		Soit $f$ une fonction num\'erique d\'efinie sur un intervalle r\'eel $I$.
	\end{flushleft}
	\begin{defi}
		$f$ est d\'erivable sur $I$ si elle est d\'erivable en tout point de $I$. Dans ce cas, on definit la fonction d\'eriv\'ee $f'$ par :
		$$
			\fonction{f'}{I}{\R}{x}{f'\left(x_0\right)}
		$$
		\begin{liste}
			\item Si $f'$ est d\'erivable en $x_0$, on note $f"\left(x_0\right)=\left(f'\right)'\left(x_0\right)$
			\item Par r\'ecurrence, on d\'efinit, si $f^{\left(p-1\right)}$ existe et est d\'erivable en $x_0$ :
				$$
					f^{\left(p\right)}\left(x_0\right)=\left(f^{\left(p-1\right)}\right)'\left(x_0\right)
				$$
		\end{liste}
	\end{defi}
	\begin{defi}
		Soit $n\in\N$.
		\begin{liste}
			\item On dit que $f$ est de classe $\mathcal{C}^n$ sur $I$ si $f$ est d\'erivable $n$ fois sur $I$ et que $f^{\left(n\right)}$ est continue sur $I$. \\
				On note parfois $\mathcal{C}^n\left(I,\R\right)$ l'ensemble de ces applications.
			\item On dit que $f$ est de classe $\mathcal{C}^\infty$ sur $I$ si $f$ admet des d\'eriv\'ees \`a tout ordre sur $I$.
		\end{liste}
	\end{defi}
	\begin{prop}
		\textbf{Formule de Leibniz} \\
		Si $f$ et $g$ admettent des d\'eriv\'ees jusqu'\`a l'ordre $n$ sur $I$, alors $f\times g$ aussi et :
		$$
			\forall x\in I,\left(f\times g\right)^{\left(n\right)}\left(x\right)=\left(\sum\limits_{k=0}^n\dbinom{n}{k}f^{\left(k\right)}\times g^{\left(n-k\right)}\right)\left(x\right)
		$$
	\end{prop}
\end{document}
