% !TeX encoding = UTF-8
\documentclass[12pt,twoside,a4paper]{article}


\def\chapitre{D\'erivabilit\'e}
\author{MPSI 2}
\def\titre{D\'erivabilit\'e d'un point}

\usepackage{amsfonts}
\usepackage{amsmath}
\usepackage{amsthm}
\usepackage{changepage}
\usepackage{color}
\usepackage{enumitem}
\usepackage{fancyhdr}
\usepackage{framed}
\usepackage[margin=1in]{geometry}
\usepackage{mathrsfs}
\usepackage{tikz, tkz-tab}
\usepackage{titling}

\newtheoremstyle{dotless}{}{}{\itshape}{}{\bfseries}{}{ }{}
\theoremstyle{dotless}

\newtheorem{defs}{Definition}[subsection]
\newenvironment{defi}{\definecolor{shadecolor}{RGB}{255,236,217}\begin{shaded}\begin{defs}\ \\}{\end{defs}\end{shaded}}

\newtheorem{pro}{Propriete}[subsection]
\newenvironment{prop}{\definecolor{shadecolor}{RGB}{230,230,255}\begin{shaded}\begin{pro}\ \\}{\end{pro}\end{shaded}}

\newtheorem{cor}{Corollaire}[subsection]
\newenvironment{coro}{\definecolor{shadecolor}{RGB}{245,250,255}\begin{shaded}\begin{cor}\ \\}{\end{cor}\end{shaded}}

\setlength{\droptitle}{-1in}
\predate{}
\postdate{}
\date{}
\title{\chapitre\\\titre\vspace{-.25in}}

\pagestyle{fancy}
\makeatletter
\lhead{\chapitre\ - \titre}
\rhead{\@author}
\makeatother

\newenvironment{preuve}{\begin{framed}\begin{proof}[\unskip\nopunct]}{\end{proof}\end{framed}}
\newenvironment{liste}{\begin{itemize}[leftmargin=*,noitemsep, topsep=0pt]}{\end{itemize}}
\newenvironment{tab}{\begin{adjustwidth}{.5cm}{}}{\end{adjustwidth}}

\newcommand{\uu}[1] {_{_{#1}}}
\newcommand{\lbracket}{[\![}
\newcommand{\rbracket}{]\!]}
\newcommand{\fonction}[5]{\begin{aligned}[t]#1\colon&#2&&\longrightarrow#3 \\&#4&&\longmapsto#5\end{aligned}}
\newcommand{\systeme}[1]{\left\{\begin{aligned}#1\end{aligned}\right.}
\newcommand{\cercle}[1]{\textcircled{\scriptsize{#1}}}

\newcommand{\lf}[1]{\left(#1\right)}
\newcommand{\C}{\mathbb{C}}
\newcommand{\R}{\mathbb{R}}
\newcommand{\K}{\mathbb{K}}
\newcommand{\N}{\mathbb{N}}
\newcommand{\I}{\mathcal{I}}
\newcommand{\F}{\mathcal{F}}
\newcommand{\E}{\mathcal{E}}
\newcommand{\G}{\mathcal{G}}
\newcommand{\et}{\text{ et }}
\newcommand{\ou}{\text{ ou }}
\newcommand{\xou}{\ \fbox{\text{ou}}\ }


%Auteur: Cl\'ement Phan, MPSI 2

\begin{document}
	\maketitle
	\begin{flushleft}
		Soit $I$ un intervalle r\'eel non r\'eduit \`a un point.\\
		Soit $f:I\rightarrow\R$ une fonction num\'erique.
	\end{flushleft}
	\section{D\'efinition}
		\begin{flushleft}
			Soit $x_0$ un \'el\'ement de $I$.\\
			On pose $\fonction{\phi}{I\setminus\{x_0\}}{\R}{x}{\frac{f(x)-f(x_0)}{x-x_0}}$
		\end{flushleft}
		\begin{defi}
			\begin{liste}
				\item On dit que $f$ est d\'erivable en $x_0$ si $\phi(x)$ admet une limite finie not\'ee $L$ lorsque $x$ tend vers $x_0$ sur $i\setminus\{x_0\}$:\\
					$$\text{On note }f'(x)=L=\lim\limits_{\substack{x\rightarrow x_0 \\x\in I\setminus\{x_0\}}}\!\!\left(\frac{f(x)-f(x_0)}{x-x_0} \right) $$\\
					Cette limite, quand elle existe, s'appelle le nombre d\'eriv\'e de $f$ en $x_0$.\\
				\item On dit que $f$ est d\'erivable \`a gauche en $x_0$ si $\phi(x)$ admet une limite finie \`a gauche, $L_g$:\\
					$$\text{On note }f'_g(x)=L_g=\lim\limits_{\substack{x\rightarrow x_0 \\x<x_0}}\!\!\left(\frac{f(x)-f(x_0)}{x-x_0} \right) $$\\
				\item On dit que $f$ est d\'erivable \`a droite en $x_0$ si $\phi(x)$ admet une limite finie \`a droite, $L_d$:\\
					$$\text{On note }f'_d(x)=L_d=\lim\limits_{\substack{x\rightarrow x_0 \\x>x_0}}\!\!\left(\frac{f(x)-f(x_0)}{x-x_0} \right) $$
			\end{liste}
		\end{defi}
		\begin{flushleft}
			\underline{Remarque:} $f$ est d\'erivable en $x_0$ ssi: $\systeme{& f\text{ est d\'erivable \`a droite en }x_0\\& f\text{ est d\'erivable \`a gauche en }x_0\\& f'_d(x_0)=f'_g(x_0)}$
		\end{flushleft}
	\section{Interpr\'etation g\'eom\'etrique}
		\begin{flushleft}
			Soit $x_0$ un \'el\'ement de $I$ qui ne soit pas une borne de $I$.\\
			Pour $x\neq x_0,\ \phi(x)=\frac{f(x)-f(x_0)}{x-x_0}$
		\end{flushleft}
		\begin{defi}
			$f$ est d\'erivable en $x_0$ si il existe un r\'eel $L$, un r\'eel $\alpha$ strictement positif et une application $\varepsilon:]x-\alpha,x+\alpha[\rightarrow\R$ tels que:
			$$\systeme{&\forall x\in I,\ x\in]x-\alpha,x+\alpha[\et x\neq x_0,\ f(x)=f(x_0)+L\,f(x-x_0)+\varepsilon (x)\,(x-x_0)\\
				&\varepsilon(x) \mathop{\longrightarrow}\limits_{x\rightarrow x_0}0}$$
		\end{defi}
		\begin{flushleft}
			On dit que $f$ admet un d\'eveloppement limit\'e \`a l'ordre $1$ au voisinage de $x_0$.
		\end{flushleft}
	\section{Fonctions \`a valeurs complexes}
		\begin{flushleft}
			Soit $\fonction{f}{I}{\C}{x}{f_1(x)+i\,f_2(x)}$
			\begin{liste}
				\item $f_1$ et $f_2$ sont a valeurs r\'eelles et d\'efinies sur $I$.
				\item Soit $x_0$ un \'el\'ement de $I$.\\
					On dit que $f$ est d\'erivable en $x_0$ si $f_1\et f_2$ le sont, et $f'(x_0)=f_1'(x)+i\,f_2'(x)$
			\end{liste}
		\end{flushleft}
\end{document}