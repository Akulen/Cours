% !TeX encoding = UTF-8
\documentclass[12pt,twoside,a4paper]{article}


\def\chapitre{Dérivabilité}
\author{MPSI 2}
\def\titre{Dérivabilité d'un point}

\usepackage{amsfonts}
\usepackage{amsmath}
\usepackage{amsthm}
\usepackage{changepage}
\usepackage{color}
\usepackage{enumitem}
\usepackage{fancyhdr}
\usepackage{framed}
\usepackage[margin=1in]{geometry}
\usepackage{mathrsfs}
\usepackage{tikz, tkz-tab}
\usepackage{titling}

\newtheoremstyle{dotless}{}{}{\itshape}{}{\bfseries}{}{ }{}
\theoremstyle{dotless}

\newtheorem{defs}{Definition}[subsection]
\newenvironment{defi}{\definecolor{shadecolor}{RGB}{255,236,217}\begin{shaded}\begin{defs}\ \\}{\end{defs}\end{shaded}}

\newtheorem{pro}{Propriete}[subsection]
\newenvironment{prop}{\definecolor{shadecolor}{RGB}{230,230,255}\begin{shaded}\begin{pro}\ \\}{\end{pro}\end{shaded}}

\newtheorem{cor}{Corollaire}[subsection]
\newenvironment{coro}{\definecolor{shadecolor}{RGB}{245,250,255}\begin{shaded}\begin{cor}\ \\}{\end{cor}\end{shaded}}

\setlength{\droptitle}{-1in}
\predate{}
\postdate{}
\date{}
\title{\chapitre\\\titre\vspace{-.25in}}

\pagestyle{fancy}
\makeatletter
\lhead{\chapitre\ - \titre}
\rhead{\@author}
\makeatother

\newenvironment{preuve}{\begin{framed}\begin{proof}[\unskip\nopunct]}{\end{proof}\end{framed}}
\newenvironment{liste}{\begin{itemize}[leftmargin=*,noitemsep, topsep=0pt]}{\end{itemize}}
\newenvironment{tab}{\begin{adjustwidth}{.5cm}{}}{\end{adjustwidth}}

\newcommand{\uu}[1] {_{_{#1}}}
\newcommand{\lbracket}{[\![}
\newcommand{\rbracket}{]\!]}
\newcommand{\fonction}[5]{\begin{aligned}[t]#1\colon&#2&&\longrightarrow#3 \\&#4&&\longmapsto#5\end{aligned}}
\newcommand{\systeme}[1]{\left\{\begin{aligned}#1\end{aligned}\right.}
\newcommand{\cercle}[1]{\textcircled{\scriptsize{#1}}}

\newcommand{\lf}[1]{\left(#1\right)}
\newcommand{\C}{\mathbb{C}}
\newcommand{\R}{\mathbb{R}}
\newcommand{\K}{\mathbb{K}}
\newcommand{\N}{\mathbb{N}}
\newcommand{\I}{\mathcal{I}}
\newcommand{\F}{\mathcal{F}}
\newcommand{\E}{\mathcal{E}}
\newcommand{\G}{\mathcal{G}}
\newcommand{\et}{\text{ et }}
\newcommand{\ou}{\text{ ou }}
\newcommand{\xou}{\ \fbox{\text{ou}}\ }


%Auteur: Cl\'ement Phan, MPSI 2

\begin{document}
	\maketitle
	\begin{flushleft}
		Soit $I$ un intervalle réel non réduit \`a un point.\\
		Soit $f:I\rightarrow\R$ une fonction numérique.
	\end{flushleft}
	\section{Définition}
	\begin{flushleft}
		Soit $x_0$ un élément de $I$.\\
		On pose $\fonction{\phi}{I\setminus\{x_0\}}{\R}{x}{\frac{f(x)-f(x_0)}{x-x_0}}$
	\end{flushleft}
	\begin{defi}
		\begin{liste}
			\item On dit que $f$ est dérivable en $x_0$ si $\phi(x)$ admet une limite finie notée $L$ lorsque $x$ tend vers $x_0$ sur $i\setminus\{x_0\}$:\\
				$$\text{On note }f'(x)=L=\lim\limits_{\substack{x\rightarrow x_0 \\x\in I\setminus\{x_0\}}}\!\!\left(\frac{f(x)-f(x_0)}{x-x_0} \right) $$\\
				Cette limite, quand elle existe, s'appelle le nombre dérivé de $f$ en $x_0$.
			\item On dit que $f$ est dérivable \`a gauche en $x_0$ si $\phi(x)$ admet une limite finie \`a gauche, $L_g$:\\
				$$\text{On note }f'_g(x)=L_g=\lim\limits_{\substack{x\rightarrow x_0 \\x<x_0}}\!\!\left(\frac{f(x)-f(x_0)}{x-x_0} \right) $$\\
			\item On dit que $f$ est dérivable \`a droite en $x_0$ si $\phi(x)$ admet une limite finie \`a droite, $L_d$:\\
				$$\text{On note }f'_d(x)=L_d=\lim\limits_{\substack{x\rightarrow x_0 \\x>x_0}}\!\!\left(\frac{f(x)-f(x_0)}{x-x_0} \right) $$\\
		\end{liste}
	\end{defi}
\end{document}