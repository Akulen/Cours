% !TeX encoding = UTF-8
\documentclass[12pt,twoside,a4paper]{article}


\def\chapitre{D\'erivabilit\'e}
\author{MPSI 2}
\def\titre{Op\'erations sur les nombres d\'eriv\'es}

\usepackage{amsfonts}
\usepackage{amsmath}
\usepackage{amsthm}
\usepackage{changepage}
\usepackage{color}
\usepackage{enumitem}
\usepackage{fancyhdr}
\usepackage{framed}
\usepackage[margin=1in]{geometry}
\usepackage{mathrsfs}
\usepackage{tikz, tkz-tab}
\usepackage{titling}

\newtheoremstyle{dotless}{}{}{\itshape}{}{\bfseries}{}{ }{}
\theoremstyle{dotless}

\newtheorem{defs}{Definition}[subsection]
\newenvironment{defi}{\definecolor{shadecolor}{RGB}{255,236,217}\begin{shaded}\begin{defs}\ \\}{\end{defs}\end{shaded}}

\newtheorem{pro}{Propriete}[subsection]
\newenvironment{prop}{\definecolor{shadecolor}{RGB}{230,230,255}\begin{shaded}\begin{pro}\ \\}{\end{pro}\end{shaded}}

\newtheorem{cor}{Corollaire}[subsection]
\newenvironment{coro}{\definecolor{shadecolor}{RGB}{245,250,255}\begin{shaded}\begin{cor}\ \\}{\end{cor}\end{shaded}}

\setlength{\droptitle}{-1in}
\predate{}
\postdate{}
\date{}
\title{\chapitre\\\titre\vspace{-.25in}}

\pagestyle{fancy}
\makeatletter
\lhead{\chapitre\ - \titre}
\rhead{\@author}
\makeatother

\newenvironment{preuve}{\begin{framed}\begin{proof}[\unskip\nopunct]}{\end{proof}\end{framed}}
\newenvironment{liste}{\begin{itemize}[leftmargin=*,noitemsep, topsep=0pt]}{\end{itemize}}
\newenvironment{tab}{\begin{adjustwidth}{.5cm}{}}{\end{adjustwidth}}

\newcommand{\uu}[1] {_{_{#1}}}
\newcommand{\lbracket}{[\![}
\newcommand{\rbracket}{]\!]}
\newcommand{\fonction}[5]{\begin{aligned}[t]#1\colon&#2&&\longrightarrow#3 \\&#4&&\longmapsto#5\end{aligned}}
\newcommand{\systeme}[1]{\left\{\begin{aligned}#1\end{aligned}\right.}
\newcommand{\cercle}[1]{\textcircled{\scriptsize{#1}}}

\newcommand{\lf}[1]{\left(#1\right)}
\newcommand{\C}{\mathbb{C}}
\newcommand{\R}{\mathbb{R}}
\newcommand{\K}{\mathbb{K}}
\newcommand{\N}{\mathbb{N}}
\newcommand{\I}{\mathcal{I}}
\newcommand{\F}{\mathcal{F}}
\newcommand{\E}{\mathcal{E}}
\newcommand{\G}{\mathcal{G}}
\newcommand{\et}{\text{ et }}
\newcommand{\ou}{\text{ ou }}
\newcommand{\xou}{\ \fbox{\text{ou}}\ }


%Auteur: Tomas Rigaux, MPSI 2

\begin{document}
	\maketitle
	\begin{flushleft}
		$I\subset\R$ est un intervalle non vide et non r\'eduit \`a un point.
	\end{flushleft}
	\begin{prop}
		Soit $x_0\in I$.
		\begin{liste}
			\item L'ensemble $\mathcal{E}$ des applications d\'efinies sur $I$ \`a valeurs r\'eelles et d\'erivables en $x_0$ est un espace vectoriel sur $\R$. ($\mathcal{E}$ est non vide et stable par combinaisons linéaires).
			\item $\fonction{\Phi}{\mathcal{E}}{\R}{f}{f'\left(x_0\right)}$ est une forme lin\'eaire.
		\end{liste}
	\end{prop}
	\begin{prop}
		Soit $x_0$ un \'el\'ement de $I$. \\
		Soit $f$ et $g$ deux fonctions num\'eriques d\'efinies sur $I$ et d\'erivables en $x_0$. Alors :
		\begin{liste}
			\item $f\times g$ est d\'erivable en $x_0$ et $\left(f\times g\right)'\left(x_0\right)=\left(f'\times g+g'\times f\right)\left(x_0\right)$.
			\item Si $g\left(x_0\right)\neq 0$ alors il existe un r\'eel strictement positif $\alpha$ tel que :
				$$
					\forall x\in\left]x_0-\alpha,x_0+\alpha\right[\cap I,g\left(x\right)\neq 0
				$$
				Sur $\left]x_0-\alpha,x_0+\alpha\right[\cap I$, $\frac{f}{g}$ \`a un sens, $\frac{f}{g}$ est d\'erivable en $x_0$ et
				$$
					\left(\frac{f}{g}\right)'\left(x_0\right)=\left(\frac{f'\times g-g'\times f}{g^2}\right)\left(x_0\right)
				$$
		\end{liste}
	\end{prop}
	\begin{prop}
		Soit $g$ d\'efinie sur $I$, d\'erivable en $x_0$. \\
		Soit $f$ d\'efinie sur un intervalle contenant $g\left(I\right)$ et d\'erivable en $g\left(x_0\right)$. \\
		Alors $f\circ g$ \`a un sens sur $I$, est d\'erivable en $x_0$ et :
		$$
			\left(f\circ g\right)'\left(x_0\right)=\left(g'\times f'\circ g\right)\left(x_0\right)
		$$
	\end{prop}
	\newpage
	\begin{theo}{des fonctions r\'eciproques}
		\begin{liste}
			\item Si $f$ est d\'efinie sur $I$, continue sur $I$ et strictement monotone sur $I$, alors $f$ r\'ealise une bijection de $I$ sur l'intervalle $f\left(I\right)$.
			\item Soit $x_0\in I$. Si de plus $f$ est d\'erivable en $x_0$ et $f'\left(x_0\right)\neq 0$ alors l'application r\'eciproque $f^{-1}$ est d\'erivable en $f\left(x_0\right)$ et on a :
			$$
				f^{-1}\left(f\left(x_0\right)\right)=\frac{1}{f'\left(x_0\right)}
			$$
		\end{liste}
	\end{theo}
\end{document}
