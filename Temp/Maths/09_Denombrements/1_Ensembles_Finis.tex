\documentclass[12pt,twoside,a4paper]{article}

\def\chapitre{Dénombrements}
\author{MPSI 2}
\def\titre{Ensembles Finis}

\usepackage{amsfonts}
\usepackage{amsmath}
\usepackage{amsthm}
\usepackage{changepage}
\usepackage{color}
\usepackage{enumitem}
\usepackage{fancyhdr}
\usepackage{framed}
\usepackage[margin=1in]{geometry}
\usepackage{mathrsfs}
\usepackage{tikz, tkz-tab}
\usepackage{titling}

\newtheoremstyle{dotless}{}{}{\itshape}{}{\bfseries}{}{ }{}
\theoremstyle{dotless}

\newtheorem{defs}{Definition}[subsection]
\newenvironment{defi}{\definecolor{shadecolor}{RGB}{255,236,217}\begin{shaded}\begin{defs}\ \\}{\end{defs}\end{shaded}}

\newtheorem{pro}{Propriete}[subsection]
\newenvironment{prop}{\definecolor{shadecolor}{RGB}{230,230,255}\begin{shaded}\begin{pro}\ \\}{\end{pro}\end{shaded}}

\newtheorem{cor}{Corollaire}[subsection]
\newenvironment{coro}{\definecolor{shadecolor}{RGB}{245,250,255}\begin{shaded}\begin{cor}\ \\}{\end{cor}\end{shaded}}

\setlength{\droptitle}{-1in}
\predate{}
\postdate{}
\date{}
\title{\chapitre\\\titre\vspace{-.25in}}

\pagestyle{fancy}
\makeatletter
\lhead{\chapitre\ - \titre}
\rhead{\@author}
\makeatother

\newenvironment{preuve}{\begin{framed}\begin{proof}[\unskip\nopunct]}{\end{proof}\end{framed}}
\newenvironment{liste}{\begin{itemize}[leftmargin=*,noitemsep, topsep=0pt]}{\end{itemize}}
\newenvironment{tab}{\begin{adjustwidth}{.5cm}{}}{\end{adjustwidth}}

\newcommand{\uu}[1] {_{_{#1}}}
\newcommand{\lbracket}{[\![}
\newcommand{\rbracket}{]\!]}
\newcommand{\fonction}[5]{\begin{aligned}[t]#1\colon&#2&&\longrightarrow#3 \\&#4&&\longmapsto#5\end{aligned}}
\newcommand{\systeme}[1]{\left\{\begin{aligned}#1\end{aligned}\right.}
\newcommand{\cercle}[1]{\textcircled{\scriptsize{#1}}}

\newcommand{\lf}[1]{\left(#1\right)}
\newcommand{\C}{\mathbb{C}}
\newcommand{\R}{\mathbb{R}}
\newcommand{\K}{\mathbb{K}}
\newcommand{\N}{\mathbb{N}}
\newcommand{\I}{\mathcal{I}}
\newcommand{\F}{\mathcal{F}}
\newcommand{\E}{\mathcal{E}}
\newcommand{\G}{\mathcal{G}}
\newcommand{\et}{\text{ et }}
\newcommand{\ou}{\text{ ou }}
\newcommand{\xou}{\ \fbox{\text{ou}}\ }


%Auteur: Cl\'ement Phan, MPSI 2

\begin{document}
	\maketitle
	\begin{prop}
		Soit $n$ et $p$ deux entiers naturels non nuls.\\
		Soit $f$ une application de $\lbracket 1,n\rbracket$ dans $\lbracket 1,p\rbracket$.\\
		\begin{liste}
			\item Si $f$ est bijective, alors $n=p$
			\item Si $f$ est injective, alors $n\leqslant p$
			\item Si $f$ est surjective, alors $n\geqslant p$
		\end{liste}
	\end{prop}
	\begin{defi}
		Soit $E$ un ensemble non vide.\\
		On dit que $E$ est \underline{fini} si il existe un entier naturel non nul $n$ et une bijection de $\lbracket 1,n\rbracket$ sur $E$\\
		Si un tel entier existe, il est unique et est le \underline{cardinal de $E$}.
	\end{defi}
	\begin{flushleft}
		\textbf{Notations:} $\card(E)$,\ \#$E$,\ $|E|$\\
		\textbf{Convention:} $\card(\varnothing)=0$
	\end{flushleft}
	\begin{prop}
		Soit $E$ un ensemble fini de cardinal $n$.\\
		Soit $F$ un sous-ensemble de $E$.\\
		Alors $F$ est \'egalement un ensemble fini \underline{et} $\card(F)\leqslant n$ \underline{et} $(\card(F)=n)\iff(E=F)$
	\end{prop}
	\begin{preuve}
		On procède par récurrence sur le cardinal de $E$.\\
		\textbf{Lemme:}
		$\begin{aligned}[t]
			&\text{Si }E\text{ est un ensemble fini de cardinal }n\geqslant 1,\\
			&\text{Et si }a\text{ est un élément de }E,\\
			&\text{Alors }E\setminus \{a\}\text{ est un ensemble fini de cardinal }n-1
		\end{aligned}$\\
		\\
		\textbf{Démonstration de la propriété}\\
		Soit $P(n):
		\begin{aligned}[t]
			&\text{Pour tout ensemble }E\text{ de cardinal }n\text{, pour tout sous-ensemble }F\text{ de }E\text{,}\\
			&\text{}F\text{ est fini \underline{et} }\card(F)\leqslant n\text{ \underline{et} }(\card(F)=n)\iff(E=F)
		\end{aligned}$\\
		\\
		\underline{$n=0$}: $E=\varnothing$ et $F=\varnothing$ et $\card(E)=\card(F)=0$\\
		\\
		Soit $n\in\N^*$ tel que $P(n-1)$ soit vérifié.\\
		Montrons $P(n)$\\
		Soit $E$ un ensemble fini de cardinal $n$.\\
		Soit $F$ une partie de $E$.\\
		\underline{$1^{\text{er}}$ cas:} $F=E$ alors $\card(F)=n$\\
		\underline{$2^{\text{ème}}$ cas:} $F\neq E$\\
		Alors $\exists a\in E,\ a\notin F$\\
		Soit $a$ un tel élément.\\
		$a\notin F$ donc $F\subset E\setminus\{a\}$\\
		Or, d'après la lemme, $E\setminus\{a\}$ est de cardinal $n-1$, donc d'après l'hypothèse de récurrence, $F$ est fini \underline{et} $\card(F)\leqslant n-1$\\
		\\
		Finalement: 
		$\begin{aligned}[t]
			& F=E\Rightarrow \card(E)=\card(F)\\
			& F\neq E\Rightarrow \card(F)<\card(E)
		\end{aligned}$\\
		D'où $(\card(F)=n)\iff(E=F)$\\
		Donc $P(n)$ est vérifié.\\
		D'après le principe de récurrence, $\forall n\in\N,\ P(n)$\\
		\\
		\textbf{Démonstration du lemme}\\
		Soit $E$ un ensemble fini de cardinal $n\geqslant 1$.\\
		Il existe une bijection de $\lbracket 1,n\rbracket$ sur $E$.\\
		\underline{$1^{\text{er}}$ cas:} $n=1$\\
		$\fonction{f}{\{1\}}{E}{1}{f(1)}$\\
		$E=\{f(1)\}$, donc $E\setminus \{f(1)\}=\varnothing$, de cardinal $0$.\\
		\underline{$2^{\text{ème}}$ cas:} $n\geqslant 2$\\
		Soit $a$ un élément de $E$.\\
		$f$ réalise une bijection de $\lbracket 1,n\rbracket$ sur $E$, donc $\exists i\in \lbracket 1,n\rbracket,\ \text{unique},\ f(i)=a$
		\begin{liste}
			\item \underline{Si $i=n$} alors $f(n)=a$\\
				$\restr{f}{\lbracket 1,n-1\rbracket}$ réalise une bijection de $\lbracket 1,n-1\rbracket$ sur $f(\lbracket 1,n-1\rbracket)$\\
				Or $\begin{aligned}
				f(\lbracket 1,n-1\rbracket)&=f(\lbracket 1,n\rbracket\setminus\{n\})\\
					&=E\setminus \{a\}
				\end{aligned}$\\
				Donc $\restr{f}{\lbracket 1,n-1\rbracket}$ réalise une bijection de $\lbracket 1,n-1\rbracket$ sur $E\setminus \{a\}$.\\
				Donc $E$ est de cardinal $n-1$
			\item \underline{Si $i\neq n$}\\
				Notons $i_0$ l'unique élément de $\lbracket 1,n-1\rbracket$ tel que $f(i_0)=a$\\
				On considère 
				$\begin{aligned}[t]
				\tau:\lbracket 1,n-1\rbracket &\longrightarrow \lbracket 1,n-1\rbracket\\
					i &\longmapsto 
					\left\{\begin{aligned}
					 i &\text{ si }(i\neq i_0\et i\neq n)\\
					 i_0 &\text{ si } i=n\\
					 n &\text{ si } i=i_0
					\end{aligned}\right.
				\end{aligned}$\\
				$\tau$ réalise un bijection de $\lbracket 1,n\rbracket$ sur $\lbracket 1,n\rbracket$.\\
				On applique ensuite le premier cas avec $\tau(\lbracket 1,n\rbracket)$ au lieu de $\lbracket 1,n\rbracket$
		\end{liste}
	\end{preuve}
	\begin{prop}
		Soit $P$ une partie finie, non vide et incluse dans $\N$, de cardinal $p$.\\
		Alors il existe une unique bijection strictement croissante de $\lbracket 1,p\rbracket$ sur $P$.
	\end{prop}
	\begin{preuve}
		\begin{flushleft}
			\textbf{Existence:}
			\begin{liste}
				\item $P$ est une partie non vide de $\N$, et admet un plus petit élément que l'on note $y_1$.\\
					On pose $\phi(1)=y_1$
				\item Soit $k\in \lbracket 1,p-1\rbracket$ tel que $(\phi(i))_{i\in\lbracket 1,k\rbracket}$ est défini \underline{et} $\phi(1)<...<\phi(k)$\\
					Soit $P_k=\{ x\in\N,\ x\in P\et x>\phi(k) \}$\\
					$P_k$ est non vide car $k<p$, donc admet un plus petit élément. On le note $\phi(k+1)$
			\end{liste}
			On construit alors $\phi$ par itération, et elle est strictement croissante.
		\end{flushleft}
		\begin{flushleft}
			\textbf{Unicité:}
			Soit $\psi$ une application strictement croissante bijective de $\lbracket 1,p\rbracket$ sut $P$.\\
			Alors $P=\{\psi(1),\,\psi(2),\,...,\psi(p) \}$ et $\psi(1)<...<\psi(p)$\\
			\begin{liste}
				\item$\psi(1)$ est le plus petit élément de $P$, donc $\psi(1)=\phi(1)$
				\item$\psi(2)$ est le plus petit élément de $P\setminus\{\psi(1) \}$. Or $P\setminus\{\psi(1) \}=P_1$, donc par définition, $\phi(2)=\psi(2)$		
				\item ...
			\end{liste}
			Conclusion: $\phi=\psi$
		\end{flushleft}
	\end{preuve}
	\begin{prop}
		Soit $E$ et $F$ deux ensembles finis de m\^eme cardinal $n$.\\
		Soit $f:E\longrightarrow F$ une application.\\
		On a alors équivalence entre:
		\begin{liste}
			\item[\cercle1] $f$ est injective
			\item[\cercle2] $f$ est surjective
			\item[\cercle3] $f$ est bijective
		\end{liste}
	\end{prop}
	\begin{preuve}
		\begin{liste}
			\item[\cercle1]Supposons $f$ injective.\\
				Montrer que $f$ est surjective.\\
				Donc montrer que $f(E)=F$\\
				Soit $\fonction{g}{E}{f(E)}{x}{f(x)}$ une application.\\
				$g$ réalise une bijection de $E$ sur $f(E)$ par définition de l'espace d'arrivée.\\
				Par ailleurs, $E$ est de cardinal $n$, donc il existe une bijection $\phi$ de $\lbracket 1,n\rbracket$ sur $E$.\\
				$g\circ \phi:\lbracket 1,n\rbracket\longrightarrow f(E)$\\
				$g\circ\phi$ est une bijection, donc $\card(f(E))=n$\\
				D'o\`u, sachant $f(E)\subset F$ et $\card(E)=\card(F)$, $E=F$\\
				Donc $f$ est surjective.
			\item[\cercle2]Supposons $f$ surjective.\\
				Montrer que $f$ est bijective de $E$ sur $F$.
				\begin{liste}
					\item[a/] Montrer que $\exists h\in\mathcal{F}()$
				\end{liste}
		\end{liste}
	\end{preuve}
\end{document}