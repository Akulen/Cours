\documentclass[12pt,twoside,a4paper]{article}

\def\chapitre{D\'enombrements}
\author{MPSI 2}
\def\titre{D\'enombrements}

\usepackage{amsfonts}
\usepackage{amsmath}
\usepackage{amsthm}
\usepackage{changepage}
\usepackage{color}
\usepackage{enumitem}
\usepackage{fancyhdr}
\usepackage{framed}
\usepackage[margin=1in]{geometry}
\usepackage{mathrsfs}
\usepackage{tikz, tkz-tab}
\usepackage{titling}

\newtheoremstyle{dotless}{}{}{\itshape}{}{\bfseries}{}{ }{}
\theoremstyle{dotless}

\newtheorem{defs}{Definition}[subsection]
\newenvironment{defi}{\definecolor{shadecolor}{RGB}{255,236,217}\begin{shaded}\begin{defs}\ \\}{\end{defs}\end{shaded}}

\newtheorem{pro}{Propriete}[subsection]
\newenvironment{prop}{\definecolor{shadecolor}{RGB}{230,230,255}\begin{shaded}\begin{pro}\ \\}{\end{pro}\end{shaded}}

\newtheorem{cor}{Corollaire}[subsection]
\newenvironment{coro}{\definecolor{shadecolor}{RGB}{245,250,255}\begin{shaded}\begin{cor}\ \\}{\end{cor}\end{shaded}}

\setlength{\droptitle}{-1in}
\predate{}
\postdate{}
\date{}
\title{\chapitre\\\titre\vspace{-.25in}}

\pagestyle{fancy}
\makeatletter
\lhead{\chapitre\ - \titre}
\rhead{\@author}
\makeatother

\newenvironment{preuve}{\begin{framed}\begin{proof}[\unskip\nopunct]}{\end{proof}\end{framed}}
\newenvironment{liste}{\begin{itemize}[leftmargin=*,noitemsep, topsep=0pt]}{\end{itemize}}
\newenvironment{tab}{\begin{adjustwidth}{.5cm}{}}{\end{adjustwidth}}

\newcommand{\uu}[1] {_{_{#1}}}
\newcommand{\lbracket}{[\![}
\newcommand{\rbracket}{]\!]}
\newcommand{\fonction}[5]{\begin{aligned}[t]#1\colon&#2&&\longrightarrow#3 \\&#4&&\longmapsto#5\end{aligned}}
\newcommand{\systeme}[1]{\left\{\begin{aligned}#1\end{aligned}\right.}
\newcommand{\cercle}[1]{\textcircled{\scriptsize{#1}}}

\newcommand{\lf}[1]{\left(#1\right)}
\newcommand{\C}{\mathbb{C}}
\newcommand{\R}{\mathbb{R}}
\newcommand{\K}{\mathbb{K}}
\newcommand{\N}{\mathbb{N}}
\newcommand{\I}{\mathcal{I}}
\newcommand{\F}{\mathcal{F}}
\newcommand{\E}{\mathcal{E}}
\newcommand{\G}{\mathcal{G}}
\newcommand{\et}{\text{ et }}
\newcommand{\ou}{\text{ ou }}
\newcommand{\xou}{\ \fbox{\text{ou}}\ }


%Auteur: Cl\'ement Phan, MPSI 2

\begin{document}
	\maketitle
	\begin{prop}
		Soit $E$ et $F$ deux ensembles finis de cardinaux respectifs $n$ et $p$.\\
		Alors $\mathcal{F}(E,F)$ est fini de cardinal $p^{n}$
	\end{prop}
	\begin{coro}
		Soit $E$ un ensemble fini de cardinal $n$.\\
		Alors $\mathcal{P}(E)$ est fini et de cardinal $2^{n}$
	\end{coro}
	\begin{preuve}
		Voir bijection de $\mathcal(P)$ sur l'ensemble des fonctions indicatrices de $\mathcal{P}$
	\end{preuve}
	\begin{prop}
		Soit $E$ et $F$ deux ensembles finis de cardinaux respectifs $n$ et $p$.\\
		Soit $\mathcal{E}$ l'ensemble des applications injectives de $E$ vers $F$.\\
		Alors $\mathcal{E}$ est fini et $\card(\mathcal{E})=\frac{n!}{(n-p)!}$
	\end{prop}
	\begin{coro}
		Soit $E$ et $F$ deux ensembles finis de m\^eme cardinal $n$.\\
		\begin{liste}
			\item Le nombre de bijections de $E$ dans $F$ est $n!$
			\item Le nombre de bijections de $E$ dans $E$ est $n!$
		\end{liste}
	\end{coro}
\end{document} %ACCENTS