% !TeX encoding = UTF-8
\documentclass[12pt,twoside,a4paper]{article}


\def\chapitre{Polyn\^omes \`a une indéterminée sur un corps $\mathbb{K}$}
\author{MPSI 2}
\def\titre{Généralités}

\usepackage{amsfonts}
\usepackage{amsmath}
\usepackage{amsthm}
\usepackage{changepage}
\usepackage{color}
\usepackage{enumitem}
\usepackage{fancyhdr}
\usepackage{framed}
\usepackage[margin=1in]{geometry}
\usepackage{mathrsfs}
\usepackage{tikz, tkz-tab}
\usepackage{titling}

\newtheoremstyle{dotless}{}{}{\itshape}{}{\bfseries}{}{ }{}
\theoremstyle{dotless}

\newtheorem{defs}{Definition}[subsection]
\newenvironment{defi}{\definecolor{shadecolor}{RGB}{255,236,217}\begin{shaded}\begin{defs}\ \\}{\end{defs}\end{shaded}}

\newtheorem{pro}{Propriete}[subsection]
\newenvironment{prop}{\definecolor{shadecolor}{RGB}{230,230,255}\begin{shaded}\begin{pro}\ \\}{\end{pro}\end{shaded}}

\newtheorem{cor}{Corollaire}[subsection]
\newenvironment{coro}{\definecolor{shadecolor}{RGB}{245,250,255}\begin{shaded}\begin{cor}\ \\}{\end{cor}\end{shaded}}

\setlength{\droptitle}{-1in}
\predate{}
\postdate{}
\date{}
\title{\chapitre\\\titre\vspace{-.25in}}

\pagestyle{fancy}
\makeatletter
\lhead{\chapitre\ - \titre}
\rhead{\@author}
\makeatother

\newenvironment{preuve}{\begin{framed}\begin{proof}[\unskip\nopunct]}{\end{proof}\end{framed}}
\newenvironment{liste}{\begin{itemize}[leftmargin=*,noitemsep, topsep=0pt]}{\end{itemize}}
\newenvironment{tab}{\begin{adjustwidth}{.5cm}{}}{\end{adjustwidth}}

\newcommand{\uu}[1] {_{_{#1}}}
\newcommand{\lbracket}{[\![}
\newcommand{\rbracket}{]\!]}
\newcommand{\fonction}[5]{\begin{aligned}[t]#1\colon&#2&&\longrightarrow#3 \\&#4&&\longmapsto#5\end{aligned}}
\newcommand{\systeme}[1]{\left\{\begin{aligned}#1\end{aligned}\right.}
\newcommand{\cercle}[1]{\textcircled{\scriptsize{#1}}}

\newcommand{\lf}[1]{\left(#1\right)}
\newcommand{\C}{\mathbb{C}}
\newcommand{\R}{\mathbb{R}}
\newcommand{\K}{\mathbb{K}}
\newcommand{\N}{\mathbb{N}}
\newcommand{\I}{\mathcal{I}}
\newcommand{\F}{\mathcal{F}}
\newcommand{\E}{\mathcal{E}}
\newcommand{\G}{\mathcal{G}}
\newcommand{\et}{\text{ et }}
\newcommand{\ou}{\text{ ou }}
\newcommand{\xou}{\ \fbox{\text{ou}}\ }


%Auteur: Cl\'ement Phan, MPSI 2

\begin{document}
	\maketitle
	\section{Définition}
		\begin{defi}
			On appelle \underline{polyn\^ome \`a une indéterminée et \`a coefficients dans $\K$} toute suite $(a_n)_{n\in\N}$ d'éléments de $\K$ dont tous les termes sont nuls \`a partir d'un certain rang.
		\end{defi}
		\begin{flushleft}
			$\K[X]=\{(a_n)_{n\in\N},\ (\forall k\in\N,\ a_k\in\K)\et (\exists p_0\in\N,\ \forall n\in\N,\ n\geqslant p_0\Rightarrow a_n=0_\K) \}$
		\end{flushleft}
		\begin{defi}
			Soit $P\in \K[X], P\neq 0_{K[X]}$, tel que $P=(a_n)_{n\in\N}$
			\begin{liste}
				\item On appelle \underline{degré de $P$}, noté $\deg(P)$, le plus grand entier $n_0$ tel que: $a_{n_0}\neq 0_{K[X]}$
				\item Si $\deg(P)=n_0$, alors on appelle \underline{coefficient dominant de $P$} $a_{n_0}$.
				\item Si $\deg(P)=n_0$, alors $P$ est unitaire si $a_{n_0}=1_\K$.
			\end{liste}
		\end{defi}
		\begin{flushleft}
			\underline{Convention:} $\deg(0_{\K[X]})=-\infty$
		\end{flushleft}
		\begin{defi}
			Soit $P$ un polynôme \`a coefficients dans $\K$ et non nul.\\
			On appelle \underline{valuation de $P$}, noté $\mathrm{val}(P)$, le plus petit indice $n$ tel que $a_n\neq0_\K$.
		\end{defi}
	\section{Structure algébrique}
		\subsection{Addition}
			\begin{prop}
				Si $P=(a_n)_{n\in\N}$ et $Q=(b_n)_{n\in\N}$ sont deux polynômes de $\K[X]$,\\
				Alors $P+Q=(a_n+b_n)_{n\in\N}$ est un polynôme de $\K[X]$
			\end{prop}
			\begin{prop}
				$(\K[X],+)$ est un groupe abélien.
			\end{prop}
			\begin{prop}
				Soient $P$ et $Q$ deux polynômes de $\K[X]$.
				\begin{liste}
					\item $\deg(P+Q)\leqslant\max(\{\deg(P),\ deg(Q) \})$
					\item Si $\deg(P)\neq \deg(Q)$, alors $\deg(P+Q)=\max(\{\deg(P),\ deg(Q) \})$
				\end{liste}
			\end{prop}
		\subsection{Loi externe}
			\begin{defi}
				Soit $\lambda\in\K\et P=(a_n)_{n\in\N}$ un élément de $\K[X]$.\\
				$\lambda\cdot P=(\lambda\times a_n)_{n\in\N}$
			\end{defi}
			\begin{prop}
				$(\K[X],+,\cdot)$ est un espace vectoriel.\\
				C'est un sous-espace vectoriel de $(\K^\N,+,\cdot)$
			\end{prop}
			\begin{prop}
				$\forall \lambda\in\K,\ \lambda\neq 0_\K\Rightarrow \deg(\lambda\cdot P)=\deg(P)$
			\end{prop}
		\subsection{Multiplication}
			\begin{defi}
				Soient $P$ et $Q$ deux éléments de $\K[X]$, avec $P=(a_n)_{n\in\N}$ et $Q=(b_n)_{n\in\N}$.\\
				On appelle \underline{produit de $P$ par $Q$}, noté $P\times Q$, la suite $(c_n)_{n\in\N}$, de $\K[X]$, définie par:
				$$\forall n\in\N,\ c_n=\sum\limits_{\substack{(i,j)\in\N^{2}\\i+j=n}}a_i\times b_j$$
			\end{defi}
			\begin{prop}
				$(\K[X],+,\times)$ est un anneau commutatif.
			\end{prop}
			\begin{prop}
				$\forall (P,Q)\in\K[X]^{2},\ \deg(P\times Q)=\deg(P)+\deg(Q)$
			\end{prop}
			\begin{coro}
				$\K[X]$ est un anneau intègre.
			\end{coro}
			\begin{defi}
				\begin{liste}
					\item $a_n$ s'appelle le \underline{coefficient de $X^{n}$} dans $P$.
					\item $a_n\,X^{n}$ s'appelle le \underline{terme de degré $n$} dans $P$.
					\item $X$ s'appelle l'indéterminée.
				\end{liste}
			\end{defi}
			\begin{prop}
				$(\K[X],+,\times,\cdot)$ est une algèbre commutative sur $\K$:
				\begin{liste}
					\item $(\K[X],+,\times)$ est un anneau commutatif.
					\item $(K[X],+,\cdot)$ est un espace vectoriel sur $K$
					\item $\forall(\lambda,P,Q)\in\K\times \K[X]^{2},\ \lambda\cdot(P\times Q)=(\lambda\cdot P)\times Q=P\times(\lambda\cdot Q)$
				\end{liste}
			\end{prop}
		\subsection{Bases et familles libres}
			\begin{prop}
				La famille $(X^{n})_{n\in\N}$ est une base de $\K[X]$.\\
				Donc tout élément de $\K[X]$ peut s'écrire de manière unique comme combinaison linéaire finie d'éléments de $(X^{n})_{n\in\N}$.
			\end{prop}
			\begin{prop}
				Soit $n\in\N$.\\
				On note $\K_n[X]=\left\lbrace P\in\K[X],deg(P)\leqslant n\right\rbrace$\\
				$\K_n[X]$ est un sous-espace vectoriel de $(\K[X],+,\cdot)$ dont une base est donnée par $(X^{i})_{i\in\lbracket0,n\rbracket}$
			\end{prop}
			\begin{defi}
				\begin{liste}
					\item $(X^{n})_{n\in\N}$ est la \underline{base canonique} de $\K[X]$.
					\item $(X^{i})_{i\in\lbracket0,n\rbracket}$ est la \underline{base canonique} de $\K_n[X]$.
				\end{liste}
			\end{defi}
			
			%13-4-1
			
		\subsection{Autres Opérations}
			\begin{defi}
				Soit $P=\sum\limits_{r=0}^Na_n\,X^{n}$ et $Q\in\K[X]$\\
				On définit le polynôme composé $P\circ Q$ par:
				$$p\circ Q=\sum\limits_{r=0}^Na_n\,Q^{n}$$
			\end{defi}
			\begin{defi}
				Soit $P=\sum\limits_{r=0}^Na_n\,X^{n}$ un polynôme de $\K[X]$.\\
				On appelle \underline{polynôme dérivé de $P$}, noté $P'$, le polynôme:
				$$P'=\sum\limits_{r=1}^Na_n\,n\,nX^{n-1} $$
			\end{defi}
\end{document}
