% !TeX encoding = UTF-8
\documentclass[12pt,twoside,a4paper]{article}


\def\chapitre{Espaces Vectoriels de Dimension finie}
\author{MPSI 2}
\def\titre{Existance de bases}

\usepackage{amsfonts}
\usepackage{amsmath}
\usepackage{amsthm}
\usepackage{changepage}
\usepackage{color}
\usepackage{enumitem}
\usepackage{fancyhdr}
\usepackage{framed}
\usepackage[margin=1in]{geometry}
\usepackage{mathrsfs}
\usepackage{tikz, tkz-tab}
\usepackage{titling}

\newtheoremstyle{dotless}{}{}{\itshape}{}{\bfseries}{}{ }{}
\theoremstyle{dotless}

\newtheorem{defs}{Definition}[subsection]
\newenvironment{defi}{\definecolor{shadecolor}{RGB}{255,236,217}\begin{shaded}\begin{defs}\ \\}{\end{defs}\end{shaded}}

\newtheorem{pro}{Propriete}[subsection]
\newenvironment{prop}{\definecolor{shadecolor}{RGB}{230,230,255}\begin{shaded}\begin{pro}\ \\}{\end{pro}\end{shaded}}

\newtheorem{cor}{Corollaire}[subsection]
\newenvironment{coro}{\definecolor{shadecolor}{RGB}{245,250,255}\begin{shaded}\begin{cor}\ \\}{\end{cor}\end{shaded}}

\setlength{\droptitle}{-1in}
\predate{}
\postdate{}
\date{}
\title{\chapitre\\\titre\vspace{-.25in}}

\pagestyle{fancy}
\makeatletter
\lhead{\chapitre\ - \titre}
\rhead{\@author}
\makeatother

\newenvironment{preuve}{\begin{framed}\begin{proof}[\unskip\nopunct]}{\end{proof}\end{framed}}
\newenvironment{liste}{\begin{itemize}[leftmargin=*,noitemsep, topsep=0pt]}{\end{itemize}}
\newenvironment{tab}{\begin{adjustwidth}{.5cm}{}}{\end{adjustwidth}}

\newcommand{\uu}[1] {_{_{#1}}}
\newcommand{\lbracket}{[\![}
\newcommand{\rbracket}{]\!]}
\newcommand{\fonction}[5]{\begin{aligned}[t]#1\colon&#2&&\longrightarrow#3 \\&#4&&\longmapsto#5\end{aligned}}
\newcommand{\systeme}[1]{\left\{\begin{aligned}#1\end{aligned}\right.}
\newcommand{\cercle}[1]{\textcircled{\scriptsize{#1}}}

\newcommand{\lf}[1]{\left(#1\right)}
\newcommand{\C}{\mathbb{C}}
\newcommand{\R}{\mathbb{R}}
\newcommand{\K}{\mathbb{K}}
\newcommand{\N}{\mathbb{N}}
\newcommand{\I}{\mathcal{I}}
\newcommand{\F}{\mathcal{F}}
\newcommand{\E}{\mathcal{E}}
\newcommand{\G}{\mathcal{G}}
\newcommand{\et}{\text{ et }}
\newcommand{\ou}{\text{ ou }}
\newcommand{\xou}{\ \fbox{\text{ou}}\ }


%Auteur: Cl\'ement Phan, MPSI 2

\begin{document}
	\maketitle
	\begin{defi}
		Soit $E$ un espace vectoriel.\\
		On dit que $E$ est \underline{de dimension finie} si $E$ admet un syst\`eme g\'en\'erateur fini.
	\end{defi}
	\begin{prop}
		\textbf{Caract\'erisation de bases}\\
		Soit $\mathcal{S}$ un syst\`eme de vecteurs de $E$.\\
		On a \'equivalence entre:
		\begin{liste}
			\item $\mathcal{S}$ est une base de $E$.
			\item $\mathcal{S}$ est un syst\`eme g\'en\'erateur minimal.
			\item $\mathcal{S}$ est un syst\`eme libre maximal.
		\end{liste}
	\end{prop}
	\begin{preuve}
		Utilisation des d\'efinitions et des axiomes.
	\end{preuve}
	\begin{prop}
		Soient $A$ et $B$ deux syst\`emes finis de $E$ tels que $A$ soit libre et $B$ g\'en\'erateur.\\ 
		Alors il existe un syst\`eme $\mathcal{S}$ de $E$ tel que:
		\begin{liste}
			\item $A\subset\mathcal{S}\subset B$
			\item $\mathcal{S}$ est une base de $E$.
		\end{liste}
	\end{prop}
	\begin{preuve}
		On pose $\mathcal{S}$ le plus grand \'el\'ement de $\{A'\subset E,\ A' \mathrm{ libre}\et A\subset A'\subset B \}$ au sens du cardinal.\\
		On montre ensuite que ce syst\`eme est g\'en\'erateur en montrant que $B$ est CL de $\mathcal{S}$.
	\end{preuve}
	\begin{theo}{d'existance de base}
		Soit $E$ un $\Kev$ de dimension finie.\\
		Alors il existe un syst\`eme fini d'\'el\'ements de $E$ qui soit une base de $E$.
	\end{theo}
	\begin{preuve}
		On applique la propri\'et\'e pr\'ec\'edente, avec $1_E$ et le syst\`eme g\'en\'erateur fini.\\
		Par convention, $\{\varnothing\}$ est la base de $\{0_E\}$
	\end{preuve}
	\begin{theo}{de l'\'echange}
		Soit $A=\{a_1,\,...\,,\,a_r\}$ une partie libre de E.\\
		Soit $A'=\{a'_1,\,...\,,\,a'_p\}$ une partie g\'en\'eratrice de E.\\
		Alors on peut remplacer $r$ \'el\'ements de $A'$ par des \'el\'ements de $A$ pour obtenir un syst\`eme g\'en\'erateur de $E$.
	\end{theo}
	\begin{preuve}
		On exprime les \'el\'ements de $A$ comme CL d'\'el\'ements de $A'$, et on remplace un \'el\'ement exprimant par l'exprim\'e.
	\end{preuve}
	\textbf{Cons\'equences en dimension finie}
	\begin{coro}
		Soit $E$ un $\Kev$ de dimension finie.\\
		Tout syst\`eme libre de $E$ a au plus autant d'\'el\'ements qu'un syst\`eme g\'en\'erateur de $E$.
	\end{coro}
	\begin{preuve}
		Application du th\'eor\`eme de l'\'echange.
	\end{preuve}
	\begin{coro}
		Si $E$ est un $\Kev$ de dimension finie, alors deux bases de $E$ ont le m\^eme nombre d'\'el\'ements.
	\end{coro}
	\begin{preuve}
		Application du th\'eor\`eme de l'\'echange.
	\end{preuve}
	\begin{defi}
		Soit $E$ un $\Kev$ de dimension finie.\\
		On appelle \underline{dimension de $E$} le cardinal d'une base de $E$.
	\end{defi}
	\begin{coro}
		Soit $E$ un $\Kev$ de dimension finie.\\
		Soit $B$ une base de $E$ de cardinal $n$.\\
		Soit $A$ une partie de $E$.\\
		Alors on a:
		\begin{liste}
			\item $\dim(E)=n$
			\item Si $A$ est libre, alors $\mathrm{card}(A)\leqslant n$
			\item Si $A$ est g\'en\'erateur de $E$, alors $\mathrm{card}(A)\geqslant n$
		\end{liste}
	\end{coro}
	\begin{coro}
		Soit $E$ un $\Kev$ de dimension $n$.\\
		Soit $A$ une partie de $E$ de cardinal $n$.\\
		Alors on a \'equivalence entre :
		\begin{liste}
			\item $A$ est une base de $E$.
			\item $A$ est libre.
			\item $A$ est g\'en\'erateur de $E$.
		\end{liste}
	\end{coro}
	\begin{preuve}
		Syst\`emes libres maximaux et g\'en\'erateurs minimaux.
	\end{preuve}
	\begin{theo}{de la base incompl\`ete}
		Soit $E$ un $\Kev$ de dimension $n$.\\
		Soit $A=\{a_1,\,...\,,\,a_r \}$ un syst\`eme libre de $E$.\\
		Soit $A'=\{a'_1,\,...\,,\,a'_p \}$ un syst\`eme g\'en\'erateur de $E$.\\
		Alors on peut compl\'eter $A$ avec $n-r$ \'el\'ements de $A'$ pour obtenir une base de $E$.
	\end{theo}
	\begin{preuve}
		$\exists \mathcal{S}\in \mathcal{P}(E),\ A\subset \mathcal{S}\subset A\cup A',\ \mathcal{S}\text{ g\'en\'erateur}$\\
		En particulier, $\mathrm{card}(\mathcal S)=n$
	\end{preuve}
\end{document} %ACCENTS!!!