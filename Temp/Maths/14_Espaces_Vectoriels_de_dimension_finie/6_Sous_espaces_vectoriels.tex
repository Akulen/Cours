% !TeX encoding = UTF-8
\documentclass[12pt,twoside,a4paper]{article}


\def\chapitre{Espaces Vectoriels de Dimension finie}
\author{MPSI 2}
\def\titre{Sous-Espaces Vectoriels}

\usepackage{amsfonts}
\usepackage{amsmath}
\usepackage{amsthm}
\usepackage{changepage}
\usepackage{color}
\usepackage{enumitem}
\usepackage{fancyhdr}
\usepackage{framed}
\usepackage[margin=1in]{geometry}
\usepackage{mathrsfs}
\usepackage{tikz, tkz-tab}
\usepackage{titling}

\newtheoremstyle{dotless}{}{}{\itshape}{}{\bfseries}{}{ }{}
\theoremstyle{dotless}

\newtheorem{defs}{Definition}[subsection]
\newenvironment{defi}{\definecolor{shadecolor}{RGB}{255,236,217}\begin{shaded}\begin{defs}\ \\}{\end{defs}\end{shaded}}

\newtheorem{pro}{Propriete}[subsection]
\newenvironment{prop}{\definecolor{shadecolor}{RGB}{230,230,255}\begin{shaded}\begin{pro}\ \\}{\end{pro}\end{shaded}}

\newtheorem{cor}{Corollaire}[subsection]
\newenvironment{coro}{\definecolor{shadecolor}{RGB}{245,250,255}\begin{shaded}\begin{cor}\ \\}{\end{cor}\end{shaded}}

\setlength{\droptitle}{-1in}
\predate{}
\postdate{}
\date{}
\title{\chapitre\\\titre\vspace{-.25in}}

\pagestyle{fancy}
\makeatletter
\lhead{\chapitre\ - \titre}
\rhead{\@author}
\makeatother

\newenvironment{preuve}{\begin{framed}\begin{proof}[\unskip\nopunct]}{\end{proof}\end{framed}}
\newenvironment{liste}{\begin{itemize}[leftmargin=*,noitemsep, topsep=0pt]}{\end{itemize}}
\newenvironment{tab}{\begin{adjustwidth}{.5cm}{}}{\end{adjustwidth}}

\newcommand{\uu}[1] {_{_{#1}}}
\newcommand{\lbracket}{[\![}
\newcommand{\rbracket}{]\!]}
\newcommand{\fonction}[5]{\begin{aligned}[t]#1\colon&#2&&\longrightarrow#3 \\&#4&&\longmapsto#5\end{aligned}}
\newcommand{\systeme}[1]{\left\{\begin{aligned}#1\end{aligned}\right.}
\newcommand{\cercle}[1]{\textcircled{\scriptsize{#1}}}

%Auteur: Cl\'ement Phan, MPSI 2

\begin{document}
	\maketitle
	\section{Dimension de sous-espaces vectoriels}
		Soit $E$ un $\Kev$ de dimension finie.
		\begin{prop}
			Soit $F$ un $\Sev$ de $E$.\\
			Alors:
			\begin{liste}
				\item $F$ est de dimension finie.
				\item $\dim(F)\leqslant\dim(E)$
				\item $F=E\iff \dim(F)=dim(E)$
			\end{liste}
		\end{prop}
		\begin{preuve}
			On raisonne sur une base de $F$.
		\end{preuve}
	\section{Somme de sous-espaces vectoriels}
		Soit $E$ un $\Kev$ de dimension finie.\\
		Soient $F$ et $G$ deux $\Sev$ de $E$.
		\begin{defi}
			On appelle \underline{somme de $F$ et $G$} le sous-espace vectoriel engendré par $F\cup G$
		\end{defi}
		\begin{flushleft}
			\textbf{Notation:} $F+G=\vect(F\cup G)$
		\end{flushleft}
		\begin{prop}
			$$F+G=\{x\in E,\ \exists(x_F,x_G)\in F\times G,\ x=x_F+x_G \}$$
		\end{prop}
		\newpage
		\begin{prop}
			$$\dim(F+G)=\dim(F)+\dim(G)-\dim(F\cap G)$$
		\end{prop}
		\begin{preuve}
			On raisonne avec les bases de $F$, $G$, et $F\cap G$, et avec le théorème de la base incomplète sur $F\cap G$.
		\end{preuve}
	\section{Somme directe, espaces supplémentaires}
		\begin{flushleft}
			Soit $E$ un $\Kev$ de dimension $n$.
		\end{flushleft}
		\begin{defi}
			Soient $F$ et $G$ deux $\Sev$ de $E$.
			\begin{liste}
				\item La somme $F+G$ est \underline{directe} si $F\cap G=\{0_E\}$
				\item $F$ et $G$ sont \underline{supplémentaires de $E$} si $F\oplus G=E$
			\end{liste}
		\end{defi}
		\begin{flushleft}
			\textbf{Notation:} Somme directe de $F$ et $G$: $F\oplus G$
		\end{flushleft}
		\begin{prop}
			$\fonction{\varphi_1}{F\times G}{F+G}{(x_F,x_G)}{x_F+x_G}$\\
			$\varphi_1$ est linéaire et surjective.\\
			\begin{liste}
				\item $F$ et $G$ sont en somme directe ssi $\varphi_1$ est injective.
				\item $F$ et $G$ sont en somme directe ssi tout élément de $F+G$ s'écrit comme manière unique comme CL d'éléments de $F$ et de $G$.
			\end{liste}
		\end{prop}
		\begin{defi}
			$$\sum\limits_{i=1}^pE_i=\vect\left( \bigcup\limits_{i=1}^pE_i\right) $$
		\end{defi}
		\begin{defi}
			Soit $\fonction{\varphi}{E_1\times...\times E_p}{E_1+...+E_p}{(x_1,\,...\,,\,x_p)}{x_1+...+x_p}$\\
			La somme $\sum\limits_{i=1}^pE_i$ est directe ssi $\varphi$ est injective, c'est \`a dire si tout élément de $E_1+...+E_p$ s'écrit comme CL unique d'éléments de $\{E_1\times...\times E_p\}$.
		\end{defi}
		\begin{flushleft}
			\textbf{Notation:} $\bigoplus\limits_{i=1}^pE_i$
		\end{flushleft}
		\begin{prop}
			$F+G$ est une somme directe ssi la réunion d'une base de $F$ et d'une base de $G$ est une base de $F+G$.
		\end{prop}
		\begin{coro}
			$$\dim(F\oplus G)=\dim(F)+\dim(G)$$
		\end{coro}
		\begin{coro}
			\begin{liste}
				\item Si $F$ et $G$ sont supplémentaires de $E$, alors $\dim(F)+\dim(G)=\dim(E)$
				\item Tous les $\Sev$ supplémentaires de $F$ dans $E$ sont de dimension $\dim(E)-\dim(F)$
			\end{liste}
			\end{coro}
\end{document} %ACCENTS!!!
