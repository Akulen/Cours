% !TeX encoding = UTF-8
\documentclass[12pt,twoside,a4paper]{article}


\def\chapitre{Espaces Vectoriels\\ de Dimension finie}
\author{MPSI 2}
\def\titre{Espaces Vectoriels}

\usepackage{amsfonts}
\usepackage{amsmath}
\usepackage{amsthm}
\usepackage{changepage}
\usepackage{color}
\usepackage{enumitem}
\usepackage{fancyhdr}
\usepackage{framed}
\usepackage[margin=1in]{geometry}
\usepackage{mathrsfs}
\usepackage{tikz, tkz-tab}
\usepackage{titling}

\newtheoremstyle{dotless}{}{}{\itshape}{}{\bfseries}{}{ }{}
\theoremstyle{dotless}

\newtheorem{defs}{Definition}[subsection]
\newenvironment{defi}{\definecolor{shadecolor}{RGB}{255,236,217}\begin{shaded}\begin{defs}\ \\}{\end{defs}\end{shaded}}

\newtheorem{pro}{Propriete}[subsection]
\newenvironment{prop}{\definecolor{shadecolor}{RGB}{230,230,255}\begin{shaded}\begin{pro}\ \\}{\end{pro}\end{shaded}}

\newtheorem{cor}{Corollaire}[subsection]
\newenvironment{coro}{\definecolor{shadecolor}{RGB}{245,250,255}\begin{shaded}\begin{cor}\ \\}{\end{cor}\end{shaded}}

\setlength{\droptitle}{-1in}
\predate{}
\postdate{}
\date{}
\title{\chapitre\\\titre\vspace{-.25in}}

\pagestyle{fancy}
\makeatletter
\lhead{\chapitre\ - \titre}
\rhead{\@author}
\makeatother

\newenvironment{preuve}{\begin{framed}\begin{proof}[\unskip\nopunct]}{\end{proof}\end{framed}}
\newenvironment{liste}{\begin{itemize}[leftmargin=*,noitemsep, topsep=0pt]}{\end{itemize}}
\newenvironment{tab}{\begin{adjustwidth}{.5cm}{}}{\end{adjustwidth}}

\newcommand{\uu}[1] {_{_{#1}}}
\newcommand{\lbracket}{[\![}
\newcommand{\rbracket}{]\!]}
\newcommand{\fonction}[5]{\begin{aligned}[t]#1\colon&#2&&\longrightarrow#3 \\&#4&&\longmapsto#5\end{aligned}}
\newcommand{\systeme}[1]{\left\{\begin{aligned}#1\end{aligned}\right.}
\newcommand{\cercle}[1]{\textcircled{\scriptsize{#1}}}

\newcommand{\lf}[1]{\left(#1\right)}
\newcommand{\C}{\mathbb{C}}
\newcommand{\R}{\mathbb{R}}
\newcommand{\K}{\mathbb{K}}
\newcommand{\N}{\mathbb{N}}
\newcommand{\I}{\mathcal{I}}
\newcommand{\F}{\mathcal{F}}
\newcommand{\E}{\mathcal{E}}
\newcommand{\G}{\mathcal{G}}
\newcommand{\et}{\text{ et }}
\newcommand{\ou}{\text{ ou }}
\newcommand{\xou}{\ \fbox{\text{ou}}\ }


%Auteur: Cl\'ement Phan, MPSI 2

\begin{document}
	\maketitle
	\section{Structure d'espace vectoriel}
		Soit $E$ un ensemble non vide.
		\begin{defi}
			$E$ est un \underline{espace vectoriel sur $\K$} (ou $\K$-espace vectoriel) si:
			\begin{liste}
				\item $(E,+)$ est un groupe abélien.
				\item $\begin{aligned}[t]\K\times E&\longrightarrow E \\(\lambda,x)&\longmapsto \lambda\cdot x\end{aligned}$ est un loi interne telle que:\\
				$\forall (x,y)\in E^{2},\ \forall (\lambda,\mu)\in\K^{2}:$
				\begin{liste}
					\item $(\lambda+\mu)\cdot x=\lambda\cdot x+\mu\cdot x$
					\item $\lambda\cdot(x+y)=\lambda\cdot x+\lambda\cdot y$
					\item $\lambda\cdot(\mu\cdot x)=(\lambda\times\mu)\cdot x$
					\item $1_\K\cdot x=x$
				\end{liste}
			\end{liste}
		\end{defi}
		\begin{flushleft}
			\underline{Règles de calcul dans un espace vectoriel:}
			\begin{liste}
				\item $(-\lambda)\cdot x=-(\lambda\cdot x)$
				\item $\lambda\cdot (-x)=-(\lambda\cdot x)$
				\item $0_\K\cdot x=0_E$
				\item $\lambda\cdot 0_E=0_E$
			\end{liste}
		\end{flushleft}
	\section{Sous-espace vectoriel}
		Soit $E$ un $\K_\text{EV}$, soit $F$ un sous-ensemble non vide de $E$.
		\begin{defi}
			On dit que $F$ est un \underline{sous-espace vectoriel de $E$} si il est stable par les lois de $E$ et si, muni des restrictions de ces lois, $F$ est un $\K_\text{EV}$
		\end{defi}
		\begin{flushleft}
			\underline{Critères de $\Sev$}
			\begin{liste}
				\item Critère 1: $F$ est un $\Sev$ de $E$ si il est non vide et stable par les lois de $E$.
				\begin{liste}
					\item $0_E\in F$
					\item $\forall(x,y)\in F^{2},\ x+y\in F$
					\item $\forall \lambda\in\K,\ \forall x\in F,\ \lambda\cdot x\in F$
				\end{liste}
				\item Critère 2: $F$ est un $\Sev$ de $E$ si il est non vide et stable par combinaison linéaire.
				\begin{liste}
					\item $0_E\in F$
					\item $\forall (\lambda,\mu)\in\K^{2},\ \forall (x,y)\in F^{2},\ \lambda\cdot x+\mu\cdot y\in F$
				\end{liste}
			\end{liste}
		\end{flushleft}
		\begin{defi}
			On appelle \underline{espace vectoriel engendré par $A$} le plus petit espace vectoriel contenant $A$.
		\end{defi}
		\begin{flushleft}
			\underline{Notation:} $\vect(A)$
		\end{flushleft}
		\begin{flushleft}
			\underline{Justification}\\
			Soit $\mathcal{F}=\left\lbrace F\subset E,\ F\ \Sev\text{ de }E\et A\subset F\right\rbrace $
			\begin{liste}
				\item $F_0=\bigcap\limits_{F\in\mathcal{F}}F$ est un $\Sev$ de $E$, et contient $A$: $\forall F\in\mathcal{F},\ A\subset F$\\
					D'o\`u $F_0\in\mathcal{F}$
				\item Par définition de $F_0$, c'est le plus petit élément de $\mathcal{F}$.
			\end{liste}
			Donc $F_0$ existe.
		\end{flushleft}
		\begin{prop}
			Soit $A=\{X_1,X_2,\ ...\ ,X_p \}$ une partie finie de $E$.\\
			Alors $\vect(A)$ est l'ensemble des combinaisons linéaires de $A$.
		\end{prop}
		\begin{preuve}
			Soit $B$ l'ensemble des combinaisons linéaires de $A$:\\
			$B=\left\lbrace x\in E,\ \exists (\lambda_i)_{i\in\lbracket1,p\rbracket}\in\K^{p},\ x=\sum\limits_{i=1}^{p}\lambda_i \cdot X_i\right\rbrace $
			%15-3-1
		\end{preuve}
\end{document} %ACCENTS!!!