\documentclass[12pt,twoside,a4paper]{article}

\def\chapitre{\'El\'ements de Logique}
\author{MPSI 2}
\def\titre{Op\'erations sur les propositions}

\usepackage{amsfonts}
\usepackage{amsmath}
\usepackage{amsthm}
\usepackage{changepage}
\usepackage{color}
\usepackage{enumitem}
\usepackage{fancyhdr}
\usepackage{framed}
\usepackage[margin=1in]{geometry}
\usepackage{mathrsfs}
\usepackage{tikz, tkz-tab}
\usepackage{titling}

\newtheoremstyle{dotless}{}{}{\itshape}{}{\bfseries}{}{ }{}
\theoremstyle{dotless}

\newtheorem{defs}{Definition}[subsection]
\newenvironment{defi}{\definecolor{shadecolor}{RGB}{255,236,217}\begin{shaded}\begin{defs}\ \\}{\end{defs}\end{shaded}}

\newtheorem{pro}{Propriete}[subsection]
\newenvironment{prop}{\definecolor{shadecolor}{RGB}{230,230,255}\begin{shaded}\begin{pro}\ \\}{\end{pro}\end{shaded}}

\newtheorem{cor}{Corollaire}[subsection]
\newenvironment{coro}{\definecolor{shadecolor}{RGB}{245,250,255}\begin{shaded}\begin{cor}\ \\}{\end{cor}\end{shaded}}

\setlength{\droptitle}{-1in}
\predate{}
\postdate{}
\date{}
\title{\chapitre\\\titre\vspace{-.25in}}

\pagestyle{fancy}
\makeatletter
\lhead{\chapitre\ - \titre}
\rhead{\@author}
\makeatother

\newenvironment{preuve}{\begin{framed}\begin{proof}[\unskip\nopunct]}{\end{proof}\end{framed}}
\newenvironment{liste}{\begin{itemize}[leftmargin=*,noitemsep, topsep=0pt]}{\end{itemize}}
\newenvironment{tab}{\begin{adjustwidth}{.5cm}{}}{\end{adjustwidth}}

\newcommand{\uu}[1] {_{_{#1}}}
\newcommand{\lbracket}{[\![}
\newcommand{\rbracket}{]\!]}
\newcommand{\fonction}[5]{\begin{aligned}[t]#1\colon&#2&&\longrightarrow#3 \\&#4&&\longmapsto#5\end{aligned}}
\newcommand{\systeme}[1]{\left\{\begin{aligned}#1\end{aligned}\right.}
\newcommand{\cercle}[1]{\textcircled{\scriptsize{#1}}}

\newcommand{\lf}[1]{\left(#1\right)}
\newcommand{\C}{\mathbb{C}}
\newcommand{\R}{\mathbb{R}}
\newcommand{\K}{\mathbb{K}}
\newcommand{\N}{\mathbb{N}}
\newcommand{\I}{\mathcal{I}}
\newcommand{\F}{\mathcal{F}}
\newcommand{\E}{\mathcal{E}}
\newcommand{\G}{\mathcal{G}}
\newcommand{\et}{\text{ et }}
\newcommand{\ou}{\text{ ou }}
\newcommand{\xou}{\ \fbox{\text{ou}}\ }


\begin{document}
	\maketitle
	\section{Codage et valeurs logiques}
		Soit $A$ une proposition. On lui associe une valeur logique (Vrai ou Faux) ou binaire \\($0$ ou $1$)
		\\
		Soient $A$ et $B$. Si $A$ et $B$ ont la m\^eme valeur logique, on note $A\vdashv B$\\
		\\
		Soient $a$ et $b$ deux codages binaires.
		\begin{liste}
			\item N\'egation de $a$: $\neg a=1-a$\\
			\item $"a\vee b"$, "$a$ ou $b$", "$a$ sup $b$"\\
				\begin{tabular}{ c | c | c  }
					$a$ & $b$ &($a\vee b$)\\\hline 
					$1$&$1$&$1$\\
					$1$&$0$&$1$\\
					$0$&$1$&$1$\\
					$0$&$0$&$0$\\
				\end{tabular}
				$a\vee b=a+b-a\,b$\\\\
			\item "$a\land b$", "$a$ et $b$", "$a$ inf $b$"\\
				\begin{tabular}{ c | c | c  }
					$a$ & $b$ &($a\land b$)\\\hline 
					$1$&$1$&$1$\\
					$1$&$0$&$0$\\
					$0$&$1$&$0$\\
					$0$&$0$&$0$\\
				\end{tabular}
				$a\land b=a\,b$
		\end{liste}
	\section{Op\'erations \'el\'ementaires sur les propositions}
		Soient $A$ et $B$ deux propositions de codage binaire $a$ et $b$. On a  alors:
		\begin{liste}
			\item N\'egation de $A$: c'est la proposition dont le codage binaire est $\neg a$.\\
				\begin{tabular}{ c | c }
					$A$ & $non(A)$\\\hline 
					$1$&$0$\\
					$0$&$1$\\
				\end{tabular}\\\\
			\item Disjonction: "$A$ ou $B$" est la proposition cod\'ee par "$a\vee b$".\\
				\begin{tabular}{ c | c | c  }
					$A$ & $B$ &($A\vee B$)\\\hline 
					$1$&$1$&$1$\\
					$1$&$0$&$1$\\
					$0$&$1$&$1$\\
					$0$&$0$&$0$\\
				\end{tabular} 
				$\begin{aligned}
					&\text{\textbullet}\ \text{"}A\text{ ou }B\text{" est Vraie si }A\text{ est Vraie ou si }B\text{ est Vraie.}\\
					&\text{\textbullet}\ \text{"}A\text{ ou }B\text{" est Fausse ssi "}A\text{ et }B\text{" est Fausse.}
				\end{aligned}$
		\end{liste}
	\section{Autres op\'erations}
		\begin{liste}
			\item La conjonction
				\begin{defi}"$A$ et $B$" est la proposition "$non(\,non(A)\ ou\ non(B))$"\end{defi}
				\begin{tabular}{ c | c | c  }
					$A$ & $B$ &($A\vee B$)\\\hline 
					$1$&$1$&$1$\\
					$1$&$0$&$0$\\
					$0$&$1$&$0$\\
					$0$&$0$&$0$\\
				\end{tabular} \textbullet\ "$A$ et $B$" est cod\'ee par "$a\land b$"\\\\
			\item L'implication 
				\begin{defi}"$A\Rightarrow B$" est la proposition "$non(A)\ ou\ B$"\end{defi}
				\begin{tabular}{ c | c | c  }
					$A$ & $B$ &($A\Rightarrow B$)\\\hline 
					$1$&$1$&$1$\\
					$1$&$0$&$0$\\
					$0$&$1$&$1$\\
					$0$&$0$&$1$\\
				\end{tabular}
				$\begin{aligned}
					&\text{\textbullet}\ \text{Si }A\text{ est Fausse, }A\Rightarrow B\text{ est Vraie par d\'efinition.}\\
					&\text{\textbullet}\ \text{Si }A\text{ est Vraie, il faut d\'emontrer que }B\text{ est Vraie.}
				\end{aligned}$
			\item La contrapos\'e: "$A\Rightarrow B$" et "$non(B)\Rightarrow non(A)$" ont la m\^eme valeur logique.\\
				D\'emonstration triviale.\\
			\item l'\'equivalence
				\begin{defi}"$A\Leftrightarrow B$" est la proposition "$(A\Rightarrow B)\ et\ (B\Rightarrow A)$" \end{defi}
				\begin{tabular}{ c | c | c  }
					$A$ & $B$ &($A\Leftrightarrow B$)\\\hline 
					$1$&$1$&$1$\\
					$1$&$0$&$0$\\
					$0$&$1$&$0$\\
					$0$&$0$&$1$\\
				\end{tabular}
		\end{liste}
		\underline{Remarques:}
		\begin{liste}
			\item[\textbf{1/}] N\'egation de "ou" et "et":
				\begin{liste}
					\item $non(a\ ou\ b)\vdashv non(A)\ et\ non(B)$
					\item $non(a\ et\ b)\vdashv non(A)\ ou\ non(B)$
				\end{liste}
			\item[\textbf{2/}] N\'egation de l'implication:
				$non(A\Rightarrow B)\vdashv A\ et\ non(B)$
		\end{liste}
\end{document}
