\documentclass[12pt,twoside,a4paper]{article}

\newcommand{\vdashv}{\mathrel{\text{\ooalign{$\vdash$\cr$\dashv$\cr}}}}

\def\chapitre{\'El\'ements de Logique}
\author{MPSI 2}
\def\titre{G\'en\'eralit\'es}

\usepackage{amsfonts}
\usepackage{amsmath}
\usepackage{amsthm}
\usepackage{changepage}
\usepackage{color}
\usepackage{enumitem}
\usepackage{fancyhdr}
\usepackage{framed}
\usepackage[margin=1in]{geometry}
\usepackage{mathrsfs}
\usepackage{tikz, tkz-tab}
\usepackage{titling}

\newtheoremstyle{dotless}{}{}{\itshape}{}{\bfseries}{}{ }{}
\theoremstyle{dotless}

\newtheorem{defs}{Definition}[subsection]
\newenvironment{defi}{\definecolor{shadecolor}{RGB}{255,236,217}\begin{shaded}\begin{defs}\ \\}{\end{defs}\end{shaded}}

\newtheorem{pro}{Propriete}[subsection]
\newenvironment{prop}{\definecolor{shadecolor}{RGB}{230,230,255}\begin{shaded}\begin{pro}\ \\}{\end{pro}\end{shaded}}

\newtheorem{cor}{Corollaire}[subsection]
\newenvironment{coro}{\definecolor{shadecolor}{RGB}{245,250,255}\begin{shaded}\begin{cor}\ \\}{\end{cor}\end{shaded}}

\setlength{\droptitle}{-1in}
\predate{}
\postdate{}
\date{}
\title{\chapitre\\\titre\vspace{-.25in}}

\pagestyle{fancy}
\makeatletter
\lhead{\chapitre\ - \titre}
\rhead{\@author}
\makeatother

\newenvironment{preuve}{\begin{framed}\begin{proof}[\unskip\nopunct]}{\end{proof}\end{framed}}
\newenvironment{liste}{\begin{itemize}[leftmargin=*,noitemsep, topsep=0pt]}{\end{itemize}}
\newenvironment{tab}{\begin{adjustwidth}{.5cm}{}}{\end{adjustwidth}}

\newcommand{\uu}[1] {_{_{#1}}}
\newcommand{\lbracket}{[\![}
\newcommand{\rbracket}{]\!]}
\newcommand{\fonction}[5]{\begin{aligned}[t]#1\colon&#2&&\longrightarrow#3 \\&#4&&\longmapsto#5\end{aligned}}
\newcommand{\systeme}[1]{\left\{\begin{aligned}#1\end{aligned}\right.}
\newcommand{\cercle}[1]{\textcircled{\scriptsize{#1}}}

\newcommand{\lf}[1]{\left(#1\right)}
\newcommand{\C}{\mathbb{C}}
\newcommand{\R}{\mathbb{R}}
\newcommand{\K}{\mathbb{K}}
\newcommand{\N}{\mathbb{N}}
\newcommand{\I}{\mathcal{I}}
\newcommand{\F}{\mathcal{F}}
\newcommand{\E}{\mathcal{E}}
\newcommand{\G}{\mathcal{G}}
\newcommand{\et}{\text{ et }}
\newcommand{\ou}{\text{ ou }}
\newcommand{\xou}{\ \fbox{\text{ou}}\ }


\begin{document}
	\maketitle
	\section{L'univers math\'ematique}
		\begin{liste}
			\item\textbf{Objets math\'ematiques}
				\begin{tab}\begin{liste}
					\item \underline{Objets \'el\'ementaires:} ensemble des r\'eels, des entiers, des points...
					\item \underline{Objets \'elabores:} constructions d'objets \'el\'ementaires: droites, cercles, fonctions...\\\\
				\end{liste}\end{tab}
			\item Une \textbf{Proposition} est un \'enonc\'e portant sur un ou plusieurs objets math\'ematiques.\\\\
			\item On appelle \textbf{D\'emonstration} une suite de propositions qui s'enchainent selon la logique math\'ematique. De plus:
				\begin{liste}
					\item La logique math\'ematique poss\`ede des r\`egles.
					\item Chaque proposition poss\`ede une valeur logique.\\
				\end{liste}
				Mais si $A$ est une proposition:
				\begin{liste}
					\item $A$ est Vraie ou$^X$ Fausse
					\item Si $A$ est Vraie et Fausse, $A$ est une proposition contradictoire.
					\item Il se peut que l'on ne puisse d\'emontrer ni $A$ ni $non(A)$. on dit que $A$ est une proposition \underline{ind\'ecidable}.
					\item On peut d\'emontrer que $A$ est ind\'ecidable.
				\end{liste}
		\end{liste}
\end{document}
