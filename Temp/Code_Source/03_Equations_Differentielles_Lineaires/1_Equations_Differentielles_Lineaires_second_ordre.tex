\documentclass[12pt,twoside,a4paper]{article}

\def\chapitre{EDL}
\author{MPSI 2}
\def\titre{Equations Differentielles Lineaires du second ordre}

\usepackage{amsfonts}
\usepackage{amsmath}
\usepackage{amsthm}
\usepackage{changepage}
\usepackage{color}
\usepackage{enumitem}
\usepackage{fancyhdr}
\usepackage{framed}
\usepackage[margin=1in]{geometry}
\usepackage{mathrsfs}
\usepackage{tikz, tkz-tab}
\usepackage{titling}

\newtheoremstyle{dotless}{}{}{\itshape}{}{\bfseries}{}{ }{}
\theoremstyle{dotless}

\newtheorem{defs}{Definition}[subsection]
\newenvironment{defi}{\definecolor{shadecolor}{RGB}{255,236,217}\begin{shaded}\begin{defs}\ \\}{\end{defs}\end{shaded}}

\newtheorem{pro}{Propriete}[subsection]
\newenvironment{prop}{\definecolor{shadecolor}{RGB}{230,230,255}\begin{shaded}\begin{pro}\ \\}{\end{pro}\end{shaded}}

\newtheorem{cor}{Corollaire}[subsection]
\newenvironment{coro}{\definecolor{shadecolor}{RGB}{245,250,255}\begin{shaded}\begin{cor}\ \\}{\end{cor}\end{shaded}}

\setlength{\droptitle}{-1in}
\predate{}
\postdate{}
\date{}
\title{\chapitre\\\titre\vspace{-.25in}}

\pagestyle{fancy}
\makeatletter
\lhead{\chapitre\ - \titre}
\rhead{\@author}
\makeatother

\newenvironment{preuve}{\begin{framed}\begin{proof}[\unskip\nopunct]}{\end{proof}\end{framed}}
\newenvironment{liste}{\begin{itemize}[leftmargin=*,noitemsep, topsep=0pt]}{\end{itemize}}
\newenvironment{tab}{\begin{adjustwidth}{.5cm}{}}{\end{adjustwidth}}

\newcommand{\uu}[1] {_{_{#1}}}
\newcommand{\lbracket}{[\![}
\newcommand{\rbracket}{]\!]}
\newcommand{\fonction}[5]{\begin{aligned}[t]#1\colon&#2&&\longrightarrow#3 \\&#4&&\longmapsto#5\end{aligned}}
\newcommand{\systeme}[1]{\left\{\begin{aligned}#1\end{aligned}\right.}
\newcommand{\cercle}[1]{\textcircled{\scriptsize{#1}}}

\newcommand{\lf}[1]{\left(#1\right)}
\newcommand{\C}{\mathbb{C}}
\newcommand{\R}{\mathbb{R}}
\newcommand{\K}{\mathbb{K}}
\newcommand{\N}{\mathbb{N}}
\newcommand{\I}{\mathcal{I}}
\newcommand{\F}{\mathcal{F}}
\newcommand{\E}{\mathcal{E}}
\newcommand{\G}{\mathcal{G}}
\newcommand{\et}{\text{ et }}
\newcommand{\ou}{\text{ ou }}
\newcommand{\xou}{\ \fbox{\text{ou}}\ }


\begin{document}
	\maketitle
	\section{Generalites}
		\begin{defi}
			Soit $\mathcal{I}$ un intervalle reel.
			$a$,$b$,$c$,$d$ des applications definies sur $\mathcal{I}$ a valeurs dans $\mathbb{K}$ et continues sur $\mathcal{I}$.
			On appelle equation differentielle lineaire du second ordre toute relation du type :
			$$\begin{aligned}
				\forall x\in\mathcal{I},a\left(x\right)y''\left(x\right)+b\left(x\right)y'\left(x\right)+c\left(x\right)y\left(x\right)&=d\left(x\right)\\
																												   &\text{ d'inconnue y}
			\end{aligned}$$
		\end{defi}
		\begin{defi}
			Une solution particuliere sur $\mathcal{I}$ de l'equation differentielle precedente est une application $\phi\colon\mathcal{I}\longrightarrow\mathbb{K}$ telle que :
			\begin{liste}
				\item $\phi$ est deux foix derivable sur $\mathcal{I}$.
				\item $\forall x\in\mathcal{I},a\left(x\right)\phi''\left(x\right)+b\left(x\right)\phi'\left(x\right)+c\left(x\right)\phi\left(x\right)=d\left(x\right)$
			\end{liste}
		\end{defi}
		\textbf{Remarque :} L'ensemble des solutions de l'equation differentielle lineaire du second ordre homogene
		$$
			\forall x\in\mathcal{I},a\left(x\right)y''\left(x\right)+b\left(x\right)y'\left(x\right)+c\left(x\right)y\left(x\right)=0
		$$
		a une structure d'espace vectoriel (non vide et stable par combinaisons lineaires).
		
	\section{Equation Differentielle Lineaire du second ordre a coefficients constants}
		\subsection{Definitions}
			\begin{defi}
				Soit $\left(a,b,c\right)\in\mathbb{K}^3$ tels que $a\neq0$.
				Soit $d\colon\mathcal{I}\longrightarrow\mathbb{K}$ une application continue sur $\mathcal{I}$.
				On appelle equation differentielle lineaire du second ordre a coefficients constants une relation du type :
				$$
					\forall x\in\mathcal{I},ay''\left(x\right)+by'\left(x\right)+cy\left(x\right)=d\left(x\right)
				$$
			\end{defi}
		\newpage
		\subsection{Etude de l'equation homogene associee}
			$$\begin{aligned}
				\forall x\in\mathcal{I},ay''\left(x\right)+by'\left(x\right)+cy\left(x\right)&=0\\
																								&\text{avec}\left(a,b,c\right)\in\mathbb{K}^3,a\neq0
			\end{aligned}$$
			\begin{prop}
				(Solutions reelles)\\
				On suppose $a,b,c$ reels et $a\neq0$\\
				Soit l'equation differentielle homogene associee :
				$$
					\forall x\in\mathcal{I},ay''\left(x\right)+by'\left(x\right)+cy\left(x\right)=0 \ \ \ \left(2\right)
				$$
				On considere l'equation $ar^2+br+c=0 \ \left(E\right)$ d'inconnue $r$.\\
				$\left(E\right)$ s'appelle l'equation caracteristique associee a $\left(2\right)$\\
				\textbf{Cas 1 :} $\left(E\right)$ admet deux solutions reelles distinctes $r\uu1$ et $r\uu2$.
				\begin{tab}
					$y$ est solution de $\left(2\right)$ ssi:
					$$
						\exists\left(k\uu1,k\uu2\right)\in\mathbb{R}^2,\forall x\in\mathbb{R},y\left(x\right)=k\uu1e^{r\uu1x}+k\uu2e^{r\uu2x}
					$$
				\end{tab}
				\textbf{Cas 2 :} $\left(E\right)$ admet une unique solution reelle $r\uu0$.
				\begin{tab}
					$y$ est solution de $\left(2\right)$ ssi:
					$$
						\exists\left(\lambda,\mu\right)\in\mathbb{R}^2,\forall x\in\mathbb{R},y\left(x\right)=\left(\lambda x+\mu\right)e^{r\uu0x}					$$
				\end{tab}
				\textbf{Cas 3 :} $\left(E\right)$ admet deux solutions complexes non reelles conjuguees $\alpha\pm\imath\beta$.
				\begin{tab}
					$y$ est solution de $\left(2\right)$ ssi:
					$$
						\exists\left(a,b\right)\in\mathbb{R}^2,\forall x\in\mathbb{R},y\left(x\right)=e^{\alpha x}\left(a\cos\beta x+\imath\sin\beta x\right)
					$$
				\end{tab}
			\end{prop}
			\begin{prop}
				(moins importante, solutions complexes)\\
				On suppose $a,b,c$ complexes et $a\neq0$.\\
				\textbf{Cas 1 :} $\left(E\right)$ admet deux solutions distinctes $r\uu1$ et $r\uu2$.
				\begin{tab}
					$y$ est solution de $\left(2\right)$ ssi:
					$$
						\exists\left(k\uu1,k\uu2\right)\in\mathbb{C}^2,\forall x\in\mathbb{R},y\left(x\right)=k\uu1e^{r\uu1x}+k\uu2e^{r\uu2x}
					$$
				\end{tab}
				\textbf{Cas 2 :} $\left(E\right)$ admet une unique solution $r\uu0$.
				\begin{tab}
					$y$ est solution de $\left(2\right)$ ssi:
					$$
						\exists\left(\lambda,\mu\right)\in\mathbb{C}^2,\forall x\in\mathbb{R},y\left(x\right)=\left(\lambda x+\mu\right)e^{r\uu2x}
					$$
				\end{tab}
			\end{prop}
\end{document}
