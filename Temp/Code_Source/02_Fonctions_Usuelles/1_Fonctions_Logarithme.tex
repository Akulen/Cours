\documentclass[12pt,twoside,a4paper]{article}

\def\chapitre{Fonctions Usuelles}
\author{MPSI 2}
\def\titre{Fonctions Logarithme}

\usepackage{amsfonts}
\usepackage{amsmath}
\usepackage{amsthm}
\usepackage{changepage}
\usepackage{color}
\usepackage{enumitem}
\usepackage{fancyhdr}
\usepackage{framed}
\usepackage[margin=1in]{geometry}
\usepackage{mathrsfs}
\usepackage{tikz, tkz-tab}
\usepackage{titling}

\newtheoremstyle{dotless}{}{}{\itshape}{}{\bfseries}{}{ }{}
\theoremstyle{dotless}

\newtheorem{defs}{Definition}[subsection]
\newenvironment{defi}{\definecolor{shadecolor}{RGB}{255,236,217}\begin{shaded}\begin{defs}\ \\}{\end{defs}\end{shaded}}

\newtheorem{pro}{Propriete}[subsection]
\newenvironment{prop}{\definecolor{shadecolor}{RGB}{230,230,255}\begin{shaded}\begin{pro}\ \\}{\end{pro}\end{shaded}}

\newtheorem{cor}{Corollaire}[subsection]
\newenvironment{coro}{\definecolor{shadecolor}{RGB}{245,250,255}\begin{shaded}\begin{cor}\ \\}{\end{cor}\end{shaded}}

\setlength{\droptitle}{-1in}
\predate{}
\postdate{}
\date{}
\title{\chapitre\\\titre\vspace{-.25in}}

\pagestyle{fancy}
\makeatletter
\lhead{\chapitre\ - \titre}
\rhead{\@author}
\makeatother

\newenvironment{preuve}{\begin{framed}\begin{proof}[\unskip\nopunct]}{\end{proof}\end{framed}}
\newenvironment{liste}{\begin{itemize}[leftmargin=*,noitemsep, topsep=0pt]}{\end{itemize}}
\newenvironment{tab}{\begin{adjustwidth}{.5cm}{}}{\end{adjustwidth}}

\newcommand{\uu}[1] {_{_{#1}}}
\newcommand{\lbracket}{[\![}
\newcommand{\rbracket}{]\!]}
\newcommand{\fonction}[5]{\begin{aligned}[t]#1\colon&#2&&\longrightarrow#3 \\&#4&&\longmapsto#5\end{aligned}}
\newcommand{\systeme}[1]{\left\{\begin{aligned}#1\end{aligned}\right.}
\newcommand{\cercle}[1]{\textcircled{\scriptsize{#1}}}

\newcommand{\lf}[1]{\left(#1\right)}
\newcommand{\C}{\mathbb{C}}
\newcommand{\R}{\mathbb{R}}
\newcommand{\K}{\mathbb{K}}
\newcommand{\N}{\mathbb{N}}
\newcommand{\I}{\mathcal{I}}
\newcommand{\F}{\mathcal{F}}
\newcommand{\E}{\mathcal{E}}
\newcommand{\G}{\mathcal{G}}
\newcommand{\et}{\text{ et }}
\newcommand{\ou}{\text{ ou }}
\newcommand{\xou}{\ \fbox{\text{ou}}\ }


\begin{document}
	\maketitle
\section{Logarithme Neperien}
	\begin{defi}
		Le logarithme neperien est l'unique primitive de $x\longmapsto\frac{1}{x}$ qui s'annule en 0:
		$$
			\fonction{ln}{\mathbb{R}^{+*}}{\mathbb{R}}{x}{\int_1^x\frac{1}{x}\ dx}
		$$
	\end{defi}
	\begin{prop}
		$\forall(x;y)\in\mathbb{R}^{+*},\ ln(xy)=ln(x)+ln(y)$
	\end{prop}
	\begin{preuve}
		On considere l'application $\fonction{h}{\mathbb{R}^{+*^2}}{\mathbb{R}}{(x;y)}{ln(xy)}$\\\\
		\begin{tab}
			Soit $y\in\mathbb{R}^{+*}$ fixe. \\
			Soit $h_y$ l'application définie par $\fonction{h_y}{\mathbb{R}^{+*}}{\mathbb{R}}{X}{ln(xy)}$\\\\\\
			$h_y$ et derivable car $ln$ est derivable, $\forall x\in\mathbb{R}^{+*},\ h_y'=\frac{y}{xy}=\frac{1}{x}$
			$ln$ et $h_y$ sont des primitives de $x\longmapsto\frac{1}{x}$, donc elles different d'une constante.
			$$\begin{aligned}
				\exists K\in\mathbb{R},\ \forall x\in\mathbb{R}^{+*},\ &h_y(x)=ln(x)+K\\
				\iff &h_y(x)-ln(x)=K\\
				\iff &h_y(1)-ln(1)=K\\
				\iff &ln(y)=K
			\end{aligned}$$
		\end{tab}
		\textbf{Conclusion:} $\forall x\in\mathbb{R}^{+*},\ h_y(x)=ln(x)+ln(y)$\\
		\\
		\begin{tab}Or, ce raisonnement est valable pout tout y de $\mathbb{R}^{+*}$, donc:\\\end{tab}
		\textbf{Conclusion Generale:} $\forall x\in\mathbb{R}^{+*^2},\ ln(xy)=ln(x)+ln(y)$
	\end{preuve}
	\begin{coro}
		\begin{liste}
			\item $\forall x\in\mathbb{R}^{+*},\ ln\left(\frac{1}{x}\right)=-ln(x)$
			\item $\forall(x;y)\in\mathbb{R}^{+*^2},\ ln\left(\frac{x}{y}\right)=ln(x)-ln(y)$
			\item $\forall n\in\mathbb{N},\ ln\left(x^n\right)=n\ ln(x)$
		\end{liste}
	\end{coro}
	\begin{prop}
		\begin{liste}
			\item $\lim\limits_{x\rightarrow+\infty}ln(x)=+\infty$
			\item $\lim\limits_{x\rightarrow0}ln(x)=-\infty$
		\end{liste}
	\end{prop}
	\begin{preuve}
		\textbf{1/} Pour $n\in\mathbb{N}$, on note $U_n=2^n$.
		\begin{tab}\begin{tab}
			$ln\left(U_n\right)=n\ ln(2)$\\
			\textbf{Donc: }$\forall A\in\mathbb{R},\ \exists n\uu0\in\mathbb{N},\ n\geq n\uu0 \Longrightarrow\ ln\left(U_n\right)>A$\\
		\end{tab}\end{tab}
		\textbf{2/} De plus, $ln$ est croissante, donc si $x\geq U_{n\uu0}$, alors:
		\begin{tab}\begin{tab}
			$ln(x)\geq ln\left(U_n\right)>A$ \\
			Posons $x\uu0=U_{n_0}$\\
			\textbf{Donc: }$\forall A\in\mathbb{R},\ \exists x\uu0\in\mathbb{R},\ x\geq x\uu0 \Longrightarrow\ ln\left(x\right)>A$\\
			\textbf{Donc: }$\lim\limits_{x\rightarrow+\infty}ln(x)=+\infty$\\
		\end{tab}\end{tab}
		\textbf{3/} En $-\infty$:
		\begin{tab}\begin{tab}
			$\systeme{&ln(x)=-ln\left(\frac{1}{x}\right)\\&x>0}$\\
			$\lim\limits_{x\rightarrow0}\frac{1}{x}=+\infty$, donc par composition de limites:\\
			$\lim\limits_{x\rightarrow0}ln(x)=-\infty$
		\end{tab}\end{tab}
	\end{preuve}
	\begin{prop}
		\begin{liste}
			\item $ln$ realise une bijection de $\mathbb{R}^{+*}$ sur $\mathbb{R}$.  (Th de la bijection)
			\item $\forall x\in\mathbb{R}^{+*},\ ln'(x)=\frac{1}{x}$, par definition.
			\item $ln$ est de classe $\mathscr{C}^\infty$ sur $\mathbb{R}^{+*}$ (Par recurrence)
		\end{liste}
	\end{prop}
\pagebreak
\section{Logatithme en base $a$}
	\begin{defi}
		Soit $a\in\mathbb{R}^{+*}\setminus\{1\}$\\
		Le logarithme en base $a$ est l'application definie par:\\
		\begin{tab}\begin{tab}$\fonction{log_a}{\mathbb{R}^{+*}}{\mathbb{R}}{x}{\frac{ln(x)}{ln(a)}}$\\\\\end{tab}\end{tab}
		\begin{liste}
			\item Si $a=10$, $log_a$ est note $log$
			\item On note $e$ l'unique reel tel que $log_e=ln$
		\end{liste}
	\end{defi}
	\begin{prop}
		\begin{liste}
			\item $log_a(1)=0$
			\item $log_a(a)=1$
			\item $\forall(x;y)\in\mathbb{R}^{+*^2},\ ln\left(xy\right)=ln(x)+ln(y)$
			\item $\forall x\in\mathbb{R}^{+*},\ log_a'(x)=\frac{1}{x\ ln(a)}$
			\item Si $a>1$,
				\begin{liste}
					\item $log_a$ est croissante sur $\mathbb{R}^{+*}$
					\item $\lim\limits_{x\rightarrow+\infty}log_a(x)=+\infty$
					\item $\lim\limits_{x\rightarrow0}log_a(x)=-\infty$
				\end{liste}
			\item Si $0<a<1$,
				\begin{liste}
					\item $log_a$ est decroissante sur $\mathbb{R}^{+*}$
					\item $\lim\limits_{x\rightarrow+\infty}log_a(x)=-\infty$
					\item $\lim\limits_{x\rightarrow0}log_a(x)=+\infty$
				\end{liste}
				\item $log_a$ realise une bijection de $\mathbb{R}^{+*}$ sur $\mathbb{R}$
		\end{liste}
	\end{prop}
\end{document}