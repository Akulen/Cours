\documentclass[12pt,twoside,a4paper]{article}

\def\chapitre{Complexes}
\author{MPSI 2}
\def\titre{Generalites}

\usepackage{amsfonts}
\usepackage{amsmath}
\usepackage{amsthm}
\usepackage{changepage}
\usepackage{color}
\usepackage{enumitem}
\usepackage{fancyhdr}
\usepackage{framed}
\usepackage[margin=1in]{geometry}
\usepackage{mathrsfs}
\usepackage{tikz, tkz-tab}
\usepackage{titling}

\newtheoremstyle{dotless}{}{}{\itshape}{}{\bfseries}{}{ }{}
\theoremstyle{dotless}

\newtheorem{defs}{Definition}[subsection]
\newenvironment{defi}{\definecolor{shadecolor}{RGB}{255,236,217}\begin{shaded}\begin{defs}\ \\}{\end{defs}\end{shaded}}

\newtheorem{pro}{Propriete}[subsection]
\newenvironment{prop}{\definecolor{shadecolor}{RGB}{230,230,255}\begin{shaded}\begin{pro}\ \\}{\end{pro}\end{shaded}}

\newtheorem{cor}{Corollaire}[subsection]
\newenvironment{coro}{\definecolor{shadecolor}{RGB}{245,250,255}\begin{shaded}\begin{cor}\ \\}{\end{cor}\end{shaded}}

\setlength{\droptitle}{-1in}
\predate{}
\postdate{}
\date{}
\title{\chapitre\\\titre\vspace{-.25in}}

\pagestyle{fancy}
\makeatletter
\lhead{\chapitre\ - \titre}
\rhead{\@author}
\makeatother

\newenvironment{preuve}{\begin{framed}\begin{proof}[\unskip\nopunct]}{\end{proof}\end{framed}}
\newenvironment{liste}{\begin{itemize}[leftmargin=*,noitemsep, topsep=0pt]}{\end{itemize}}
\newenvironment{tab}{\begin{adjustwidth}{.5cm}{}}{\end{adjustwidth}}

\newcommand{\uu}[1] {_{_{#1}}}
\newcommand{\lbracket}{[\![}
\newcommand{\rbracket}{]\!]}
\newcommand{\fonction}[5]{\begin{aligned}[t]#1\colon&#2&&\longrightarrow#3 \\&#4&&\longmapsto#5\end{aligned}}
\newcommand{\systeme}[1]{\left\{\begin{aligned}#1\end{aligned}\right.}
\newcommand{\cercle}[1]{\textcircled{\scriptsize{#1}}}

\newcommand{\lf}[1]{\left(#1\right)}
\newcommand{\C}{\mathbb{C}}
\newcommand{\R}{\mathbb{R}}
\newcommand{\K}{\mathbb{K}}
\newcommand{\N}{\mathbb{N}}
\newcommand{\I}{\mathcal{I}}
\newcommand{\F}{\mathcal{F}}
\newcommand{\E}{\mathcal{E}}
\newcommand{\G}{\mathcal{G}}
\newcommand{\et}{\text{ et }}
\newcommand{\ou}{\text{ ou }}
\newcommand{\xou}{\ \fbox{\text{ou}}\ }


\begin{document}
	\maketitle

	\section{Plan Affixe $\mathcal{A}_2$}
		On note $\mathcal{A}_2$ l'ensemble des points du plan.

		\subsection{Angles}
			\begin{defi}
				Soit $\vec{u}$ et $\vec{v}$ deux vecteurs unitaires du plan.\\
				On appelle  mesure de l'angle oriente $\left(\vec{u}, \vec{v}\right)$ tout reel $\theta$ tel que : $\vec{v} = \cos\left(\theta\right)\vec{u} + \sin\left(\theta\right)\vec{u'}$ avec $\vec{u'}$ est le vecteur unitaire directement orthogonal a $\vec{u}$.\\
				On note mesure de $\vec{u}$, $\vec{v}$ : $mes\left(\vec{u}, \vec{v}\right) \equiv \theta \ \left[2\pi\right]$
			\end{defi}
			\begin{defi}
				Soit $\vec{u}$ et $\vec{v}$ deux vecteurs non nuls.\\
				On appelle mesure de l'angle oriente $\left(\vec{u}, \vec{v}\right)$ toute mesure de l'angle oriente des vecteurs unitaires associes.\\
				On a donc : $mes\left(\vec{u}, \vec{v}\right) \equiv mes\left(\frac{\vec{u}}{\|\vec{u}\|}, \frac{\vec{v}}{\|\vec{v}\|}\right) \ \left[2\pi\right]$
			\end{defi}
		
		\subsection{Definition sur les nombres complexes}
			\begin{defi}
				Soit $\mathcal{A}_2$ l'ensemble des points du plan muni d'un repere orthonorme $\left(\mathcal{O}, \vec{i}, \vec{j}\right)$.\\
				On note $\mathcal{E}_2$ l'ensemble des vecteurs du plan.\\
				L'ensemble des complexes peut etre mis en bijection avec $\mathcal{E}_2$ ou avec $\mathcal{A}_2$.
			\end{defi}
			\begin{preuve}
				$$ \begin{aligned}[t]
					\mathbb{C} &\longrightarrow \mathcal{A}_2 \\
					x+iy &\longmapsto M\text{, de coordonnees $\left(x, y\right)$ dans le repere $\left(\mathcal{O}, \vec{i}, \vec{j}\right)$} \\
					\mathbb{C} &\longrightarrow \mathcal{E}_2 \\
					x+iy &\longmapsto \vec{OM}\text{, de coordonnees $\left(x, y\right)$ dans la base $\left(\mathcal{O}, \vec{i}, \vec{j}\right)$}
				\end{aligned} $$
				Sont deux applications bijectives.
			\end{preuve}
			\begin{defi}
				\begin{liste}
					\item On dit que $M$ de coordonnees $\left(x, y\right)$ est l'image affixe du complexe $z=x+\imath y$.
					\item On dit que $z=x+\imath y$ est l'affixe du point $M$ / du vecteur $\vec{OM}$.
				\end{liste}
			\end{defi}
			\begin{defi}
				Soit $z=x+\imath y$.
				\begin{liste}
					\item Le module de $z$ : $|z|=\left\|\vec{OM}\right\|=\sqrt{x^2+y^2}$.
					\item Un argument de $z$ pour $z$ non nul : $\arg\left(z\right)=mes\left(\vec{i}, \vec{OM}\right)$.
				\end{liste}
			\end{defi}
			\begin{prop}
				Soit $z$ un complexe non nul.
				\begin{liste}
					\item $z$ est reel $\begin{aligned}[t]
											&\iff \arg\left(z\right)\equiv 0 \ \left[2\pi\right] \\
											&\iff \exists k\in\mathbb{Z},\arg\left(z\right)=k\pi
										\end{aligned}$
					\item $z$ est imaginaire $\begin{aligned}[t]
												&\iff \arg\left(z\right)\equiv \frac{\pi}{2} \ \left[\pi\right]\\
												&\iff \exists k\in\mathbb{Z},\arg\left(z\right)=\frac{\pi}{2}+k\pi
											  \end{aligned}$
				\end{liste}
			\end{prop}
			\begin{prop}
				\begin{liste}
					\item $|z\times z'|=|z|\times|z'|$
					\item $\left\{\begin{aligned}&\arg\left(zz'\right)\equiv\arg\left(z\right)+\arg\left(z'\right) \ \left[2\pi\right] \\
													  &z\text{ et }z'\text{ non nuls}\end{aligned}\right.$
				\end{liste}
			\end{prop}
			\begin{defi}
				Conjuge de $z=x+\imath y$ : $\bar{z}=x-\imath y$
			\end{defi}
			\begin{prop}
				\begin{liste}
					\item $z$ est reel $\iff z=\bar{z}$
					\item $z$ est imaginaire $\iff z=-\bar{z}$
					\item $\overline{\left(\bar{z}\right)}=z$
					\item $\left|z\right|^2=z\bar{z}$
				\end{liste}
			\end{prop}
			\textbf{Notation :}\\
			Si $z=x+\imath y$, on note:
			\begin{liste}
				\item[] $\mathcal{R}e\left(z\right)=x$ \ \ \ la partie reelle de $z$.
				\item[] $\mathcal{I}m\left(z\right)=y$ \ \ \ la partie imaginaire de $z$.
			\end{liste}
			On a :
			$$ \begin{aligned}
				\mathcal{R}e\left(z\right)=\frac{z+\bar{z}}{2} \\
				\mathcal{I}m\left(z\right)=\frac{z-\bar{z}}{2\imath}
			\end{aligned}$$
	
	\section{Exponentielle complexe}
		Soit $\begin{aligned}[t]
				\psi\colon\mathbb{R}&\longrightarrow\mathbb{C} \\
						  x&\longmapsto\cos x+\imath\sin x
		      \end{aligned}$ \\
		C'est une fonction d'une variable reelle et a valeurs complexes dont les applications composantes sont $\cos$ et $\sin$. \\
		Ces applications composantes sont derivables sur $\mathbb{R}$, donc $\psi$ est derivable sur $\mathbb{R}$ et on a :
		$$
			\forall x\in\mathbb{R},\psi'\left(x\right)=-\sin x+\imath\cos x
		$$
		\textbf{Notation :} $\psi\left(x\right)=e^{\imath x}$ \\
		\textbf{Remarque :}
		$$\begin{aligned}
			\forall x\in\mathbb{R},\psi'\left(x\right)&=\imath\left(\cos x+\imath\sin x\right)\\
													&=\imath\psi\left(x\right)\\
			\forall x\in\mathbb{R},\psi'\left(x\right)&=\imath e^{\imath x}
		\end{aligned}$$
		\begin{preuve} 
			On a pour $\alpha$ et $\beta$ reels :
			$$\begin{aligned}
				\psi\left(\alpha+\beta\right)&=\cos\left(\alpha+\beta\right)+\imath\sin\left(\alpha+\beta\right) \\
											&=\cos\alpha\cos\beta-\sin\alpha\sin\beta+\imath\left(\sin\alpha\cos\beta+\sin\beta\cos\alpha\right) \\
											&=\cos\beta\left(\cos\alpha+\imath\sin\alpha\right)+\sin\beta\left(\imath\cos\alpha-\sin\alpha\right) \\
											&=\left(\cos\alpha+\imath\sin\alpha\right)\left(\cos\beta+\imath\sin\beta\right) \\
											&=\psi\left(\alpha\right)\psi\left(\beta\right)
			\end{aligned}$$
		\end{preuve}\newpage
		On a demontre :
		\begin{prop}
			$$\begin{aligned}
				\forall\left(\alpha, \beta\right)\in\mathbb{R},\psi\left(\alpha+\beta\right)&=\psi\left(\alpha\right)\psi\left(\beta\right) \\
														 e^{\imath\left(\alpha+\beta\right)}&=e^{\imath\alpha}e^{\imath\beta}
			\end{aligned}$$
		\end{prop}
		\begin{coro}
			\begin{liste}
				\item $\left(\alpha=\beta\right) \ \ \ e^{\imath\left(\alpha+\beta\right)}=\left(e^{\imath\alpha}\right)^2$
				\item Par recurrence, on obtient :
					$$\begin{aligned}
						\forall n\in\mathbb{N},\forall\alpha\in\mathbb{R}, \ &e^{\imath n\alpha}=\left(e^{\imath\alpha}\right)^n \\
									 &\cos\left(n\alpha\right)+\imath\sin\left(n\alpha\right)=\left(\cos\alpha+\imath\sin\alpha\right)^n
					\end{aligned}$$
					Formule de MOIVRE
				\item $\left(\beta=-\alpha\right) \ \ \ e^{\imath\left(\alpha+\beta\right)}=e^{\imath\alpha}e^{-\imath\alpha}\iff e^{-\imath\alpha}=\frac{1}{e^{\imath\alpha}}$
				\item En utilisant $\frac{1}{z^n}=z^{-n}$ on obtient :
					$$\begin{aligned}
						\forall n\in\mathbb{Z},\forall\alpha\in\mathbb{R},e^{-\imath n\alpha}&=\frac{1}{e^{\imath n\alpha}} \\
																					 e^{\imath n\alpha}&=\left(e^{\imath\alpha}\right)^n
					\end{aligned}$$
			\end{liste}
		\end{coro}
		\textbf{Formules et calculs a connaitre} \\
		\begin{liste}
			\item $\cos\alpha=\frac{e^{\imath\alpha}+e^{-\imath\alpha}}{2}$ et $\sin\alpha=\frac{e^{\imath\alpha}-e^{-\imath\alpha}}{2\imath}$
			\item Pour $n\in\mathbb{N}^*$ : $\begin{aligned}[t]\cos^n\alpha&=\left(\frac{e^{\imath\alpha}+e^{-\imath\alpha}}{2}\right)^n \\
																					&=\frac{1}{2^n}\sum\limits_{k=0}^n{n\choose{k}}e^{\imath k\alpha}e^{-\imath\left(n-k\right)\alpha}\end{aligned}$ \\
				En regroupant les termes d'indice $k$ et $n-k$, on obtient une expression du type :
				$$\cos^n\alpha=\sum\limits_{k=0}^na\uu{k}\cos\left(k\alpha\right)$$
				Avec $a\uu{k}$ des coefficients calculables.
		\end{liste}\newpage
		\begin{defi}
			On note $\mathbb{U}=\left\{z\in\mathbb{C},\left|z\right|=1\right\}$
			\begin{liste}
				\item $\mathbb{U}$ est represente dans le plan affine par le cercle unite (de centre $\mathcal{O}$ et de rayon $1$).
				\item $1\in\mathbb{U}$
				\item $\mathbb{U}$ est stable par multiplication.
				\item $\mathbb{U}$ est stable par passage a l'inverse.
			\end{liste}
			On dit que $\left(\mathbb{U},\times\right)$ est un groupe.
		\end{defi}
		\begin{prop}
			Inegalite triangulaire : $\forall\left(z,z'\right)\in\mathbb{C}^2,\left|z+z'\right|\leq\left|z\right|+\left|z'\right|$
		\end{prop}
		\begin{coro}
			$\forall\left(z,z'\right)\in\mathbb{C}^2,\left|\left|z\right|-\left|z'\right|\right|\leq\left|z-z'\right|$
		\end{coro}
		\begin{preuve}
			Montrer que $\forall\left(z,z'\right)\in\mathbb{C}^2,\left|z+z'\right|\leq\left|z\right|+\left|z'\right|$ \\
			\textbf{Cas 1 :} $z'=0$
			\begin{tab}
				Alors $\forall z\in\mathbb{C},\begin{aligned}[t]&\left|z+z'\right|=\left|z\right| \\
																		 &\left|z\right|+\left|z'\right|=\left|z\right|\end{aligned}$
			\end{tab}
			\textbf{Cas 2 :} $z'\neq0$
			\begin{tab}
				$$\left|z+z'\right|\leq\left|z\right|+\left|z'\right|\iff\left|1+\frac{z}{z'}\right|\leq 1+\left|\frac{z}{z'}\right|$$
				On est amene a demontrer :
				$$\forall u\in\mathbb{C},\left|1+u\right|\leq 1+\left|u\right|$$
				Soit $u$ un complexe fixe :
				$$\begin{aligned}
					\left|1+u\right|\leq 1+\left|u\right|&\iff\left|1+u\right|^2\leq\left(1+\left|u\right|\right)^2 \\
																&\iff\left(1+u\right)\left(1+\bar{u}\right)\leq1+2\left|u\right|+\left|u\right|^2 \\
																&\iff1+u+\bar{u}+u\bar{u}\leq1+2\left|u\right|+\left|u\right|^2 \\
																&\iff1+2\mathcal{R}e\left(u\right)+\left|u\right|\leq1+2\left|u\right|+\left|u\right|^2 \\
																&\iff\mathcal{R}e\left(u\right)\leq\left|u\right|
				\end{aligned}$$
				Or cette derniere propriete est vraie pour tout complexe u. \\
				\textbf{Conclusion :} $\forall u\in\mathbb{C},\left|1+u\right|\leq 1+\left|u\right|$
			\end{tab}
			\textbf{Conclusion Generale :} $\forall\left(z,z'\right)\in\mathbb{C}^2,\left|z+z'\right|\leq\left|z\right|+\left|z'\right|$
		\end{preuve}
\end{document}
