\documentclass[12pt,twoside,a4paper]{article}

\def\chapitre{Circuits Electriques dans l'ARQS}
\author{MPSI 2}
\def\titre{Tension \'electrique}

\usepackage{amsfonts}
\usepackage{amsmath}
\usepackage{amsthm}
\usepackage{changepage}
\usepackage{color}
\usepackage{enumitem}
\usepackage{fancyhdr}
\usepackage{framed}
\usepackage[margin=1in]{geometry}
\usepackage{mathrsfs}
\usepackage{tikz, tkz-tab}
\usepackage{titling}

\newtheoremstyle{dotless}{}{}{\itshape}{}{\bfseries}{}{ }{}
\theoremstyle{dotless}

\newtheorem{defs}{Definition}[subsection]
\newenvironment{defi}{\definecolor{shadecolor}{RGB}{255,236,217}\begin{shaded}\begin{defs}\ \\}{\end{defs}\end{shaded}}

\newtheorem{pro}{Propriete}[subsection]
\newenvironment{prop}{\definecolor{shadecolor}{RGB}{230,230,255}\begin{shaded}\begin{pro}\ \\}{\end{pro}\end{shaded}}

\newtheorem{cor}{Corollaire}[subsection]
\newenvironment{coro}{\definecolor{shadecolor}{RGB}{245,250,255}\begin{shaded}\begin{cor}\ \\}{\end{cor}\end{shaded}}

\setlength{\droptitle}{-1in}
\predate{}
\postdate{}
\date{}
\title{\chapitre\\\titre\vspace{-.25in}}

\pagestyle{fancy}
\makeatletter
\lhead{\chapitre\ - \titre}
\rhead{\@author}
\makeatother

\newenvironment{preuve}{\begin{framed}\begin{proof}[\unskip\nopunct]}{\end{proof}\end{framed}}
\newenvironment{liste}{\begin{itemize}[leftmargin=*,noitemsep, topsep=0pt]}{\end{itemize}}
\newenvironment{tab}{\begin{adjustwidth}{.5cm}{}}{\end{adjustwidth}}

\newcommand{\uu}[1] {_{_{#1}}}
\newcommand{\lbracket}{[\![}
\newcommand{\rbracket}{]\!]}
\newcommand{\fonction}[5]{\begin{aligned}[t]#1\colon&#2&&\longrightarrow#3 \\&#4&&\longmapsto#5\end{aligned}}
\newcommand{\systeme}[1]{\left\{\begin{aligned}#1\end{aligned}\right.}
\newcommand{\cercle}[1]{\textcircled{\scriptsize{#1}}}

\newcommand{\lf}[1]{\left(#1\right)}
\newcommand{\C}{\mathbb{C}}
\newcommand{\R}{\mathbb{R}}
\newcommand{\K}{\mathbb{K}}
\newcommand{\N}{\mathbb{N}}
\newcommand{\I}{\mathcal{I}}
\newcommand{\F}{\mathcal{F}}
\newcommand{\E}{\mathcal{E}}
\newcommand{\G}{\mathcal{G}}
\newcommand{\et}{\text{ et }}
\newcommand{\ou}{\text{ ou }}
\newcommand{\xou}{\ \fbox{\text{ou}}\ }


%Auteur: Tomas Rigaux, MPSI 2

\begin{document}
	\maketitle\ \\
	La diff\'erence de potentiel (ddp) est la tension \'electrique.
	\begin{center}	
		\begin{circuitikz}[scale=1.5]
			\draw (0,0)
				to [battery1] (0,2)
				-- (0.5,2)
				to [generic,v=$U_{AB}$] (2.5,2)
				-- (3,2)
				-- (3,1.75)
				to [generic,v^>=$U_{CB}$] (3,0.25)
				-- (3,0)
				-- (0,0);
			\draw (0.5,2) node {$\bullet$} node [above] {A};
			\draw (2.5,2) node {$\bullet$} node [above] {B};
			\draw (3,0.25) node {$\bullet$} node [right] {C};
			\node[anchor=east, left] at (-1,1)
				{$\underbrace{U_{AB}}_{\text{tension}}=\underbrace{V_A-V_B}_{potentiels}$};
		\end{circuitikz}
		\end{center}
	\textbf{G\'en\'eralisation :} entre deux points $A$ et $B$ du circuit, il existe une tension, un ddp. \\ \\
	En pratique, on mesure des ddp. \\
	On choisit l'origine des tensions de fa\c con arbitraire. \\
	Ce point de potentiel nul est appell\'e la masse du circuit et est symbolys\'e par \begin{circuitikz}\draw node[ground] {};\end{circuitikz}. \\
	Le potentiel \'electrique s'exprime en Volts ($V$). \\ \\
	\textbf{Ordres de grandeur :}
	\begin{liste}
		\item lignes HT : $10^5V$
		\item eclair : $10^6V$
	\end{liste}
\end{document}
