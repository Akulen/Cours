\documentclass[12pt,twoside,a4paper]{article}

\def\chapitre{Circuits Electriques dans l'ARQS}
\author{MPSI 2}
\def\titre{Lois de Kirchhoff}

\usepackage{amsfonts}
\usepackage{amsmath}
\usepackage{amsthm}
\usepackage{changepage}
\usepackage{color}
\usepackage{enumitem}
\usepackage{fancyhdr}
\usepackage{framed}
\usepackage[margin=1in]{geometry}
\usepackage{mathrsfs}
\usepackage{tikz, tkz-tab}
\usepackage{titling}

\newtheoremstyle{dotless}{}{}{\itshape}{}{\bfseries}{}{ }{}
\theoremstyle{dotless}

\newtheorem{defs}{Definition}[subsection]
\newenvironment{defi}{\definecolor{shadecolor}{RGB}{255,236,217}\begin{shaded}\begin{defs}\ \\}{\end{defs}\end{shaded}}

\newtheorem{pro}{Propriete}[subsection]
\newenvironment{prop}{\definecolor{shadecolor}{RGB}{230,230,255}\begin{shaded}\begin{pro}\ \\}{\end{pro}\end{shaded}}

\newtheorem{cor}{Corollaire}[subsection]
\newenvironment{coro}{\definecolor{shadecolor}{RGB}{245,250,255}\begin{shaded}\begin{cor}\ \\}{\end{cor}\end{shaded}}

\setlength{\droptitle}{-1in}
\predate{}
\postdate{}
\date{}
\title{\chapitre\\\titre\vspace{-.25in}}

\pagestyle{fancy}
\makeatletter
\lhead{\chapitre\ - \titre}
\rhead{\@author}
\makeatother

\newenvironment{preuve}{\begin{framed}\begin{proof}[\unskip\nopunct]}{\end{proof}\end{framed}}
\newenvironment{liste}{\begin{itemize}[leftmargin=*,noitemsep, topsep=0pt]}{\end{itemize}}
\newenvironment{tab}{\begin{adjustwidth}{.5cm}{}}{\end{adjustwidth}}

\newcommand{\uu}[1] {_{_{#1}}}
\newcommand{\lbracket}{[\![}
\newcommand{\rbracket}{]\!]}
\newcommand{\fonction}[5]{\begin{aligned}[t]#1\colon&#2&&\longrightarrow#3 \\&#4&&\longmapsto#5\end{aligned}}
\newcommand{\systeme}[1]{\left\{\begin{aligned}#1\end{aligned}\right.}
\newcommand{\cercle}[1]{\textcircled{\scriptsize{#1}}}

\newcommand{\lf}[1]{\left(#1\right)}
\newcommand{\C}{\mathbb{C}}
\newcommand{\R}{\mathbb{R}}
\newcommand{\K}{\mathbb{K}}
\newcommand{\N}{\mathbb{N}}
\newcommand{\I}{\mathcal{I}}
\newcommand{\F}{\mathcal{F}}
\newcommand{\E}{\mathcal{E}}
\newcommand{\G}{\mathcal{G}}
\newcommand{\et}{\text{ et }}
\newcommand{\ou}{\text{ ou }}
\newcommand{\xou}{\ \fbox{\text{ou}}\ }


%Auteur: Tomas Rigaux, MPSI 2

\begin{document}
	\maketitle\ \\
	Les fils de connection sont id\'eaux. \\
	\textbf{Noeud :} point o\`u se rencontrent au moins 3 fils. \\
	\textbf{Branche :} partition de circuit entre 2 noeuds cons\'ecutifs. \\
	\textbf{Maille :} ensemble de branches successives qui reviennent au point de d\'epart. \\
	2 branches sont en \textbf{s\'erie} si elles sont parcourures par le m\^eme courant. \\
	2 branches sont en \textbf{d\'erivation} si elles ont la m\^eme tension \`a leurs bornes. \\
	\section{Loi des noeuds}
		La somme des courants entrants dans un noeud = La somme des courants sortants.
	\section{Loi des mailles}
		\begin{center}
			\begin{tikzpicture}
				\draw (0,4)
					to [generic,v=$U_{AB}$] (6,4)
					to [generic,v=$U_{BC}$] (6,0)
					to [generic,v=$U_{CD}$] (0,0)
					to [generic,v=$U_{DA}$] (0,4);
				\draw (0,4) node {$\bullet$} node [above left] {A};
				\draw (6,4) node {$\bullet$} node [above right] {B};
				\draw (6,0) node {$\bullet$} node [below right] {C};
				\draw (0,0) node {$\bullet$} node [below left] {D};
				\node[anchor=west, right] at (7,2)
					{$\underbrace{V_A-V_B}_{V_{AB}}+\underbrace{V_B-V_C}_{V_{BC}}+\underbrace{V_C-V_D}_{V_{CD}}+\underbrace{V_D-V_A}_{V_{DA}}=0$};
			\end{tikzpicture}
		\end{center}
\end{document}
