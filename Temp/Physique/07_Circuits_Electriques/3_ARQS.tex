\documentclass[12pt,twoside,a4paper]{article}

\def\chapitre{Circuits Electriques dans l'ARQS}
\author{MPSI 2}
\def\titre{Approximation des r\'egimes quasi-permanents}

\usepackage{amsfonts}
\usepackage{amsmath}
\usepackage{amsthm}
\usepackage{changepage}
\usepackage{color}
\usepackage{enumitem}
\usepackage{fancyhdr}
\usepackage{framed}
\usepackage[margin=1in]{geometry}
\usepackage{mathrsfs}
\usepackage{tikz, tkz-tab}
\usepackage{titling}

\newtheoremstyle{dotless}{}{}{\itshape}{}{\bfseries}{}{ }{}
\theoremstyle{dotless}

\newtheorem{defs}{Definition}[subsection]
\newenvironment{defi}{\definecolor{shadecolor}{RGB}{255,236,217}\begin{shaded}\begin{defs}\ \\}{\end{defs}\end{shaded}}

\newtheorem{pro}{Propriete}[subsection]
\newenvironment{prop}{\definecolor{shadecolor}{RGB}{230,230,255}\begin{shaded}\begin{pro}\ \\}{\end{pro}\end{shaded}}

\newtheorem{cor}{Corollaire}[subsection]
\newenvironment{coro}{\definecolor{shadecolor}{RGB}{245,250,255}\begin{shaded}\begin{cor}\ \\}{\end{cor}\end{shaded}}

\setlength{\droptitle}{-1in}
\predate{}
\postdate{}
\date{}
\title{\chapitre\\\titre\vspace{-.25in}}

\pagestyle{fancy}
\makeatletter
\lhead{\chapitre\ - \titre}
\rhead{\@author}
\makeatother

\newenvironment{preuve}{\begin{framed}\begin{proof}[\unskip\nopunct]}{\end{proof}\end{framed}}
\newenvironment{liste}{\begin{itemize}[leftmargin=*,noitemsep, topsep=0pt]}{\end{itemize}}
\newenvironment{tab}{\begin{adjustwidth}{.5cm}{}}{\end{adjustwidth}}

\newcommand{\uu}[1] {_{_{#1}}}
\newcommand{\lbracket}{[\![}
\newcommand{\rbracket}{]\!]}
\newcommand{\fonction}[5]{\begin{aligned}[t]#1\colon&#2&&\longrightarrow#3 \\&#4&&\longmapsto#5\end{aligned}}
\newcommand{\systeme}[1]{\left\{\begin{aligned}#1\end{aligned}\right.}
\newcommand{\cercle}[1]{\textcircled{\scriptsize{#1}}}

%Auteur: Tomas Rigaux, MPSI 2

\begin{document}
	\maketitle\ \\
	Un circuit doit \^etre ferm\'e pour qu'un courant se cr\'ee. \\
	Si le courant au point $A$ charge, le point $B$ se charge avec un retard : $\tau=\underbrace{\frac{AB}{c}}_{\text{vitesse de la lumi\`ere}}$.
	\begin{center}
		\begin{tikzpicture}
			\draw [very thin,gray!10] (0,0) grid[step=0.1] (7,3);
			\draw [very thin,black!10] (0,0) grid[step=0.5] (7,3);
			\draw [very thin,black!20] (0,0) grid[step=1] (7,3);
			\draw [->,>=stealth] (0,0) -- (7,0);
			\draw [->,>=stealth] (0,0) -- (0,3);
			\draw (1,0.1) -- (1,-0.1) node [below] {{\scriptsize $\frac{T}{2}$}};
			\draw (2,0.1) -- (2,-0.1) node [below] {{\scriptsize $T$}};
			\draw [ultra thick,red] (0,2.5)
				-- (1,2.5)
				-- (1,0.5)
				-- (2,0.5)
				-- (2,2.5)
				-- (3,2.5)
				-- (3,0.5)
				-- (4,0.5)
				-- (4,2.5)
				-- (5,2.5)
				-- (5,0.5)
				-- (6,0.5)
				-- (6,2.5)
				-- (7,2.5);
			\node[anchor=east, left] at (-1,1.5)
				{Au point $A$ :};
		\end{tikzpicture}
	\end{center}
	Si $\tau\ll\frac{T}{2}\implies i_B(t)=i_A(t)$ \\
	Si $\tau<\frac{T}{2}\implies i_B(t)\neq i_A(t)$ \\
	Se placer dans l'ARQS signifie n\'egliger le ph\'enom\`ne de propagation. \\
	En TP, on travaille donc avec des fr\'equences inf\'erieures \`a $300MHz$.
\end{document}
