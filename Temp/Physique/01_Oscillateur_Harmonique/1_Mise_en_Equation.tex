\documentclass[12pt,twoside,a4paper]{article}

\def\chapitre{Oscillateur Harmonique}
\author{MPSI 2}
\def\titre{Mise en \'equation}

\usepackage{amsfonts}
\usepackage{amsmath}
\usepackage{amsthm}
\usepackage{changepage}
\usepackage{color}
\usepackage{enumitem}
\usepackage{fancyhdr}
\usepackage{framed}
\usepackage[margin=1in]{geometry}
\usepackage{mathrsfs}
\usepackage{tikz, tkz-tab}
\usepackage{titling}

\newtheoremstyle{dotless}{}{}{\itshape}{}{\bfseries}{}{ }{}
\theoremstyle{dotless}

\newtheorem{defs}{Definition}[subsection]
\newenvironment{defi}{\definecolor{shadecolor}{RGB}{255,236,217}\begin{shaded}\begin{defs}\ \\}{\end{defs}\end{shaded}}

\newtheorem{pro}{Propriete}[subsection]
\newenvironment{prop}{\definecolor{shadecolor}{RGB}{230,230,255}\begin{shaded}\begin{pro}\ \\}{\end{pro}\end{shaded}}

\newtheorem{cor}{Corollaire}[subsection]
\newenvironment{coro}{\definecolor{shadecolor}{RGB}{245,250,255}\begin{shaded}\begin{cor}\ \\}{\end{cor}\end{shaded}}

\setlength{\droptitle}{-1in}
\predate{}
\postdate{}
\date{}
\title{\chapitre\\\titre\vspace{-.25in}}

\pagestyle{fancy}
\makeatletter
\lhead{\chapitre\ - \titre}
\rhead{\@author}
\makeatother

\newenvironment{preuve}{\begin{framed}\begin{proof}[\unskip\nopunct]}{\end{proof}\end{framed}}
\newenvironment{liste}{\begin{itemize}[leftmargin=*,noitemsep, topsep=0pt]}{\end{itemize}}
\newenvironment{tab}{\begin{adjustwidth}{.5cm}{}}{\end{adjustwidth}}

\newcommand{\uu}[1] {_{_{#1}}}
\newcommand{\lbracket}{[\![}
\newcommand{\rbracket}{]\!]}
\newcommand{\fonction}[5]{\begin{aligned}[t]#1\colon&#2&&\longrightarrow#3 \\&#4&&\longmapsto#5\end{aligned}}
\newcommand{\systeme}[1]{\left\{\begin{aligned}#1\end{aligned}\right.}
\newcommand{\cercle}[1]{\textcircled{\scriptsize{#1}}}

\newcommand{\lf}[1]{\left(#1\right)}
\newcommand{\C}{\mathbb{C}}
\newcommand{\R}{\mathbb{R}}
\newcommand{\K}{\mathbb{K}}
\newcommand{\N}{\mathbb{N}}
\newcommand{\I}{\mathcal{I}}
\newcommand{\F}{\mathcal{F}}
\newcommand{\E}{\mathcal{E}}
\newcommand{\G}{\mathcal{G}}
\newcommand{\et}{\text{ et }}
\newcommand{\ou}{\text{ ou }}
\newcommand{\xou}{\ \fbox{\text{ou}}\ }


%Auteur: Tomas Rigaux, MPSI 2

\begin{document}
	\maketitle\ \\
	\section{Inventaire des forces}
		Dans $R_T$ galil\'een	, le solide $M$ est soumis \`a :
		\begin{liste}
			\item Son poids : $\vec{P}=m\vec{g}$, $||\vec{g}||=9.81 N.kg^{-1}$.
			\item La r\'eaction du support (axe $(Ox)$) sur la masse : $\vec{R}$, perpendiculaire au suport car les frottements solides sont n\'eglig\'es.
			\item La force de rappel exerc\'ee par le ressort $\vec{T}$.
		\end{liste}
		On a vu en TP que $||\vec{T}||$ est proportionelle \`a l'alllongement du ressort $\Delta l$ et da direction est colin\'eaire au ressort :
		$$\vec{T}=\pm k(l-l_0)\vec{e}_x$$
		\textbf{Remarques :}
		\begin{liste}
			\item La formule est la m\^eme si le ressort est \'etir\'e ou comprim\'e.
			\item \textbf{Unit\'e de k :} $||\vec{T}||=k(l-l_0)\implies k$ est en $N.m^{-1}$.
			\item On n\'eglige les frottements fluides.
		\end{liste}
	\section{Equations du mouvement}
		On applique la seconde loi de Newton :
		$$\begin{aligned}
			\frac{d\vec{P}}{dt}=\Sigma\vec{F}&\text{ avec }\vec{P}=m\vec{v} \\
											 &\text{ ici, }m=constante \\
											 &\text{ donc }\frac{d\vec{P}}{dt}=m\vec{a}
		\end{aligned}$$
		D'ou $m\vec{a}=\vec{P}+\vec{R}+\vec{T}\ (E)$ \\
		Le mouvement se fait sur l'axe $(Ox)$ donc on projette $(E)$ sur $Ox$ :
	$$\begin{aligned}
		m\ddot x&=0+0-k(l-l_0) \\
		\ddot x+\frac{k}{m}x&=\frac{k}{m}l_0
	\end{aligned}$$
	On pose $\omega_0^2=\frac{k}{m}$ de telle sorte que : \\
	$(E)\iff\fbox{$\ddot x+\omega_0^2x=\omega_0^2l_0$}$ : \'equation de l'oscillateur harmonique.
\end{document}
