\documentclass[12pt,twoside,a4paper]{article}

\def\chapitre{Circuits Electriques dans l'ARQS}
\author{MPSI 2}
\def\titre{Courant \'electrique}

\usepackage{amsfonts}
\usepackage{amsmath}
\usepackage{amsthm}
\usepackage{changepage}
\usepackage{color}
\usepackage{enumitem}
\usepackage{fancyhdr}
\usepackage{framed}
\usepackage[margin=1in]{geometry}
\usepackage{mathrsfs}
\usepackage{tikz, tkz-tab}
\usepackage{titling}

\newtheoremstyle{dotless}{}{}{\itshape}{}{\bfseries}{}{ }{}
\theoremstyle{dotless}

\newtheorem{defs}{Definition}[subsection]
\newenvironment{defi}{\definecolor{shadecolor}{RGB}{255,236,217}\begin{shaded}\begin{defs}\ \\}{\end{defs}\end{shaded}}

\newtheorem{pro}{Propriete}[subsection]
\newenvironment{prop}{\definecolor{shadecolor}{RGB}{230,230,255}\begin{shaded}\begin{pro}\ \\}{\end{pro}\end{shaded}}

\newtheorem{cor}{Corollaire}[subsection]
\newenvironment{coro}{\definecolor{shadecolor}{RGB}{245,250,255}\begin{shaded}\begin{cor}\ \\}{\end{cor}\end{shaded}}

\setlength{\droptitle}{-1in}
\predate{}
\postdate{}
\date{}
\title{\chapitre\\\titre\vspace{-.25in}}

\pagestyle{fancy}
\makeatletter
\lhead{\chapitre\ - \titre}
\rhead{\@author}
\makeatother

\newenvironment{preuve}{\begin{framed}\begin{proof}[\unskip\nopunct]}{\end{proof}\end{framed}}
\newenvironment{liste}{\begin{itemize}[leftmargin=*,noitemsep, topsep=0pt]}{\end{itemize}}
\newenvironment{tab}{\begin{adjustwidth}{.5cm}{}}{\end{adjustwidth}}

\newcommand{\uu}[1] {_{_{#1}}}
\newcommand{\lbracket}{[\![}
\newcommand{\rbracket}{]\!]}
\newcommand{\fonction}[5]{\begin{aligned}[t]#1\colon&#2&&\longrightarrow#3 \\&#4&&\longmapsto#5\end{aligned}}
\newcommand{\systeme}[1]{\left\{\begin{aligned}#1\end{aligned}\right.}
\newcommand{\cercle}[1]{\textcircled{\scriptsize{#1}}}

\newcommand{\lf}[1]{\left(#1\right)}
\newcommand{\C}{\mathbb{C}}
\newcommand{\R}{\mathbb{R}}
\newcommand{\K}{\mathbb{K}}
\newcommand{\N}{\mathbb{N}}
\newcommand{\I}{\mathcal{I}}
\newcommand{\F}{\mathcal{F}}
\newcommand{\E}{\mathcal{E}}
\newcommand{\G}{\mathcal{G}}
\newcommand{\et}{\text{ et }}
\newcommand{\ou}{\text{ ou }}
\newcommand{\xou}{\ \fbox{\text{ou}}\ }


%Auteur: Tomas Rigaux, MPSI 2

\begin{document}
	\maketitle
	\section{Charge \'electrique}
		C'est une propri\'et\'e intrinseque d'un corps (comme la masse). \\
		La charge est quantifi\'ee ($n*$charge \'el\'ementaire). \\
		\textbf{Notation :} $q$ ou $Q$. \\
		\textbf{Unit\'e :} Le Coulomb ($C$)
	\section{Le courant \'electrique}
		Il est du \`a un d\'eplacement de charges \'electriques :
		\begin{liste}
			\item \'electrons dans les conducteurs m\'etalliques.
			\item ions (anions ou cations) dans les solutions.
		\end{liste}
		Par convention, le sens positif du courant correspond au sens de d\'eplacement des charges positives. \\
		On travaille avec des intensit\'es alg\'ebriques.
	\section{Intensit\'e du courant}
		On se place en un point et on compte les charges passant par cette section en une seconde. \\
		Intensit\'e moyenne : $I=\frac{Q}{\Delta t}$ \\
		Intensit\'e instantan\'ee : $i=\frac{dq}{dt}$ \\
		$\frac{dq}{dt}$ d\'esigne \`a la fois la d\'eriv\'ee de $q$ par rapport \`a $t$ et le rapport de deux quantit\'es tr\`es petites.
		$$i=\frac{dq}{dt}=\dot{q}(t)$$
		\textbf{Unit\'e :} L'Amp\`ere ($A$) \\
		\textbf{Ordres de grandeurs :}
		\begin{liste}
			\item quelques $mA$ sur les circuits de TP.
			\item eclairs : $10kA$.
			\item corps humain :
				\begin{liste}
					\item $1mA$ : picotements
					\item $10mA$ : contraction des muscles
					\item $30mA$ : paralysie respiratoire
					\item $75mA$ : fibrillations cardiaques
					\item $1A$ : Arret du coeur.
				\end{liste}
		\end{liste}
		\textbf{Mesure :} avec un amp\`erem\`etre en s\'erie avec une faible r\'esistance interne.
\end{document}
