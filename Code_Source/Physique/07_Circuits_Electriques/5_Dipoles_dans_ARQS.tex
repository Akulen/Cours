\documentclass[12pt,twoside,a4paper]{article}

\def\chapitre{Circuits Electriques dans l'ARQS}
\author{MPSI 2}
\def\titre{Dipoles dans l'ARQS}

\usepackage{amsfonts}
\usepackage{amsmath}
\usepackage{amsthm}
\usepackage{changepage}
\usepackage{color}
\usepackage{enumitem}
\usepackage{fancyhdr}
\usepackage{framed}
\usepackage[margin=1in]{geometry}
\usepackage{mathrsfs}
\usepackage{tikz, tkz-tab}
\usepackage{titling}

\newtheoremstyle{dotless}{}{}{\itshape}{}{\bfseries}{}{ }{}
\theoremstyle{dotless}

\newtheorem{defs}{Definition}[subsection]
\newenvironment{defi}{\definecolor{shadecolor}{RGB}{255,236,217}\begin{shaded}\begin{defs}\ \\}{\end{defs}\end{shaded}}

\newtheorem{pro}{Propriete}[subsection]
\newenvironment{prop}{\definecolor{shadecolor}{RGB}{230,230,255}\begin{shaded}\begin{pro}\ \\}{\end{pro}\end{shaded}}

\newtheorem{cor}{Corollaire}[subsection]
\newenvironment{coro}{\definecolor{shadecolor}{RGB}{245,250,255}\begin{shaded}\begin{cor}\ \\}{\end{cor}\end{shaded}}

\setlength{\droptitle}{-1in}
\predate{}
\postdate{}
\date{}
\title{\chapitre\\\titre\vspace{-.25in}}

\pagestyle{fancy}
\makeatletter
\lhead{\chapitre\ - \titre}
\rhead{\@author}
\makeatother

\newenvironment{preuve}{\begin{framed}\begin{proof}[\unskip\nopunct]}{\end{proof}\end{framed}}
\newenvironment{liste}{\begin{itemize}[leftmargin=*,noitemsep, topsep=0pt]}{\end{itemize}}
\newenvironment{tab}{\begin{adjustwidth}{.5cm}{}}{\end{adjustwidth}}

\newcommand{\uu}[1] {_{_{#1}}}
\newcommand{\lbracket}{[\![}
\newcommand{\rbracket}{]\!]}
\newcommand{\fonction}[5]{\begin{aligned}[t]#1\colon&#2&&\longrightarrow#3 \\&#4&&\longmapsto#5\end{aligned}}
\newcommand{\systeme}[1]{\left\{\begin{aligned}#1\end{aligned}\right.}
\newcommand{\cercle}[1]{\textcircled{\scriptsize{#1}}}

%Auteur: Tomas Rigaux, MPSI 2

\begin{document}
	\maketitle\ \\
	\begin{defi}
		Un dipole est un composant qui poss\`ede 2 bornes.
	\end{defi}
	\section{Conventions}
		\begin{center}
			\begin{tikzpicture}
				\draw (0,0)
					to [short,i_>=$i_{AB}$] (1,0)
					to [generic,l=$D$] (3,0)
					to [short,i_<=$i_{BA}$] (4,0);
				\draw (0,0) node {$\bullet$} node [above] {A};
				\draw (4,0) node {$\bullet$} node [above] {B};
				\node[anchor=west, right] at (5,0)
					{$i_{AB}=-i_{BA}$};
			\end{tikzpicture}
		\end{center}
		$i_{AB}$ : courant de $A$ vers $B$. \\
		\textbf{Convention recepteur :} \\ \ \\
		\begin{tikzpicture}
			\draw (0,0)
				to [short,i_>=$i_{AB}$] (1,0)
				to [generic,l_=$D$,v^=$U_{AB}{=}V_A-V_B$] (3,0)
				to [short] (4,0);
			\draw (0,0) node {$\bullet$} node [above] {A};
			\draw (4,0) node {$\bullet$} node [above] {B};
		\end{tikzpicture}
		\textbf{Convention g\'en\'erateur :} \\ \ \\
		\begin{tikzpicture}
			\draw (0,0)
				to [short] (1,0)
				to [generic,l_=$D$,v^>=$U_{BA}{=}V_B-V_A$] (3,0)
				to [short,i_>=$i_{AB}$] (4,0);
			\draw (0,0) node {$\bullet$} node [above] {A};
			\draw (4,0) node {$\bullet$} node [above] {B};
		\end{tikzpicture}
	\section{Puissance recue par un dip\^ole $D$}
\end{document}
