\documentclass[12pt,twoside,a4paper]{article}

\def\chapitre{Complexes}
\author{MPSI 2}
\def\titre{Applications Geometriques des Complexes}

\usepackage{amsfonts}
\usepackage{amsmath}
\usepackage{amsthm}
\usepackage{changepage}
\usepackage{color}
\usepackage{enumitem}
\usepackage{fancyhdr}
\usepackage{framed}
\usepackage[margin=1in]{geometry}
\usepackage{mathrsfs}
\usepackage{tikz, tkz-tab}
\usepackage{titling}

\newtheoremstyle{dotless}{}{}{\itshape}{}{\bfseries}{}{ }{}
\theoremstyle{dotless}

\newtheorem{defs}{Definition}[subsection]
\newenvironment{defi}{\definecolor{shadecolor}{RGB}{255,236,217}\begin{shaded}\begin{defs}\ \\}{\end{defs}\end{shaded}}

\newtheorem{pro}{Propriete}[subsection]
\newenvironment{prop}{\definecolor{shadecolor}{RGB}{230,230,255}\begin{shaded}\begin{pro}\ \\}{\end{pro}\end{shaded}}

\newtheorem{cor}{Corollaire}[subsection]
\newenvironment{coro}{\definecolor{shadecolor}{RGB}{245,250,255}\begin{shaded}\begin{cor}\ \\}{\end{cor}\end{shaded}}

\setlength{\droptitle}{-1in}
\predate{}
\postdate{}
\date{}
\title{\chapitre\\\titre\vspace{-.25in}}

\pagestyle{fancy}
\makeatletter
\lhead{\chapitre\ - \titre}
\rhead{\@author}
\makeatother

\newenvironment{preuve}{\begin{framed}\begin{proof}[\unskip\nopunct]}{\end{proof}\end{framed}}
\newenvironment{liste}{\begin{itemize}[leftmargin=*,noitemsep, topsep=0pt]}{\end{itemize}}
\newenvironment{tab}{\begin{adjustwidth}{.5cm}{}}{\end{adjustwidth}}

\newcommand{\uu}[1] {_{_{#1}}}
\newcommand{\lbracket}{[\![}
\newcommand{\rbracket}{]\!]}
\newcommand{\fonction}[5]{\begin{aligned}[t]#1\colon&#2&&\longrightarrow#3 \\&#4&&\longmapsto#5\end{aligned}}
\newcommand{\systeme}[1]{\left\{\begin{aligned}#1\end{aligned}\right.}
\newcommand{\cercle}[1]{\textcircled{\scriptsize{#1}}}

\newcommand{\lf}[1]{\left(#1\right)}
\newcommand{\C}{\mathbb{C}}
\newcommand{\R}{\mathbb{R}}
\newcommand{\K}{\mathbb{K}}
\newcommand{\N}{\mathbb{N}}
\newcommand{\I}{\mathcal{I}}
\newcommand{\F}{\mathcal{F}}
\newcommand{\E}{\mathcal{E}}
\newcommand{\G}{\mathcal{G}}
\newcommand{\et}{\text{ et }}
\newcommand{\ou}{\text{ ou }}
\newcommand{\xou}{\ \fbox{\text{ou}}\ }


\begin{document}
	\maketitle

	\section{Modules et Arguments}
		Soit $A$, $B$ et $M$ trois points du plan deux a deux distincts d'affixe respectives $a$, $b$ et $z$.
		\begin{liste}
			\item $\left|\frac{z-a}{z-b}\right|=\frac{\left\|\vec{AM}\right\|}{\left\|\vec{BM}\right\|}=\frac{AM}{BM}$
			\item Soit $r$ un reel strictement positif :
				$$\begin{aligned}
					&\left\{z\in\mathbb{C},\left|z-a\right|=r\right\}\ \ \ \text{est le cercle de centre }A\text{ et de rayon }r \\
					&\left\{z\in\mathbb{C},\left|z-a\right|<r\right\}\ \ \ \text{est le disque ouvert de centre }A\text{ et de rayon }r
				\end{aligned}$$
			\item $\begin{aligned}[t]
					\arg\left(\frac{z-a}{z-b}\right)&\equiv\arg\left(z-a\right)-\arg\left(z-b\right)\ \left[2\pi\right] \\
													 &\equiv mes\left(\vec{i}, \vec{AM}\right)-mes\left(\vec{i}, \vec{BM}\right)\ \left[2\pi\right] \\
													 &\equiv mes\left(\vec{BM}, \vec{AM}\right)\ \left[2\pi\right] \\
													 &\equiv mes\left(\vec{MB}, \vec{MA}\right)\ \left[2\pi\right]
				\end{aligned}$
		\end{liste}
	
	\section{Interpretation geometrique des operations dans $\mathbb{C}$}
		\begin{liste}
			\item $z\longmapsto z+b$\\
				$\begin{aligned}\text{Notons }&M&&\text{ l'image affine de }z \\
											   &M'&&\text{ l'image affine de }z' \\
											   &B&&\text{ l'image affine de }b\end{aligned}$
				$$\begin{aligned}
					z'=z+b&\iff z'-z=b \\
						  &\iff\vec{MM'}=\vec{OB}
				\end{aligned}$$
				Translation par le vecteur $\vec{OB}$.
			\item $z\longmapsto az$ avec $a\in\mathbb{C}$\\
				\textbf{Cas 1 :} $a\in\mathbb{R}^*$
				\begin{tab}
					$$
						z'=az\iff\vec{OM'}=a\vec{OM}
					$$
					$M'$ est l'image de $M$ par l'homothetie de centre $\mathcal{O}$ et de rapport $\left|a\right|$.
				\end{tab}
				\textbf{Cas 2 :} $\left|a\right|=1\ \ \ \exists\theta\in\mathbb{R},a=e^{\imath\theta}$
				\begin{tab}
					$$\begin{aligned}
						z'=az&\iff\left\{\begin{aligned}&\left|z'\right|=\left|z\right| \\
														 &\arg\left(z'\right)\equiv\arg\left(a\right)+\arg\left(z\right)\ \ \ \left[2\pi\right]\end{aligned}\right. \\
							 &\iff\left\{\begin{aligned}&OM'=OM \\
														 &mes\left(\vec{OM}, \vec{OM'}\right)&\equiv\theta\ \ \ \left[2\pi\right]\end{aligned}\right. \\
					\end{aligned}$$
					$M'$ est l'image de $M$ par la rotation de centre $\mathcal{O}$ et d'angle de mesure $\theta$.
				\end{tab}
				\textbf{Cas 3 :} $a\in\mathbb{C}^*$
				\begin{tab}
					$M'$ est l'image de $M$ par la similitude de centre $\mathcal{O}$, de rapport $\left|a\right|$ et d'angle de mesure $\theta$.
				\end{tab}
		\end{liste}
		\textbf{Retour sur l'equation $z^n=a$}
		\begin{tab}
			$$\begin{aligned}
				&z^n=a\iff\exists k\in\lbracket0, n-1\rbracket,z=\rho\uu0^\frac{1}{n}e^{\imath\frac{\alpha}{n}}\omega\uu{k} \\
				&z\longmapsto z\uu0\times z
			\end{aligned}$$
			Les images affines des solutions de $z^n=a$ sont les images par la similitude de centre $\mathcal{O}$, de rapport $\rho\uu0^\frac{1}{n}$ et d'angle de mesure $\frac{\alpha}{n}$ des images affines des racines$^n$ de l'unite.
		\end{tab}
		\textbf{Plus generalement}
		\begin{liste}
			\item L'homothetie de centre $\mathcal{A}$ et de rapport $\lambda\in\mathbb{R}^*$ est la transformation du plan affine qui a un point $M$ associe le point $M'$ tel que $\vec{AM'}=\lambda\vec{AM}$. \\
				Expression analytique de l'homothetie :
				$$\begin{aligned}
					\vec{AM'}=\lambda\vec{AM}&\iff\left\{\begin{aligned}x'-x\uu0&=\lambda\left(x-x\uu0\right) \\
																		y'-y\uu0&=\lambda\left(y-y\uu0\right)\end{aligned}\right.\\
											 &\iff z'-a=\lambda\left(z-a\right)\\
											 &\iff z'=\lambda\left(z-a\right)+a
				\end{aligned}$$
				\item La rotation de centre $\mathcal{A}$ et d'angle de mesure $\theta$ est la transformation du plan affine qui a un point $M$ associe le point $M'$ : \\
				$$
					\left\{\begin{aligned}&AM=AM'\\
										  &mes\left(\vec{AM}, \vec{AM'}\right)\equiv\theta\ \ \ \left[2\pi\right]\end{aligned}\right.\iff\left\{\begin{aligned}&\left|z-a\right|=\left|z'-a\right| \\
									  																																			  &\arg\left(z'-a\right)\equiv\arg\left(z-a\right)+\theta\ \ \ \left[2\pi\right]\end{aligned}\right.
				$$
			\item $z'=az+b\ \ \ \left(a, b\right)\in\mathbb{C}^2,b\neq0$ \\
				\textbf{Recherche de points fixes :} $z\uu0=az+b$
				$$
					z=az+b\iff\left(a-1\right)z=-b
				$$
				\textbf{Cas 1 :} $a=1\Rightarrow0=-b$
				\begin{tab}
					On a $b\neq0$, donc il n'y a pas de points fixes.
				\end{tab}
				\textbf{Cas 2 :} $a\neq1$
				\begin{tab}
					$$\begin{aligned}
						z=az+b\iff z&=\frac{-b}{a-1} \\
								   &=\frac{b}{1-a}
					\end{aligned}$$
					Il existe un unique point fixe $x\uu0=\frac{b}{1-a}$.
				\end{tab}
				$$
					\left\{\begin{aligned}z'&=az+b \\
										z\uu0&=az\uu0+b\end{aligned}\right.\iff z'-z\uu0=a\left(z-z\uu0\right)
				$$
				$\Rightarrow$ Similitude de centre $z\uu0$, de rapport $\left|a\right|$ et d'angle de mesure $\arg\left(a\right)$.
		\end{liste}
\end{document}
