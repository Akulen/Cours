\documentclass[12pt,twoside,a4paper]{article}

\def\chapitre{Fonctions Usuelles}
\author{MPSI 2}
\def\titre{Fonctions Puissance}

\usepackage{amsfonts}
\usepackage{amsmath}
\usepackage{amsthm}
\usepackage{changepage}
\usepackage{color}
\usepackage{enumitem}
\usepackage{fancyhdr}
\usepackage{framed}
\usepackage[margin=1in]{geometry}
\usepackage{mathrsfs}
\usepackage{tikz, tkz-tab}
\usepackage{titling}

\newtheoremstyle{dotless}{}{}{\itshape}{}{\bfseries}{}{ }{}
\theoremstyle{dotless}

\newtheorem{defs}{Definition}[subsection]
\newenvironment{defi}{\definecolor{shadecolor}{RGB}{255,236,217}\begin{shaded}\begin{defs}\ \\}{\end{defs}\end{shaded}}

\newtheorem{pro}{Propriete}[subsection]
\newenvironment{prop}{\definecolor{shadecolor}{RGB}{230,230,255}\begin{shaded}\begin{pro}\ \\}{\end{pro}\end{shaded}}

\newtheorem{cor}{Corollaire}[subsection]
\newenvironment{coro}{\definecolor{shadecolor}{RGB}{245,250,255}\begin{shaded}\begin{cor}\ \\}{\end{cor}\end{shaded}}

\setlength{\droptitle}{-1in}
\predate{}
\postdate{}
\date{}
\title{\chapitre\\\titre\vspace{-.25in}}

\pagestyle{fancy}
\makeatletter
\lhead{\chapitre\ - \titre}
\rhead{\@author}
\makeatother

\newenvironment{preuve}{\begin{framed}\begin{proof}[\unskip\nopunct]}{\end{proof}\end{framed}}
\newenvironment{liste}{\begin{itemize}[leftmargin=*,noitemsep, topsep=0pt]}{\end{itemize}}
\newenvironment{tab}{\begin{adjustwidth}{.5cm}{}}{\end{adjustwidth}}

\newcommand{\uu}[1] {_{_{#1}}}
\newcommand{\lbracket}{[\![}
\newcommand{\rbracket}{]\!]}
\newcommand{\fonction}[5]{\begin{aligned}[t]#1\colon&#2&&\longrightarrow#3 \\&#4&&\longmapsto#5\end{aligned}}
\newcommand{\systeme}[1]{\left\{\begin{aligned}#1\end{aligned}\right.}
\newcommand{\cercle}[1]{\textcircled{\scriptsize{#1}}}

\begin{document}
	\maketitle
	\section{D\'efinition}
		\subsection{D\'efinition}
			\begin{defi}
				Soit $\alpha\in\mathbb{R},$\\
				$\fonction{f_\alpha}{\mathbb{R}^{+*}}{\mathbb{R}}{x}{x^\alpha=exp(\alpha\ ln(x))}$\\
				$f_\alpha$ s'appelle la fonction puissance $\alpha$
			\end{defi}
			\begin{prop}
				Pour tout $(x,y)$ strictement positifs, pour tout $(\alpha,\beta)$ r\'eels:\\
				\begin{liste}
					\item $x^{\alpha+\beta}=x^\alpha\ x^\beta$
					\item $\left(x^\alpha\right)^\beta=x^{\alpha\beta}$
					\item $(xy)^\alpha=x^\alpha y^\alpha$
					\item $ln\left(x^\alpha\right)=\alpha ln(x)$
				\end{liste}
			\end{prop}
		\subsection{\'Etude de $f_\alpha$}
			\textbf{D\'erivabilit\'e} $f$ est d\'erivable par compos\'ee de fonctions d\'erivables
				\begin{tab}
					$\begin{aligned}\forall x\in\mathbb{R}^{+*},\ f'_\alpha(x)&=\frac{\alpha}{x}exp(\alpha ln(x))\\
						&=\alpha exp((\alpha-1)ln(x))\end{aligned}$\\\\\\
						Donc $\forall x\in\mathbb{R}^{+*},\ f'_\alpha(x)=\alpha x^{\alpha-1}$
				\end{tab}\ \\
			\textbf{Etude des limites}
				\begin{tab}
					$\forall x\in\mathbb{R}^{+*},\ f_\alpha(x)=exp(\alpha\ ln(x))$
					\begin{liste}
						\item Si $\alpha<0$, alors $\lim\limits_{x\rightarrow+\infty}f_\alpha(x)=0$
						\item Si $\alpha>0$, alors $\lim\limits_{x\rightarrow+\infty}f_\alpha(x)=+\infty$\\
							\textbf{De quelle mani\`ere $f_\alpha$ tend-t-elle vers $+\infty?$}\\
								On \'etudie $\lim\limits_{x\rightarrow+\infty}\frac{f_\alpha(x)}{x}$\\
								Or $\frac{f_\alpha(x)}{x}=exp((\alpha-1)ln(x))$
								\begin{liste}
									\item Si $0<\alpha<1$ alors $\lim\limits_{x\rightarrow+\infty}\frac{f_\alpha(x)}{x}=0$,\\
											Donc $\mathscr{C}_{f_\alpha}$ admet une \textbf{branche parabolique de direction asymptotique} $Ox$
									\item Si $1<\alpha$ alors $\lim\limits_{x\rightarrow+\infty}\frac{f_\alpha(x)}{x}=\infty$,\\
										Donc $\mathscr{C}_{f_\alpha}$ admet une \textbf{branche parabolique de direction asymptotique} $Oy$
								\end{liste}
					\end{liste}
				\end{tab}\pagebreak
			\textbf{Etude en 0}
				\begin{liste}
					\item Si $\alpha<0$, $\lim\limits_{x\rightarrow0}f_\alpha(x)=+\infty$
					\item Si $\alpha>0$, $\lim\limits_{x\rightarrow0}f_\alpha(x)=0$\\
							Dans ce cas, notons $\tilde{f}_\alpha$ l'application d\'efinie sur $\mathbb{R}$ par:\\
							$\tilde{f}_\alpha(x)=\systeme{&f_\alpha(x) &\text{si } x>0\\
							&0 &\text{si } x=0}$\\
							D'apr\`es le taux d'accroissement:\begin{liste}
								\item Si $\alpha>1$, $\tilde{f}'_\alpha(0)=0$, donc $\mathscr{C}_{\tilde{f}_\alpha}$ admet une demi-tangente d'\'equation $y=0$
								\item Si $0<\alpha<1$, $\lim\limits_{x\rightarrow0}exp((\alpha-1)ln(x))=+\infty$, donc $\mathscr{C}_{\tilde{f}_\alpha}$ admet une demi-tangente d'\'equation $x=0$
							\end{liste}
				\end{liste}
	\section{Fonctions Puissance Vs. Fonctions racines$^n$}
		\begin{liste}
			\item Soit $n$ un entier \textbf{pair non nul}
					\begin{tab}
						$\fonction{g_n}{\mathbb{R}}{\mathbb{R}}{x}{x^n}$\\
						$g_n$ r\'ealise une bijection de $\mathbb{R}^+$ sur $\mathbb{R}^+$
					\end{tab}
					Son application r\'eciproque s'appelle la fonction \textbf{racine$^n$}
					$\fonction{}{\mathbb{R}^+}{\mathbb{R}^+}{x}{\sqrt[n]{x}}$
			\item Soit $n$ un entier\textbf{impair non nul}
					\begin{tab}
						$\fonction{g_n}{\mathbb{R}}{\mathbb{R}}{x}{x^n}$\\
						$g_n$ r\'ealise une bijection de $\mathbb{R}$ sur $\mathbb{R}$
					\end{tab}
					Son application r\'eciproque s'appelle la fonction \textbf{racine$^n$}
					$\fonction{}{\mathbb{R}}{\mathbb{R}}{x}{\sqrt[n]{x}}$
			\item Par ailleurs, pour tout $n$ entier naturel non nul, $f_\frac{1}{n}$ est l'application r\'eciproque de $\fonction{f_n}{\mathbb{R}^{+*}}{\mathbb{R}}{x}{\sqrt[n]{x}}$\\
			Donc par unicit\'e de la r\'eciproque on peut \'ecrire: $\forall x\in\mathbb{R}^{+*},\ \sqrt[n]{x}=x^\frac{1}{n}$\\
		\end{liste}
	\section{Th\'eor\`eme de comparaison}
		\begin{prop}
			Soient $\alpha$ et $\beta$ deux r\'eels strictement positifs.
			\begin{liste}
				\item $\lim\limits_{x\rightarrow+\infty}\frac{\left(ln(x)\right)^\beta}{x^\alpha}=0$
				\item $\lim\limits_{x\rightarrow+\infty}\frac{\left(exp(x)\right)^\alpha}{x^\beta}=+\infty$
				\item $\lim\limits_{x\rightarrow0^+}x^\alpha\left|ln(x)\right|^\beta=0$
				\item $\lim\limits_{x\rightarrow-\infty}|x|^\beta \left(exp(x)\right)^\alpha=0$
			\end{liste}
		\end{prop}
		\begin{preuve}
			\begin{liste}
				\item Pour $t\geq1$, on a: $\begin{aligned}[t]&0<\sqrt{t}\leq t\\&0<\frac{1}{t}\leq\frac{1}{\sqrt{t}}\end{aligned}$\\
						Donc pour $x\geq1$ on a: $\begin{aligned}[t]&0\leq\int\limits_1^x\frac{1}{t}\,dt\leq2\int\limits_1^x\frac{1}{2\sqrt{x}}\,dt\\
						\iff&0\leq ln(x)\leq2\sqrt{x}-2\\
						\iff&0\leq\frac{ln(x)}{x}\leq\frac{2\sqrt{x}-2}{x}\end{aligned}$\\\\
						Par cons\'equent, $\lim\limits_{x\rightarrow+\infty}\frac{ln(x)}{x}=0$
				\item Soient $\alpha$ et $\beta$ deux r\'eels strictement positifs. Pour $x>1$:\\
						$\begin{aligned}
							\frac{ln(x)^\beta}{x^\alpha}&=\left(\frac{ln(x)}{x^\frac{\alpha}{\beta}}\right)^\beta\\
							&=\left(\frac{\frac{\beta}{\alpha}\,ln\left(x^\frac{\alpha}{\beta}\right)}{x^\frac{\alpha}{\beta}}\right)^\beta\\
							&=\left(\frac{\beta}{\alpha}\right)^\beta\,\left(\frac{ln\left(x^\frac{\alpha}{\beta}\right)}{x^\frac{\alpha}{\beta}}\right)^\beta
						\end{aligned}$
						\begin{tab}
							de plus:
							\begin{liste}
								\item $\lim\limits_{x\rightarrow+\infty}x^\frac{\alpha}{\beta}=+\infty$
								\item $\lim\limits_{t\rightarrow+\infty}\frac{ln(t)}{t}=0$
								\item $\lim\limits_{u\rightarrow0}u^\beta=0$
							\end{liste}
							Donc par compos\'ees de limites, $\lim\limits_{x\rightarrow+\infty}\frac{\left(ln(x)\right)^\beta}{x^\alpha}=0$
						\end{tab}
				\item Passage a l'exponentielle et passage a l'inverse.
				\item etc.
			\end{liste}
		\end{preuve}
\end{document}
