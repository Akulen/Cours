\documentclass[12pt,twoside,a4paper]{article}

\def\chapitre{EDL}
\author{MPSI 2}
\def\titre{Equations Differentielles Lineaires du second ordre}

\usepackage{amsfonts}
\usepackage{amsmath}
\usepackage{amsthm}
\usepackage{changepage}
\usepackage{color}
\usepackage{enumitem}
\usepackage{fancyhdr}
\usepackage{framed}
\usepackage[margin=1in]{geometry}
\usepackage{mathrsfs}
\usepackage{tikz, tkz-tab}
\usepackage{titling}

\newtheoremstyle{dotless}{}{}{\itshape}{}{\bfseries}{}{ }{}
\theoremstyle{dotless}

\newtheorem{defs}{Definition}[subsection]
\newenvironment{defi}{\definecolor{shadecolor}{RGB}{255,236,217}\begin{shaded}\begin{defs}\ \\}{\end{defs}\end{shaded}}

\newtheorem{pro}{Propriete}[subsection]
\newenvironment{prop}{\definecolor{shadecolor}{RGB}{230,230,255}\begin{shaded}\begin{pro}\ \\}{\end{pro}\end{shaded}}

\newtheorem{cor}{Corollaire}[subsection]
\newenvironment{coro}{\definecolor{shadecolor}{RGB}{245,250,255}\begin{shaded}\begin{cor}\ \\}{\end{cor}\end{shaded}}

\setlength{\droptitle}{-1in}
\predate{}
\postdate{}
\date{}
\title{\chapitre\\\titre\vspace{-.25in}}

\pagestyle{fancy}
\makeatletter
\lhead{\chapitre\ - \titre}
\rhead{\@author}
\makeatother

\newenvironment{preuve}{\begin{framed}\begin{proof}[\unskip\nopunct]}{\end{proof}\end{framed}}
\newenvironment{liste}{\begin{itemize}[leftmargin=*,noitemsep, topsep=0pt]}{\end{itemize}}
\newenvironment{tab}{\begin{adjustwidth}{.5cm}{}}{\end{adjustwidth}}

\newcommand{\uu}[1] {_{_{#1}}}
\newcommand{\lbracket}{[\![}
\newcommand{\rbracket}{]\!]}
\newcommand{\fonction}[5]{\begin{aligned}[t]#1\colon&#2&&\longrightarrow#3 \\&#4&&\longmapsto#5\end{aligned}}
\newcommand{\systeme}[1]{\left\{\begin{aligned}#1\end{aligned}\right.}
\newcommand{\cercle}[1]{\textcircled{\scriptsize{#1}}}

\newcommand{\lf}[1]{\left(#1\right)}
\newcommand{\C}{\mathbb{C}}
\newcommand{\R}{\mathbb{R}}
\newcommand{\K}{\mathbb{K}}
\newcommand{\N}{\mathbb{N}}
\newcommand{\I}{\mathcal{I}}
\newcommand{\F}{\mathcal{F}}
\newcommand{\E}{\mathcal{E}}
\newcommand{\G}{\mathcal{G}}
\newcommand{\et}{\text{ et }}
\newcommand{\ou}{\text{ ou }}
\newcommand{\xou}{\ \fbox{\text{ou}}\ }


\begin{document}
	\maketitle
	\section{Generalites}
		\begin{defi}
			Soit $\mathcal{I}$ un intervalle reel.
			$a$,$b$,$c$,$d$ des applications definies sur $\mathcal{I}$ a valeurs dans $\mathbb{K}$ et continues sur $\mathcal{I}$.
			On appelle equation differentielle lineaire du second ordre toute relation du type :
			$$\begin{aligned}
				\forall x\in\mathcal{I},a\left(x\right)y''\left(x\right)+b\left(x\right)y'\left(x\right)+c\left(x\right)y\left(x\right)&=d\left(x\right)\\
																												   &\text{ d'inconnue y}
			\end{aligned}$$
		\end{defi}
		\begin{defi}
			Une solution particuliere sur $\mathcal{I}$ de l'equation differentielle precedente est une application $\phi\colon\mathcal{I}\longrightarrow\mathbb{K}$ telle que :
			\begin{liste}
				\item $\phi$ est deux foix derivable sur $\mathcal{I}$.
				\item $\forall x\in\mathcal{I},a\left(x\right)\phi''\left(x\right)+b\left(x\right)\phi'\left(x\right)+c\left(x\right)\phi\left(x\right)=d\left(x\right)$
			\end{liste}
		\end{defi}
		\textbf{Remarque :} L'ensemble des solutions de l'equation differentielle lineaire du second ordre homogene
		$$
			\forall x\in\mathcal{I},a\left(x\right)y''\left(x\right)+b\left(x\right)y'\left(x\right)+c\left(x\right)y\left(x\right)=0
		$$
		a une structure d'espace vectoriel (non vide et stable par combinaisons lineaires).
		
	\section{Equation Differentielle Lineaire du second ordre a coefficients constants}
		\subsection{Definitions}
			\begin{defi}
				Soit $\left(a,b,c\right)\in\mathbb{K}^3$ tels que $a\neq0$.
				Soit $d\colon\mathcal{I}\longrightarrow\mathbb{K}$ une application continue sur $\mathcal{I}$.
				On appelle equation differentielle lineaire du second ordre a coefficients constants une relation du type :
				$$
					\forall x\in\mathcal{I},ay''\left(x\right)+by'\left(x\right)+cy\left(x\right)=d\left(x\right)
				$$
			\end{defi}
		\newpage
		\subsection{Etude de l'equation homogene associee}
			$$\begin{aligned}
				\forall x\in\mathcal{I},ay''\left(x\right)+by'\left(x\right)+cy\left(x\right)&=0\\
																								&\text{avec}\left(a,b,c\right)\in\mathbb{K}^3,a\neq0
			\end{aligned}$$
			\begin{prop}
				(Solutions reelles)\\
				On suppose $a,b,c$ reels et $a\neq0$\\
				Soit l'equation differentielle homogene associee :
				$$
					\forall x\in\mathcal{I},ay''\left(x\right)+by'\left(x\right)+cy\left(x\right)=0 \ \ \ \left(2\right)
				$$
				On considere l'equation $ar^2+br+c=0 \ \left(E\right)$ d'inconnue $r$.\\
				$\left(E\right)$ s'appelle l'equation caracteristique associee a $\left(2\right)$\\
				\textbf{Cas 1 :} $\left(E\right)$ admet deux solutions reelles distinctes $r\uu1$ et $r\uu2$.
				\begin{tab}
					$y$ est solution de $\left(2\right)$ ssi:
					$$
						\exists\left(k\uu1,k\uu2\right)\in\mathbb{R}^2,\forall x\in\mathbb{R},y\left(x\right)=k\uu1e^{r\uu1x}+k\uu2e^{r\uu2x}
					$$
				\end{tab}
				\textbf{Cas 2 :} $\left(E\right)$ admet une unique solution reelle $r\uu0$.
				\begin{tab}
					$y$ est solution de $\left(2\right)$ ssi:
					$$
						\exists\left(\lambda,\mu\right)\in\mathbb{R}^2,\forall x\in\mathbb{R},y\left(x\right)=\left(\lambda x+\mu\right)e^{r\uu0x}					$$
				\end{tab}
				\textbf{Cas 3 :} $\left(E\right)$ admet deux solutions complexes non reelles conjuguees $\alpha\pm\imath\beta$.
				\begin{tab}
					$y$ est solution de $\left(2\right)$ ssi:
					$$
						\exists\left(a,b\right)\in\mathbb{R}^2,\forall x\in\mathbb{R},y\left(x\right)=e^{\alpha x}\left(a\cos\beta x+\imath\sin\beta x\right)
					$$
				\end{tab}
			\end{prop}
			\begin{prop}
				(moins importante, solutions complexes)\\
				On suppose $a,b,c$ complexes et $a\neq0$.\\
				\textbf{Cas 1 :} $\left(E\right)$ admet deux solutions distinctes $r\uu1$ et $r\uu2$.
				\begin{tab}
					$y$ est solution de $\left(2\right)$ ssi:
					$$
						\exists\left(k\uu1,k\uu2\right)\in\mathbb{C}^2,\forall x\in\mathbb{R},y\left(x\right)=k\uu1e^{r\uu1x}+k\uu2e^{r\uu2x}
					$$
				\end{tab}
				\textbf{Cas 2 :} $\left(E\right)$ admet une unique solution $r\uu0$.
				\begin{tab}
					$y$ est solution de $\left(2\right)$ ssi:
					$$
						\exists\left(\lambda,\mu\right)\in\mathbb{C}^2,\forall x\in\mathbb{R},y\left(x\right)=\left(\lambda x+\mu\right)e^{r\uu2x}
					$$
				\end{tab}
			\end{prop}
			\begin{preuve}
				Soit $\left(a,b,c\right)\in\mathbb(C)^3$ et $a\neq0$.\\
				Soit l'equation differentielle homogene :
				$$
					\forall x\in\mathbb{R},ay''\left(x\right)+by'\left(x\right)+cy\left(x\right)=0 \ \ \ \left(2\right)
				$$
				\begin{liste}
					\item Recherche d'une solution exponentielle.\\
						Soit $r\in\mathbb{C}$ fixe.\\
						Soit $z\uu0\colon x\longmapsto e^{rx}$ une application deux foix derivable.
						$$\begin{aligned}
							&z\uu0\left(x\right)&&=e^{rx}\\
							&z\uu0'\left(x\right)&&=re^{rx}\\
							&z\uu0''\left(x\right)&&=r^2e^{rx}
						\end{aligned}$$
						$z\uu0$ est solution de $\left(2\right)$ sur $\mathbb{R}$ :
						$$\begin{aligned}
							&\iff\forall x\in\mathbb{R},az\uu0''\left(x\right)+bz\uu0'\left(x\right)+cz\uu0\left(x\right)=0\\
							&\iff\forall x\in\mathbb{R},e^{rx}\left(ar^2+br+c\right)=0\\
							&\iff\forall x\in\mathbb{R},ar^2+br+c=0 \ \ \ \text{car }\forall x\in\mathbb{R},e^{rx}>0
						\end{aligned}$$
						\textbf{Conclusion :} $z\uu0$ est solution de $\left(2\right)$ ssi $r$ est racine de l'equation caracteristique.
					\item Determination des autres solutions.\\
						Soit $g\colon\mathbb{R}\longrightarrow\mathbb{C}$ une application deux foix derivable sur $\mathbb{R}$.\\
						Alors il existe une application $u$ definie sur $\mathbb{R}$ a valeurs dans $\mathbb{C}$ telle que :
						$$\begin{aligned}
							\forall x\in\mathbb{R},y\left(x\right)&=u\left(x\right)e^{rx} \\
												    y\left(x\right)&=e^{rx}u\left(x\right) \\
												   y'\left(x\right)&=e^{rx}\left(ru\left(x\right)+u'\left(x\right)\right) \\
												  y''\left(x\right)&=e^{rx}\left(r^2u\left(x\right)+2ru'\left(x\right)+u''\left(x\right)\right) \\
						\end{aligned}$$
						$y$ est solution de $\left(2\right)$ sur $\mathbb{R}$ :
						$$\begin{aligned}
							&\iff\forall x\in\mathbb{R},a\left(r^2u\left(x\right)+ru'\left(x\right)+u''\left(x\right)\right)+b\left(ru\left(x\right)+u'\left(x\right)\right)+cu\left(x\right)=0 \\
							&\iff\forall x\in\mathbb{R},au''\left(x\right)+\left(2ra+b\right)u'\left(x\right)+\left(ar^2+br+c\right)u\left(x\right)=0 \\
							&\iff\forall x\in\mathbb{R},au''\left(x\right)+\left(2ra+b\right)u'\left(x\right)=0
						\end{aligned}$$
						On pose $v=u'$ pour obtenir une equation differentielle d'ordre un. \\
						$y$ est solution de $\left(2\right)$ sur $\mathbb{R}$ :
						$$\begin{aligned}
							&\forall x\in\mathbb{R},av'\left(x\right)+\left(2ra+b\right)v\left(x\right)=0 \\
							&\forall x\in\mathbb{R},av'\left(x\right)=-\frac{2ra+b}{a}v\left(x\right) \\
							&\exists\lambda\in\mathbb{C},\forall x\in\mathbb{R},v\left(x\right)=\lambda\exp\left(-\left(2r+\frac{b}{a}\right)x\right) \\
						\end{aligned}$$
						\textbf{Cas 1 :} L'equation caracteristique admet deux racines $r\uu1$ et $r\uu2$.
						\begin{tab}
							On applique ce qui precede avec $r=r\uu1$ et $r\uu1+r\uu2=-\frac{b}{a}$. \\
							$y$ est solution de $\left(2\right)$ sur $\mathbb{R}$ :
							$$\begin{aligned}
								&\iff\exists\lambda\in\mathbb{C},\forall x\in\mathbb{R},&v\left(x\right)&=\lambda e^{\left(-2r\uu1+r\uu1+r\uu2\right)x} \\
								&\iff\exists\lambda\in\mathbb{C},\forall x\in\mathbb{R},&v\left(x\right)&=\lambda e^{\left(-r\uu1+r\uu2\right)x} \\
								&\iff\exists\lambda\in\mathbb{C},\exists\mu\in\mathbb{C},\forall x\in\mathbb{R},&u\left(x\right)&=\frac{\lambda}{r\uu2-r\uu1}e^{\left(r\uu2-r\uu1\right)x}+\mu \\
								&\iff\exists\left(\lambda,\mu\right)\in\mathbb{C}^2,\forall x\in\mathbb{R},&y\left(x\right)&=\frac{\lambda}{r\uu2-r\uu1}e^{r\uu2x}+\mu e^{r\uu1x} \\
								&\iff\exists\left(k\uu1,k\uu2\right)\in\mathbb{C}^2,\forall x\in\mathbb{R},&y\left(x\right)&=k\uu1e^{r\uu2x}+k\uu2 e^{r\uu1x}
							\end{aligned}$$
						\end{tab}
						\textbf{Cas 2:} L'equation caracteristique admet une racine unique $r\uu0$.
						\begin{tab}
							Alors $r\uu0=-\frac{b}{2a}$. On applique ce qui precede le cas 1 avec $r=r\uu0$. \\
							$y$ est solution de $\left(2\right)$ sur $\mathbb{R}$ :
							$$\begin{aligned}
								&\iff\exists\lambda\in\mathbb{C},\forall x\in\mathbb{R},&v\left(x\right)&=\lambda \\
								&\iff\exists\left(\lambda,\mu\right)\in\mathbb{C}^2,\forall x\in\mathbb{R},&u\left(x\right)&=\lambda x+\mu \\
								&\iff\exists\left(\lambda,\mu\right)\in\mathbb{C}^2,\forall x\in\mathbb{R},&y\left(x\right)&=\left(\lambda x+\mu\right)e^{r\uu0x}
							\end{aligned}$$
						\end{tab}
				\end{liste}
				\textbf{Conclusion :} Ceci demontre :
				\begin{liste}
					\item La propriete sur les solutions complexes
					\item Si $\left(a,b,c\right)\in\mathbb{R}^3$, la propriete sur les solutions reelles dans les deux premiers cas, en remplacant "$\exists\lambda\in\mathbb{C}/\exists\left(\lambda,\mu\right)\in\mathbb{C}^2/\exists\left(k\uu1,k\uu2\right)\in\mathbb{C}^2$" par "$\exists\lambda\in\mathbb{R}/\exists\left(\lambda,\mu\right)\in\mathbb{R}^2/\exists\left(k\uu1,k\uu2\right)\in\mathbb{R}^2$"
				\end{liste}
			\end{preuve}
			\textbf{Structure de l'ensemble des solutions de $\left(2\right)$} \\
			\textbf{Cas reel :} $\left(a,b,c\right)\in\mathbb{R}^3,a\neq0$
			\begin{tab}
				L'ensemble des solutions reelles de $\left(2\right)$ est un espace vectoriel reel de dimension deux dont une base est $\left(z\uu1,z\uu2\right)$ non vide et stable par combinaisons lineaires a coefficients reels avec :
				\begin{liste}
					\item $z\uu1\colon x\longmapsto e^{r\uu1x}$ et $z\uu2\colon x\longmapsto e^{r\uu2x}$ si l'equation caracteristique admet deux racines reelles $r\uu1$ et $r\uu2$
						$$
							\mathcal{S}=\left\{y\in\mathcal{F}\left(\mathbb{R},\mathbb{R}\right),\exists\left(k\uu1,k\uu2\right)\in\mathbb{R}^2,y=k\uu1z\uu1+k\uu2z\uu2\text{ sur }\mathbb{R}\right\}
						$$
					\item $z\uu1\colon x\longmapsto e^{r\uu0x}$ et $z\uu2\colon x\longmapsto xe^{r\uu0x}$ si l'equation caracteristique admet une racine reelle unique $r\uu0$
						$$
							\mathcal{S}=\left\{\lambda z\uu1+\mu z\uu2,\left(\lambda,\mu\right)\in\mathbb{R}^2\right\}
						$$
					\item $z\uu1\colon x\longmapsto e^{\alpha x}\cos\beta x$ et $z\uu2\colon x\longmapsto e^{\alpha x}\sin\beta x$ si l'equation caracteristique admet deux racine complexes conjuguees $\alpha\pm\imath\beta$
						$$
							\mathcal{S}=\left\{Az\uu1+Bz\uu2,\left(A,B\right)\in\mathbb{C}^2\right\}
						$$
				\end{liste}
			\end{tab}
			\textbf{Cas complexe :} $\left(a,b,c\right)\in\mathbb{C}^3,a\neq0$
			\begin{tab}
				L'ensemble des solutions complexes de $\left(2\right)$ est un espace vectoriel complexe de dimension deux dont une base est $\left(z\uu1,z\uu2\right)$ avec :
				\begin{liste}
					\item $z\uu1\colon x\longmapsto e^{r\uu1x}$ et $z\uu2\colon x\longmapsto e^{r\uu2x}$ si l'equation caracteristique admet deux racines $r\uu1$ et $r\uu2$
					\item $z\uu1\colon x\longmapsto e^{r\uu0x}$ et $z\uu2\colon x\longmapsto xe^{r\uu0x}$ si l'equation caracteristique admet une racine unique $r\uu0$
				\end{liste}
			\end{tab}
			\textbf{Remarque :} Toute solution de $\left(2\right)$ sur $\mathbb{R}$ existe de maniere unique comme combinaison lineaire de $z\uu1$ et $z\uu2$
		\subsection{Resolution de l'equation differentielle}
			$$\begin{aligned}
				\forall x\in\mathcal{I},ay''\left(x\right)+by'\left(x\right)+cy\left(x\right)=d\left(x\right)& \ \ \ \left(1\right)\\
																								\text{avec}&\left(a,b,c\right)\in\mathbb{K}^3,a\neq0 \\
																												&d\colon\mathcal{I}\longmapsto\mathbb{K}\text{ continue}
			\end{aligned}$$
			\subsubsection{Structure de l'ensemble des solutions}
				Soit $y\uu0$ une solution particuliere de $\left(1\right)$ sur $\mathcal{I}$.\\
				Soit $y$ ube application definie sur $\mathcal{I}$ a valeurs dans $\mathbb{K}$, deux foix derivable sur $\mathcal{I}$.
				$y$ est solution de $\lf1$ sur $\I$ :
				$$\begin{aligned}
					&\iff\forall x\in\I,ay''\lf{x}+by'\lf{x}+cy\lf{x}=b\lf{x} \\
					&\iff\forall x\in\I,ay''\lf{x}+by'\lf{x}+cy\lf{x}=ay''\uu0\lf{x}+by'\uu0\lf{x}+cy\uu0\lf{x} \\
					&\iff\forall x\in\I,a(y-y\uu0)''\lf{x}+b(y-y\uu0)'\lf{x}+c(y-y\uu0)\lf{x}=0 \\
					&\iff y-y\uu0\text{ est solution de }\lf2\text{ sur }\I \\
					&\iff\exists\lf{\lambda,\mu}\in\K^2,y-y\uu0=\lambda z\uu1+\mu z\uu2\text{ sur }\I \\
					&\iff\exists\lf{\lambda,\mu}\in\K^2,\forall x\in\I,y\lf{x}=y\uu0\lf{x}+\lambda z\uu1\lf{x}+\mu z\uu2\lf{x}
				\end{aligned}$$
				\textbf{Conclusion :} L'ensemble $S$ des solutions de $\lf1$ sur $\I$ est :
				$$S=\left\{y\uu0+\lambda z\uu1+\mu z\uu2,\lf{\lambda,\mu}\in\K^2\right\}$$
				Structure d'espace affine d'espace vectoriel sous-jacent l'ensemble $S$ des solutions de l'equation differentielle homogene associee $\lf2$.
			\subsubsection{Etude de $d$}
				\begin{liste}
					\item Si $d$ est une fonction polynomiale, on cherche $y\uu0$ sous forme polynomiale de degre $\deg\lf{d}+$ un nombre dependant de $a$, $b$ et $c$.
				\end{liste}
\end{document}
