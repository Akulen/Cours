\documentclass[12pt,twoside,a4paper]{article}

\def\chapitre{EDL}
\author{MPSI 2}
\def\titre{Equations Differentielles Lineaires du premier ordre}

\usepackage{amsfonts}
\usepackage{amsmath}
\usepackage{amsthm}
\usepackage{changepage}
\usepackage{color}
\usepackage{enumitem}
\usepackage{fancyhdr}
\usepackage{framed}
\usepackage[margin=1in]{geometry}
\usepackage{mathrsfs}
\usepackage{tikz, tkz-tab}
\usepackage{titling}

\newtheoremstyle{dotless}{}{}{\itshape}{}{\bfseries}{}{ }{}
\theoremstyle{dotless}

\newtheorem{defs}{Definition}[subsection]
\newenvironment{defi}{\definecolor{shadecolor}{RGB}{255,236,217}\begin{shaded}\begin{defs}\ \\}{\end{defs}\end{shaded}}

\newtheorem{pro}{Propriete}[subsection]
\newenvironment{prop}{\definecolor{shadecolor}{RGB}{230,230,255}\begin{shaded}\begin{pro}\ \\}{\end{pro}\end{shaded}}

\newtheorem{cor}{Corollaire}[subsection]
\newenvironment{coro}{\definecolor{shadecolor}{RGB}{245,250,255}\begin{shaded}\begin{cor}\ \\}{\end{cor}\end{shaded}}

\setlength{\droptitle}{-1in}
\predate{}
\postdate{}
\date{}
\title{\chapitre\\\titre\vspace{-.25in}}

\pagestyle{fancy}
\makeatletter
\lhead{\chapitre\ - \titre}
\rhead{\@author}
\makeatother

\newenvironment{preuve}{\begin{framed}\begin{proof}[\unskip\nopunct]}{\end{proof}\end{framed}}
\newenvironment{liste}{\begin{itemize}[leftmargin=*,noitemsep, topsep=0pt]}{\end{itemize}}
\newenvironment{tab}{\begin{adjustwidth}{.5cm}{}}{\end{adjustwidth}}

\newcommand{\uu}[1] {_{_{#1}}}
\newcommand{\lbracket}{[\![}
\newcommand{\rbracket}{]\!]}
\newcommand{\fonction}[5]{\begin{aligned}[t]#1\colon&#2&&\longrightarrow#3 \\&#4&&\longmapsto#5\end{aligned}}
\newcommand{\systeme}[1]{\left\{\begin{aligned}#1\end{aligned}\right.}
\newcommand{\cercle}[1]{\textcircled{\scriptsize{#1}}}

\newcommand{\lf}[1]{\left(#1\right)}
\newcommand{\C}{\mathbb{C}}
\newcommand{\R}{\mathbb{R}}
\newcommand{\K}{\mathbb{K}}
\newcommand{\N}{\mathbb{N}}
\newcommand{\I}{\mathcal{I}}
\newcommand{\F}{\mathcal{F}}
\newcommand{\E}{\mathcal{E}}
\newcommand{\G}{\mathcal{G}}
\newcommand{\et}{\text{ et }}
\newcommand{\ou}{\text{ ou }}
\newcommand{\xou}{\ \fbox{\text{ou}}\ }


\begin{document}
	\maketitle
	\section{Generalites}
		\begin{defi}
			Soit $I$ un intervalle reel.\\
			Soient $a$, $b$ et $c$ trois fonctions definies sur $I$ a valeurs reelles ou complexes.\\
			$$\begin{aligned}&&\fonction{a}{I}{\mathbb{K}}{x}{a(x)} &&\fonction{b}{I}{\mathbb{K}}{x}{b(x)} &&\fonction{c}{I}{\mathbb{K}}{x}{c(x)} \end{aligned}$$\\
			On suppose $a$ $b$ et $c$ continues sur $I$\\
			\\
			On appelle equation differentielle lineaire du premier ordre une relation du type:\\
			$$\forall x\in I,\ a(x)\,y'(x)+b(x)\,y(x)=c(x)$$
		\end{defi}
		\begin{defi}
			\begin{liste}
				\item $c$ est le second membre de l'equation differentielle.
				\item $\forall x\in I,\ a(x)\,y'(x)+b(x)\,y(x)=0$ est le second membre de l'equation.
			\end{liste}
		\end{defi}
		\begin{defi}
			Soit $J$ un sous-intervalle de $I$.\\
			On appelle solution de l'equation differentielle toute application $\Phi$ telle que:\\
			$\fonction{\Phi}{J}{\mathbb{K}}{x}{\Phi(x)}$\\
			Telle que :\begin{liste}
							\item $\Phi$ soit derivable sur $J$
							\item $\forall x\in J,\ a(x)\Phi'(x)+b(x)\Phi(x)=c(x)$
							\end{liste}
		\end{defi}
		\textbf{Remarque:} l'ensemble $\mathcal{S}\uu0$ des solutions de l'EDHA sur $I$ est stable par combinaison lineaire et non vide ($y=0$ est solution)\\
		On dit alors que $S_0$ a une structure d'espace vectoriel
	\section{Etude de l'equation $\forall x\in I,\ y'(x)+\alpha y(x)=0$}
	Pour $\alpha$ continue sur I.\\
	\\
	\begin{prop}
		L'ensemble $\mathcal{S}\uu0$ des solutions de l'ED $\forall x\in I,\ y'(x)+\alpha y(x)=0$ est:\\
		$$\mathcal{S}_0=\left\{\lambda\times g,\ \lambda\in\mathbb{R}\right\}$$\\
		avec $g(x)=exp\left(\int\limits^x_{x_0}A(x)\right)$ et $A(x)$ une primitive de $\alpha$ sur I
	\end{prop}
	\begin{preuve}
		On etudie l'expression $y'(x)=-\alpha y(x)$, et on cherche une primitive de $-\alpha$.\\
		L'exponentielle de cette primitive est solution de l'expression ($g(x)$).\\
		\\
		On cherche une fonction $u$ telle que $y=u\,g$ soit solution de l'expression precedente.\\
		Par calcul, on trouve: $\forall x\in I,\ u'(x)=0$, donc u est constante sur I.\\
		\\
		Ainsi, toutes les solutions de l'expression sont de la forme de la propriete.
	\end{preuve}
	\textbf{Remarques:}\\
	\begin{liste}
		\item $S_0$ est un espace vectoriel sur $\mathbb{K}$ de dimention 1, dont une base est donnee par $g$. On parle de droite affine.
		\item Il est possible de caracteriser la faction exponentielle par l'unique solution du systeme\\
			$\left\{\begin{aligned}&\forall x\in\mathbb{R},\ y'(x)=y(x)\\
				&y(0)=1\end{aligned}\right.$\\
		\item Si y est solution de l'ED, de deux choses l'une:
			\begin{liste}
				\item $y$ est l'application nulle sur $I$.
				\item $y$ ne s'annule jamais sur $I$ 
			\end{liste}
	\end{liste}
	\section{Cas general: $\forall x\in I,\ a(x)y'(x)+b(x)y(x)=c(x)$}
		\subsection{Resolution de l'EDHA}
			$a$ est continue sur $I$. Soit $J$ un intervalle ou $a$ ne s'annule pas. Alors:\\\\
			$\begin{aligned}&\forall x \in J,\ q(x)\,y'(x)+b(x)\,y(x)=0\\
				\iff&\forall x \in J,\ y'(x)+\frac{b(x)}{a(x)}y(x)=0\end{aligned}$\\
			D'apres la propriete precedente, l'ensemble des solutions est un espace vectriel de dimention 1.\\
			\\
			Pour $x\uu0\in J$, on note $Z_0$ l'application définie sur $J$ par:\\
			$\forall x\in J,\ Z_0(x)=-exp\left(\int\limits_{x\uu0}^x\frac{b(t)}{a(t)}\,dt\right)$\\
			$Z_0$ est une solution de l'EDHA.
		\subsection{Resolution de l'ED}
			Soit $y\uu0$ une solution particuliere de l'ED. Donc $y\uu0$ est derivable sur $J$ et:\\
			$\forall x\in J,\ a(x)\,y\uu0'(x)+b(x)\,y\uu0(x)=c(x)$\\
			\\
			$y$ est solution de l'ED\\
			$\begin{aligned}&\iff\forall x\in J,\ a(x)\,y'(x)+b(x)\,y(x)=c(x)\\
				&\iff\forall x\in J,\ a(x)\,y'(x)+b(x)\,y(x)=a(x)\,y\uu0'(x)+b(x)\,y\uu0(x)\\
				&\iff\forall x\in J,\ a(x)(y-y\uu0)(x)+b(x)(y-y\uu0)(x)=0\\
				&\iff(y-y\uu0)\text{ est solution de l'EDHA}\\
				&\iff\exists\lambda\in\mathbb{K},\ \forall x\in J,\ (y-y\uu0)(x)=\lambda Z_0(x)\\
				&\iff\exists\lambda\in\mathbb{K},\ \forall x\in J,\ y(x)=y\uu0+\lambda Z_0\end{aligned}$\\
			\\
			On a demontre:
			\begin{prop}
				\begin{liste}
					\item l'ensemble des solutions de l'ED $a(x)\,y'(x)+b(x)\,y(x)=c(x)$ est :\\
						$\mathcal{S}_J=\left\{y\uu0+\lambda Z_0,\ \lambda\in\mathbb{K}\right\}$\\
						avec $\fonction{Z_0}{J}{\mathbb{K}}{x}{-exp\left(-\int\limits_{x\uu0}^x\frac{b(t)}{a(t)}\,dt\right)}$\\
						et $y\uu0$ solution particuliere de l'ED.
					\item $\mathcal{S}_J$est un espace affine d'espace vectoriel sous-jacent l'ensemble des solutions de l'EDHA.\\
		c'est un espace vectoriel de dimension 1, on parle de droite affine.
				\end{liste}
			\end{prop}
		\subsection{Determination d'une solution particuliere}
			Il existe deux fonctions $y$ et $u$ derivables sur $J$ telles que:\\
			$\forall x\in J,\ y(x)=u(x)\,Z_0(x)$\\
			$y$ est solution de l'ED $\iff \forall x\in J,\ u'(x)=\frac{c(x)}{a(x)\,Z_0(x)}$\\
			On choisit une primitive $u\uu0$ de $u'$ et $y\uu0=u\uu0\,Z_0$, les solutions de l'ED sont donc de la forme:\\
			$y=y\uu0+\lambda Z_0$
		\subsection{Probleme de Cauchy}
			Soit $I$ un intervalle reel et $a$; $b$, $c$ trois applications continues sur $I$ a valeurs dans $\mathbb{K}$.
			Soit l'ED: $\forall s\in I,\ a(x)\,y'(x)+b(x)\,y(x)=c(x)$\\
			\\
			\textbf{Soit $(x\uu0,y\uu0\in I\times\mathbb{K}$. Existe-t-il une solution $y:x\rightarrow\mathbb{K}$ satisfaisant $y(x\uu0)=y\uu0$ ?}
			\begin{prop}
				Si $a$ ne s'annule pas sur $I$, alors il existe une unique solution $y:I\rightarrow\mathbb{K}$ telle que:\\
				$\left\{\begin{aligned}&\forall x\in I,\ a(x)\,y'(x)+b(x)\,y(x)=c(x)\\&y(x\uu0)=y\uu0\end{aligned}\right.$
			\end{prop}
			\textbf{Remarques si $\mathbb{K=R}$}
			\begin{liste}
				\item Si $a$ ne s'annule pas sur $I$, il existe une unique \textbf{courbe solution} passant oar le pint de coordonnees $(x\uu0,y\uu0)$
				\item par application de cette propriete, deux courbes integrales ne se coupent jamais.
			\end{liste}
			\begin{preuve}
				Determiner les solutions litterales de l'ED, et exprimer $y(x\uu0)=y\uu0$ en fonction des expressions precedentes. En deduire $\lambda$ unique, donc $y$ unique.
			\end{preuve}
\end{document} 