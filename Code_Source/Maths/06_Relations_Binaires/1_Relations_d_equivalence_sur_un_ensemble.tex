\documentclass[12pt,twoside,a4paper]{article}

\def\chapitre{Relations Binaires}
\author{MPSI 2}
\def\titre{Relations d'\'equivalence sur un ensemble}

\usepackage{amsfonts}
\usepackage{amsmath}
\usepackage{amsthm}
\usepackage{changepage}
\usepackage{color}
\usepackage{enumitem}
\usepackage{fancyhdr}
\usepackage{framed}
\usepackage[margin=1in]{geometry}
\usepackage{mathrsfs}
\usepackage{tikz, tkz-tab}
\usepackage{titling}

\newtheoremstyle{dotless}{}{}{\itshape}{}{\bfseries}{}{ }{}
\theoremstyle{dotless}

\newtheorem{defs}{Definition}[subsection]
\newenvironment{defi}{\definecolor{shadecolor}{RGB}{255,236,217}\begin{shaded}\begin{defs}\ \\}{\end{defs}\end{shaded}}

\newtheorem{pro}{Propriete}[subsection]
\newenvironment{prop}{\definecolor{shadecolor}{RGB}{230,230,255}\begin{shaded}\begin{pro}\ \\}{\end{pro}\end{shaded}}

\newtheorem{cor}{Corollaire}[subsection]
\newenvironment{coro}{\definecolor{shadecolor}{RGB}{245,250,255}\begin{shaded}\begin{cor}\ \\}{\end{cor}\end{shaded}}

\setlength{\droptitle}{-1in}
\predate{}
\postdate{}
\date{}
\title{\chapitre\\\titre\vspace{-.25in}}

\pagestyle{fancy}
\makeatletter
\lhead{\chapitre\ - \titre}
\rhead{\@author}
\makeatother

\newenvironment{preuve}{\begin{framed}\begin{proof}[\unskip\nopunct]}{\end{proof}\end{framed}}
\newenvironment{liste}{\begin{itemize}[leftmargin=*,noitemsep, topsep=0pt]}{\end{itemize}}
\newenvironment{tab}{\begin{adjustwidth}{.5cm}{}}{\end{adjustwidth}}

\newcommand{\uu}[1] {_{_{#1}}}
\newcommand{\lbracket}{[\![}
\newcommand{\rbracket}{]\!]}
\newcommand{\fonction}[5]{\begin{aligned}[t]#1\colon&#2&&\longrightarrow#3 \\&#4&&\longmapsto#5\end{aligned}}
\newcommand{\systeme}[1]{\left\{\begin{aligned}#1\end{aligned}\right.}
\newcommand{\cercle}[1]{\textcircled{\scriptsize{#1}}}

\newcommand{\lf}[1]{\left(#1\right)}
\newcommand{\C}{\mathbb{C}}
\newcommand{\R}{\mathbb{R}}
\newcommand{\K}{\mathbb{K}}
\newcommand{\N}{\mathbb{N}}
\newcommand{\I}{\mathcal{I}}
\newcommand{\F}{\mathcal{F}}
\newcommand{\E}{\mathcal{E}}
\newcommand{\G}{\mathcal{G}}
\newcommand{\et}{\text{ et }}
\newcommand{\ou}{\text{ ou }}
\newcommand{\xou}{\ \fbox{\text{ou}}\ }


\begin{document}
	\maketitle
	\section{G\'en\'eralit\'es}
		Soit $E$ un ensemble non vide.
		\begin{defi}
			On appelle \underline{relation binaire sur $E$} le couple $(E,G)$ o\`u $G$ est un graphe de $E$ dans $E$.
		\end{defi}\ \\
		\textbf{Notations}: $\begin{aligned}[t]&(E,G), \mathcal{R}\\
					&\forall(x,y)\in E^2,\ x\,\mathcal{R}\,y\iff(x,y)\in G\end{aligned}$
		\begin{tab}
			Notons $\Delta_E=\left\{(x,x),x\in E\right\}$\\
			$\Delta_E$ s'appelle la diagonale de $E$\\
			On en d\'efinit une relation binaire:
			\begin{tab}
				$\begin{aligned}\forall(x,y)\in E^2,\ x\,\mathcal{R}\,y&\iff(x,y)\in\Delta_E\\
					&\iff x=y\end{aligned}$
			\end{tab}
		\end{tab}
	\section{Relations d'\'equivalences}
		Soit $\mathcal{R}$ une relation binaire sur $E$
		\begin{defi}$\mathcal{R}$ est \underline{r\'eflexive} si $\forall x\in E,\ x\,\mathcal{R}\, x$\end{defi}
		\begin{defi}$\mathcal{R}$ est \underline{sym\'etrique} si $\forall(x,y)\in E^2,\ (x\,\mathcal{R}\,y)\Rightarrow(y\,\mathcal{R}\,x)$\end{defi}
		\begin{defi}$\mathcal{R}$ est \underline{transitive} si $\forall(x,y,z)\in E^3,\ (x\,\mathcal{R}\,y\ et\ y\,\mathcal{R}\,z)\Rightarrow(x\,\mathcal{R}\,z)$\end{defi}
		\begin{defi}$\mathcal{R}$ est une \underline{relation d'\'equivalence} sur $E$ si $\mathcal{R}$ est r\'eflexive, sym\'etrique et transitive.\end{defi}
		\begin{defi}
			Soit $\mathcal{R}$ une relation d'\'equivalence sur $E$.\\
			Soit $x$ un \'el\'ement de $E$.\\
			\\
			On appelle \underline{classe d'\'equivalence de $x$ suivant $\mathcal{R}$} le sous ensemble de $E$:\\
			$\mathcal{C_R}(x)=\left\{y\in E,\ x\,\mathcal{R}\,y\right\}$
		\end{defi}
		\begin{prop}
			La famille des classes d'\'equivalences suivant $\mathcal{R}$, $\left(\mathcal{C_R}(x)\right)_{x\in E}$ est une partition de $E$.
		\end{prop}
		\begin{preuve}
			\begin{liste}
				\item[\cercle{1}] Montrer que: $\mathcal{C_R}(x)\neq\varnothing$\\
					$\mathcal{R}$ est r\'eflexive, donc $x\,\mathcal{R}\,x$\\
					Autrement dit, $x\in \mathcal{C_R}(x)$ donc $\mathcal{C_R}(x)\neq\varnothing$\\
				\item[\cercle{2}] $\begin{aligned}[t]\text{Montrer que: }&\bigcup\limits_{x\in E}\mathcal{C_R}(x)=E\\
					\iff&\bigcup\limits_{x\in E}\mathcal{C_R}(x)\subset E\text{ et que }E\subset\bigcup\limits_{x\in E}\mathcal{C_R}(x)\end{aligned}$
					\begin{liste}
						\item[a)] Les classes d'\'equivalences sont des sous-ensembles de $E$.\\
							$\forall x\in E,\ \mathcal{C_R}(x)\subset E$\\
							Ainsi, $\bigcup\limits_{x\in E}\mathcal{C_R}(x)\subset E$
						\item[b)] $\begin{aligned}[t]\text{Montrer que: }&E\subset\bigcup\limits_{x\in E}\mathcal{C}\mathcal{R}(x)\\
							\iff&\forall t\in E,\ t\in\bigcup\limits_{x\in E}\mathcal{C_R}(x)\end{aligned}$\\
							$\mathcal{R}$ est r\'eflexive, donc $t\in \mathcal{C_R}(t)$\\
							En posant $t=x)$ on d\'emontre la proposition.\\
							Cela \'etant vrai pour tout $x$, on obtient $E\subset\bigcup\limits_{x\in E}\mathcal{C_R}(x)$
					\end{liste}\ \\
				\item[\cercle{3}]$\begin{aligned}[t]\text{Montrer que: }&\forall(x,y)\in E^2,\ \mathcal{C_R}(x)=\mathcal{C_R}(y)\text{ ou } \mathcal{C_R}(x)\cap \mathcal{C_R}(y)=\varnothing\\
					\iff&\forall(x,y)\in E^2,\ \left(\mathcal{C_R}(x)\cap \mathcal{C_R}(y)\neq\varnothing\right)\Rightarrow \left(\mathcal{C_R}(x)=\mathcal{C_R}(y)\right)\end{aligned}$\\
					\underline{H$_1$}: Soit $(x,y)$ un couple d'\'el\'ements de $E$ tels que $\mathcal{C_R}(x)\cap \mathcal{C_R}(y)\neq\varnothing$\\
					$\begin{aligned}\text{Montrer que: }&\mathcal{C_R}(x)=\mathcal{C}\mathcal{R}(y)\\
					\iff&\mathcal{C_R}(x)\subset \mathcal{C_R}(y)\text{ et }\mathcal{C_R}(y)\subset \mathcal{C_R}(x)\end{aligned}$
					\begin{liste}
						\item[a)] $\begin{aligned}[t]\text{Montrer que: }&\mathcal{C_R}(x)\subset \mathcal{C_R}(y)\\
							\iff&\forall z\in E,\ z\in \mathcal{C_R}(x)\Rightarrow z\in \mathcal{C_R}(y)\end{aligned}$\\
							\underline{H$_2$}: Soit $z$ un \'el\'ement de $\mathcal{C_R}(x)$\\
							Montrer que $z\in \mathcal{C_R}(y)$\\
							D'apr\`es H$_1$, $\exists t\in E,\ t\in \mathcal{C_R}(x)\cap \mathcal{C_R}(y)$\\
							\underline{H$_3$}: Soit $t\uu0\in E$ tel que $t\uu0\in \mathcal{C_R}(x)$ et $t\uu0\in\mathcal{C_R}(y)$\\
							$\begin{aligned}\text{Montrer que: }&z\in \mathcal{C_R}\\
							\iff&z\,\mathcal{R}\,y\end{aligned}$\\
							D'apr\`es H$_2$: $z\,\mathcal{R}\,x$\\
							D'apr\`es H$_3$: $t\uu0\,\mathcal{R}\,x$\\
							Par sym\'etrie et transitivit\'e: $z\,\mathcal{R}\,t\uu0$\\
							D'apr\`es H$_2$: $t\uu0\,\mathcal{R}\,y$\\
							Par transitivit\'e: $z\,\mathcal{R}\,y$\\
							\\
							Conclusion 1: $z\in\mathcal{C_R}\Rightarrow z\in\mathcal{C_R}$\\
							Conclusion 2: $\begin{aligned}[t]&\text{Ceci \'etant vrai pour tout }z\text{ dans }E\text{:}\\
							&\mathcal{C_R}(x)\subset\mathcal{C_R}(y)\end{aligned}$
						\item[b)] Montrer que $\mathcal{C_R}(y)\subset \mathcal{C_R}(x)$\\
							En \'echangeant les r\^oles de $x$ et $y$, et par une d\'emonstration analogue, on obtient:\\
							$\mathcal{C_R}(y)\subset \mathcal{C_R}(x)$
					\end{liste}
					Finalement: $\mathcal{C_R}(x)=\mathcal{C_R}(y)$\\
					Conclusion G\'en\'erale: $\forall(x,y)\in E^2,\ \mathcal{C_R}(x)\cap\mathcal{C_R}(y)\neq\varnothing\Rightarrow\mathcal{C_R}=\mathcal{C_R}$
			\end{liste}\ \\
			La famille $(\mathcal{C_R}(x))_{x\in E}$ est une partition de $E$.
		\end{preuve}
		\begin{prop}
			Soit $\left(A_i\right)_{i\in I}$ une partition de $E$, alors, il existe une relation d'\'equivalence $\mathcal{R}$ dont la famille des classes d'\'equivalences est cette partition.
		\end{prop}
		\begin{preuve}
			Soit $\mathcal{R}$ une relation binaire d\'efinie par:\\
			$\forall(x,y)\in E^2,\ x\,\mathcal{R}\,y\iff\exists i\in I,\ x\in A_i\text{ et }y\in A_i$
			\begin{liste}
				\item[\cercle{1}] Montrer que $\mathcal{R}$ est une relation d'\'equivalence sur $E$.\\
				\begin{liste}
					\item[a)] $\begin{aligned}[t]\text{Montrer que }&\mathcal{R}\text{ est r\'eflectve}\\
						\iff&\forall x\in E,\ x\,\mathcal{R}\,x\end{aligned}$\\
						\underline{H$_1$}: Soit $x$ un \'el\'ement de $E$\\
						Montrer que: $\exists i\in I,\ x\in A_i$\\
						$A_i$ est une partition de de $E$, donc d'apr\`es H$_1$,\\
						$\exists i\in I,\ x\in A_i$
					\item[b)]  $\begin{aligned}[t]\text{Montrer que }&\mathcal{R}\text{ est sym\'etrique}\\
						\iff&\forall (x,y)\in E^2,\ (x\,\mathcal{R}\,y)\Rightarrow(y\,\mathcal{R}\,x)\end{aligned}$\\
						\underline{H$_1$}: Soit $(x,y)\in E^2$ tel que $x\,\mathcal{R}\,y$\\
						H$_2$: $\exists i\in I,\ x\in A_i\text{ et }y\in A_i\iff\exists i\in I,\ y\in A_i \text{ et } x\in A_i$\\
						On a donc $y\,\mathcal{R}\,x$\\
						Donc $\mathcal{R}$ est sym\'etrique.
					\item[c)] $\begin{aligned}[t]\text{Montrer que }&\mathcal{R}\text{ est transitive}\\
						\iff&\forall (x,y,z)\in E^3,\ (x\,\mathcal{R}\,y\text{ et }y\,\mathcal{R}\,z)\Rightarrow(x\,\mathcal{R}\,z)\end{aligned}$\\
						\underline{H$_1$}: Soit $x$, $y$ et $z$ trois \'el\'ements de $E$ tels que $x\,\mathcal{R}\,y$ et $y\,\mathcal{R}\,z$
						H$_1$: $\exists i\in I,\ (x\in A_I$ et $y\in A_I)$ et ($\exists j\in I,\ y\in A_j$ et $z\in A_j$)\\
						\underline{H$_2$}: Soit $i\uu0$ et $i'\uu0$ deux \'el\'ements de $I$ tels que $\systeme{& x\in A_{i_0},& y\in A_{i_0}\\& y\in A_{i'_0},& z\in A_{i'_0}}$\\
						Donc $y\in A_{i_0}\cap A_{i'_0}$\\
						Or $(A_i)_{i\in I}$ est une partition de E\\
						Donc $A_{i_0}=A_{i'_0}$\\
						Donc $x$, $y$ et $z$ sont des \'el\'ements de $A_{i_0}$,\\
						Donc $x\,\mathcal{R}\,z$
						Donc $\mathcal{R}$ est transitive.
				\end{liste}
				Conclusion \cercle{1}: $\mathcal{R}$ est une relation d'\'equivalence.\\
				\item[\cercle{2}]$\begin{aligned}[t]\text{Montrer que}&\text{ les } A_i\text{ sont les classes d'\'equivalences suivant }\mathcal{R}\\
					\iff& \forall i\in I,\ \exists x\in E,\ A_i=C_\mathcal{R}(x)\end{aligned}$\\
					\underline{H$_1$}: Soit $i$ un \'el\'ement de $I$.\\
					Montrer que $\exists x\in E, A_i=C_\mathcal{R}(x)$\\
					$(A_i)_{i\in I}$ est une partition de E, donc en particulier\\
					$A_i$ non vide, \'ecrit:\\
					$\exists x\in E, x\in A_i$\\
					\underline{H$_2$}: Soit $x\uu0$ un \'el\'ement de $A_i$ fix\'e.\\
					$\begin{aligned}[t]\text{Montrer que }& A_i=C_\mathcal{R}(x\uu0)\\
					\iff& \left(A_i\subset C_\mathcal{R}(x\uu0)\right)\text{ et }\left(C_\mathcal{R}(x\uu0)\subset A_i\right)\end{aligned}$\\
					\begin{liste}
						\item[a)] $\begin{aligned}[t]\text{Montrer que }& A_i\subset C_\mathcal{R}(x\uu0)\\
							\iff& \forall y\in E,\ y\in A_i\Rightarrow y\in C_\mathcal{R}(x\uu0)\end{aligned}$\\
							\underline{H$_3$}: Soit $y$ un \'el\'ement de $A_i$.\\
							Montrer que $y\in C_\mathcal{R}(x\uu0)$\\
							D'apr\`es H$_1$ et H$_2$, on a $y\in A_i$ et $x\in A_i$\\
							Donc $x\,\mathcal{R}\,y$ par d\'efinition de $\mathcal{R}$\\
							Cela \'etant valable pour tout $i$ dans $I$ et pour tout $y$ dans $A_i$, \\
							$A_i\subset C_\mathcal{R}(x\uu0)$
						\item[b)] $\begin{aligned}[t]\text{Montrer que }& C_\mathcal{R}(x\uu0)\subset A_i\\
							\iff& \forall j\in E,\ y\in C_\mathcal{R}(x\uu0)\Rightarrow y\in A_j\end{aligned}$\\
							\underline{H$_4$}: Soit $y$ un \'el\'ement de $C_\mathcal{R}(x\uu0)$\\
							Montrer que $y\in A_i$\\
							H$_4$: $y\,\mathcal{R}\,x\uu0$, autrement dit:\\
							$\exists j\in I,\ y\in A_j\text{ et }x\uu0\in A_j$\\
							\underline{H$_5$}: Soit $j\uu0$ un \'el\'ement de $I$ tel que $y\in A_j\text{ et }x\uu0\in A_j$\\
							D'apr\`es H$_2$: $x\uu0\in A_i$\\
							Avec H$_2$ et H$_3$, on en d\'eduit que $x\uu0\in A_i\cap A_{j\uu0}$\\
							Or $(A_i)_{i\in I}$ est une partition de $E$, donc $A_i=A_{j\uu0}$\\
							Montrer que $y\in A_i$\\
							Or, $y\in A_{j\uu0}$, donc $y\in A_i$\\
							Cela \'etant valable pour tout y dans $A_i$,\\
							$ C_\mathcal{R}(x\uu0)\subset A_i$
					\end{liste}
				Conclusion \cercle{2}: $\forall i\in I,\ \exists x\in E,\ A_i=C_\mathcal{R}(x)$
			\end{liste}
			Conclusion g\'en\'erale: Par raisonnement sur des conditions n\'ecessaires et suffisantes, la propri\'et\'e est d\'emontr\'ee
		\end{preuve}
	\section{Partition associ\'ee a une application}
		Soit $E$ un ensemble non vide, soit $F$ un ensemble.\\
		Soit une application $\fonction{f}{E}{F}{x}{f(x)}$
		\begin{defi}
			On appelle \underline{relation d'\'equivalence associ\'ee a $f$} la relation d\'efinie par:\\
			$$\forall(x,y)\in E^2,\ x\,\mathcal{R}\,y\iff f(x)=f(y)$$
		\end{defi}
		\begin{defi}
			On appelle \underline{partition associ\'ee a $f$} la famille des classes d'\'equivalences suivant $\mathcal{R}_f$
		\end{defi}
\end{document}
