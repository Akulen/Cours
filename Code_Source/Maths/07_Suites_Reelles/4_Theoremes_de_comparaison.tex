\documentclass[12pt,twoside,a4paper]{article}

\def\chapitre{Suites R\'eelles}
\author{MPSI 2}
\def\titre{Th\'eor\`emes de comparaison}

\usepackage{amsfonts}
\usepackage{amsmath}
\usepackage{amsthm}
\usepackage{changepage}
\usepackage{color}
\usepackage{enumitem}
\usepackage{fancyhdr}
\usepackage{framed}
\usepackage[margin=1in]{geometry}
\usepackage{mathrsfs}
\usepackage{tikz, tkz-tab}
\usepackage{titling}

\newtheoremstyle{dotless}{}{}{\itshape}{}{\bfseries}{}{ }{}
\theoremstyle{dotless}

\newtheorem{defs}{Definition}[subsection]
\newenvironment{defi}{\definecolor{shadecolor}{RGB}{255,236,217}\begin{shaded}\begin{defs}\ \\}{\end{defs}\end{shaded}}

\newtheorem{pro}{Propriete}[subsection]
\newenvironment{prop}{\definecolor{shadecolor}{RGB}{230,230,255}\begin{shaded}\begin{pro}\ \\}{\end{pro}\end{shaded}}

\newtheorem{cor}{Corollaire}[subsection]
\newenvironment{coro}{\definecolor{shadecolor}{RGB}{245,250,255}\begin{shaded}\begin{cor}\ \\}{\end{cor}\end{shaded}}

\setlength{\droptitle}{-1in}
\predate{}
\postdate{}
\date{}
\title{\chapitre\\\titre\vspace{-.25in}}

\pagestyle{fancy}
\makeatletter
\lhead{\chapitre\ - \titre}
\rhead{\@author}
\makeatother

\newenvironment{preuve}{\begin{framed}\begin{proof}[\unskip\nopunct]}{\end{proof}\end{framed}}
\newenvironment{liste}{\begin{itemize}[leftmargin=*,noitemsep, topsep=0pt]}{\end{itemize}}
\newenvironment{tab}{\begin{adjustwidth}{.5cm}{}}{\end{adjustwidth}}

\newcommand{\uu}[1] {_{_{#1}}}
\newcommand{\lbracket}{[\![}
\newcommand{\rbracket}{]\!]}
\newcommand{\fonction}[5]{\begin{aligned}[t]#1\colon&#2&&\longrightarrow#3 \\&#4&&\longmapsto#5\end{aligned}}
\newcommand{\systeme}[1]{\left\{\begin{aligned}#1\end{aligned}\right.}
\newcommand{\cercle}[1]{\textcircled{\scriptsize{#1}}}

%Auteur: Cl\'ement Phan, MPSI 2

\begin{document}
	\maketitle
	\begin{theo}{des suites monotones}
		Soit $u$ une suite croissante. On a:
		\begin{liste}
			\item Si $u$ n'est pas major\'ee, $u$ diverge vers $+\infty$.
			\item Si $u$ est major\'ee, $u$ converge vers la borne sup\'erieure de ses valeurs.
		\end{liste}
		On proc\`ede de m\^eme pour les suites d\'ecroissantes.
	\end{theo}
	\begin{preuve}
		Soit $u$ une suite d\'ecroissante, et $A$ l'ensemble de ses valeurs.
		\begin{liste}
			\item[\cercle1]Supposons $u$ non major\'ee.\\
				Donc $A$ n'est pas major\'e.\\
				Donc $\forall M\in\R,\ \exists x\in A,\ x>M$\\
				Donc $\forall M\in\R,\ \exists n_0\in \N,\ u_{n_0}>M$\\
				Or, sachant $u$ croissante, $\forall M\in\R,\ \exists n_0\in \N,\ \forall n\in\N,\ n\geqslant n_0\Rightarrow u_n>M$\\
				Donc $u$ diverge vers $+\infty$ par d\'efinition.
			\item[\cercle2]Supposons $u$ major\'ee.\\
				Donc $A$ est major\'e et non vide. Donc $A$ admet une borne sup\'erieure, not\'ee $\alpha$.\\
				$\alpha$ est borne sup\'erieure de $A$, donc:
				\begin{liste}
					\item $\alpha$ est un majorant de $A$: $\forall n\in\N,\ u_n\leqslant\alpha$
					\item $\forall \varepsilon\in\R^{+*},\ \exists x\in\R,\ (x\in A)\et(\alpha-\varepsilon<x\leqslant\alpha)$\\
						C'est \`a dire: $\forall \varepsilon\in\R^{+*},\ \exists n_0\in\N,\ \alpha-\varepsilon<u_{n_0}\leqslant\alpha$\\
						Donc, sachant $u$ croissante:\\
						$\forall \varepsilon\in\R^{+*},\ \exists n_0\in\N,\ \forall n\in\N,\ n\geqslant n_0\Rightarrow\alpha-\varepsilon<u_{n_0}\leqslant\alpha<\alpha+\varepsilon$\\
						Donc $u$ converge vers $\alpha$
				\end{liste}
		\end{liste}
	\end{preuve}
	\begin{theo}{des suites adjacentes}
		Soit $u$ et $v$ deux suites r\'eelles telles que:
		\begin{liste}
			\item $u$ est croissante, $v$ est d\'ecroissante.
			\item $u-v$ tend vers $0$
		\end{liste}
		Alors $u$ et $v$ convergent vers une m\^eme limite $L$\\Et $\forall n\in\N,\ u_n\leqslant L\leqslant v_n$
	\end{theo}
	\newpage
	\begin{preuve}
		Soit $u$ et $v$ deux suites adjacentes
		\begin{liste}
			\item[\cercle1] Montrer que $\forall n\in\N,\ u_n\leqslant v_n$\\
				\underline{HA:} Supposons $n_0$ tel que $v_{n_0}<u_{n_0}$\\
				$u$ est croissante et $v$ est d\'ecroissante:\\
				$\forall n\in\N,\ n\geqslant n_0\Rightarrow u_n\geqslant u_{n_0}$\\
				$\forall n\in\N,\ n\geqslant n_0\Rightarrow v_n\leqslant v_{n_0}$\\
				Donc $\forall n\in\N,\ n\geqslant n_0\Rightarrow u_{n_0}-v_{n_0}\geqslant u_n-v_n$\\
				Or, sachant $u_{n_0}-v_{n_0}<0$, on a:$\forall n\in\N,\ n\geqslant n_0\Rightarrow |u_{n_0}-v_{n_0}|\leqslant |u_n-v_n|$\\
				Donc $|u-v|$ est minor\'e par un r\'eel strictement positif, et ne tend donc pas vers $0$.\\
				Donc il n'existe aucun $n_0$ de $\N$ tel que $v_{n_0}<u_{n_0}$\\
				Donc $\forall n\in\N,\ u_n\leqslant v_n$
			\item[\cercle2] $u$ est d\'ecroissante et minor\'ee (par tout terme de $v$), elle admet une limite not\'ee $l_1$\\
				$u$ est croissante et major\'ee (par tout terme de $u$), elle admet une limite not\'ee $l_2$\\
				De plus,
				$\left\{\begin{aligned}
					& u_n-v_n\mathop{\longrightarrow}\limits_{n\rightarrow+\infty}0\\
					& u_n-v_n\mathop{\longrightarrow}\limits_{n\rightarrow+\infty}l_1-l_2
				\end{aligned}\right.$\\
				Donc, par unicit\'e de la limite, $l_1=l_2$
		\end{liste}
		Donc $\forall n\in\N,\ u_n\leqslant L\leqslant v_n$
	\end{preuve}
	\begin{theo}{des segments emboit\'es}
		Si $a$ et $b$ deux suites adjacentes, avec $a_0\leqslant b_0$, on pose $\forall n\in\N,\ I_n=[a_n,b_n]$\\
		Alors $\bigcap\limits_{n\in\N}I_n=\{L\}$\\
		Avec $L$ la limite de $a$ et $b$
	\end{theo}
	\begin{preuve}
		\begin{liste}
			\item Par le th\'eor\`eme des suites adjacentes, $a$ et $b$ tendent vers $L$, et $\forall n\in\N,\ u_n\leqslant L\leqslant v_n$
			\item Montrer que $\forall x\in\R,\ (x\in\bigcap\limits_{n\in\N}I_n)\Rightarrow(x=L)$\\
				Ou bien montrer que $\forall x\in\R,\ (x\neq L)\Rightarrow(x\notin\bigcap\limits_{n\in\N}I_n)$\\
				Soit $x$ un réel différent de $\R$
				\begin{liste}
					\item \underline{$1^{\text{er}}$ cas:} $x>l$\\
						Soit $\varepsilon=\frac{L-x}{2}$\\
						$a$ converge vers $L$ donc: $\exists n_0\in\N,\ \forall n\in\N,\ n\geqslant n_0\Rightarrow L-\varepsilon<a_n<l+\varepsilon$\\
						Donc $\exists n_0\in\N,\ \forall n\in\N,\ \frac{L+x}{2}<a_n<\frac{3L+x}{2}$\\
						Soit $n_0$ un tel entier.\\
						Donc $x<\frac{x+L}{2}<a_{n_0}\leqslant b_{n_0}$, donc $x\notin I_{n_0}$\\
						Par suite, $(x\neq L)\Rightarrow(x\notin\bigcap\limits_{n\in\N}I_n)$
					\item \underline{$2^{\text{\`eme}}$ cas:} $x<l$\\
						Procéder de m\^eme.
				\end{liste}
		\end{liste}
		Donc $\forall x\in\R,\ (x\in\bigcap\limits_{n\in\N}I_n)\Rightarrow(x=L)$
	\end{preuve}
\end{document}