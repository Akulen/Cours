\documentclass[12pt,twoside,a4paper]{article}

\def\chapitre{Suites R\'eelles}
\author{MPSI 2}
\def\titre{Comparaison avec une suite géométrique}

\usepackage{amsfonts}
\usepackage{amsmath}
\usepackage{amsthm}
\usepackage{changepage}
\usepackage{color}
\usepackage{enumitem}
\usepackage{fancyhdr}
\usepackage{framed}
\usepackage[margin=1in]{geometry}
\usepackage{mathrsfs}
\usepackage{tikz, tkz-tab}
\usepackage{titling}

\newtheoremstyle{dotless}{}{}{\itshape}{}{\bfseries}{}{ }{}
\theoremstyle{dotless}

\newtheorem{defs}{Definition}[subsection]
\newenvironment{defi}{\definecolor{shadecolor}{RGB}{255,236,217}\begin{shaded}\begin{defs}\ \\}{\end{defs}\end{shaded}}

\newtheorem{pro}{Propriete}[subsection]
\newenvironment{prop}{\definecolor{shadecolor}{RGB}{230,230,255}\begin{shaded}\begin{pro}\ \\}{\end{pro}\end{shaded}}

\newtheorem{cor}{Corollaire}[subsection]
\newenvironment{coro}{\definecolor{shadecolor}{RGB}{245,250,255}\begin{shaded}\begin{cor}\ \\}{\end{cor}\end{shaded}}

\setlength{\droptitle}{-1in}
\predate{}
\postdate{}
\date{}
\title{\chapitre\\\titre\vspace{-.25in}}

\pagestyle{fancy}
\makeatletter
\lhead{\chapitre\ - \titre}
\rhead{\@author}
\makeatother

\newenvironment{preuve}{\begin{framed}\begin{proof}[\unskip\nopunct]}{\end{proof}\end{framed}}
\newenvironment{liste}{\begin{itemize}[leftmargin=*,noitemsep, topsep=0pt]}{\end{itemize}}
\newenvironment{tab}{\begin{adjustwidth}{.5cm}{}}{\end{adjustwidth}}

\newcommand{\uu}[1] {_{_{#1}}}
\newcommand{\lbracket}{[\![}
\newcommand{\rbracket}{]\!]}
\newcommand{\fonction}[5]{\begin{aligned}[t]#1\colon&#2&&\longrightarrow#3 \\&#4&&\longmapsto#5\end{aligned}}
\newcommand{\systeme}[1]{\left\{\begin{aligned}#1\end{aligned}\right.}
\newcommand{\cercle}[1]{\textcircled{\scriptsize{#1}}}

\newcommand{\lf}[1]{\left(#1\right)}
\newcommand{\C}{\mathbb{C}}
\newcommand{\R}{\mathbb{R}}
\newcommand{\K}{\mathbb{K}}
\newcommand{\N}{\mathbb{N}}
\newcommand{\I}{\mathcal{I}}
\newcommand{\F}{\mathcal{F}}
\newcommand{\E}{\mathcal{E}}
\newcommand{\G}{\mathcal{G}}
\newcommand{\et}{\text{ et }}
\newcommand{\ou}{\text{ ou }}
\newcommand{\xou}{\ \fbox{\text{ou}}\ }


%Auteur: Clément Phan, MPSI 2

\begin{document}
	\maketitle
	\section{Suites et séries géométriques}
		\begin{defi}
			On appelle \underline{suite géométrique de raison $a$} toute suite telle que:\\
			$\systeme{&u_0\in\R\\\forall n\in\N,\ u_{n+1}=a\,u_n}$
		\end{defi}
		\begin{flushleft}
			\textbf{Remarques:}
			\begin{liste}
				\item C'est équivalent a dire $\forall n\in\N,\ u_n=a^n\,u_0$
				\item Si $u_0=0$ alors la suite est nulle.
				\item Si $u_0\neq 0$ alors l'étude de la suite est ramenée a l'étude de $a^n$ \`a une constante multiplicative près.
			\end{liste}
		\end{flushleft}
		\begin{flushleft}
			\textbf{Étude de $\left(a^n\right)_{n\in\N}$}
			\begin{liste}
				\item \underline{Si $a>1$} alors $\exists h\in\R^{+*},\ a=1+h$\\
					De plus, $(h+1)^n=\sum\limits_{k=0}^{n}\dbinom{n}{k}\,h^k$\\
					En particulier, $(1+h)^n\geqslant nh$\\
					Or par minoration, $a^n\mathop{\longrightarrow}\limits_{n\rightarrow+\infty}+\infty$
				\item \underline{Si $a=1$} alors $(a_n)_{n\in\N}$ est stationnaire a $1$.
				\item \underline{Si $a=-1$} alors $(a_n)_{n\in\N}$ diverge.
				\item \underline{Si $-1<a<1$} alors $|a|^n=\frac{1}{\left(\frac{1}{|a|}\right)^n}$\\
					et $\frac{1}{|a|^n}>1$ donc d'après le premier point, $\frac1{|a|^n}\mathop{\longrightarrow}\limits_{n\rightarrow+\infty}+\infty$\\
					D'o\`u $|a|^n\mathop{\longrightarrow}\limits_{n\rightarrow+\infty}0$\\
					Ainsi, comme $|a|^n=|a^n|$, $a^n\mathop{\longrightarrow}\limits_{n\rightarrow+\infty}0$
				\item \underline{Si $a<-1$} alors $|a|^n\mathop{\longrightarrow}\limits_{n\rightarrow+\infty}+\infty$ et $a^n$ change de ighe en fonction de la parité de $n$.\\
					Donc $(a^n)_{n\in\N}$ diverge.
			\end{liste}
		\end{flushleft}
		\begin{defi}
			On appelle \underline{série géométrique} toute suite de terme général:\\
			$\systeme{& u_n=\sum\limits_{k=0}^na^k\\& n\in\N\\& a\in\R}$
		\end{defi}
		\begin{flushleft}
			\textbf{Étude de $u_n$}\\
			\begin{liste}
				\item \underline{Si $a=1$} alors $u_n=n+1$ et $u_n\mathop{\longrightarrow}\limits_{n\rightarrow+\infty}+\infty$
			\end{liste}
		\end{flushleft}
\end{document}