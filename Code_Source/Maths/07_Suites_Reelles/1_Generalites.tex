\documentclass[12pt,twoside,a4paper]{article}

\def\chapitre{Suites R\'eelles}
\author{MPSI 2}
\def\titre{G\'en\'eralit\'es}

\usepackage{amsfonts}
\usepackage{amsmath}
\usepackage{amsthm}
\usepackage{changepage}
\usepackage{color}
\usepackage{enumitem}
\usepackage{fancyhdr}
\usepackage{framed}
\usepackage[margin=1in]{geometry}
\usepackage{mathrsfs}
\usepackage{tikz, tkz-tab}
\usepackage{titling}

\newtheoremstyle{dotless}{}{}{\itshape}{}{\bfseries}{}{ }{}
\theoremstyle{dotless}

\newtheorem{defs}{Definition}[subsection]
\newenvironment{defi}{\definecolor{shadecolor}{RGB}{255,236,217}\begin{shaded}\begin{defs}\ \\}{\end{defs}\end{shaded}}

\newtheorem{pro}{Propriete}[subsection]
\newenvironment{prop}{\definecolor{shadecolor}{RGB}{230,230,255}\begin{shaded}\begin{pro}\ \\}{\end{pro}\end{shaded}}

\newtheorem{cor}{Corollaire}[subsection]
\newenvironment{coro}{\definecolor{shadecolor}{RGB}{245,250,255}\begin{shaded}\begin{cor}\ \\}{\end{cor}\end{shaded}}

\setlength{\droptitle}{-1in}
\predate{}
\postdate{}
\date{}
\title{\chapitre\\\titre\vspace{-.25in}}

\pagestyle{fancy}
\makeatletter
\lhead{\chapitre\ - \titre}
\rhead{\@author}
\makeatother

\newenvironment{preuve}{\begin{framed}\begin{proof}[\unskip\nopunct]}{\end{proof}\end{framed}}
\newenvironment{liste}{\begin{itemize}[leftmargin=*,noitemsep, topsep=0pt]}{\end{itemize}}
\newenvironment{tab}{\begin{adjustwidth}{.5cm}{}}{\end{adjustwidth}}

\newcommand{\uu}[1] {_{_{#1}}}
\newcommand{\lbracket}{[\![}
\newcommand{\rbracket}{]\!]}
\newcommand{\fonction}[5]{\begin{aligned}[t]#1\colon&#2&&\longrightarrow#3 \\&#4&&\longmapsto#5\end{aligned}}
\newcommand{\systeme}[1]{\left\{\begin{aligned}#1\end{aligned}\right.}
\newcommand{\cercle}[1]{\textcircled{\scriptsize{#1}}}

\newcommand{\lf}[1]{\left(#1\right)}
\newcommand{\C}{\mathbb{C}}
\newcommand{\R}{\mathbb{R}}
\newcommand{\K}{\mathbb{K}}
\newcommand{\N}{\mathbb{N}}
\newcommand{\I}{\mathcal{I}}
\newcommand{\F}{\mathcal{F}}
\newcommand{\E}{\mathcal{E}}
\newcommand{\G}{\mathcal{G}}
\newcommand{\et}{\text{ et }}
\newcommand{\ou}{\text{ ou }}
\newcommand{\xou}{\ \fbox{\text{ou}}\ }


\begin{document}
	\maketitle
	\section{Droite numérique achevée $\overline{\mathbb{R}}$}
		\begin{defi}
			On note $\overline{\mathbb{R}}$ la réunion de $\mathbb{R}$ et de deux éléments distincts: $-\infty$ et $+\infty$\\
			$\overline{\mathbb{R}}=\mathbb{R}\cup\{-\infty,+\infty\}$
		\end{defi}
		\begin{liste}
			\item On peut prolonger partiellement les lois internes $+$ et $\times$ \`a $\overline{\mathbb{R}}$, mais il existe des opération indéfinies.
			\item On peut prolonger la relation d'ordre naturelle de $\mathbb{R}$ \`a $\overline{\mathbb{R}}$: $\overline{\mathbb{R}}$ est ordonné.
		\end{liste}
		Utilisation: une suite tend vers un élément de $\overline{\mathbb{R}}$
	\section{Définitions}
		\begin{defi}
			On appelle suite réelle toute application $\fonction{\phi}{\mathbb{N}}{\mathbb{R}}{n}{\phi(n)}$
		\end{defi}
		\begin{flushleft}
			Une suite réelle est une famille d'éléments indexée par $\mathbb{N}$\\
			Notations: $\begin{aligned}& u_n=\phi(n)\\ & u=(u_n)_{n\in\mathbb{N}}=\phi \end{aligned}$
		\end{flushleft}
		\begin{defi}
			On appelle ensemble des valeurs de $(u_n)_{n\in\mathbb{N}}$ le sous ensemble de $\mathbb{R}$:
			$$A=\{x\in\mathbb{R},\ \exists n\in\mathbb{N},\ x=u_n \}$$
		\end{defi}
		\begin{defi}
			On dit que $u$ est une suite monotone si $\phi$ est monotone.\\
			De m\^eme avec croissante et décroissante.
		\end{defi}
		\begin{defi}
			On dit que $u$ est majorée si $A$ est majoré dans $\mathbb{R}$\\
			De m\^eme avec minorée et bornée.
		\end{defi}
	\section{Notations et limites}
		\subsection{Limites réelles}
			\begin{defi}
				Soit $u=(u_n)_{n\in\mathbb{R}}$ une suite réelle, et $l$ un réel.\\
				On dit que \underline{$u$ converge vers $l$} si pour tout intervalle $I$ centré en $l$, il existe un rang $n_0$ \`a partir duquel tous les $u_n$ sont dans $I$:
				$$\forall\epsilon\in\mathbb{R}^{+*},\ \exists n_0\in\mathbb{N},\ n\geqslant n_0\Rightarrow|u_n-l|<\epsilon$$
			\end{defi}
			\begin{prop}
				Si $u$ converge vers un $l$ réel, alors \underline{$l$ est unique.}
			\end{prop}
			\begin{preuve}
				Utiliser les définitions, raisonner par l'absurde avec $\epsilon=\frac{l_2-l_1}{2}$
			\end{preuve}
			%Feuille 8-2-1
\end{document}