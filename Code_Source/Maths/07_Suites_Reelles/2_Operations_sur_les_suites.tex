\documentclass[12pt,twoside,a4paper]{article}

\def\chapitre{Suites R\'eelles}
\author{MPSI 2}
\def\titre{Op\'erations sur les suites}

\usepackage{amsfonts}
\usepackage{amsmath}
\usepackage{amsthm}
\usepackage{changepage}
\usepackage{color}
\usepackage{enumitem}
\usepackage{fancyhdr}
\usepackage{framed}
\usepackage[margin=1in]{geometry}
\usepackage{mathrsfs}
\usepackage{tikz, tkz-tab}
\usepackage{titling}

\newtheoremstyle{dotless}{}{}{\itshape}{}{\bfseries}{}{ }{}
\theoremstyle{dotless}

\newtheorem{defs}{Definition}[subsection]
\newenvironment{defi}{\definecolor{shadecolor}{RGB}{255,236,217}\begin{shaded}\begin{defs}\ \\}{\end{defs}\end{shaded}}

\newtheorem{pro}{Propriete}[subsection]
\newenvironment{prop}{\definecolor{shadecolor}{RGB}{230,230,255}\begin{shaded}\begin{pro}\ \\}{\end{pro}\end{shaded}}

\newtheorem{cor}{Corollaire}[subsection]
\newenvironment{coro}{\definecolor{shadecolor}{RGB}{245,250,255}\begin{shaded}\begin{cor}\ \\}{\end{cor}\end{shaded}}

\setlength{\droptitle}{-1in}
\predate{}
\postdate{}
\date{}
\title{\chapitre\\\titre\vspace{-.25in}}

\pagestyle{fancy}
\makeatletter
\lhead{\chapitre\ - \titre}
\rhead{\@author}
\makeatother

\newenvironment{preuve}{\begin{framed}\begin{proof}[\unskip\nopunct]}{\end{proof}\end{framed}}
\newenvironment{liste}{\begin{itemize}[leftmargin=*,noitemsep, topsep=0pt]}{\end{itemize}}
\newenvironment{tab}{\begin{adjustwidth}{.5cm}{}}{\end{adjustwidth}}

\newcommand{\uu}[1] {_{_{#1}}}
\newcommand{\lbracket}{[\![}
\newcommand{\rbracket}{]\!]}
\newcommand{\fonction}[5]{\begin{aligned}[t]#1\colon&#2&&\longrightarrow#3 \\&#4&&\longmapsto#5\end{aligned}}
\newcommand{\systeme}[1]{\left\{\begin{aligned}#1\end{aligned}\right.}
\newcommand{\cercle}[1]{\textcircled{\scriptsize{#1}}}

\newcommand{\lf}[1]{\left(#1\right)}
\newcommand{\C}{\mathbb{C}}
\newcommand{\R}{\mathbb{R}}
\newcommand{\K}{\mathbb{K}}
\newcommand{\N}{\mathbb{N}}
\newcommand{\I}{\mathcal{I}}
\newcommand{\F}{\mathcal{F}}
\newcommand{\E}{\mathcal{E}}
\newcommand{\G}{\mathcal{G}}
\newcommand{\et}{\text{ et }}
\newcommand{\ou}{\text{ ou }}
\newcommand{\xou}{\ \fbox{\text{ou}}\ }


%Auteur: Clément Phan, MPSI 2

\begin{document}
	\maketitle
	\section{Structure d'alg\`ebre des suites}
		\begin{flushleft}
			Soit $\mathcal{E}$ l'ensemble des suites r\'eelles.\\
			Soit $(u_n)_{n\in\mathbb{N}}$ et $(v_n)_{n\in\mathbb{N}}$ deux suites r\'eelles.\\
		\end{flushleft}
		\begin{flushleft}
			\textbf{Addition}\\
			On pose $u+v=(u_n+v_n)_{n\in\mathbb{N}}$\\
			$(\mathcal{E},+)$ est un \underline{groupe ab\'elien} (commutatif):
			\begin{liste}
				\item Il poss\`ede un \'el\'ement neutre: $(0)_{n\in\mathbb{N}}$
				\item Il poss\`ede un \'el\'ement sym\'etrique: $-u_n=(-u_n)_{n\in\mathbb{N}}$
			\end{liste}
		\end{flushleft}
		\begin{flushleft}
			\textbf{Multiplication}\\
			On pose $u\times v=(u_n\times v_n)_{n\in\mathbb{N}}$\\
			\begin{liste}
				\item[$\times$] est associative
				\item[$\times$] est distributive sur $+$
				\item[$\times$] admet un élément neutre: $(1)_{n\in\mathbb{N}}$
			\end{liste}
			On dit que $(\mathcal{E},+,\times)$ est un \underline{anneau commutatif} (car pas complètement symétrique)
		\end{flushleft}
		\begin{flushleft}
			\textbf{Multiplication externe}\\
			On pose $\forall\lambda\in\mathbb{R},\ \lambda\cdot u=(\lambda\times u_n)_{n\in\mathbb{N}}$\\
			On a de plus $\lambda\cdot(u\times v)=(\lambda\cdot u)\times v=u\times(v\cdot\lambda)$\\
			On dit que $(\mathcal{E},+,\cdot)$ est un espace vectoriel.\\
			On dit que $(\mathcal{E},+,\times,\cdot)$ est une \underline{algèbre commutative}.
		\end{flushleft}
		\begin{prop}
			On note $\mathcal{E}_b$ l'ensemble des suites bornées.\\
			On note $\mathcal{E}_c$ l'ensemble des suites convergentes.\\
			On a:
			\begin{liste}
				\item$\mathcal{E}_c\subset\mathcal{E}_b\subset\mathcal{E}$
				\item$(\mathcal{E}_b,+,\times,\cdot)$ et $(\mathcal{E}_b,+,\times,\cdot)$ sont des algèbres commutatives.
			\end{liste}
		\end{prop}
		\newpage
		\begin{flushleft}
			\textbf{Limites Réelles}
		\end{flushleft}
		\begin{prop}
			\begin{liste}
				\item Si $u$ converge vers $l$ et $v$ converge vers $l'$, alors $u+v$ converge vers $l+l'$.
				\item Si $u$ converge vers $l$ et $v$ converge vers $l'$, alors $u\times v$ converge vers $l\times l'$.
				\item Si $u$ converge vers $l$ et $\lambda\in\mathbb{R}$, alors $\lambda\cdot u$ converge vers $\lambda\,l$
				\item Si $u$ converge vers $l\neq0$, alors il existe un rang $n_0$ a partir duquel $\left(\frac{1}{u_n}\right)_{n\geq n_0}$ ait un sens, et $\left(\frac{1}{u_n}\right)_{n\geq n_0}$ converge vers $\frac{1}{l}$
				\item Si $u$ converge vers $l$ et $v$ converge vers $l'\neq0$, alors il existe un rang $n_0$ a partir duquel $\left(\frac{u_n}{v_n}\right)_{n\geq n_0}$ ait un sens, et $\left(\frac{u_n}{v_n}\right)_{n\geq n_0}$ converge vers $\frac{l}{l'}$
			\end{liste}
		\end{prop}
		\begin{preuve}
			\begin{liste}
				\item\textbf{Premier point}:\\
					Utiliser les définitions des limites avec $\frac{\varepsilon}{2}$, \`a $\varepsilon$ fixé.\\
					Puis, avec l'addition des deux, utiliser l'inégalité triangulaire.
				\item\textbf{Deuxième point}:\\
					Soit $u$ et $v$ tendant vers $l$ et $l'$.\\
					Montrer que $u\,v\mathop{\longrightarrow}\limits_{n\rightarrow+\infty}l\,l'$\\
					Soit $\varepsilon$ un réel strictement positif.\\
					Montrer que $\exists n_0\in\mathbb{N},\ \forall n\in\mathbb{N},\ n\geqslant n_0\Rightarrow|_n\,v_n-l\,l'|<\varepsilon$\\
					De plus: $\begin{aligned}[t]
						|u_n\,v_n-l\,l'|&=|u_n\,v_n-u_n\,l'+u_n\,l'-l\,l'|\\
						                &=|u_n(v_n-l)+l'(u_n-l)|\\
						|u_n\,v_n-l\,l'|&\leqslant|u_n|\,|v_n-l|+|l'|\,|u_n-l|
					\end{aligned}$
					$u$ est convergente, donc bornée. Soit $M$ le majorant de $|u|$.\\
					Donc $|u_n\,v_n-l\,l'|\leqslant M\,|v_n-l|+|l'|\,|u_n-l|$\\
					On utilise ensuite la convergence de $u$ avec $\frac{\varepsilon}{2(|l'|+1)}$ et de $v$ avec $\frac{\varepsilon}{2(M+1)}$\\
					Donc: 
					$\begin{aligned}[t]
						&\exists n_1\in\mathbb{N},\ \forall n\in\mathbb{N},\ n\geqslant n_1\Rightarrow|u_n-l|<\frac{\varepsilon}{2(|l'|+1)}\\
						&\exists n_2\in\mathbb{N},\ \forall n\in\mathbb{N},\ n\geqslant n_2\Rightarrow|v_n-l|<\frac{\varepsilon}{2(M+1)}
					\end{aligned}$
					Soit $n_1$ et $n_2$ deux tels réels. Posons $n_0=\max(\{n_1,n_2\})$\\
					Donc $\forall n\in\mathbb{N},\ n\geqslant n_0\Rightarrow|_n\,v_n-l\,l'|<M\frac{\varepsilon}{2(M+1)}+|l'|\frac{\varepsilon}{2(|l'|+1)}<\frac{\varepsilon}{2}+\frac{\varepsilon}{2}<\varepsilon$\\\\
					Ce raisonnement étant vrai pour tout $\varepsilon$, $u\,v\mathop{\longrightarrow}\limits_{n\rightarrow+\infty}l\,l'$
				\item\textbf{Quatrième point}:\\
					Soit $u$ une suite tendant vers un réel $l$ différent de 0.\\
					\begin{liste}
						\item Démontrer l'existence de $\left(\frac{1}{u_n}\right)_{n\geqslant n_0}$\\
							$\exists n_2\in\mathbb{N},\ \forall n\in\mathbb{N},\ n\geqslant n_2\Rightarrow \frac{|l|}{2}<u_n<\frac{3\,|l|}{2}$\\
							Donc il existe un rang $n_0$ tel que l'inverse de $u_n$ soit défini.
						\item $\begin{aligned}[t]\text{Montrer que:}&\left(\frac{1}{u_n}\right)_{n\geqslant n_0}\text{converge vers}\frac{1}{l}\\
								\iff&\forall\varepsilon\in\mathbb{R}^{+*},\ \exists n_1\in\mathbb{N},\ \forall n\in\mathbb{N},\ n\geqslant n_1\Rightarrow\left|\frac{1}{u_n}-\frac{1}{l}\right|\end{aligned}$
							$\begin{aligned}\left|\frac{1}{u_n}-\frac{1}{l}\right|&=\left|\frac{l-u_n}{l\,u_n}\right|
								&=\frac{\left| l-u_n\right|}{\left| u_n\right| \left| l\right| }\end{aligned}$\\
							Or $\left| u_n\right| >\frac{\left| l\right| }{2}$\\
							Donc $\forall n\in\mathbb{N},\ n\geqslant n_0\Rightarrow\left|\frac{1}{u_n}-\frac{1}{l}\right|<\frac{\left| u_n-l\right| }{\frac{\left| l\right| }{2}\times\left| l\right| }\leqslant \frac{2}{\left| l\right| ^2}\left| u_n-l\right|$\\
							On applique la convergence de $u$ avec $\frac{\left| l\right| ^2}{2}\varepsilon$\\
							Donc $\exists n_1\in\mathbb{N},\ \forall n\in\mathbb{N},\ n\geqslant n_1\Rightarrow\left|\frac{1}{u_n}-\frac{1}{l}\right|< \frac{2}{\left| l\right| ^2}\times\frac{\left| l\right| ^2}{2}\varepsilon \leqslant \varepsilon$\\
							Donc il existe un rang $n_1$ a partir duquel $\left|\frac{1}{u_n}-\frac{1}{l}\right|$ est inférieur \`a $\varepsilon$\\
							Cela étant vrai pour tout $\varepsilon$, $\frac{1}{u_n}\mathop{\longrightarrow}\limits_{n\rightarrow+\infty}\frac{1}{l}$
					\end{liste}
			\end{liste}
		\end{preuve}
		\begin{coro}
			$\begin{aligned}u_n\mathop{\longrightarrow}\limits_{n\rightarrow+\infty}l&\iff u_n-l\mathop{\longrightarrow}\limits_{n\rightarrow+\infty}0\\
				&\iff \frac{u_n}{l}\mathop{\longrightarrow}\limits_{n\rightarrow+\infty}1\end{aligned}$
		\end{coro}
		\begin{flushleft}
			\textbf{Limites Infinies}
		\end{flushleft}
		\begin{prop}
			\underline{Somme}:
			\begin{liste}
				\item Si $u\mathop{\longrightarrow}\limits_{n\rightarrow+\infty}+\infty$
				\begin{liste}
					\item Si $v$ est minorée dans $\mathbb{R}$, $u+v\mathop{\longrightarrow}\limits_{n\rightarrow+\infty}+\infty$\\
						En particulier si $v\mathop{\longrightarrow}\limits_{n\rightarrow+\infty}+\infty$ ou si $v\mathop{\longrightarrow}\limits_{n\rightarrow+\infty}l$ avec $l\in\mathbb{R}$
					\item On ne peut pas conclure immédiatement si $v\mathop{\longrightarrow}\limits_{n\rightarrow+\infty}-\infty$
				\end{liste}
				\item De m\^eme avec $u\mathop{\longrightarrow}\limits_{n\rightarrow+\infty}-\infty$
			\end{liste}
			\underline{Produit}
			\begin{liste}
				\item Si $u\mathop{\longrightarrow}\limits_{n\rightarrow+\infty}+\infty$
				\begin{liste}
					\item Si $v$ n'admet pas $0$ pour limite, et si $v_n$ garde un signe constant a partir d'un certain rang, alors:\\
						Si $v_n<0,\ u_n\,v_n\mathop{\longrightarrow}\limits_{n\rightarrow+\infty}-\infty$\\
						Si $v_n>0,\ u_n\,v_n\mathop{\longrightarrow}\limits_{n\rightarrow+\infty}+\infty$
					\item Si $v_n\mathop{\longrightarrow}\limits_{n\rightarrow+\infty}0$, on ne peut pas conclure a priori,\\
					\textbf{Sauf si} $\exists n_0\in\N,\ \forall n\in\N,\ n\geqslant n_0\Rightarrow v_n=0$ Dans ce cas, $u_n\,v_n\mathop{\longrightarrow}\limits_{n\rightarrow+\infty}0$
				\end{liste}
				\item De m\^eme avec $u\mathop{\longrightarrow}\limits_{n\rightarrow+\infty}-\infty$
			\end{liste}
			\underline{Inverse}
			\begin{liste}
				\item Si $|u_n|\mathop{\longrightarrow}\limits_{n\rightarrow+\infty}+\infty$, alors il existe un rang $n_0$ a partir duquel $\left(\frac{1}{u_n}\right)_{n\geqslant n_0}$ existe et $\left(\frac{1}{u_n}\right)\mathop{\longrightarrow}\limits_{n\rightarrow+\infty}0$\\
					Cela s'applique aussi a $u_n\mathop{\longrightarrow}\limits_{n\rightarrow+\infty}-\infty$
				\item Si $u\mathop{\longrightarrow}\limits_{n\rightarrow+\infty}0$
				\begin{liste}
					\item Si a partir d'un certain rang $n_0$, tous les $u_n$ sont non nuls, alors $\left(\frac{1}{u_n}\right)_{n\geqslant n_0}$existe.
					\item Si de plus $u_n$ garde un signe constant a partir d'un rang $n_1\geqslant n_0$, $\left(\frac{1}{u_n}\right)\mathop{\longrightarrow}\limits_{n\rightarrow+\infty}\pm\infty$ selon le signe de $u_n$
				\end{liste}
			\end{liste}
		\end{prop}
		\begin{preuve}
			\textbf{Inverse}\\
			Montrer que si $u_n\mathop{\longrightarrow}\limits_{n\rightarrow+\infty}0$ et, \`a partir d'un certain rang, $u_n<0$, alors $\frac{1}{u_n}\mathop{\longrightarrow}\limits_{n\rightarrow+\infty}-\infty$\\
			\\
			Soit $u$ une suite convergeant vers $0$ et strictement négative a partir d'un rang $n_1$.\\
			Donc: $\forall\varepsilon\in\R^{+*},\ \exists n_2\in\N,\ \forall n\in\N,\ n\geqslant n_2\Rightarrow|u_n|<\varepsilon$\\
			Soit $n_2$ un tel réel, $\varepsilon$ un réel positif, et $n_3=\max(\{n_1,n_2\})$\\
			Pour tout $n$ supérieur a $n_3$, $\frac{1}{u_n}$ existe.\\
			Montrer que $\frac{1}{u_n}$ diverge vers $-\infty$\\
			Soit $K$ un réel.
			\begin{liste}
				\item Si $K\geqslant0$: $\forall n\in\N,\ n\geqslant n_3\Rightarrow \frac1{u_n}<0\geqslant K$
				\item Si $K<0$\\
					En appliquant la convergence de $u$ avec le réel $\frac1{|K|}$:\\
					$\exists n_0\in\N,\ \forall n\in\N,\ n\geqslant n_4\Rightarrow -\frac1{|K|}<u_n<\frac1{u_n}$\\
					Soit $n_0$ un tel réel.\\
					$\begin{aligned}
						\forall n\in\N,\ n\geqslant n_0&\Rightarrow \frac1{K}<u_n\\
							&\Rightarrow \frac{1}{u_n}<K
					\end{aligned}$
			\end{liste}
			Ce raisonnement étant valable pour tout $K$ réel, $\frac1{u_n}\mathop{\longrightarrow}\limits_{n\rightarrow+\infty}-\infty$
		\end{preuve}
		\begin{prop}
			Tout réel est limite d'une suite de rationnels.
		\end{prop}
		\begin{preuve}
			Montrer que $\forall x\in\R,\ \exists (q_n)_{n\in\N}\in\mathcal{E},\ (\forall n\in\N,\ q_n\in\mathbb{Q})\et(q_n\mathop{\longrightarrow}\limits_{n\rightarrow+\infty}x)$\\
			Soit $x$ un réel.
			\begin{liste}
				\item Existence de la suite:\\
					$\Q$
			\end{liste}
		\end{preuve}
\end{document}