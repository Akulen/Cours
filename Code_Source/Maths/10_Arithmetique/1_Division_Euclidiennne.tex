\documentclass[12pt,twoside,a4paper]{article}

\def\chapitre{Arithm\'etique}
\author{MPSI 2}
\def\titre{Division Euclidienne}

\usepackage{amsfonts}
\usepackage{amsmath}
\usepackage{amsthm}
\usepackage{changepage}
\usepackage{color}
\usepackage{enumitem}
\usepackage{fancyhdr}
\usepackage{framed}
\usepackage[margin=1in]{geometry}
\usepackage{mathrsfs}
\usepackage{tikz, tkz-tab}
\usepackage{titling}

\newtheoremstyle{dotless}{}{}{\itshape}{}{\bfseries}{}{ }{}
\theoremstyle{dotless}

\newtheorem{defs}{Definition}[subsection]
\newenvironment{defi}{\definecolor{shadecolor}{RGB}{255,236,217}\begin{shaded}\begin{defs}\ \\}{\end{defs}\end{shaded}}

\newtheorem{pro}{Propriete}[subsection]
\newenvironment{prop}{\definecolor{shadecolor}{RGB}{230,230,255}\begin{shaded}\begin{pro}\ \\}{\end{pro}\end{shaded}}

\newtheorem{cor}{Corollaire}[subsection]
\newenvironment{coro}{\definecolor{shadecolor}{RGB}{245,250,255}\begin{shaded}\begin{cor}\ \\}{\end{cor}\end{shaded}}

\setlength{\droptitle}{-1in}
\predate{}
\postdate{}
\date{}
\title{\chapitre\\\titre\vspace{-.25in}}

\pagestyle{fancy}
\makeatletter
\lhead{\chapitre\ - \titre}
\rhead{\@author}
\makeatother

\newenvironment{preuve}{\begin{framed}\begin{proof}[\unskip\nopunct]}{\end{proof}\end{framed}}
\newenvironment{liste}{\begin{itemize}[leftmargin=*,noitemsep, topsep=0pt]}{\end{itemize}}
\newenvironment{tab}{\begin{adjustwidth}{.5cm}{}}{\end{adjustwidth}}

\newcommand{\uu}[1] {_{_{#1}}}
\newcommand{\lbracket}{[\![}
\newcommand{\rbracket}{]\!]}
\newcommand{\fonction}[5]{\begin{aligned}[t]#1\colon&#2&&\longrightarrow#3 \\&#4&&\longmapsto#5\end{aligned}}
\newcommand{\systeme}[1]{\left\{\begin{aligned}#1\end{aligned}\right.}
\newcommand{\cercle}[1]{\textcircled{\scriptsize{#1}}}

\newcommand{\lf}[1]{\left(#1\right)}
\newcommand{\C}{\mathbb{C}}
\newcommand{\R}{\mathbb{R}}
\newcommand{\K}{\mathbb{K}}
\newcommand{\N}{\mathbb{N}}
\newcommand{\I}{\mathcal{I}}
\newcommand{\F}{\mathcal{F}}
\newcommand{\E}{\mathcal{E}}
\newcommand{\G}{\mathcal{G}}
\newcommand{\et}{\text{ et }}
\newcommand{\ou}{\text{ ou }}
\newcommand{\xou}{\ \fbox{\text{ou}}\ }


%Auteur: Tomas Rigaux, MPSI 2

\begin{document}
	\maketitle
	\begin{prop}
		Soit $a$ un entier naturel, $b$ un entier naturel non nul. \\
		Alors il existe un unique couple d'entiers naturels $q$ et $r$ tels que : \\
		\begin{tab}
			$\left\{\begin{aligned}&a=b*q+r\\
									&0\leq r<b\end{aligned}\right.$
		\end{tab}
		\begin{liste}
			\item $q$ est le \textbf{quotient} de la division euclidienne de $a$ par $b$.
			\item $r$ est le \textbf{reste} de la division euclidienne de $a$ par $b$.
		\end{liste}
	\end{prop}
	\begin{preuve}
		\cercle{1} \textbf{Existence}
		\begin{tab}
			\begin{liste}
				\item si $a=0,0=0*b+0$ \\
					Le couple $(q,r)=(0,0)$ convient.
				\item si $b=1,a=1*a+0$ \\
					Le couple $(q,r)=(a,0)$ convient.
				\item \textbf{Cas g\'eneral :} \\
					Supposons $a\geq 1$ et $b\geq 2$. \\
					\textbf{Cas 1 :} $a<b$
					\begin{tab}
						Alors $a=b*0+a$ \\
						Le couple $(0,a)$ convient.
					\end{tab}
					\textbf{Cas 2 :} $a\geq b$
					\begin{tab}
						Soit $E=\left\{q\in\N^*,b*q>a\right\}$
						\begin{liste}
							\item $E$ est une partie de $\N$.
							\item $E$ est non vide car $a\in E$ ($a\neq 0$ et $b\geq 2$).
						\end{liste}
						Donc $E$ admet un plus petit \'el\'ement not\'e $q_1$. \\
						$q1\in\N^*$ et $1\notin E$ car $a\geq b$. \\
						Donc $q_1\geq 2$. \\
						Ainsi, $q_1-1\in\N^*$. Posons $q_0=q_1-1$. \\
						On a : $q_0\in\N$ et $q_0<q_1$ donc $q_0\notin E$. \\
						On en deduit que $b*q_0\leq a$. \\
						Par ailleurs, $q_1\in E$, donc $a<bq_1$. \\
						Donc $bq_0\leq a<b(q_0+1)$. \\
						Posons $r=a-bq_0$. \\
						Donc $0\leq r<b$. \\
						\textbf{Conclusion :} Le couple $(q_0,r)$ satisfait : \\
						\begin{tab}
							$q_0\in\N,r\in\N,\left\{\begin{aligned}&a=b*q_0+r\\
																   &0\leq r<b\end{aligned}\right.$
						\end{tab}
					\end{tab}
			\end{liste}
		\end{tab}
		\cercle{2} \textbf{Unicit\'e}
		\begin{tab}
			Soit $a$ un entier naturel et $b$ un entier naturel non nul. \\
			Supposons qu'il existe deux couples d'entiers naturels $(q,r)$ et $(q',r')$ tels que :
			\begin{tab}
				$\left\{\begin{aligned}&a=bq+r \\
									   &a=bq'+r' \\
									   &0\leq r<b \\
									   &0\leq r'<b\end{aligned}\right.$
			\end{tab}
			Montrons que $(q,r)=(q',r')$. \\
			On remarque que : $b(q'-q)=r-r'$. \\
			Ainsi, $q=q'\iff r=r'$. \\
			Il suffit donc de montrer que $q=q'$. \\
			\fbox{HA} Supposons $q=q'$. Par exemple, $q<q'$.
			\begin{tab}
				Alors $q-q'>0$. \\
				Donc $q-q'\geq 1$ car $(q,q')\in\N^2$. \\
				On en deduit que $r-r'\geq b$. \\
				Par ailleurs, $\systeme{&0\leq r<b\\
										&0\leq r'<b}$. \\
				D'o\`u $-b<r-r<b$. \\
				Or $r-r'\geq b$, donc contradiction. \\
				Donc $q=q'$, d'o\`u $r=r'$.
			\end{tab}
			On a donc existence et unicit\'e de l'\'ecriture.
		\end{tab}
	\end{preuve}
	\textbf{Remarque :} Avec les notations de la d\'emonstration, on a : $bq_0\leq a<b(q_0+1)$. \\
	Ou encore, sachant $b\in\N^*$, $q_0\leq \frac{a}{b}<q_0+1$. \\
	Donc $q_0=\floor{\frac{a}{b}}$.
	\begin{coro}
		\cercle{1} Soit $a\in\Z$ et $b\in\N^*$, alors :
		\begin{tab}
			\begin{tab}
				$\exists!(q,r)\in\Z\times\N,\systeme{&a=bq+r\\
													 &0\leq r<b}$
			\end{tab}
		\end{tab}
		\cercle{2} Soit $a\in\Z$ et $b\in\Z*$, alors :
		\begin{tab}
			\begin{tab}
				$\exists!(q,r)\in\Z\times\N,\systeme{&a=bq+r\\
													 &0\leq r<|b|}$
			\end{tab}
		\end{tab}
	\end{coro}
\end{document}
