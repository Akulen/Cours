% !TeX encoding = UTF-8
\documentclass[12pt,twoside,a4paper]{article}


\def\chapitre{D\'erivabilit\'e}
\author{MPSI 2}
\def\titre{Th\'eor\`me de Rolle et applications}

\usepackage{amsfonts}
\usepackage{amsmath}
\usepackage{amsthm}
\usepackage{changepage}
\usepackage{color}
\usepackage{enumitem}
\usepackage{fancyhdr}
\usepackage{framed}
\usepackage[margin=1in]{geometry}
\usepackage{mathrsfs}
\usepackage{tikz, tkz-tab}
\usepackage{titling}

\newtheoremstyle{dotless}{}{}{\itshape}{}{\bfseries}{}{ }{}
\theoremstyle{dotless}

\newtheorem{defs}{Definition}[subsection]
\newenvironment{defi}{\definecolor{shadecolor}{RGB}{255,236,217}\begin{shaded}\begin{defs}\ \\}{\end{defs}\end{shaded}}

\newtheorem{pro}{Propriete}[subsection]
\newenvironment{prop}{\definecolor{shadecolor}{RGB}{230,230,255}\begin{shaded}\begin{pro}\ \\}{\end{pro}\end{shaded}}

\newtheorem{cor}{Corollaire}[subsection]
\newenvironment{coro}{\definecolor{shadecolor}{RGB}{245,250,255}\begin{shaded}\begin{cor}\ \\}{\end{cor}\end{shaded}}

\setlength{\droptitle}{-1in}
\predate{}
\postdate{}
\date{}
\title{\chapitre\\\titre\vspace{-.25in}}

\pagestyle{fancy}
\makeatletter
\lhead{\chapitre\ - \titre}
\rhead{\@author}
\makeatother

\newenvironment{preuve}{\begin{framed}\begin{proof}[\unskip\nopunct]}{\end{proof}\end{framed}}
\newenvironment{liste}{\begin{itemize}[leftmargin=*,noitemsep, topsep=0pt]}{\end{itemize}}
\newenvironment{tab}{\begin{adjustwidth}{.5cm}{}}{\end{adjustwidth}}

\newcommand{\uu}[1] {_{_{#1}}}
\newcommand{\lbracket}{[\![}
\newcommand{\rbracket}{]\!]}
\newcommand{\fonction}[5]{\begin{aligned}[t]#1\colon&#2&&\longrightarrow#3 \\&#4&&\longmapsto#5\end{aligned}}
\newcommand{\systeme}[1]{\left\{\begin{aligned}#1\end{aligned}\right.}
\newcommand{\cercle}[1]{\textcircled{\scriptsize{#1}}}

%Auteur: Tomas Rigaux, MPSI 2

\begin{document}
	\maketitle
	\section{Extremums relatifs}
		\begin{defi}
			Soit $I$ un intervalle, soit $x_0\in I$. \\
			Soit $f\colon R\longrightarrow\R$ une fonction num\'erique. \\
			On dit que $f$ pr\'esente un extremum relatif en $x_0$ sur $I$ si la quantit\'e $f\left(x\right)-f\left(x_0\right)$ garde un signe constant au voisinage de $x_0$.
		\end{defi}
		\begin{prop}
			Si $x_0$ n'est pas une borne de $I$, si $f$ pr\'esente un extremum relatif en $x_0$ et si $f$ est d\'erivable en $x_0$ alors $f'\left(x_0\right)=0$
		\end{prop}
	\section{Th\'eor\`eme de Rolle}
		\begin{flushleft}
			Soit $a$ et $b$ deux r\'eels tels que $a<b$.
		\end{flushleft}
		\begin{theo}{de Rolle}
			Soit $f$ une application continue sur $\left[a,b\right]$, d\'erivable sur $\left]a,b\right[$ et telle que $f\left(a\right)=f\left(b\right)$. \\
			Alors il existe $c$ appartenant \`a $\left]a,b\right[$ tel que  $f'\left(c\right)=0$.
		\end{theo}
	\section{Th\'eor\`eme des accroissements finis}
		\begin{theo}{des accroissements finis}
			Soit $f$ une application continue sur $\left[a,b\right]$ et d\'erivable sur $\left]a,b\right[$. \\
			Alors il existe $c$ appartenant \`a $\left]a,b\right[$ tel que : $f(b)-f(a)=f'(c)\times(b-a)$
		\end{theo}
	\section{Cons\'equences du Th\'eor\`eme des accroissements \\ finis}
		\subsection{In\'egalit\'e des accroissements finis}
			\begin{prop}
				Soit $f$ une application continue sur $\left[a,b\right]$, d\'erivable sur $\left]a,b\right[$ et de d\'eriv\'ee born\'ee sur $\left]a,b\right[$.
				$$
					\exists\left(k,K\right)\in\R^2,\forall x\in\left]a,b\right[,k\leq f'(x)\leq K
				$$
				Alors : $k\times(b-a)\leq f(b)-f(a)\leq K\times(b-a)$.
			\end{prop}
		\subsection{Th\'eor\`eme de prolongement}
			\begin{prop}
				Soit $f$ une fonction continue sur $\left[a,b\right]$, d\'erivable sur $\left]a,b\right[$. \\
				On suppose $f'$ continue sur $\left]a,b\right[$ et admet des limites finies en $a$ et en $b$. \\
				Alors $f$ est de classe $\mathcal{C}^1$ sur $\left[a,b\right]$.
			\end{prop}
		\subsection{Fonctions Lipschitziennes}
			Sous les hypoth\`eses de l'in\'egalit\'e des accroissements finis, $f$ est lipschitzienne.
		\subsection{Variations d'une fonction}
			\begin{prop}
				Soit $f$ une application continue sur $\left[a,b\right]$, d\'erivable sur $\left]a,b\right[$. \\
				Soit $\left[\alpha,\beta\right]\subset\left[a,b\right]$. Alors :
				\begin{liste}
					\item $f$ est croissante sur $\left[\alpha,\beta\right]$ si et seulement si $f'\geq 0$ sur $\left]\alpha,\beta\right[$.
					\item $f$ est d\'ecroissante sur $\left[\alpha,\beta\right]$ si et seulement si $f'\leq 0$ sur $\left]\alpha,\beta\right[$.
					\item $f$ est constante sur $\left[\alpha,\beta\right]$ si et seulement si $f'=0$ sur $\left]\alpha,\beta\right[$.
				\end{liste}
			\end{prop}
		\subsection{Application au calcul de primitive}
			\begin{defi}
				Soit $f$ une fonction num\'erique. On dit que $F$ est une primitive de $f$ sur $[a,b]$ si $F$ est une fonction d\'erivable sur $[a,b]$ et si $F'=f$ sur $[a,b]$.
			\end{defi}
			\begin{prop}
				Soit $F_1$ et $F_2$ deux primitives de $f$ sur $[a,b]$. Alors :
				$$
					\exists\lambda\in\R,\forall x\in[a,b],F_2(x)-F_1(x)=\lambda
				$$
			\end{prop}
\end{document}
