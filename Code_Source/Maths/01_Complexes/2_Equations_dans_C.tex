\documentclass[12pt,a4paper]{article}

\def\chapitre{Complexes}
\author{MPSI 2}
\def\titre{Equations dans $\mathbb{C}$}

\usepackage{amsfonts}
\usepackage{amsmath}
\usepackage{amsthm}
\usepackage{changepage}
\usepackage{color}
\usepackage{enumitem}
\usepackage{fancyhdr}
\usepackage{framed}
\usepackage[margin=1in]{geometry}
\usepackage{mathrsfs}
\usepackage{tikz, tkz-tab}
\usepackage{titling}

\newtheoremstyle{dotless}{}{}{\itshape}{}{\bfseries}{}{ }{}
\theoremstyle{dotless}

\newtheorem{defs}{Definition}[subsection]
\newenvironment{defi}{\definecolor{shadecolor}{RGB}{255,236,217}\begin{shaded}\begin{defs}\ \\}{\end{defs}\end{shaded}}

\newtheorem{pro}{Propriete}[subsection]
\newenvironment{prop}{\definecolor{shadecolor}{RGB}{230,230,255}\begin{shaded}\begin{pro}\ \\}{\end{pro}\end{shaded}}

\newtheorem{cor}{Corollaire}[subsection]
\newenvironment{coro}{\definecolor{shadecolor}{RGB}{245,250,255}\begin{shaded}\begin{cor}\ \\}{\end{cor}\end{shaded}}

\setlength{\droptitle}{-1in}
\predate{}
\postdate{}
\date{}
\title{\chapitre\\\titre\vspace{-.25in}}

\pagestyle{fancy}
\makeatletter
\lhead{\chapitre\ - \titre}
\rhead{\@author}
\makeatother

\newenvironment{preuve}{\begin{framed}\begin{proof}[\unskip\nopunct]}{\end{proof}\end{framed}}
\newenvironment{liste}{\begin{itemize}[leftmargin=*,noitemsep, topsep=0pt]}{\end{itemize}}
\newenvironment{tab}{\begin{adjustwidth}{.5cm}{}}{\end{adjustwidth}}

\newcommand{\uu}[1] {_{_{#1}}}
\newcommand{\lbracket}{[\![}
\newcommand{\rbracket}{]\!]}
\newcommand{\fonction}[5]{\begin{aligned}[t]#1\colon&#2&&\longrightarrow#3 \\&#4&&\longmapsto#5\end{aligned}}
\newcommand{\systeme}[1]{\left\{\begin{aligned}#1\end{aligned}\right.}
\newcommand{\cercle}[1]{\textcircled{\scriptsize{#1}}}

\newcommand{\lf}[1]{\left(#1\right)}
\newcommand{\C}{\mathbb{C}}
\newcommand{\R}{\mathbb{R}}
\newcommand{\K}{\mathbb{K}}
\newcommand{\N}{\mathbb{N}}
\newcommand{\I}{\mathcal{I}}
\newcommand{\F}{\mathcal{F}}
\newcommand{\E}{\mathcal{E}}
\newcommand{\G}{\mathcal{G}}
\newcommand{\et}{\text{ et }}
\newcommand{\ou}{\text{ ou }}
\newcommand{\xou}{\ \fbox{\text{ou}}\ }


\begin{document}
	\maketitle
	
	\section{Racine carree d'un complexe}
		\subsection{Methode trigonometrique}
			Soit $z\uu0$ un complexe non nul. Resolvons $z^2 = z\uu0$ \\
			Notons $z=\rho e^{i\theta}$ et $z\uu0=\rho\uu0e^{i\alpha}$ \\ \\
			$$\begin{aligned}
				z=z\uu0&\iff\left\{\begin{aligned}				
					&\rho^2=\rho\uu0 \\
					&2\theta\equiv\alpha\ [2\pi]\end{aligned}\right. \\
				&\iff\left\{\begin{aligned}
					&\rho=\sqrt{\rho\uu0} \\
					\exists k\in\mathbb{Z},\ &\theta=\frac{\alpha}{2}+k\pi
					\end{aligned}\right. \\
				&\iff z=\sqrt{\rho\uu0}e^{i\frac{\alpha}{2}}\text{ ou }z=-\sqrt{\rho\uu0}e^{i\frac{\alpha}{2}}
			\end{aligned}$$
			Les solutions sont opposees \\
		\subsection{Methode Algebrique}
			Notons $z=x+iy$ et $z\uu0=a+ib$. Resolvons $z^2=z\uu0$ \\
			$$\begin{aligned}
				z^2=z\uu0 &\iff x^2+2ixy-y^2=a+ib \\
				&\iff \left\{\begin{aligned}
					x^2-y^2&=a \\
					2xy&=b
				\end{aligned}\right. \\
				&\iff \left\{\begin{aligned}
					x^2-y^2&=a \\
					-x^2y^2&=\frac{-b}{4} \\
					2xy&=b
				\end{aligned}\right. \\
				&\iff \left\{\begin{aligned}
					&x^2\text{ et }y^2\text{ sont les racines du polynome }X^2-aX-\frac{b^2}{4} \\
					&2xy=b
				\end{aligned}\right.
			\end{aligned}$$ \\
	\section{Equation du 2$^{\text{nd}}$ degre}
		Resolution de $az^2+bz+c=0$ avec $\iff \left\{\begin{aligned}(a,b,c)\in\mathbb{C}^2\\a\neq0\end{aligned}\right.$\\
		$$\begin{aligned}
			az^2+bz+c=a\left[\left(z-\frac{b}{2a}\right)^2-\frac{b^2}{4a^2}+\frac{c}{a}\right] \\
			az^2+bz+c=0\iff\left(z-\frac{b}{2a}\right)^2=\frac{b^2}{4a^2}+\frac{c}{a}
		\end{aligned}$$
		On pose $\Delta=b^2-4ac$ et $\delta=\sqrt{\Delta}$ \\
		$$\begin{aligned}
			az^2+bz+c=0 &\iff \left(z+\frac{b}{2a}\right)=\frac{\Delta}{4a^2} \\
			&\iff z=\frac{-b-\delta}{2a}\text{ ou }z=\frac{-b+\delta}{2a}
		\end{aligned}$$
		De plus, $\begin{aligned}[t]\text{produit des racines}&=\frac{c}{a}\\\text{somme des racines}&=-\frac{b}{a}\end{aligned}$ \\
	\section{Resolution d'equations du type $z^n=a$}
		\subsection{Racines$^n$ de l'unite}
			\begin{defi}
				Soit $n$ un entier naturel non nul.\\
				Les racines$^n$ de l'unite sont les solutions de l'equation $z^n=1$
			\end{defi}
			Cas particuliers :
			\begin{liste}
				\item$n=2\ \iff\omega\uu0=1$ ou $\omega\uu1=-1$
				\item$n=3\ \iff\omega\uu0=1$ ou $\omega\uu1=e^{i\frac{2\pi}{3}}=j$ ou $\omega\uu2=j=\bar{j}$
				\item$n=4\ \iff\omega\uu0=1$ ou $\omega\uu1=i$ ou $\omega\uu3=-1$ ou $\omega\uu3=-i$\\
			\end{liste}
			On note $U_{n}=\left\{\omega\uu{k},\ \forall k\in\lbracket0;n-1\rbracket\right\}$. $U$ muni de la multiplication est un groupe cyclique car $\omega\uu1$ engendre le groupe.
			\begin{prop}
				\begin{liste}
					\item$\forall n\in\mathbb{N},\ U_n=\left\{e^{i\frac{2k\pi}{n}},\ k\in\lbracket0;n-1\rbracket\right\}$ \\
					\item $U$ est l'ensemble des racines$^n$ de l'unite
					\item Les images M$_k$ affixes de $\omega\uu{k}$ forment un polygone regulier a n cotes.
				\end{liste}
			\end{prop}
			\begin{preuve}
				Etudions la position relative de M$_{n-1}$ par rapport a M$_k$
				$$\begin{aligned}
					\forall k\in\lbracket0;n-1\rbracket, \omega\uu{k+1}&=e^{i\frac{2(k+1)\pi}{n}} \\
					&=\omega\uu{k}\ e^{i\frac{2\pi}{n}} \\
					\text{et : }\omega\uu0&=\omega\uu{n-1}\ e^{i\frac{2\pi}{n}} \\\\
				\end{aligned}$$
				$$\begin{aligned}
					\omega\uu{k+1}=\omega\uu{k}\ e^{i\frac{2\pi}{n}}
					&\iff\left\{\begin{aligned}
						&\left|\omega\uu{k+1}\right|=\left|\omega\uu{k}\right| \\
						&arg\left(\frac{\omega\uu{k+1}}{\omega\uu{k}}\right)\equiv\frac{2\pi}{n}\ \left[2\pi\right]
					\end{aligned}\right.\\
					&\iff\left\{\begin{aligned}
						&\left|\omega\uu{k+1}\right|=\left|\omega\uu{k}\right| \\
						&mes\left(\overrightarrow{\text{OM}}_{k}\ ;\overrightarrow{\text{OM}}_{k+1}\right)\equiv\frac{2\pi}{n}\ \left[2\pi\right]
					\end{aligned}\right.
				\end{aligned}$$
				\begin{tab}
					M$_{k+1}$ est donc l'image de M$_k$ par la rotation d'angle $\frac{2\pi}{n}$ autour du centre O. \\
					De meme, M$_0$ est l'image de M$_{n-1}$ par la meme rotation. \\\\
					On en deduit que, pour tout $k$, le triangle OM$_k$M$_{k+1}$ a pour image par cette rotation le triangle OM$_{k+1}$M$_{k+2}$\\
					En particulier, $\left|\left|\overrightarrow{\text{M}_k\text{M}_{k+1}}\right|\right|=\left|\left|\overrightarrow{\text{M}_{k+1}\text{M}_{k+2}}\right|\right|$. \\\\
				\end{tab}
				\textbf{Conclusion:} Donc M$_0$M$_1$...M$_{n-1}$ est un polygone regulier.
			\end{preuve}
			\begin{prop}
				Le polygone regulier a n cotes est symetrique par rapport a l'axe reel.\\
				Les $\omega\uu{k}$ sont deux a deux conjugues.
			\end{prop}
			\begin{preuve}
				Pour cette demonstation, $n\in\mathbb{N}^*$ et $\left(k;k'\right)\in\lbracket0;n-1\rbracket^2$ \\
				$$\begin{aligned}
					\left(\mathcal{S}\right):\ \omega\uu{k}=\overline{\omega\uu{k'}}&\iff e^{i\frac{2k\pi}{n}}=e^{i\frac{2k'\pi}{n}} \\
					&\iff\frac{2k\pi}{n}\equiv\frac{2k'\pi}{n}\ \left[2\pi\right] \\
					&\iff\left\{\begin{aligned}
						\exists p\in\mathbb{Z},\ k+k'=np\\
						0\leq k+k'\leq2n-2
					\end{aligned}\right.
				\end{aligned}$$ \\
				Si $p\notin\left\{0;1\right\}$, le systeme est incompatible. \\
				On a donc :
				$$\begin{aligned}
					\mathcal{S}\iff&\left\{\begin{aligned}
						&k=-k'\\
						&\left(k;k'\right)\in\lbracket0;n-1\rbracket^2
					\end{aligned}\right.
					&&\text{  ou  }
					&&\left\{\begin{aligned}
						&k=-k'+n\\
						&\left(k;k'\right)\in\lbracket0;n-1\rbracket^2
					\end{aligned}\right. \\
					\iff &k=k'=0
					&&\text{  ou  }
					&&\ \ k'=n-k
				\end{aligned}$$
			\end{preuve}
			\begin{prop}
				Soit n un entier naturel superieur ou egal a 1. \\
				Alors la somme des racines$^n$ de l'unite vaut 0
			\end{prop}
			\begin{preuve}
				Somme triviale des termes d'une suite geometrique avec $k$ allant de $0$ a $n-1$
			\end{preuve}
		\subsection{Racines$^n$ d'un complexe $a$}
			Resolution de $z^n=a$ avec $a$ et $z$ non nuls:\\
			\begin{tab}
				Formes trigonometriques: $z=\rho e^{i\theta}$ et $a=\rho\uu0e^{i\alpha}$
				$$\begin{aligned}
					z^n=a&\iff \systeme{\rho^n&=\rho\uu0\\ n\theta&\equiv\alpha\ \left[2\pi\right]}\\
					&\iff\systeme{&\rho=\rho\uu0^{\frac{1}{n}}\\&\exists k\in\mathbb{Z},\ \theta\equiv\frac{a}{n}+\frac{2k\pi}{n}}\\
					&\iff\exists k\in\lbracket0;n-1\rbracket,\ z=\rho\uu0^\frac{1}{n}\ e^{i\left(\frac{\alpha}{n}+\frac{2k\pi}{n}\right)} \\
					&\iff\exists k\in\lbracket0;n-1\rbracket,\ z=\rho\uu0^\frac{1}{n}\ e^{i\frac{\alpha}{n}}\ \omega\uu{k}
				\end{aligned}$$
			\end{tab}
			L'ensemble des solutions realise une bijection sur $U_n$.\\
			Les affixes des solutions forment un polygone regulier obtenu a par une similitude du polygone regulier a n cotes.
			
\end{document}
