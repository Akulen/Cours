\documentclass[12pt,twoside,a4paper]{article}

\def\chapitre{Fractions Rationnelles}
\author{MPSI 2}
\def\titre{Fonctions fraction Rationnelles}

\usepackage{amsfonts}
\usepackage{amsmath}
\usepackage{amsthm}
\usepackage{changepage}
\usepackage{color}
\usepackage{enumitem}
\usepackage{fancyhdr}
\usepackage{framed}
\usepackage[margin=1in]{geometry}
\usepackage{mathrsfs}
\usepackage{tikz, tkz-tab}
\usepackage{titling}

\newtheoremstyle{dotless}{}{}{\itshape}{}{\bfseries}{}{ }{}
\theoremstyle{dotless}

\newtheorem{defs}{Definition}[subsection]
\newenvironment{defi}{\definecolor{shadecolor}{RGB}{255,236,217}\begin{shaded}\begin{defs}\ \\}{\end{defs}\end{shaded}}

\newtheorem{pro}{Propriete}[subsection]
\newenvironment{prop}{\definecolor{shadecolor}{RGB}{230,230,255}\begin{shaded}\begin{pro}\ \\}{\end{pro}\end{shaded}}

\newtheorem{cor}{Corollaire}[subsection]
\newenvironment{coro}{\definecolor{shadecolor}{RGB}{245,250,255}\begin{shaded}\begin{cor}\ \\}{\end{cor}\end{shaded}}

\setlength{\droptitle}{-1in}
\predate{}
\postdate{}
\date{}
\title{\chapitre\\\titre\vspace{-.25in}}

\pagestyle{fancy}
\makeatletter
\lhead{\chapitre\ - \titre}
\rhead{\@author}
\makeatother

\newenvironment{preuve}{\begin{framed}\begin{proof}[\unskip\nopunct]}{\end{proof}\end{framed}}
\newenvironment{liste}{\begin{itemize}[leftmargin=*,noitemsep, topsep=0pt]}{\end{itemize}}
\newenvironment{tab}{\begin{adjustwidth}{.5cm}{}}{\end{adjustwidth}}

\newcommand{\uu}[1] {_{_{#1}}}
\newcommand{\lbracket}{[\![}
\newcommand{\rbracket}{]\!]}
\newcommand{\fonction}[5]{\begin{aligned}[t]#1\colon&#2&&\longrightarrow#3 \\&#4&&\longmapsto#5\end{aligned}}
\newcommand{\systeme}[1]{\left\{\begin{aligned}#1\end{aligned}\right.}
\newcommand{\cercle}[1]{\textcircled{\scriptsize{#1}}}

\newcommand{\lf}[1]{\left(#1\right)}
\newcommand{\C}{\mathbb{C}}
\newcommand{\R}{\mathbb{R}}
\newcommand{\K}{\mathbb{K}}
\newcommand{\N}{\mathbb{N}}
\newcommand{\I}{\mathcal{I}}
\newcommand{\F}{\mathcal{F}}
\newcommand{\E}{\mathcal{E}}
\newcommand{\G}{\mathcal{G}}
\newcommand{\et}{\text{ et }}
\newcommand{\ou}{\text{ ou }}
\newcommand{\xou}{\ \fbox{\text{ou}}\ }


\DeclareMathOperator{\PGCD}{pgcd}

%Auteur: Tomas Rigaux, MPSI 2

\begin{document}
	\maketitle
	Soit $F=\frac PQ$ o\`u $(P,Q)\in\K[X]\times\K[X]^*,P\wedge Q=1$ \\
	On note $p=\deg(Q)(\in\N)$. \\
	On pose $\mathcal{D}_F\left\{x\in\K,\widetilde Q(x)\neq0\right\}$ \\
	$\mathcal D_F$ est l'ensemble $\K$ prive d'un nombre fini d'\'el\'ements.
	\begin{defi}
		\begin{liste}
			\item On note $\fonction{\widetilde{F}}{\mathcal{D}_F}{\R}{x}{\frac{\widetilde{P}(x)}{\widetilde{Q}(x)}}$ \\
				$\widetilde F$ s'appelle la fonction fraction rationnelle associ\'ee \`a $F$.
			\item Un \'el\'ement $x\in\K\setminus\mathcal D_F$ est un p\^ole de F.
			\item Un \'el\'ement $x\in\mathcal D_F,\widetilde P(x)=0$ s'appelle une racine de $F$.
		\end{liste}
	\end{defi}
	\begin{prop}
		On note $E$ l'ensemble des fraction rationnelles \`a coefficients dans $\K$. \\
		Soit $\fonction{\psi}{\K(X)}{E}{F}{\widetilde F}$
		\begin{liste}
			\item $\psi$ est surjective.
			\item Si $\K$ est un corps infini, alors $\phi$ est injective.
		\end{liste}
	\end{prop}
	\begin{preuve}
		\begin{liste}
			\item $\psi$ est surjective par d\'efinition de $E$.
			\item Pour montrer que si $E$ est infini, $\psi$ est injective, on utilise le fait que $\frac{\widetilde P_1}{\widetilde Q_1}=\frac{\widetilde P_2}{\widetilde Q_2}\iff\widetilde P_1\widetilde Q_2=\widetilde P_2\widetilde Q_1$ puis on montre que $P_1Q_2-P_2Q_1$ est le polynome nul en montrant qu'il poss\`ede une infinit\'e de racines.
		\end{liste}
	\end{preuve}
\end{document}
