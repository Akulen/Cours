% !TeX encoding = UTF-8
\documentclass[12pt,twoside,a4paper]{article}


\def\chapitre{Espaces Vectoriels de Dimension finie}
\author{MPSI 2}
\def\titre{Sous-Espaces Vectoriels}

\usepackage{amsfonts}
\usepackage{amsmath}
\usepackage{amsthm}
\usepackage{changepage}
\usepackage{color}
\usepackage{enumitem}
\usepackage{fancyhdr}
\usepackage{framed}
\usepackage[margin=1in]{geometry}
\usepackage{mathrsfs}
\usepackage{tikz, tkz-tab}
\usepackage{titling}

\newtheoremstyle{dotless}{}{}{\itshape}{}{\bfseries}{}{ }{}
\theoremstyle{dotless}

\newtheorem{defs}{Definition}[subsection]
\newenvironment{defi}{\definecolor{shadecolor}{RGB}{255,236,217}\begin{shaded}\begin{defs}\ \\}{\end{defs}\end{shaded}}

\newtheorem{pro}{Propriete}[subsection]
\newenvironment{prop}{\definecolor{shadecolor}{RGB}{230,230,255}\begin{shaded}\begin{pro}\ \\}{\end{pro}\end{shaded}}

\newtheorem{cor}{Corollaire}[subsection]
\newenvironment{coro}{\definecolor{shadecolor}{RGB}{245,250,255}\begin{shaded}\begin{cor}\ \\}{\end{cor}\end{shaded}}

\setlength{\droptitle}{-1in}
\predate{}
\postdate{}
\date{}
\title{\chapitre\\\titre\vspace{-.25in}}

\pagestyle{fancy}
\makeatletter
\lhead{\chapitre\ - \titre}
\rhead{\@author}
\makeatother

\newenvironment{preuve}{\begin{framed}\begin{proof}[\unskip\nopunct]}{\end{proof}\end{framed}}
\newenvironment{liste}{\begin{itemize}[leftmargin=*,noitemsep, topsep=0pt]}{\end{itemize}}
\newenvironment{tab}{\begin{adjustwidth}{.5cm}{}}{\end{adjustwidth}}

\newcommand{\uu}[1] {_{_{#1}}}
\newcommand{\lbracket}{[\![}
\newcommand{\rbracket}{]\!]}
\newcommand{\fonction}[5]{\begin{aligned}[t]#1\colon&#2&&\longrightarrow#3 \\&#4&&\longmapsto#5\end{aligned}}
\newcommand{\systeme}[1]{\left\{\begin{aligned}#1\end{aligned}\right.}
\newcommand{\cercle}[1]{\textcircled{\scriptsize{#1}}}

\newcommand{\lf}[1]{\left(#1\right)}
\newcommand{\C}{\mathbb{C}}
\newcommand{\R}{\mathbb{R}}
\newcommand{\K}{\mathbb{K}}
\newcommand{\N}{\mathbb{N}}
\newcommand{\I}{\mathcal{I}}
\newcommand{\F}{\mathcal{F}}
\newcommand{\E}{\mathcal{E}}
\newcommand{\G}{\mathcal{G}}
\newcommand{\et}{\text{ et }}
\newcommand{\ou}{\text{ ou }}
\newcommand{\xou}{\ \fbox{\text{ou}}\ }


%Auteur: Cl\'ement Phan, MPSI 2

\begin{document}
	\maketitle
	\section{Dimension de sous-espaces vectoriels}
		Soit $E$ un $\Kev$ de dimension finie.
		\begin{prop}
			Soit $F$ un $\Sev$ de $E$.\\
			Alors:
			\begin{liste}
				\item $F$ est de dimension finie.
				\item $\dim(F)\leqslant\dim(E)$
				\item $F=E\iff \dim(F)=dim(E)$
			\end{liste}
		\end{prop}
		\begin{preuve}
			On raisonne sur une base de $F$.
		\end{preuve}
	\section{Somme de sous-espaces vectoriels}
		Soit $E$ un $\Kev$ de dimension finie.\\
		Soient $F$ et $G$ deux $\Sev$ de $E$.
		\begin{defi}
			On appelle \underline{somme de $F$ et $G$} le sous-espace vectoriel engendr\'e par $F\cup G$
		\end{defi}
		\begin{flushleft}
			\textbf{Notation:} $F+G=\vect(F\cup G)$
		\end{flushleft}
		\begin{prop}
			$$F+G=\{x\in E,\ \exists(x_F,x_G)\in F\times G,\ x=x_F+x_G \}$$
		\end{prop}
		\newpage
		\begin{prop}
			$$\dim(F+G)=\dim(F)+\dim(G)-\dim(F\cap G)$$
		\end{prop}
		\begin{preuve}
			On raisonne avec les bases de $F$, $G$, et $F\cap G$, et avec le th\'eor\`eme de la base incompl\`ete sur $F\cap G$.
		\end{preuve}
	\section{Somme directe, espaces suppl\'ementaires}
		\begin{flushleft}
			Soit $E$ un $\Kev$ de dimension $n$.
		\end{flushleft}
		\begin{defi}
			Soient $F$ et $G$ deux $\Sev$ de $E$.
			\begin{liste}
				\item La somme $F+G$ est \underline{directe} si $F\cap G=\{0_E\}$
				\item $F$ et $G$ sont \underline{suppl\'ementaires de $E$} si $F\oplus G=E$
			\end{liste}
		\end{defi}
		\begin{flushleft}
			\textbf{Notation:} Somme directe de $F$ et $G$: $F\oplus G$
		\end{flushleft}
		\begin{prop}
			$\fonction{\varphi_1}{F\times G}{F+G}{(x_F,x_G)}{x_F+x_G}$\\
			$\varphi_1$ est lin\'eaire et surjective.\\
			\begin{liste}
				\item $F$ et $G$ sont en somme directe ssi $\varphi_1$ est injective.
				\item $F$ et $G$ sont en somme directe ssi tout \'el\'ement de $F+G$ s'\'ecrit comme mani\`ere unique comme CL d'\'el\'ements de $F$ et de $G$.
			\end{liste}
		\end{prop}
		\begin{defi}
			$$\sum\limits_{i=1}^pE_i=\vect\left( \bigcup\limits_{i=1}^pE_i\right) $$
		\end{defi}
		\begin{defi}
			Soit $\fonction{\varphi}{E_1\times...\times E_p}{E_1+...+E_p}{(x_1,\,...\,,\,x_p)}{x_1+...+x_p}$\\
			La somme $\sum\limits_{i=1}^pE_i$ est directe ssi $\varphi$ est injective, c'est \`a dire si tout \'el\'ement de $E_1+...+E_p$ s'\'ecrit comme CL unique d'\'el\'ements de $\{E_1\times...\times E_p\}$.
		\end{defi}
		\begin{flushleft}
			\textbf{Notation:} $\bigoplus\limits_{i=1}^pE_i$
		\end{flushleft}
		\begin{prop}
			$F+G$ est une somme directe ssi la r\'eunion d'une base de $F$ et d'une base de $G$ est une base de $F+G$.
		\end{prop}
		\begin{coro}
			$$\dim(F\oplus G)=\dim(F)+\dim(G)$$
		\end{coro}
		\begin{coro}
			\begin{liste}
				\item Si $F$ et $G$ sont suppl\'ementaires de $E$, alors $\dim(F)+\dim(G)=\dim(E)$
				\item Tous les $\Sev$ suppl\'ementaires de $F$ dans $E$ sont de dimension $\dim(E)-\dim(F)$
				\item Tous les $\Sev$ suppl\'ementaires de $F$ dans $E$ sont isomorphes.
			\end{liste}
		\end{coro}
		\begin{flushleft}
			\textbf{Crit\`eres de suppl\'ementarit\'e}
			$$ \left\lbrace \begin{aligned}
			& F\cap G=\{O_E\}\\ & F+G=E
			\end{aligned}\right. \ou 
			\left\lbrace \begin{aligned}
			& \dim(F)+\dim(G)=\dim(E)\\& F+G=E
			\end{aligned}\right. \ou
			\left\lbrace \begin{aligned}
			& \dim(F)+\dim(G)=\dim(E)\\ & F\cap G=\{0_E \}
			\end{aligned}\right. $$
		\end{flushleft}
		\begin{prop}
			\textbf{Existence d'un suppl\'ementaire d'un $\Sev$ de $E$}\\
			Soit $F$ un $\Sev$ de $E$.\\
			$F$ admet au moins un $\Sev$ suppl\'ementaire dans $E$.
		\end{prop}
		\begin{preuve}
			On utilise le th\'eor\`eme de la base incompl\`ete, et on forme l'espace vectoriel engendr\'e par les termes ajout\'es \`a la base de $F$.\\
			On a donc l'union des deux bases \'egale a une base de $E$.
		\end{preuve}
		\begin{defi}
			Soit $F$ et $G$ deux $\Sev$ suppl\'ementaires de $E$.\\
			On a: $\forall x\in E,\ \exists (x_F,x_G)\in F\times G,\ uniques,\ x=x_F+x_G$
			\begin{liste}
				\item $\fonction{P_F}{E}{E}{x}{x_F}$ est la \underline{projection sur $F$ parall\`element \`a $G$}.
				\item $\fonction{S_F}{E}{E}{x}{x_F-x_G}$ est la \underline{sym\'etrie par rapport \`a $F$ et parall\`element \`a $G$}.
			\end{liste}
		\end{defi}
		\begin{prop}
			\begin{liste}
				\item $P_F$ est lin\'eaire, $P_F\circ P_F=P_F$, $\mathrm{Im}(P_F)=F$, $\ker(P_F)=G$
				\item $S_F$ est lin\'eaire, $S_F\circ S_F=\mathrm{Id}_F$, $\mathrm{Im}(S_F)=E$, $\ker(P_F)=\{0_E\}$
			\end{liste}
		\end{prop}
	\section{Th\'eor\`eme du Rang}
		\begin{theo}{du Rang}
			Soit $E$ un $\Kev$ de dimension $n$.\\
			Soit $F$ un $\Kev$.\\
			Soit $f:E\rightarrow F$ lin\'eaire.\\
			On a alors:
			$$
			\underbrace{\dim(\mathrm{Im}(f))}_{\text{Rang de }f}+\dim(\ker(f))=n
			$$
		\end{theo}
		\begin{preuve}
			\begin{liste}
				\item Soit $(e_1,\,...\,,\,e_n)$ une base de $E$.\\
					Soit $(\varepsilon_1,\,...\,,\,\varepsilon_k)$ une base de $\ker(f)$.\\
					D'apr\`es le th\'eor\`eme de la base incompl\`ete:\\
					il existe $i_1,\,...\,,\,i_{n-k}$ indices distincts de $\lbracket1,n\rbracket$ tels que $(\varepsilon_1,\,...\,,\,\varepsilon_k\,,e_{i_1},\,...\,,\,e_{i_{n-k}})$ soit une base de $E$.\\
					Soit $E_1=\vect(\{e_{i_1},\,...\,,\,e_{i_{n-k}}\})$\\
					Alors $E=\ker(f)\oplus E_1$, et $\restr{f}{E_1}$ est injective.
				\item $\mathrm{Im}(\restr{f}{E_1})=\vect(\{f(e_{i_1}),\,...\,,\,f(e_{i_{n-k}})\})$ car $\{e_{i_1},\,...\,,\,e_{i_{n-k}}\}$ est g\'en\'erateur de $E_1$.\\
					De plus, $\{e_{i_1},\,...\,,\,e_{i_{n-k}}\}$ est libre \underline{et} $f$ est injective, donc $\{f(e_{i_1}),\,...\,,\,f(e_{i_{n-k}})\}$ est libre.\\
					\textbf{Ccl:} $\{f(e_{i_1}),\,...\,,\,f(e_{i_{n-k}})\}$ est une base de $\mathrm{Im}(\restr{f}{E_1})$
				\item $\begin{aligned}[t]
					\mathrm{Im}(f) & =\vect(\{\underbrace{\varepsilon_1}_{=0_E},\,...\,,\,\underbrace{\varepsilon_k}_{=0_E}\,,e_{i_1},\,...\,,\,e_{i_{n-k}}\})\\
						& =\vect(\{e_{i_1},\,...\,,\,e_{i_{n-k}} \}) \\
						& =\mathrm{Im}(\restr{f}{E_1})
					\end{aligned}$\\
					\textbf{Ccl:} $\mathrm{Im}(f)=\mathrm{Im}(\restr{f}{E_1})$
			\end{liste}
			\ \\
			Donc $\begin{aligned}[t]
				\dim(\mathrm{Im}(f)) & = \dim(\mathrm{Im}(\restr{f}{E_1}))\\
					& = n-k\\
					& = n-\dim(\ker(f))
			\end{aligned}$\\\\
			\textbf{Conclusion g\'en\'erale:} $\dim(\mathrm{Im}(f)) +\dim(\ker(f))=n$
		\end{preuve}
\end{document} %ACCENTS!!!
