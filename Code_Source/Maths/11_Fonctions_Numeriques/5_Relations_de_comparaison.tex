% !TeX encoding = UTF-8
\documentclass[12pt,twoside,a4paper]{article}


\def\chapitre{Fonctions Num\'eriques}
\author{MPSI 2}
\def\titre{Relations de comparaison}

\usepackage{amsfonts}
\usepackage{amsmath}
\usepackage{amsthm}
\usepackage{changepage}
\usepackage{color}
\usepackage{enumitem}
\usepackage{fancyhdr}
\usepackage{framed}
\usepackage[margin=1in]{geometry}
\usepackage{mathrsfs}
\usepackage{tikz, tkz-tab}
\usepackage{titling}

\newtheoremstyle{dotless}{}{}{\itshape}{}{\bfseries}{}{ }{}
\theoremstyle{dotless}

\newtheorem{defs}{Definition}[subsection]
\newenvironment{defi}{\definecolor{shadecolor}{RGB}{255,236,217}\begin{shaded}\begin{defs}\ \\}{\end{defs}\end{shaded}}

\newtheorem{pro}{Propriete}[subsection]
\newenvironment{prop}{\definecolor{shadecolor}{RGB}{230,230,255}\begin{shaded}\begin{pro}\ \\}{\end{pro}\end{shaded}}

\newtheorem{cor}{Corollaire}[subsection]
\newenvironment{coro}{\definecolor{shadecolor}{RGB}{245,250,255}\begin{shaded}\begin{cor}\ \\}{\end{cor}\end{shaded}}

\setlength{\droptitle}{-1in}
\predate{}
\postdate{}
\date{}
\title{\chapitre\\\titre\vspace{-.25in}}

\pagestyle{fancy}
\makeatletter
\lhead{\chapitre\ - \titre}
\rhead{\@author}
\makeatother

\newenvironment{preuve}{\begin{framed}\begin{proof}[\unskip\nopunct]}{\end{proof}\end{framed}}
\newenvironment{liste}{\begin{itemize}[leftmargin=*,noitemsep, topsep=0pt]}{\end{itemize}}
\newenvironment{tab}{\begin{adjustwidth}{.5cm}{}}{\end{adjustwidth}}

\newcommand{\uu}[1] {_{_{#1}}}
\newcommand{\lbracket}{[\![}
\newcommand{\rbracket}{]\!]}
\newcommand{\fonction}[5]{\begin{aligned}[t]#1\colon&#2&&\longrightarrow#3 \\&#4&&\longmapsto#5\end{aligned}}
\newcommand{\systeme}[1]{\left\{\begin{aligned}#1\end{aligned}\right.}
\newcommand{\cercle}[1]{\textcircled{\scriptsize{#1}}}

%Auteur: Cl\'ement Phan, MPSI 2

\begin{document}
	\maketitle
	Soit $x_0$ un \'el\'ement de $\overline{\R}$.
	\begin{defi}
		Soient $f$ et $g$ deux fonctions d\'efinies sur un voisinage $V$ de $x_0$, sauf \'eventuellement en $x_0$.\\
		On dit que \underline{$f$ est n\'egligeable devant $g$} si il existe une application $\varepsilon:V\rightarrow\R^+$ telle que:
		$$\systeme{& \varepsilon(x)\mathop{\longrightarrow}\limits_{\substack{x\rightarrow x_0\\x\in V}}0\\&\forall x\in V\setminus \{x_0\},\ |f(x)|<\varepsilon(x)\,|g(x)| }$$
	\end{defi}
	\begin{flushleft}
		\underline{Notation:} $f(x)\mathop{=}\limits_{x\rightarrow x_0} o(g(x))$
	\end{flushleft}
	\begin{flushleft}
		\underline{R\'esultats usuels:}
		\begin{liste}
			\item $\forall \alpha\in\R^{+*},\ \ln(x)\mathop{=}\limits_{\substack{x\rightarrow 0\\x>0}}o(\frac1{x^\alpha})$\\\ \\
			\item $\forall \alpha\in\R^{+*},\ \ln(x)\mathop{=}\limits_{x\rightarrow +\infty} o(x^\alpha)$\\\ \\
			\item $\forall \alpha\in\R^{+*},\ x^\alpha\mathop{=}\limits_{x\rightarrow +\infty} o(e^x)$
		\end{liste}
	\end{flushleft}
	\begin{defi}
		Soient $f$ et $g$ deux fonctions d\'efinies sur un voisinage $V$ de $x_0$, sauf \'eventuellement en $x_0$.\\
		On dit que \underline{$f$ est domin\'ee par $g$} si:
		$$\exists M\in\R^{+*},\ \forall x\in V\setminus \{x_0\},\ |f(x)|<M\,|g(x)|$$
	\end{defi}
	\begin{flushleft}
		\underline{Notation:} $f(x)\mathop{=}\limits_{x\rightarrow x_0} O(g(x))$
	\end{flushleft}
	\begin{defi}
		Soient $f$ et $g$ deux fonctions d\'efinies sur un voisinage $V$ de $x_0$, sauf \'eventuellement en $x_0$.\\
		On dit que \underline{$f$ et $g$ sont \'equivalentes en $x_0$} si il existe une application $h:V\rightarrow\R$ telle que:
		$$\systeme{& h(x)\mathop{\longrightarrow}\limits_{\substack{x\rightarrow x_0\\x\in V}}1\\&\forall x\in V\setminus \{x_0\},\ f(x)=h(x)\,g(x) }$$
	\end{defi}
	\begin{flushleft}
		\underline{Notation:} $f(x)\mathop{\sim}\limits_{x\rightarrow x_0}g(x)$
	\end{flushleft}
	\newpage
	\begin{flushleft}
		\underline{R\'esultats usuels en $0$:}
		\begin{liste}
			\item $\sin(x)\sim x,\ \ \ \ \ \cos(x)\sim 1,\ \ \ \ \  1-\cos(x)\sim \frac{x^2}2,\ \ \ \ \  \tan(x)\sim x$
			\item $\arcsin \sim x,\ \ \ \ \ \arctan(x)\sim x$
			\item $\ln(1+x)\sim x,\ \ \ \ \ (1+x)^\alpha\sim 1,\ \ \ \ \ (1+x)^\alpha-1\sim\alpha\,x $
			\item $\text{sh}(x)\sim x,\ \ \ \ \ \text{ch}(x)\sim 1,\ \ \ \ \ \text{th}(x)\sim 1,\ \ \ \ \ \text{ch}(x)-1\sim \frac{x^2}2$
		\end{liste}
	\end{flushleft}
	\begin{prop}
		\textbf{Op\'erations sur les \'equivalences}\\
		Soient $f$ et $g$ deux fonctions d\'efinies au voisinage de $x_0$ telles que $f(x)\mathop{\sim}\limits_{x\rightarrow x_0}g(x)$.
		\begin{liste}
			\item Si $\phi$ est une fonction ne s'annulant pas au voisinage de $x_0$, sauf \'eventuellement en $x_0$,\\
				Alors $f(x)\,\phi(x)\mathop{\sim}\limits_{x\rightarrow x_0}g(x)\,\phi(x)$
			\item Si $f_1(x)\mathop{\sim}\limits_{x\rightarrow x_0}g_1(x)$, alors $f(x)\,f_1(x)\mathop{\sim}\limits_{x\rightarrow x_0}g(x)\,g_2(x)$
			\item $\forall n\in\N^{*},(f(x))^n\mathop{\sim}\limits_{x\rightarrow x_0}(g(x))^n$
			\item Si $f_1$ ne s'annule pas au voisinage de $x_0$, sauf \'eventuellement en $x_0$, et si $f_1(x)\mathop{\sim}\limits_{x\rightarrow x_0}g_1(x)$\\
				Alors $\frac{f(x)}{f_1(x)}\mathop{\sim}\limits_{x\rightarrow x_0}\frac{g(x)}{g_1(x)}$
			\item Si $f$ ne s'annule pas au voisinage de $x_0$, alors $\forall n\in\Z,\ (f(x))^n\mathop{\sim}\limits_{x\rightarrow x_0}(g(x))^n $
		\end{liste}
	\end{prop}
	\begin{prop}
		\begin{liste}
			\item Si $f(x)\mathop{\sim}\limits_{x\rightarrow x_0}g(x)$ et si $f(x)\mathop{\longrightarrow}\limits_{x\rightarrow x_0}l\in\overline{\R}$,\\
				Alors $g(x)\mathop{\longrightarrow}\limits_{x\rightarrow x_0}l$
			\item Si $f(x)\mathop{\longrightarrow}\limits_{x\rightarrow x_0}l$ et $g(x)\mathop{\longrightarrow}\limits_{x\rightarrow x_0}l$ et $lin\R^{*}$,\\
				Alors $f(x)\mathop{\sim}\limits_{x\rightarrow x_0}g(x)$ 
		\end{liste}
	\end{prop}
\end{document}