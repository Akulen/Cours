\documentclass[12pt,twoside,a4paper]{article}

\usepackage[francais]{babel}
\usepackage[ansinew]{inputenc}

\def\chapitre{D\'enombrements}
\author{MPSI 2}
\def\titre{Proprit\'es des coefficients binomiaux}

\usepackage{amsfonts}
\usepackage{amsmath}
\usepackage{amsthm}
\usepackage{changepage}
\usepackage{color}
\usepackage{enumitem}
\usepackage{fancyhdr}
\usepackage{framed}
\usepackage[margin=1in]{geometry}
\usepackage{mathrsfs}
\usepackage{tikz, tkz-tab}
\usepackage{titling}

\newtheoremstyle{dotless}{}{}{\itshape}{}{\bfseries}{}{ }{}
\theoremstyle{dotless}

\newtheorem{defs}{Definition}[subsection]
\newenvironment{defi}{\definecolor{shadecolor}{RGB}{255,236,217}\begin{shaded}\begin{defs}\ \\}{\end{defs}\end{shaded}}

\newtheorem{pro}{Propriete}[subsection]
\newenvironment{prop}{\definecolor{shadecolor}{RGB}{230,230,255}\begin{shaded}\begin{pro}\ \\}{\end{pro}\end{shaded}}

\newtheorem{cor}{Corollaire}[subsection]
\newenvironment{coro}{\definecolor{shadecolor}{RGB}{245,250,255}\begin{shaded}\begin{cor}\ \\}{\end{cor}\end{shaded}}

\setlength{\droptitle}{-1in}
\predate{}
\postdate{}
\date{}
\title{\chapitre\\\titre\vspace{-.25in}}

\pagestyle{fancy}
\makeatletter
\lhead{\chapitre\ - \titre}
\rhead{\@author}
\makeatother

\newenvironment{preuve}{\begin{framed}\begin{proof}[\unskip\nopunct]}{\end{proof}\end{framed}}
\newenvironment{liste}{\begin{itemize}[leftmargin=*,noitemsep, topsep=0pt]}{\end{itemize}}
\newenvironment{tab}{\begin{adjustwidth}{.5cm}{}}{\end{adjustwidth}}

\newcommand{\uu}[1] {_{_{#1}}}
\newcommand{\lbracket}{[\![}
\newcommand{\rbracket}{]\!]}
\newcommand{\fonction}[5]{\begin{aligned}[t]#1\colon&#2&&\longrightarrow#3 \\&#4&&\longmapsto#5\end{aligned}}
\newcommand{\systeme}[1]{\left\{\begin{aligned}#1\end{aligned}\right.}
\newcommand{\cercle}[1]{\textcircled{\scriptsize{#1}}}

\newcommand{\lf}[1]{\left(#1\right)}
\newcommand{\C}{\mathbb{C}}
\newcommand{\R}{\mathbb{R}}
\newcommand{\K}{\mathbb{K}}
\newcommand{\N}{\mathbb{N}}
\newcommand{\I}{\mathcal{I}}
\newcommand{\F}{\mathcal{F}}
\newcommand{\E}{\mathcal{E}}
\newcommand{\G}{\mathcal{G}}
\newcommand{\et}{\text{ et }}
\newcommand{\ou}{\text{ ou }}
\newcommand{\xou}{\ \fbox{\text{ou}}\ }


%Auteur: Cl\'ement Phan, MPSI 2

\begin{document}
	\maketitle
	\begin{prop}
		$\sum\limits_{k=0}^{n}\binom{n}{k}=2^{n}$
	\end{prop}
	\begin{prop}
		\textbf{Triangle de Pascal}\\
		$\binom{n+1}{p}=\binom{n}{p}+\binom{n}{p-1}$
	\end{prop}
	\begin{prop}
		\textbf{G\'en\'eralisation du triangle de Pascal}\\
		$\sum\limits_{k=p}^{n}\binom{k}{p}=\binom{n+1}{p+1}$
	\end{prop}
	\begin{prop}
		\textbf{Formule du bin\^ome}\\
		Soit $A$ est un anneau, et $a$ et $b$ deux \'el\'ements de $A$ qui commutent.\\
		Soit $n\in\N$.\\
		Alors $(a+b)^{n}=\sum\limits^{n}_{k=0}\binom{n}{k}a^{k}b^{n-k}$
	\end{prop}
\newpage
	\begin{prop}
		\textbf{Formule de Vandermonde}\\
		$\binom{n+m}{p}=\sum\limits^{p}_{k=0}\binom{n}{k}\binom{m}{p-k}$
	\end{prop}
	\begin{preuve}
		\begin{liste}
			\item $\binom{n+m}{p}$ est le coefficient du terme de degr\'e $p$ dans $(1+x)^{n+m}$
			\item $\begin{aligned}[t]
				(1+x)^{n+m} & = (1+x)^{n}\,(1+x)^{m}\\
				& = \left(\sum\limits_{k=0}^{n}\binom{n}{x}x^{k} \right)\,\left(\sum\limits_{k=0}^{m}\binom{n}{x}x^{k} \right)\\
				& = \!\!\!\!\!\!\sum\limits_{(i,j)\in \lbracket 0,n\rbracket\times \lbracket 0,m\rbracket}\!\!\!\!\!\! \binom{n}{i}\,\binom{m}{j}\,x^{i}\,x^{j}
			\end{aligned}$\\
			On veut proc\'eder par identification, on r\'e\'ecrit donc la somme:\\
			$\begin{aligned}[t]
				(1+x)^{n+m} & =\sum\limits_{p=0}^{n+m}\sum\limits^{p}_{k=0}\binom{n}{k}\binom{m}{p-k} x^{p}\\
				& = \sum\limits_{p=0}^{n+m} \left( \sum\limits^{p}_{k=0}\binom{n}{k}\binom{m}{p-k}\right)  x^{p}
			\end{aligned}$\\
			D'o\`u le coefficient du terme de degr\'e $p$.\\
			N.B. Les termes limites de la somme sont nuls.
		\end{liste}
	\end{preuve}
\end{document} %ACCENTS