\documentclass[12pt,twoside,a4paper]{article}

\def\chapitre{Structures Alg\'ebriqes}
\author{MPSI 2}
\def\titre{Lois Internes}

\usepackage{amsfonts}
\usepackage{amsmath}
\usepackage{amsthm}
\usepackage{changepage}
\usepackage{color}
\usepackage{enumitem}
\usepackage{fancyhdr}
\usepackage{framed}
\usepackage[margin=1in]{geometry}
\usepackage{mathrsfs}
\usepackage{tikz, tkz-tab}
\usepackage{titling}

\newtheoremstyle{dotless}{}{}{\itshape}{}{\bfseries}{}{ }{}
\theoremstyle{dotless}

\newtheorem{defs}{Definition}[subsection]
\newenvironment{defi}{\definecolor{shadecolor}{RGB}{255,236,217}\begin{shaded}\begin{defs}\ \\}{\end{defs}\end{shaded}}

\newtheorem{pro}{Propriete}[subsection]
\newenvironment{prop}{\definecolor{shadecolor}{RGB}{230,230,255}\begin{shaded}\begin{pro}\ \\}{\end{pro}\end{shaded}}

\newtheorem{cor}{Corollaire}[subsection]
\newenvironment{coro}{\definecolor{shadecolor}{RGB}{245,250,255}\begin{shaded}\begin{cor}\ \\}{\end{cor}\end{shaded}}

\setlength{\droptitle}{-1in}
\predate{}
\postdate{}
\date{}
\title{\chapitre\\\titre\vspace{-.25in}}

\pagestyle{fancy}
\makeatletter
\lhead{\chapitre\ - \titre}
\rhead{\@author}
\makeatother

\newenvironment{preuve}{\begin{framed}\begin{proof}[\unskip\nopunct]}{\end{proof}\end{framed}}
\newenvironment{liste}{\begin{itemize}[leftmargin=*,noitemsep, topsep=0pt]}{\end{itemize}}
\newenvironment{tab}{\begin{adjustwidth}{.5cm}{}}{\end{adjustwidth}}

\newcommand{\uu}[1] {_{_{#1}}}
\newcommand{\lbracket}{[\![}
\newcommand{\rbracket}{]\!]}
\newcommand{\fonction}[5]{\begin{aligned}[t]#1\colon&#2&&\longrightarrow#3 \\&#4&&\longmapsto#5\end{aligned}}
\newcommand{\systeme}[1]{\left\{\begin{aligned}#1\end{aligned}\right.}
\newcommand{\cercle}[1]{\textcircled{\scriptsize{#1}}}

\newcommand{\lf}[1]{\left(#1\right)}
\newcommand{\C}{\mathbb{C}}
\newcommand{\R}{\mathbb{R}}
\newcommand{\K}{\mathbb{K}}
\newcommand{\N}{\mathbb{N}}
\newcommand{\I}{\mathcal{I}}
\newcommand{\F}{\mathcal{F}}
\newcommand{\E}{\mathcal{E}}
\newcommand{\G}{\mathcal{G}}
\newcommand{\et}{\text{ et }}
\newcommand{\ou}{\text{ ou }}
\newcommand{\xou}{\ \fbox{\text{ou}}\ }


%Auteur: Cl\'ement Phan, MPSI 2

\begin{document}
	\maketitle
	\section{D\'efinitions}
		\begin{defi}
			On appelle \underline{loi interne sur E} toute application $f$ d\'efinie sur $E\times E$ dans $E$:\\
			$$\fonction{f}{E\times E}{E}{(x,y)}{f(x,y)}$$
		\end{defi}
		\begin{flushleft}
			\textbf{Notation:} $f(x,y)=f*y$
		\end{flushleft}

	\section{Propri\'et\'es des lois internes}
		\begin{defi}
			Soit $E$ un ensemble non vide mini d'une loi interne $*$. $(E,*)$ est un magma.\\
			$*$ est une loi associative si $\forall (x,y,z)\in\E\times E\times E,\ (x*y)*z=x*(y*z)$
		\end{defi}
		\begin{prop}
			Soit $*$ une \underline{loi associative} sur $E$, et $\{x_1,x_2,..,x_n\}$ un sous-ensemble a $n$ \'el\'ements de $E$.
			\begin{liste}
				\item $x_1*...*x_n$ est d\'efini par r\'ecurrence: $(x_1*...*x_{n-1})*x_n=x_1*...*x_n$
				\item $x_1*...*x_n=(x_1*...*x_{n_1})*(x_{n_{1}+1}*...*x_{n_2})*...*(x_{n_p}*...*x_n)$\\est d\'efini avec $p$ parenth\`eses.
			\end{liste}
		\end{prop}
		\begin{defi}
			$*$ est une \underline{loi commutative} sur $E$ si $\forall (x,y)\in E\times E,\ x*y=y*x$
		\end{defi}
		\begin{defi}
			On dit que $*$ admet un \underline{\'el\'ement neutre} not\'e $e$ si $\forall x\in\E,\ x*e=x\et e*x=x$
		\end{defi}
		\begin{prop}
			Si $*$ admet un \'el\'ement neutre, alors il est unique
		\end{prop}
		\begin{preuve}
			Soit $e_1$ et $e_2$ deux \'el\'ements neutres de $(E,*)$.\\
			alors $\begin{aligned}[t]
				& e_1*e_2=e_1\text{ car }e_1\text{ est un \'el\'emnet neutre}\\
				& e_1*e_2=e_2\text{ car }e_2\text{ est un \'el\'emnet neutre}
			\end{aligned}$\\
			Or, $*$ est une application, donc $e_1=e_2$
		\end{preuve}
		\begin{defi}
			Soit $(E,*)$ un magma. Si $*$ est associative et admet un \'el\'ement neutre dans $E$, alors $(E,*)$ est un \underline{mono\"ide}
		\end{defi}
		\begin{flushleft}
			\textbf{Notations:}
			Soit $(E,*)$ un mono\"ide.\\
			\underline{Notation multiplicative:}
			\begin{liste}
				\item $x*y=x\times y$
				\item L'\'el\'ement neutre est not\'e $1$ ou $1_E$
				\item $x_1*...*x_n=\prod\limits_{i=1}^{n}x_i$\\
				\item S'utilise en g\'en\'eral lorsque $*$ n'est pas commutative.
			\end{liste}
			\underline{Notation additive:}
			\begin{liste}
				\item $x*y=x+y$
				\item L'\'el\'ement neutre est not\'e $0$ ou $0_E$
				\item $x_1*...*x_n=\sum\limits_{i=1}^{n}x_i$\\
				\item S'utilise en g\'en\'eral lorsque $*$ est commutative.
			\end{liste}
		\end{flushleft}
		\begin{defi}
			Soit $(E,*)$ un mono\"ide d'\'el\'ement neutre $e$. Soit $x$ un \'el\'ement de $E$.\\
			$x$ est \underline{sym\'etrisable} dans $E$ si il existe un \'el\'ement $y$ de $E$ tel que:
			$$x*y=e\et y*x=e$$
		\end{defi}
		\begin{prop}
			Si $x$ est sym\'etrisable, alors son sym\'etrique est unique.
		\end{prop}
		\newpage
		\begin{flushleft}
			\textbf{Notation:}
			\begin{liste}
				\item Multiplicative: le sym\'etrique de $x$ s'appelle inverse et se note $x^{-1}$
				\item Additive: le sym\'etrique de $x$ s'appelle oppos\'e et se note $-x$
			\end{liste}
		\end{flushleft}
		\begin{flushleft}
			\textbf{Remarque:}
			Soit $(E,*)$ un mono\"ide.\\
			Si $x$ et $y$ sont sym\'etrisables de sym\'etriques $x'$ et $y'$\\
			Alors $x*y$ est sym\'etrisable de sym\'etrique $y'*x'$
		\end{flushleft}
\end{document}