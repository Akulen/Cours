\documentclass[12pt,twoside,a4paper]{article}

\def\chapitre{Structures Alg\'ebriqes}
\author{MPSI 2}
\def\titre{Structure de Groupe}

\usepackage{amsfonts}
\usepackage{amsmath}
\usepackage{amsthm}
\usepackage{changepage}
\usepackage{color}
\usepackage{enumitem}
\usepackage{fancyhdr}
\usepackage{framed}
\usepackage[margin=1in]{geometry}
\usepackage{mathrsfs}
\usepackage{tikz, tkz-tab}
\usepackage{titling}

\newtheoremstyle{dotless}{}{}{\itshape}{}{\bfseries}{}{ }{}
\theoremstyle{dotless}

\newtheorem{defs}{Definition}[subsection]
\newenvironment{defi}{\definecolor{shadecolor}{RGB}{255,236,217}\begin{shaded}\begin{defs}\ \\}{\end{defs}\end{shaded}}

\newtheorem{pro}{Propriete}[subsection]
\newenvironment{prop}{\definecolor{shadecolor}{RGB}{230,230,255}\begin{shaded}\begin{pro}\ \\}{\end{pro}\end{shaded}}

\newtheorem{cor}{Corollaire}[subsection]
\newenvironment{coro}{\definecolor{shadecolor}{RGB}{245,250,255}\begin{shaded}\begin{cor}\ \\}{\end{cor}\end{shaded}}

\setlength{\droptitle}{-1in}
\predate{}
\postdate{}
\date{}
\title{\chapitre\\\titre\vspace{-.25in}}

\pagestyle{fancy}
\makeatletter
\lhead{\chapitre\ - \titre}
\rhead{\@author}
\makeatother

\newenvironment{preuve}{\begin{framed}\begin{proof}[\unskip\nopunct]}{\end{proof}\end{framed}}
\newenvironment{liste}{\begin{itemize}[leftmargin=*,noitemsep, topsep=0pt]}{\end{itemize}}
\newenvironment{tab}{\begin{adjustwidth}{.5cm}{}}{\end{adjustwidth}}

\newcommand{\uu}[1] {_{_{#1}}}
\newcommand{\lbracket}{[\![}
\newcommand{\rbracket}{]\!]}
\newcommand{\fonction}[5]{\begin{aligned}[t]#1\colon&#2&&\longrightarrow#3 \\&#4&&\longmapsto#5\end{aligned}}
\newcommand{\systeme}[1]{\left\{\begin{aligned}#1\end{aligned}\right.}
\newcommand{\cercle}[1]{\textcircled{\scriptsize{#1}}}

\newcommand{\lf}[1]{\left(#1\right)}
\newcommand{\C}{\mathbb{C}}
\newcommand{\R}{\mathbb{R}}
\newcommand{\K}{\mathbb{K}}
\newcommand{\N}{\mathbb{N}}
\newcommand{\I}{\mathcal{I}}
\newcommand{\F}{\mathcal{F}}
\newcommand{\E}{\mathcal{E}}
\newcommand{\G}{\mathcal{G}}
\newcommand{\et}{\text{ et }}
\newcommand{\ou}{\text{ ou }}
\newcommand{\xou}{\ \fbox{\text{ou}}\ }


%Auteur: Cl\'ement Phan, MPSI 2

\begin{document}
	\maketitle
	\section{D\'efinition}
		\begin{defi}
			Soit $(G,*)$ un magma.\\
			On dit que $(G,*)$ est un \underline{groupe} si:
			\begin{liste}
				\item $*$ est associative
				\item $*$ admet un \'el\'ement neutre
				\item tout \'el\'ement de $G$ est sym\'etrisable par $*$
			\end{liste}
			Si de plus, $*$ esr commutative sur $G$, on dit que $(G,*)$ est un \underline{groupe ab\'elien}.
		\end{defi}
		\begin{flushleft}
			\textbf{Cons\'equences:}
			\begin{liste}
				\item\underline{R\`egle de simplification:} $a*x=a*y\Rightarrow x=y$
					D\'emonstration:
					\begin{tab}
						Soit $a$, $x$ et $y$ trois \'el\'ements de $G$ tels que $a*x=a*y$\\
						Notons $a'$ le sym\'etrique de $a$ (car $G$ est un groupe)\\
						On a alors: $a'*(a*x)=a'*(a*y)$\\
						Par associativit\'e, on a: $(a'*a)*x=(a'*a)*y$\\
						Par sym\'etrie, on a: $e*x=e*y$\\
						Par d\'efinition de l'\'el\'ement neutre: $x=y$						
					\end{tab}
				\item\underline{R\'esolution d'\'equations:} $a*x=b\iff x=a'*b$
			\end{liste}
		\end{flushleft}
	\section{Sous-groupes}
		\subsection{D\'efinition et crit\`eres}
			\begin{flushleft}
				Soit $(G,*)$ un groupe.
			\end{flushleft}
			\begin{defi}
				Soit $F$ un sous-ensemble de $G$\\
				On dit que $(F,*)$ est un sous-groupe de $(G,*)$ si:
				\begin{liste}
					\item $\forall (x,y)\in G\times G,\ (x\in F\et y\in F)\Rightarrow (x*y\in F)$
					\item $(F,*')$ est un groupe o\`u $*'$ est la loi induite de $G$ sur $F$.
				\end{liste}
			\end{defi}
			\begin{flushleft}
				\textbf{Remarques:} Soit $(F,*')$ un sous-groupe de $(G,*)$
				\begin{liste}
					\item $e_G=e_F$
					\item $F$ est non vide: $e\in F$
					\item Si $x'$ et $x''$ sont les sym\'etriques de $x\in F$ dans $(G,*)$ et $(F,*')$ respectivement,\\
						Alors $x'=x''$
				\end{liste}
			\end{flushleft}
			\begin{flushleft}
				\textbf{Crit\`eres de sous-groupe}\\
				Soit $F$ un sous-ensemble non vide de $G$.
				\begin{liste}
					\item \underline{Crit\`ere $0$:} $F$ est un sous-groupe de $G$ ssi:
						\begin{liste}
							\item[\cercle1] $\forall (x,y)\in G\times G,\ (x\in F\et y\in F)\Rightarrow (x*y\in F)$
							\item[\cercle2] $e\in F$
							\item[\cercle3] $\forall x\in G,\ (x\in F)\Rightarrow(x^-1\in F)$
						\end{liste}\ \\
					\item \underline{Crit\`ere $1$:} $F$ est un sous-groupe de $G$ ssi:
						\begin{liste}
							\item[\cercle1] $\forall (x,y)\in G\times G,\ (x\in F\et y\in F)\Rightarrow (x*y\in F)$
							\item[\cercle2] $\forall x\in G,\ (x\in F)\Rightarrow(x^-1\in F)$
						\end{liste}\ \\
					\item \underline{Crit\`ere $2$:} $F$ est un sous-groupe de $G$ ssi:
						\begin{liste}
							\item[\cercle1] $\forall (x,y)\in G\times G,\ (x\in F\et y\in F)\Rightarrow (x*y^{-1}\in F)$
						\end{liste}
				\end{liste}
			\end{flushleft}
			\begin{preuve}
				D\'emonstration des crit\`eres de sous-groupe
				\begin{liste}
					\item Crit\`ere $1$: Soit $F$ un sous-ensemble non vide de $G$ v\'erifiant le crit\`ere $1$.\\
					D'apr\`es \cercle2, $x^{-1}\in F$\\
					D'apr\`es \cercle1, $x*x^{-1}\in F$\\
					Or $x*x^{_1}=e$, donc $e\in F$
					On a v\'erifi\'e le crit\`ere $0$, donc $(F,*)$ est un sous-groupe de $(G,*)$.
					\item Crit\`ere $2$: Soit $F$ un sous-ensemble non vide de $G$ v\'erifiant le crit\`ere $2$.\\
					\begin{liste}
						\item $F$ est non vide: Soit $x$ un \'el\'ement de $F$\\
						D'apr\`es \cercle1, $x*x^-1\in F\Rightarrow e\in F$\\
						Le point \cercle2 du crit\`ere $0$ est v\'erifi\'e.
						\item D'apr\`es \cercle1 avec $e$ et $x$: $e*x^{-1}\in F\Rightarrow x^{-1}\in F$\\
						Le pont \cercle3 du crit\`ere $0$ est v\'erifi\'e.
						\item Soit $x$ et $y$ deux \'el\'ements de $F$. De plus, $y^{-1}\in F$\\
						donc $x*\left(y^{-1}\right)^{-1}\in F\rightarrow x*y\in F$
					\end{liste}
				\end{liste}
			\end{preuve}
\end{document}