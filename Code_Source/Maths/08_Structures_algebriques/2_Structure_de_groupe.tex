\documentclass[12pt,twoside,a4paper]{article}

\def\chapitre{Structures Alg\'ebriqes}
\author{MPSI 2}
\def\titre{Structure de Groupe}

\usepackage{amsfonts}
\usepackage{amsmath}
\usepackage{amsthm}
\usepackage{changepage}
\usepackage{color}
\usepackage{enumitem}
\usepackage{fancyhdr}
\usepackage{framed}
\usepackage[margin=1in]{geometry}
\usepackage{mathrsfs}
\usepackage{tikz, tkz-tab}
\usepackage{titling}

\newtheoremstyle{dotless}{}{}{\itshape}{}{\bfseries}{}{ }{}
\theoremstyle{dotless}

\newtheorem{defs}{Definition}[subsection]
\newenvironment{defi}{\definecolor{shadecolor}{RGB}{255,236,217}\begin{shaded}\begin{defs}\ \\}{\end{defs}\end{shaded}}

\newtheorem{pro}{Propriete}[subsection]
\newenvironment{prop}{\definecolor{shadecolor}{RGB}{230,230,255}\begin{shaded}\begin{pro}\ \\}{\end{pro}\end{shaded}}

\newtheorem{cor}{Corollaire}[subsection]
\newenvironment{coro}{\definecolor{shadecolor}{RGB}{245,250,255}\begin{shaded}\begin{cor}\ \\}{\end{cor}\end{shaded}}

\setlength{\droptitle}{-1in}
\predate{}
\postdate{}
\date{}
\title{\chapitre\\\titre\vspace{-.25in}}

\pagestyle{fancy}
\makeatletter
\lhead{\chapitre\ - \titre}
\rhead{\@author}
\makeatother

\newenvironment{preuve}{\begin{framed}\begin{proof}[\unskip\nopunct]}{\end{proof}\end{framed}}
\newenvironment{liste}{\begin{itemize}[leftmargin=*,noitemsep, topsep=0pt]}{\end{itemize}}
\newenvironment{tab}{\begin{adjustwidth}{.5cm}{}}{\end{adjustwidth}}

\newcommand{\uu}[1] {_{_{#1}}}
\newcommand{\lbracket}{[\![}
\newcommand{\rbracket}{]\!]}
\newcommand{\fonction}[5]{\begin{aligned}[t]#1\colon&#2&&\longrightarrow#3 \\&#4&&\longmapsto#5\end{aligned}}
\newcommand{\systeme}[1]{\left\{\begin{aligned}#1\end{aligned}\right.}
\newcommand{\cercle}[1]{\textcircled{\scriptsize{#1}}}

\newcommand{\lf}[1]{\left(#1\right)}
\newcommand{\C}{\mathbb{C}}
\newcommand{\R}{\mathbb{R}}
\newcommand{\K}{\mathbb{K}}
\newcommand{\N}{\mathbb{N}}
\newcommand{\I}{\mathcal{I}}
\newcommand{\F}{\mathcal{F}}
\newcommand{\E}{\mathcal{E}}
\newcommand{\G}{\mathcal{G}}
\newcommand{\et}{\text{ et }}
\newcommand{\ou}{\text{ ou }}
\newcommand{\xou}{\ \fbox{\text{ou}}\ }


%Auteur: Cl\'ement Phan, MPSI 2

\begin{document}
	\maketitle
	\section{D\'efinition}
		\begin{defi}
			Soit $(G,*)$ un magma.\\
			On dit que $(G,*)$ est un \underline{groupe} si:
			\begin{liste}
				\item $*$ est associative
				\item $*$ admet un \'el\'ement neutre
				\item tout \'el\'ement de $G$ est sym\'etrisable par $*$
			\end{liste}
			Si de plus, $*$ est commutative sur $G$, on dit que $(G,*)$ est un \underline{groupe ab\'elien}.
		\end{defi}
		\begin{flushleft}
			\textbf{Cons\'equences:}
			\begin{liste}
				\item\underline{R\`egle de simplification:} $a*x=a*y\Rightarrow x=y$\\
					D\'emonstration:
					\begin{tab}
						Soit $a$, $x$ et $y$ trois \'el\'ements de $G$ tels que $a*x=a*y$\\
						Notons $a'$ le sym\'etrique de $a$ (car $G$ est un groupe)\\
						On a alors: $a'*(a*x)=a'*(a*y)$\\
						Par associativit\'e, on a: $(a'*a)*x=(a'*a)*y$\\
						Par sym\'etrie, on a: $e*x=e*y$\\
						Par d\'efinition de l'\'el\'ement neutre: $x=y$						
					\end{tab}
				\item\underline{R\'esolution d'\'equations:} $a*x=b\iff x=a'*b$
			\end{liste}
		\end{flushleft}
	\section{Sous-groupes}
		\subsection{D\'efinition et crit\`eres}
			\begin{flushleft}
				Soit $(G,*)$ un groupe.
			\end{flushleft}
			\begin{defi}
				Soit $F$ un sous-ensemble de $G$\\
				On dit que $(F,*)$ est un sous-groupe de $(G,*)$ si:
				\begin{liste}
					\item $\forall (x,y)\in G\times G,\ (x\in F\et y\in F)\Rightarrow (x*y\in F)$
					\item $(F,*')$ est un groupe o\`u $*'$ est la loi induite de $G$ sur $F$.
				\end{liste}
			\end{defi}
			\begin{flushleft}
				\textbf{Remarques:} Soit $(F,*')$ un sous-groupe de $(G,*)$
				\begin{liste}
					\item $e_G=e_F$
					\item $F$ est non vide: $e\in F$
					\item Si $x'$ et $x''$ sont les sym\'etriques de $x\in F$ dans $(G,*)$ et $(F,*')$ respectivement,\\
						Alors $x'=x''$
				\end{liste}
			\end{flushleft}
			\begin{flushleft}
				\textbf{Crit\`eres de sous-groupe}\\
				Soit $F$ un sous-ensemble non vide de $G$.
				\begin{liste}
					\item \underline{Crit\`ere $0$:} $F$ est un sous-groupe de $G$ ssi:
						\begin{liste}
							\item[\cercle1] $\forall (x,y)\in G\times G,\ (x\in F\et y\in F)\Rightarrow (x*y\in F)$
							\item[\cercle2] $e\in F$
							\item[\cercle3] $\forall x\in G,\ (x\in F)\Rightarrow(x^-1\in F)$
						\end{liste}\ \\
					\item \underline{Crit\`ere $1$:} $F$ est un sous-groupe de $G$ ssi:
						\begin{liste}
							\item[\cercle1] $\forall (x,y)\in G\times G,\ (x\in F\et y\in F)\Rightarrow (x*y\in F)$
							\item[\cercle2] $\forall x\in G,\ (x\in F)\Rightarrow(x^-1\in F)$
						\end{liste}\ \\
					\item \underline{Crit\`ere $2$:} $F$ est un sous-groupe de $G$ ssi:
						\begin{liste}
							\item[\cercle1] $\forall (x,y)\in G\times G,\ (x\in F\et y\in F)\Rightarrow (x*y^{-1}\in F)$
						\end{liste}
				\end{liste}
			\end{flushleft}
			\begin{preuve}
				D\'emonstration des crit\`eres de sous-groupe
				\begin{liste}
					\item Crit\`ere $1$: Soit $F$ un sous-ensemble non vide de $G$ v\'erifiant le crit\`ere $1$.\\
					D'apr\`es \cercle2, $x^{-1}\in F$\\
					D'apr\`es \cercle1, $x*x^{-1}\in F$\\
					Or $x*x^{_1}=e$, donc $e\in F$
					On a v\'erifi\'e le crit\`ere $0$, donc $(F,*)$ est un sous-groupe de $(G,*)$.
					\item Crit\`ere $2$: Soit $F$ un sous-ensemble non vide de $G$ v\'erifiant le crit\`ere $2$.\\
					\begin{liste}
						\item $F$ est non vide: Soit $x$ un \'el\'ement de $F$\\
						D'apr\`es \cercle1, $x*x^-1\in F\Rightarrow e\in F$\\
						Le point \cercle2 du crit\`ere $0$ est v\'erifi\'e.
						\item D'apr\`es \cercle1 avec $e$ et $x$: $e*x^{-1}\in F\Rightarrow x^{-1}\in F$\\
						Le pont \cercle3 du crit\`ere $0$ est v\'erifi\'e.
						\item Soit $x$ et $y$ deux \'el\'ements de $F$. De plus, $y^{-1}\in F$\\
						donc $x*\left(y^{-1}\right)^{-1}\in F\rightarrow x*y\in F$
					\end{liste}
				\end{liste}
			\end{preuve}
		\subsection{Propri\'et\'es des sous-groupes}
			\begin{flushleft}
				Soit $(G,*)$ un groupe.
			\end{flushleft}
			\begin{prop}
				\begin{liste}
					\item Si $F$ et $H$ sont des sous-groupes de $(G,*)$,\\
						Alors $F\cap H$ est un sous-groupe de $(G,*)$.
					\item Si $(F_i)_{i\in I}$ est une famille de sous-groupes de $(G,*)$,\\
						Alors $\bigcap\limits_{i\in I}F_i$ est un sous-groupe de $(G,*)$
				\end{liste}
			\end{prop}
\newpage
			\begin{preuve}
				\begin{liste}
					\item $f\cap H$ est non vide: $e\in F\cap H$
					\item Montrer que $\forall (x,y)\in G\times G,\ (x\in F\cap H)\et(y\in F\cap H)\Rightarrow (x*y^{-1})\in F\cap H$\\
						Soit $(x,y)\in F\cap H$\\
						$x$ et $y$ sont \'el\'ements de $F$ et $F$ est un groupe: $x*y^{-1}\in F$\\
						De m\^eme, $x*y^{-1}\in H$\\
						Donc $x*y^{-1}\in F\cap H$\\
						Vrai pour tout $(x,y)$ de $(F\cap H)^2$, donc on en d\'eduit que $F\cap H$ est un sous-groupe de $(G,*)$
				\end{liste}
			\end{preuve}
			\begin{defi}
				Soit $(G,*)$ un groupe.\\
				Soit $B$ une partie non vide de $G$.\\
				On appelle \underline{sous-groupe de $g$ engendr\'e par $B$} le plus petit sous groupe de $(G,*)$ contenant $B$, au sens de l'inclusion.
			\end{defi}
			\begin{flushleft}
				\textbf{Justification:} On munit $\mathcal{P}(G)$ de la relation d'inclusion: $(\mathcal{P}(G),\subset)$ est un ensemble ordonn\'e.\\
				Soit $\mathcal{F}$ l'ensemble des groupes contenant $B$.
				\begin{liste}
					\item $\mathcal{F}$ est non vide car $G\in\mathcal{F}$
					\item D'apr\`es la propri\'et\'e pr\'ec\'edente, $H=\bigcap\limits_{F\in\mathcal{F}}F$ est un sous-groupe de $(G,*)$.\\
						$\forall f\in\mathcal{F},\ b\subset F$, donc $B\subset H$
					\item Reste \`a montrer que $H$ est le plus petit \'el\'ement de $\mathcal{F}$\\
						Donc montrer que $\forall F\in\mathcal{F},\ H\subset F$\\
						Par d\'efinition de $H$, la proposition pr\'ec\'edente est vraie.
				\end{liste}
			\end{flushleft}
			\begin{flushleft}
				\textbf{Notation:}\\
				\textbullet\ $<B>$\\\textbullet\ Si $B$ est un singleton $b=\{b\}$, on \'ecrit $<b>$
			\end{flushleft}
\newpage
		\subsection{Morphismes de groupes}
			Soit $(G,*)$ et $(G',*')$ deux groupes.\\
			Soit $f:G\rightarrow G'$ une application.
			\begin{defi}
				\begin{liste}
					\item $f$ est un \underline{homomorphisme de groupes} si:\\
					$\forall(x,y)\in G\times G,\ f(x*y)=f(x)*'f(y)$
					\item $f$ est un \underline{endomorphisme de groupes} si:\\
					$f$ est un homomorphisme de groupes et que $(G,*)=(G',*')$
					\item $f$ est un \underline{isomorphisme de groupes} si:\\
					$f$ est un homomorphisme de groupes bijectif de $G$ sur $G'$
					\item $f$ est un \underline{automorphisme de groupes} si:\\
					$f$ est un endomorphisme de groupes bijectif de $G$ sur $G'$
				\end{liste}
			\end{defi}
			\begin{flushleft}
				\textbf{Remarques:} Soit $f$ un homomorphisme de groupes de $G$ dans $G'$\\
				\begin{liste}
					\item Soit $e$ et $e'$ les \'el\'ements neutres de $G$ et $G'$ respectivement.\\
						Alors $f(e*e)=f(e)*'f(e)$\\
						En utilisant le sym\'etrique de $f(e)$ a gauche, on obtient:\\
						$f(e)^{-1}*'f(e)=f(e)^{-1}*'f(e)*'f'(e)$\\
						$\iff e'=f(e)$
					\item Soit $x$ un \'el\'ement de $G$.\\
						$f(x*x^{-1})=f(x)*'f(x^{-1})$\\
						Par ailleurs, $f(x*x^{-1})=f(e)=e'$\\
						Donc $f(x)*'f(x^{-1})=e$\\
						On en d\'eduit que $f(x^{-1})=f(x)^{-1}$
				\end{liste}
			\end{flushleft}
			\begin{prop}
				\begin{liste}
					\item Si $F$ est un sous-groupe de $(G,*)$ et si $f$ est un homomorphisme de groupes de $G$ dans $G'$,\\
					Alors $f(F)$ est un sous-groupe de $(G',*')$
					\item Si $H'$ est un sous-groupe de $(G',*')$ et si $f$ est un homomorphisme de groupes de $G$ dans $G'$,\\
					Alors $f^{-1}<H'>$ est un sous-groupe de $(G,*)$
				\end{liste}
			\end{prop}
			\begin{preuve}
				\begin{liste}
					\item \underline{$1^{\text{er}}$ point:} Montrer que $f(F)$ est un sous-groupe de $(G',*')$\\
						Donc montrer que $\forall(x',y')\in f(F),\ x'*'y'^{-1}\in f(F)$\\
						$F$ non vide, donc $f(F)$ non vide.\\
						Soit $x'$ et $y'$ deux \'el\'ements de $f(F)$.\\
						Donc $\exists (x,y)\in F,\ f(x)=x'\et f(y)=y'$. Soit $x$ et $y$ deux tels \'el\'ements.\\
						$\begin{aligned}x'*'(y')^{-1}&=f(x)*'f(y)^{-1}\\&=f(x*y^{-1})\end{aligned}$\\
						Or, $F$ est un groupe, donc $x*y^{-1}\in F$, donc $x'*'(y')^{-1}\in f(F)$\\
						Cela \'etant vrai pour tout $x'$ et $y'$ de $f(F)$, on en conclut que $f(F)$ est un sous-groupe de $(G',*')$.
					\item \underline{$2^{\text{\`eme}}$ point:} Montrer que $f^{-1}<H'>$ est un sous-groupe de $(g,*)$.\\
						\begin{liste}
							\item $f$ est un homomorphisme de groupes donc $f(e)=e'$\\
								De plus, $e'\in H'$ car $H'$ est un sous-groupe de $(G,*)$\\
								Donc $e\in f^{-1}<H'>$, donc $f^{-1}<H'>$ est non vide.
							\item Montrer que $\forall (x,y)\in G\times G,\ (x\in f^{-1}<H'>\et x\in f^{-1}<H'>)\Rightarrow x*y^{-1}\in f^{-1}<H'>$
								Soit $x$ et $y$ deux \'el\'ements de $f^{-1}<H'>$.\\
								Donc montrons que $x*y^{-1}\in f^{-1}<H'>$\\
								Donc montrons que $f(x*y^{-1})\in H'$\\
								$f(x*y^{-1})=f(x)*f(y)^{-1}$ car $f$ est un homomorphisme de groupes .\\
								Or: 
								$\begin{aligned}[t]
								&\text{- }f(x)\in H'\text{ car }x\in f^{-1}<H'>\\
								&\text{- }f(y)\in H'\text{ car }y\in f^{-1}<H'>\text{.}\\
								&\text{- De plus, }H'\text{ est un sous-groupe donc }f(y)^{-1}\in H'
								\end{aligned}$\\
								Sachant $H'$ un sous-groupe, $f(x*y^{-1})\in H'$\\
								Ce raisonnement \'etant valable pour tout $x,y$ de $f^{-1}<H'>$, on en d\'eduit que $f^{-1}<H'>$ est un sous-groupe de $(G,*)$ d'apr\`es la crit\`ere $2$.
						\end{liste} 
				\end{liste}
			\end{preuve}
			\begin{coro}
				\begin{liste}
					\item L'image $f(G)$ est un sous-groupe de $(G',*')$
					\item $f^{-1}<\{e\}>$ est un sous-groupe de $(G,*)$. On l'appelle le \underline{noyau de $f$}
				\end{liste}
			\end{coro}
			\begin{flushleft}
				\textbf{Notation:} $\ker(f)=f^{-1}<\{e\}>$
			\end{flushleft}
			\begin{prop}
				Soit $f$ un homomorphisme de groupes de $(G,*)$ dans $(G',*')$.\\
				Alors $f$ est injective si et seulement si $\ker(f)=\{e\}$
			\end{prop}
\newpage
			\begin{preuve}
				\begin{liste}
					\item[\cercle1]Supposons $f$ injective.\\
						Montrer que $\ker(f)=\{e\}$\\
						\begin{liste}
							\item $\ker(f)$ est un sous-groupe de $(G,*)$, donc $\{e\}\subset\ker(f)$
							\item montrer que $\ker(f)\subset \{e\}$\\
								Donc montrer que $\forall x\in G,\ x\in\ker(f)\Rightarrow x=e$\\
								Soit $x$ un \'el\'ement de $\ker(f)$\\
								Montrer que $x=e$\\
								$x\in\ker(f)$, donc $f(x)=e'$\\
								$f$ est un homomorphisme, donc $f(e)=e'$\\
								D'o\`u $f(x)=f(e)$\\
								Donc, sachant $f$ injective, $x=e$
						\end{liste}
						Donc $\ker(f)=\{e\}$
					\item[\cercle2]Supposons que $\ker(f)=\{e\}$\\
						Montrer que $\forall(x,y)\in G\times G,\ f(x)=f(y)\Rightarrow x=y$\\
						Soit $x$ et $y$ deux \'el\'ements de $G$ tels que $f(x)=f(y)$\\
						$\begin{aligned}
							f(x)=f(y)&\iff f(x)*'f(y)^{-1}=e'\\
								&\iff f(x*y^{-1})=e'
						\end{aligned}$\\
						Or, $\ker(f)=\{e\}$\\
						Donc $x*y^{-1}=e$\\
						Donc $x=y$\\
						Ce raisonnement \'etant valable pour tout $x$ et $y$ de $G$, $f$ est injective.
				\end{liste}
			\end{preuve}
\end{document}